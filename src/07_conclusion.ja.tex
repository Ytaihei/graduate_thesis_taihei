\chapter{結論と展望}
\label{chap:conclusion}
\section{本研究のまとめ}
本研究では, ContactPlanの更新はDTNにおけるBundleをよりよく配達できるように
する効果があることを確認するとともに, その効果は更新の範囲を天体内に限定した場合にも
十分に有効であることが示された. また天体間にも拡散する既存手法の場合, 
提案手法よりも天体間通信リンクを消費することが確認できた. 地球のインターネットにおける
Byte数を考慮するとかなり小さい値ではあるものの, リソースが制約された宇宙ノードにおいては
このリンク消費を必要としないことはメリットとなりうる. 

\section{今後の課題と展望}
本研究ではシミュレーションに用いるトポロジーを抽象化しており, 
さらにそれぞれのノードのContactはランダムに生成を行っている. 
今後は, 衛星のContactのタイミングについて軌道シミュレーターを用いるなど, 
より実際の宇宙環境におけるトポロジーを考慮したシミュレーションを行うことが必要である. 
また本研究では単一のContactの失敗のみに着目しているが, 
実際のDTNの運用においては, 単一のContactの失敗だけでなく, 
長時間にわたる単一のContactの一定時間だけの障害や, 
複数のContactの連続した失敗なども考慮する必要がある. 
これらまで対象を拡大し, 包括的な検証を行うことが今後の課題である. 
さらに, 本研究はContact Planの臨時更新を天体内に限定することを提案したが, 
今後DTNの本格的な運用に置いては, \ref{section:Contact Planの定期的・継続的な配布}節の
Contact Planの定期的・継続的な更新もさらなる研究・標準化が必要である. 
定期的・継続的な更新においてはContact Planのサイズ増加に対する対策が必要であり, 
天体ごとに管理を行うという本研究の発想は, この問題にも適用できる可能性がある. 
