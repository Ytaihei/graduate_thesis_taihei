\chapter{結論と展望}
% --- 章アウトライン・TODO・参考文献引用例 ---
% この章では、研究のまとめ・今後の課題・展望を記述する。
% TODO: docker_open5gs_sXGP-5G の成果と今後の発展可能性を記述。
% TODO: 参考文献を本文中で引用する(例: \cite{artemis_agreement3})。
% 例: 今後の宇宙通信の展望は \cite{artemis_agreement3} などを参照。
% ------------------------------------------
\label{chap:conclusion}

\section{本研究のまとめ}
本研究では、実装ベース標準化を支援する実機検証環境として、免許不要帯で運用可能なsXGP(TD-LTE互換)をeNBとして活用し、4G RAN(UE・eNB)と5G Core(5GC)を接続するコンバータを実装した。本環境により、電波法上の制約がある中でも法令遵守の範囲で実機検証が可能となり、標準仕様と実装の乖離を早期に検出する基盤を提供した。

具体的には、以下の成果を達成した:

\begin{itemize}
	\item \textbf{プロトコル変換機能の実装}: S1AP/NGAP/NASの信令変換とGTP-Uトンネル中継により、4G RANと5GC間の相互接続を実現した。登録・PDUセッション確立・データ転送などの基本機能が正常に動作することを確認した。

	\item \textbf{実装レベルの検証項目の確立}: PCOパラメータ処理、QoS/5QI分離、SSCモード設定など、コンバータレベルで実装した相互運用性検証項目を動作確認した。さらに、IPv6-only + NAT64/464XLAT、UL-CL、GTP-U PMTUD、SMS over NAS、Paging、複数DNNなど、標準仕様は存在するが実装の成熟度や相互運用性に課題がある項目について、実機を用いた検証が可能な基盤を構築した。

	\item \textbf{再現性の高い実験環境の整備}: Docker化により、環境構築手順を標準化し、異なる環境でも同一の実験を再現可能にした。これにより、継続的インテグレーション・回帰テストへの適用が可能となる。

	\item \textbf{実機検証の有効性の実証}: 実UEを用いることで、シミュレータでは再現困難な実機特有の挙動(端末OS依存の処理、NAS再送シーケンス、無線条件依存の挙動など)を検証できることを示した。
\end{itemize}

本研究の貢献は、3GPPの仕様ファースト型標準化プロセスを補完する形で、IETFの"rough consensus and running code"原則に基づく実装ベース標準化の実現可能性を示したことにある。ソフトウェア中心の5GC実装が普及する中で、既存4G RAN資産を活用しつつ次世代コアへの移行を支援する実践的アプローチを確立した。

\section{今後の課題}

本研究で構築した環境は、実装ベース標準化の基盤として有効性を示したが、以下の課題と発展方向が存在する:

\subsection{5G NR対応への拡張}
現在の実装はLTE互換のsXGPをRANとして用いているため、5G NR特有のPHY機能(ビームフォーミング、massive MIMO等)やスケジューリング最適化は対象外である。今後、免許不要帯で運用可能な5G NR基地局(sub-6GHz帯またはミリ波帯)が利用可能になれば、NR固有の無線レイヤー機能の検証にも対応できる。ただし、現時点でも制御面(NGAP/NAS)およびユーザ面(GTP-U)のプロトコルレベルでの相互運用性検証は十分に可能であり、標準化フィードバックの主要な価値はこのレイヤーにある。

\subsection{より高度な検証ケースの実装}
本研究で提案した10項目の検証ケース(Case A~J)のうち、コンバータレベルで実装済みの項目(PCO、QoS/5QI、SSC mode)については動作確認を行ったが、UL-CL、DSCP/ECN保存、GTP-U PMTUDなどの高度な機能については、Open5GSやUPFの拡張が必要である。今後、これらの機能を段階的に実装し、実機環境での検証を進めることで、標準化団体やOSSプロジェクトへのより具体的なフィードバックが可能となる。

\subsection{標準化への実際のフィードバック実績}
本研究では実機検証環境の構築と基本的な動作確認を行ったが、実際に標準化団体(3GPP)やOSSプロジェクト(Open5GS等)へのフィードバックを行い、その効果を定量的に評価することは今後の課題である。検出した相互運用性問題を再現手順・パケットキャプチャとともにドキュメント化し、標準化議論やOSS開発に貢献することで、実装ベース標準化の実効性を実証する必要がある。

\subsection{自動テスト・CI/CD環境への統合}
現在の実験環境は手動での実行を前提としているが、継続的インテグレーション(CI)環境に統合し、コード変更のたびに自動的に実機検証を実行する仕組みを構築することで、回帰テストと品質保証を強化できる。GitHub ActionsやGitLab CI等のCI/CDプラットフォームとの連携により、標準化とOSS開発の両面でアジャイルな開発サイクルを支援する。

\subsection{6G時代への展望}
6G時代に向けては、標準化サイクルのさらなる高速化が求められる。本研究で確立した実装ベース検証の手法は、6G標準化においても有効であり、仕様策定の初期段階から実装を動かしながら標準化を進めるアプローチを促進する。特に、AIネイティブな通信システムやテラヘルツ帯通信など、新しい技術要素が導入される6Gにおいて、実機検証と標準化の並行進行は不可欠である。免許不要帯を活用した研究・教育基盤として、6G標準化における実装ベース検証の促進に貢献することが期待される。
