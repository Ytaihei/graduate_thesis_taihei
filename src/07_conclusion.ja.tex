\chapter{結論と展望}
\label{chap:conclusion}
\section{本研究のまとめ}
近年の宇宙開発の進展に伴い宇宙インターネットが必要とされ, 
DTNがその中心的なコンセプトとして考えられている. 
DTNではCGRをベースとした経路制御が考えられているが, 
その運用に必要なContact Planの定期的・継続的な配布に関する手法, 
予定されたContactに何かしらの障害が生じた際のContact Planの臨時更新に
関する技術の標準化はなされておらず, 今後宇宙インターネットの事業者などが
参入する際の障壁となる.

BezirgiannidisらによってContact Planの臨時更新の有効性は示されており,
到達遅延の低減などに寄与することが明らかになっているが,
この手法では天体間にまたがる全てのノードに情報を拡散することが前提となっており, 
天体間の貴重なリンクを消費する. これに対して
本研究ではContact Planの臨時更新を行う対象をContactの失敗が起きた天体内の
ノードに限定することを提案しており, この手法であれば天体間の情報拡散は行わないため
臨時更新に伴う天体間のリンク消費は全くない. 

\ref{section:要件2に対する更新メッセージによるリンク消費}節で述べた通り, 
既存手法における臨時更新のメッセージサイズは比較的小さなものになるということがわかっているが,
これはノードのエネルギーリソースが制約された宇宙ノードにおいては
可能な限り避けられるべきものであり, さらに運用するDTNのノード数やそれに伴う
Contactの失敗回数が増加すればより多い回数の臨時更新メッセージの送信が発生しその分天体間のリンクが消費される. 

一方で臨時更新のそもそもの目的である, DTNの最新の状況を各ノードに認識させ配送能力を向上させる点については, 
\ref{section:地球・月間のシミュレーション結果に対する考察}項で
述べたとおり地球-月間のシミュレーションでは到達率が完全一致, 到達遅延もほぼ一致する結果が得られ, 
\ref{section:地球・火星間のシミュレーション結果に対する考察}項で
述べたとおり地球-火星間のシミュレーションでは到達率・到達遅延ともに完全一致する結果が得られた. 
これにより, Contact Planの更新はDTNにおけるBundleをよりよく配送できるように
する効果は, 更新の範囲を天体内に限定した場合にも
十分に有効であることが示された. 

よって本研究で提案した手法は, 既存手法に比べて天体間のリンク消費を抑えることができ,
かつContact Planの臨時更新による効果を損なうことのない手法であることが示された.

\section{今後の課題と展望}
本研究ではシミュレーションに用いるトポロジーを抽象化しており, 
さらにそれぞれのノードのContactはランダムに生成を行っている. 
今後は, 衛星のContactのタイミングについて軌道シミュレーターを用いるなど, 
より実際の宇宙環境におけるトポロジーを考慮したシミュレーションを行うことが必要である. 
また本研究では単一のContactの失敗のみに着目しているが, 
実際のDTNの運用においては, 単一のContactの失敗だけでなく, 
長時間にわたる単一のContactの一定時間だけの障害や, 
複数のContactの連続した失敗なども考慮する必要がある. 
これらまで対象を拡大し, 包括的な検証を行うことが今後の課題である. 
さらに, 本研究はContact Planの臨時更新を天体内に限定することを提案したが, 
今後DTNの本格的な運用に置いては, \ref{section:Contact Planの定期的・継続的な配布}節の
Contact Planの定期的・継続的な更新もさらなる研究・標準化が必要である. 
定期的・継続的な更新においてはContact Planのサイズ増加に対する対策が必要であり, 
天体ごとに管理を行うという本研究の発想は, この問題にも適用できる可能性がある. 
