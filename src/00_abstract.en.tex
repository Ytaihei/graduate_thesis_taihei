Abstract of Bachelor's Thesis - Academic Year 2024
\begin{center}
\begin{large}
\begin{tabular}{|p{0.97\linewidth}|}
    \hline
      \etitle \\
    \hline
\end{tabular}
\end{large}
\end{center}

~ \\
In this paper, we propose a new concept of contact plan update required for Contact Graph Routing (CGR) in Delay/Disruption Tolerant Networks (DTN) in space. Contact Plan update required for Contact Graph Routing (CGR), which is a basic concept of routing We propose to limit the propagation of the contact plan in the case of contact failure. Space development has made great progress in recent years, and in particular, the NASA-centered ARTEMIS project is planning to establish a manned base on the Moon and Mars from the late 2020s onward. In particular, the NASA-centered ARTEMIS project plans to construct manned bases and orbiting stations on the Moon and Mars in the late 2020s and beyond, and to establish a space station on the Moon and Mars in the 2030s. The number of communication nodes on the Moon and Mars is expected to increase after the 2030s. However, the current communication with extraterrestrial nodes is a direct one-to-one communication with ground antennas. This is not possible when the number of nodes on the Moon and Mars increases in the future. Therefore, we should not rely only on direct communication. The need for a communication network using space nodes such as satellites, i.e., the Space Internet, has been recognized, and the existing Internet technologies are being used for the communication network. The application of existing Internet technologies to such a communication network is being considered. However, from the viewpoint of communication, the space environment is subject to large latency and frequent disconnection. The above-mentioned concept of DTN has been conceived. Therefore, the concept of DTN is proposed as described above. In addition, in the space Internet, most of the nodes that constitute the network are spacecrafts, and the Although the communication links change with time, the timing of the changes can be predicted by orbital calculations. For this reason, the routing in the DTN is basically planned to be based on CGR as described above. In CGR, each node is responsible for all satellites including itself. Contact Plan, which is a list of possible communication opportunities (Contact) between two nodes, is maintained in advance for all satellites including the node itself. Based on the Contact Plan, a route is calculated by an algorithm to determine the DTN node that will be the next hop and performs forwarding. Therefore, it is necessary that the Contact Plan is properly distributed to the nodes and updated as necessary. However, there is no standardization on the method of contact plan distribution. In this study, the contact plan is updated periodically based on This study categorizes Contact Plan updates into periodic/continuous distribution and occasional updates based on the timing of when updates are necessary, and classifies them into the following categories The advantages of occasional updates in the routing of DTNs have already been clarified in previous studies. However, in the previous studies, the information propagation of the occasional update is also performed between different inter-object networks. In this study, we propose to keep this propagation within a celestial body. Simulation results show that the proposed method in this study is as effective as the method in the previous study. The simulation results show that the proposed method is as effective as the method used in the previous study, and that the operational disadvantages of the information diffusion among celestial objects are avoided. The simulation results show that the proposed method is as effective as the previous method, and that it is possible to perform effective temporary updating while avoiding operational disadvantages caused by the information diffusion among the celestial bodies.
~ \\
Keywords : \\
\underline{1. Delay/Disruption Tolerant Network} 
\underline{2. Contact Graph Routing} 
\begin{flushright}
\edept \\
\eauthor
\end{flushright}