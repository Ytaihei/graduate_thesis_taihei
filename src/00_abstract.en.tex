Abstract of Bachelor's Thesis - Academic Year 2024
\begin{center}
\begin{large}
\begin{tabular}{|p{0.97\linewidth}|}
    \hline
      \etitle \\
    \hline
\end{tabular}
\end{large}
\end{center}

~ \\
This paper is concerned with updating the Contact Plan required for Contact Graph Routing (CGR), which is the basic concept of routing in the Delay/Disruption Tolerant Network (DTN) in space, We propose to limit the propagation of information within a celestial body in the event of a link failure. Space development has made great progress in recent years, and in particular, the construction of manned bases and orbiting stations on the Moon and Mars is planned after the late 2020s under the NASA-centered Artemis Project, and the number of nodes communicating to the Moon and Mars is expected to increase sequentially after the 2030s. However, the current communication with extraterrestrial nodes is direct one-to-one communication with ground antennas, which cannot cope with the increase in the number of nodes on the Moon and Mars in the future. Therefore, the need for a communication network using space nodes such as satellites, i.e., the Space Internet, rather than relying only on direct communication, has been recognized, and the application of existing Internet technologies to such a communication network is being considered. However, from a communications perspective, the space environment is subject to large delays and frequent disconnections, and the assumptions underlying the existing Internet on Earth are very different, so the concept of the DTN described above is being considered. In the space Internet, many of the nodes that make up the network are spacecraft, and although the links that can communicate with each other change with time, the timing of such changes can be predicted by orbital calculations. For this reason, routing in the DTN is basically planned to use CGR as described above. In CGR, each node maintains in advance a Contact Plan, which is a list of possible communication opportunities (Contacts) between two specific nodes for all satellites including itself, and calculates routes based on this Contact Plan using an algorithm to determine the DTN node that will be the next hop and performs forwarding. Therefore, it is necessary for the Contact Plan to be properly distributed to the nodes and updated as necessary, but there is no standardization of the distribution method. In this study, contact plan updates are classified into periodic updates and occasional updates based on the timing of when updates are necessary, and the benefits of occasional updates for DTN routing have already been clarified in previous studies. However, previous studies have also propagated the information of occasional updates between different inter-object networks. We propose to keep this within a celestial body. 
~ \\
Keywords : \\
\underline{1. Delay/Disruption Tolerant Network} 
\underline{2. Contact Graph Routing} 
\begin{flushright}
\edept \\
\eauthor
\end{flushright}