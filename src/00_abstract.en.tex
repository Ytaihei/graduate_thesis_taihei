Abstract of Bachelor's Thesis - Academic Year 2025
\begin{center}
\begin{large}
\begin{tabular}{|p{0.97\linewidth}|}
    \hline
      \etitle \\
    \hline
\end{tabular}
\end{large}
\end{center}
\begin{spacing}{1.1}

This thesis presents a practical and reproducible 5G experimentation environment by leveraging sXGP (TD-LTE compatible, license-free band in Japan) as an eNB and implementing a converter that connects a 4G RAN (UE/eNB) to a 5G Core (5GC). Traditional 5G research often requires licensed spectrum and expensive RAN equipment, which raises the barrier to entry for universities, labs, and small teams. Our contributions are threefold: (1) building an sXGP-based RAN, (2) designing and implementing a converter that handles S1AP/NGAP/NAS signaling and relays GTP-U, and (3) establishing reproducible build and experiment procedures via Docker.

We validate basic functions such as Registration and PDU session establishment and provide a measurement toolkit for latency, throughput, and resource utilization. The proposed environment benefits several use cases, including (a) education and training, (b) rapid prototyping of protocol conversion, (c) interoperability and regression testing, (d) performance evaluation and bottleneck analysis, (e) reproducible research and collaboration, and (f) operational studies and fault scenario reproduction. The environment facilitates legal, low-cost, and hands-on 5G research while keeping proximity to realistic operational settings.

\end{spacing}
Keywords : \\
\underline{1. Mobile Systems}
\underline{2. sXGP}
\underline{3. 5G Core}
\underline{4. Open5GS}
\underline{5. GTP-U}
\begin{flushright}
\edept \\
\eauthor
\end{flushright}
