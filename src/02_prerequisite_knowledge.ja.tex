\chapter{前提知識}
\label{chap:prerequisite-knowledgement}
本章では,
\section{Service Function Chaining}
\label{section:sfc}

SFC とは,エンドツーエンド通信を提供するために必要なさまざまなサービスファンクション (SF) を決定及び順序付けし,それらを介するようにトラフィックを操作することを指す.
図~\ref{fig:sfc-arc} に,SFC アーキテクチャの概略図を示す.
SF には,ファイアウォールや IP ネットワークアドレストランスレータ (NAT) などのネットワークサービスファンクションや,アプリケーション固有の機能が含まれる.
SFC アーキテクチャは,基礎となるネットワークトポロジから独立したトポロジを前提としている.
この基礎となるネットワークトポロジをアンダーレイネットワークといい,独立した SFC のためのネットワークトポロジをオーバレイネットワークという.
図~\ref{fig:sfc-arc} では灰点線のパスが物理的なパスであるアンダーレイネットワークを示し,その他の実線が SFC として選択可能なパスの例であるオーバレイネットワークを示している.
SFC アーキテクチャでは,パケットは通信の入口となるノードで事前に定義されたポリシとパケット内の情報から分類され,SFC 対応ドメイン内で適用する SF のセットを決める.
その後,任意の順番で各 SF でパケット処理が適用されるように転送される.
例えば,Logging をした後 FW Service を適用,Filtering を適用すような場合は,図~\ref{fig:sfc-arc} における緑のパスを辿るようになる.

SFC アーキテクチャはネットワークの用途や運用計画などのコンテキストに依存しない汎用的な場面で利用可能な技術であり,
SFC アーキテクチャは固定ネットワークやモバイルネットワーク,多くのデータセンターアプリケーションに適用できる.
SFC の構築に関して,すべての SF が満たさなければならない標準の定義や特性は存在しない.
各 SF は単に「パケットに対する特定の処理を適用できる要素」として扱われ,特定のネットワークで常に有効な SF を静的に列挙することはできない.
なぜなら,適用する SF の集合はその瞬間に有効な SF であり,それらが有効かどうかはその時々のネットワーク環境によって異なる場合があるからである.
SF のチェインとそれらを呼び出す基準は,SF 対応ドメインを運用する各ネットワーク毎に固有である.

SFC における SF は,受信したパケットの特定の処理を担当する機能である.
SF はプロトコルスタックのさまざまなレイヤで動作し,論理的なコンポーネントとして仮想要素として実現されることもあれば,物理的な筐体としてネットワークの中に組み込まれることもある.
近年では,SF が動作するマシンは物理的な筐体ではなく,汎用マシンにインストールされたハイパーバイザ上の VM の中で動作することも多い.
また,コンテナ技術の台頭により,SF 自体をコンテナに閉じ込めてデプロイすることも一般的になっている.
さらに,それらのコンテナを kubernetes~\cite{k8s} に代表されるコンテナオーケストレータによって管理する手法も提案されている~\cite{sfc-with-k8s}
このように,仮想化されたネットワーク上の機能 (Network Function / NF) を NFV という.
SFC は NFV の利用例として NFV のコンテキストでも研究されてきた技術である.

\begin{figure}[t]
    \centering
    \includegraphics[width=0.95\linewidth]{img/SFC.pdf}
    \caption{SFC アーキテクチャ}
    \label{fig:sfc-arc}
\end{figure}

\section{従来のパケットルーティングとトラフィックエンジニアリング}
\label{section:src-rtng}

章\ref*{section:sfc} で述べた通り,SFC を実現するためには,エンドツーエンド通信を提供するために必要な複数の SF を決定,及び順序付けし,それらを介するようにトラフィックを操作する必要がある.
しかし,従来のパケットルーティングでこのようなトラフィック操作を実現することは難しい.

ルータが IP パケットを転送するとき,ルータは自身の持つルーティングテーブルを参照する.
このルーティングテーブルは通常,BGP~\cite{rfc4271} や OSPF~\cite{rfc2328},IS-IS~\cite{rfc1142} などのルーティングプロトコルを通じて交換した経路情報から作成される.
一般的に,多くのルーティングプロトコルでは「経由するノードの数を最小にする経路を最も良いものとする」という基本設計をもとに,オペレータが任意に決定したコスト情報などを含めて最も良い経路を計算する.
このように決定された最も良い経路をベストパスといい,ルーティングテーブルには「ある宛先アドレスを持つパケットは次にどのノードに転送するのがベストパスなのか」が書かれている.
ルータは,自身に接続されているノードやルーティングプロトコルを通じて受け取った経路情報が変更されたとき,その変更を近接ルータに通知し,自身のルーティングテーブルを更新する.
ルーティングテーブルはオペレータが静的に構築することもできる.しかし,静的に経路を決定してしまうとノードの近接情報が変わるたびにオペレータ自身が設定し直す必要があり,これは手間がかかったりオペレーションミスを誘発したりする問題がある.
そのため,静的な経路設定が利用される場面は限定的である.

図~\ref*{fig:exp-src-rtng} に,あるネットワークのトポロジを示す.
このネットワークにおいて,User-B の通信は青いパスを通るように,User-A から Server に向かう通信は Security appliance を経由させるような経路制御を行いたい.
しかし,Router-B から Router-C までの経路は青いパスを通ると 1 hop だが,Security appliance を通る赤い経路は 2 hop である.
つまり,Router-B から Server までの経路は青いパスの方が経由するノードの数が少ない.
そのため,通常のルーティングプロトコルで経路を学習すると,Router-B のルーティングテーブルには Server に向かう経路として青いパスがベストパスとして採択される.
ルーティングプロトコルの設定でコストを変更することで赤いパスをベストパスにすることは可能である.
ベストパスを赤いパスに変更すると,User-A の通信は意図通り赤いパスを通るようになる.
しかし,ベストパスを変更してしまうと User-B の通信についても赤いパスを通るようになり,これは意図した通信経路にならない.

ベストパスによらずに,また特定の通信毎に選択して経路を制御することを,トラフィックステアリング,トラフィックエンジニアリング (TE) という.
TE が可能なパケット転送メカニズムとして,いくつかの候補が存在する.
例えば OpenFlow~\cite{openflow},Network Service Header (NSH)~\cite{rfc8300},MPLS~\cite{rfc3031}などである.

\begin{figure}[t]
    \centering
    \includegraphics[width=0.95\linewidth]{img/ExplainSrcRtng.pdf}
    \caption{トラフィックステアリングが必要なネットワーク例}
    \label{fig:exp-src-rtng}
\end{figure}

\subsection{OpenFlow}
\label{sbsection:openflow}
OpenFlow のによるネットワーク構成の概略図を 図~\ref{fig:openflow} に示す.
OpenFlow では,OpenFlow スイッチと呼ばれる OpenFlow に対応した専用のネットワーク機器をパケットを転送するノードとして使用し,OpenFlow コントローラと呼ばれる専用のマシンが OpenFlow スイッチの経路情報を集中管理する.
OpenFlow のアーキテクチャは一般のパケットルーティングアーキテクチャとは大きく異なる.
一般的なネットワークでは先に挙げた BGP や OSPF などのルーティングプロトコルを利用して経路情報を交換し,ルーティングテーブルを作成する.
そして,ルータは自身が作成したルーティングテーブルに基づいてパケットを転送する.
対照的に,OpenFlow では BGP や OSPF などのルーティングプロトコルを利用してルーティングテーブルを作成することはしない.
経路情報は OpenFlow コントローラが集中管理し,OpenFlow コントローラは自身が決定した経路情報を実際にパケットを転送する OpenFlow スイッチへインストールする.
また,OpenFlow ではルーティングテーブル自体も一般的なルーティングで作成されるものとは異なり,OpenFlow で利用されるルーティングテーブルに対応するテーブルのことをフローテーブルという.
OpenFlow のフローテーブルには,宛先の IP アドレスだけではなく,送信元アドレスや通信を受信したポート,独自定義の専用パケットヘッダのフィールドなどもマッチングルールとして含まれる.
これにより,IP 的なベストパスによらずに,かつ同じ宛先であっても別のパスを選択できる.

\begin{figure}[t]
    \centering
    \includegraphics[width=0.95\linewidth]{img/OpenFlow.pdf}
    \caption{OpenFlow アーキテクチャ}
    \label{fig:openflow}
\end{figure}

\subsection{NSH}
\label{sbsection:nsh}
NSH は,SFC を実現する 1 つの手法として考えられたプロトコルで,NSH と呼ばれるヘッダでパケットをカプセル化する.
OpenFlow が汎用的なパケット転送アーキテクチャとして考案されたのとは対象的に,NSH は SFC を前提として考案された.
NSH で転送されるパケットの構造を図~\ref*{fig:nsh}として示す.
NSH パケットは,大きく分けて 3 のパートに分けられる.
Original Packet は,実際のエンドツーエンドでやり取りするパケットのことを指す.
そのパケットを,NSH でカプセル化している.
NSH の中にはいくつかのフィールドが存在し,その中には SFC の中でどのパスを通過するのか通過するのかや,ユーザ定義のメタデータ,オプショナルなフィールドなどが含まれる.
NSH でカプセル化されたパケットを,更に Transport Encapsulation というヘッダがカプセル化している.
この Transport Encapsulation は特定のフォーマットである必要はなく,GRE,VXLAN などの一般的なトンネリングプロトコルや通常のイーサフレームである.
Transport Encapsulation の目的は,オーバーレイネットワークを通じて適切な SF ノードまでパケットを転送することである.
NSH に対応したノードでは,SF が適用されたパケットに対して,そのパケットの NSH を参照し次の SF のノードを決定する.
次の SF ノードが決まったらそのノードに届くよう,対応する Transport Encapsulation でカプセル化することで TE ができる.
\begin{figure}[t]
    \centering
    \includegraphics[width=0.95\linewidth]{img/nsh.pdf}
    \caption{NSH のパケットヘッダ構造}
    \label{fig:nsh}
\end{figure}

\subsection{MPLS}
\label{sbsection:mpls}
MPLS とは Multiprotocol Label Switching の略称であり,MPLS ヘッダに含まれるラベル情報に基づいてパケットを転送するプロトコルである.
MPLS ヘッダは Layer 2 ヘッダと Layer 3 ヘッダの間に挿入され,MPLS ヘッダの中にラベル情報を始めとするいくつかのフィールドが埋め込まれる.
MPLS では,宛先の IP アドレスではなく,MPLS ヘッダに含まれるラベル情報を参照してパケットを転送する.
つまり MPLS では,MPLS ヘッダ内に埋め込まれたタグ情報を使ってパケットを転送するため,IP 的なベストパスによらないルールでパケットを転送できる.
また,MPLS は古くから VPN を構成するために用いられてきた一般的なプロトコルである.
そのため,多くのネットワーク機器でサポートされている.
OpenFlow は特別な機器やコントローラが必要であり,NSH も比較的新しいプロトコルで,かつ機能も SFC に特化しているため,NSH をサポートする機器は少ない.
対象的に MPLS は既に多くの機器でサポートされているため,OpenFlow や NSH と比較して導入が容易であるという特徴を持つ.
また,RFC8595~\cite{rfc8595} のように,NSH と MPLS を組み合わせて SFC を実現する手法も提案されている.
このアーキテクチャでは,NSH の Transport Layer として MPLS を利用している.

\section{SRv6}
\label{section:srv6}
先に挙げた技術だけではなく,Segment Routing over IPv6 (SRv6) も TE を適用できる技術の 1 つである.
MPLS が独自のラベルを使って使ってパケットを転送するアーキテクチャであるのに対し,SRv6では MPLS のラベルに対応する概念として IPv6 アドレスを利用する.
SRv6 で利用される識別子はセグメント識別子 (SID) と呼ばれ,各 SID はネットワーク内の特定の場所で実行される特定の機能を表す.
この SID は IPv6 と全く同じフォーマットをしている.
SRv6 では,SRv6 ヘッダ (SRH) と呼ばれる IPv6 拡張ヘッダに SID の一連の集合からなるリストを埋め込むことで,ネットワークオペレータやアプリケーションはパケットが通過する中間地点を指定できる.

SRv6 ヘッダには通過するネットワーク上のノードの順番がリストとして埋め込まれ,ルータはそのリストに基づいてパケットを転送する.
図~\ref*{fig:srv6} において,例えば SID リストの要素が \textbf{A,FW,C} である場合,パケットは緑色のパスを通るように転送される.
また,SRv6 はあるパケットが経由するノードを指定できるだけでなく,パケットに対してパケットに対して特定の操作を適用できる.
このようなパケット操作の種類のことを \textbf{SRv6 ビヘイビア} という.
現在 RFC8986~\cite{rfc8986} では 15 種類の End ビヘイビアが定義されている.

\begin{figure}[t]
    \centering
    \includegraphics[width=0.95\linewidth]{img/SRv6Arch.pdf}
    \caption{SRv6 アーキテクチャ}
    \label{fig:srv6}
\end{figure}

\subsection{SRv6 を利用した layer-3 VPN の構築例と SRv6 によるパケット転送の具体的な動作}
\label{sbsection:srv6-vpn}

SRv6 ビヘイビアを組み合わせることで,layer-3 VPN を構成することもできる~\cite{rfc9252}.
SRv6 を利用した layer-3 VPN の動作を図~\ref*{fig:srv6-vpn} に示す.

図~\ref*{fig:srv6-vpn} \circled{1} において,PE-1 は受信したパケットを SRH でカプセル化する.
このように SRH でパケットをカプセル化する,という操作も SRv6 ではビヘイビアとして定義されており,この操作のことを H.Encaps という.
ここでは,PE-1 は受信したパケットに対して \textbf{A,FW,C} を意味する SID リストを付加したものとする.
このとき,SRv6 でカプセル化されたパケットの宛先アドレスは,内部パケットの宛先アドレスに関わらず,次に到達すべきノードを示す SID である FW Service になる.

PE-1 は H.Encaps で SRH を付加したパケットを Router-A へ送信する.
このとき,パケットは宛先 IPv6 アドレスが Router-A である単なる IPv6 パケットとして扱われる.
PE-1 は自身の持つルーティングテーブルを参照し,Router-A へのネクストホップを決定し,パケットを送出する.
図~\ref*{fig:srv6-vpn} \circled{2} は,Router-A が受信したパケットに対して SID を 1 つ進めて FW Service にパケットを転送している様子を示している.
リスト状になっている SID の中でどれが現在有効な SID であるかを指定するために,SRH にはセグメントレフト (segleft) と呼ばれるフィールドが定義されている.
segleft は SID リストのインデックスであり, $(SID の合計)-1$ から始まり,$0$ で終わる.

Router-A は受信した SRv6 パケットの segleft を 1 つデクリメントし,FW Service の SID が次に有効な SID であることを示すようにする.
また,Router-A はパケットの宛先アドレスを新しく有効になった FW Service の SID に書き換える.
このように,segleft を 1 つ進め,宛先アドレスを新たに有効になった SID で書き換える動作のことを End ビヘイビアといい,これも SRv6 ビヘイビアの 1 つである.
End ビヘイビアは SRv6 の中で最も基本的なビヘイビアである.

宛先アドレスを書き換えたあと,Router-A は自身のルーティングテーブルから新たな宛先アドレス (FW Service の SID) を検索し,ネクストホップへ転送する.
図~\ref*{fig:srv6-vpn} \circled{3} では,\circled{2} と同様に FW Service が End ビヘイビアを実行して segleft デクリメントし,新たに有効になった SID に基づいて Router-C へパケットを転送している.
図~\ref*{fig:srv6-vpn} \circled{4} では,PE-2 が受信したパケットに対して End.DT4 というビヘイビアを実行している.
このビヘイビアは,SRH を取り除き,特定の VRF を参照して SRv6 でカプセル化されていた内部パケットを転送する,という動作を実行する.
End.DT4 により,パケットから SRH は取り除かれ,PE-1 でカプセル化される前のパケットを得ることができる.

\begin{figure}[t]
    \centering
    \includegraphics[width=0.95\linewidth]{img/ExplainEndDT4.pdf}
    \caption{SRv6 を利用した layer-3 VPN の動作例}
    \label{fig:srv6-vpn}
\end{figure}

\subsection{SRv6 パケットと SID の構造}
\label{sbsection:srv6-sid-struct}
図~\ref*{fig:srv6-packet} に,SRv6 パケットの詳細な構造を示す.
図~\ref*{fig:srv6-vpn} では,簡単のためにパケット構造の図を簡略化して表現した.
実際の SRv6 パケットは,一般的な IPv6 ヘッダの下に SRH が位置する.
IPv6 ヘッダ内の Next Header (NH) には 43 が設定され,この 43 という数字は次に来るヘッダのタイプが IPv6 ルーティングヘッダであることを示す.

先に述べた通り,SID は SRv6 で利用される識別子で,SRH にはいくつかの SID がリスト状になって含まれている.
現在どの SID が有効であるかは,segleft の値によって決まる.
図~\ref*{fig:srv6-packet} の例では,segleft は 1 である.
よって,Segment List のインデックスが 1 である,最初から 2 番目の SID が有効化されている.
宛先アドレスには現在有効な SID の値が設定されるため,このパケットの宛先アドレスは Segment List の 2 番目の SID となる.

図~\ref*{fig:srv6-packet} で示されているように,SID は \texttt{LOC:FUNCT:ARG} という 3 つのパートに分かれている.
また,SID は IPv6 アドレスであるため,それら 3 つパートの長さの合計は 128bit である.
\texttt{LOC} はロケータを表す.ロケータとは,ある SID に対応するノードの場所を表す.
\texttt{FUNCT} は SID に関連付けられた SRv6 ビヘイビアの識別子であり,\texttt{ARG} はビヘイビアの動作に必要な追加情報をエンコードする領域である.
例えば,End.DT4 では \texttt{ARG} フィールドにはルックアップするべき VRF テーブルを識別するための情報がエンコードされる.

SRv6 の SID は IPv6 アドレスであるため,既存のルーティングプロトコルを用いて SID を経路情報として既存のルーティングプロトコルを利用して広告できる.
SRv6 ビヘイビアは,SID の \texttt{FUNCT} パートで表現される.
ただし,「特定の SRv6 ビヘイビアならば \texttt{FUNCT} パートはこの値である」というような一般的な定義は存在しない.
例えば,IS-IS の TLV の Type フィールドの値は IANA によって策定されいる~\cite{iana-isis}.
それに対して,SRv6 ではビヘイビアに対応する値は標準化されず,オペレータが自由に決めることができる.
図~\ref*{fig:srv6-vpn} \circled{4} では,Router-C は受信したパケットに対して End.DT4 を実行している.
Router-C は \texttt{LOC:FUNCT:ARG} のフォーマットに従って \textbf{[Router-C のロケータを示すブロック]:[Router-C 上で End.DT4 として定義されたブロック]:[利用可能な VRF table 番号]} を自由に決定し,それを経路情報として周りのノードへ広告する.

SRv6 ビヘイビアが実装されていないネットワーク機器が SRv6 でカプセル化されたパケットを受信した際,その機器は SRv6 パケットを IPv6 パケットとして転送できる.
SRH の前には一般的な IPv6 ヘッダが挿入されているため,そのパケットは IPv6 パケットとして転送可能であるからである.
つまり,SRv6 ビヘイビアを実行せず,単にパケットを現在有効な SID へ転送するだけであれば,IPv6 パケットフォワーディングが実装されていれば適切にパケットを転送できる.

\newpage
\begin{figure}[t]
    \centering
    \includegraphics[width=0.95\linewidth]{img/SRv6Packet.pdf}
    \caption{詳細な SRv6 のパケット構造}
    \label{fig:srv6-packet}
\end{figure}
