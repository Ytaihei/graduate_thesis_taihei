\chapter{提案手法:sXGP-5Gコンバータ}
% --- 章アウトライン・TODO・参考文献引用例 ---
% この章では、提案するsXGP-5Gコンバータの設計方針・要件・実装案・制約を述べる。
% TODO: docker_open5gs_sXGP-5G の設計思想や工夫点を記述。
% TODO: 参考文献を本文中で引用する(例: \cite{rfc9171})。
% 例: プロトコル変換の設計指針は \cite{rfc5326} などを参照。
% ------------------------------------------
\label{chap:proposal}

\section{要求仕様と設計方針}
\subsection{ユースケース}
提案環境は、実装ベース標準化を支援する以下のユースケースを想定する。

\begin{enumerate}
	\item \textbf{標準仕様の実装検証と相互運用性テスト}: 3GPP仕様に基づくOSS実装を実機RANで動作させ、仕様書上の記述と実装の乖離、異なる実装間の相互運用性問題を早期に発見する。

	\item \textbf{標準化へのタイムリーなフィードバック}: 検出した問題を再現手順・パケットキャプチャとともにドキュメント化し、標準化団体(3GPP等)や関連OSSプロジェクトへ報告・改善提案を行う。

	\item \textbf{継続的インテグレーション・回帰テスト}: 再現性の高い実験環境により、コード変更やプロトコル拡張が既存機能を破壊していないかを自動的に検証する。

	\item \textbf{性能評価とボトルネック分析}: 実機環境での遅延・スループット測定により、プロトコル設計や実装の性能ボトルネックを特定し、標準化・実装の両面から改善を図る。

	\item \textbf{プロトコル拡張の試作・検証}: 新機能や実験的プロトコルを実装し、実機環境で動作検証を行うことで、標準化前の技術検証を加速する。

	\item \textbf{教育・トレーニング用途}: 実機相当の環境で5G/6Gプロトコルスタックの動作を学習し、標準化・実装の実践的スキルを習得する。
\end{enumerate}

\subsection{非機能要件(再現性・法令遵守・安全性)}
\begin{itemize}
	\item \textbf{再現性}: Docker等のコンテナ技術により、環境構築手順を標準化し、異なる環境でも同一の実験を再現可能にする。
	\item \textbf{法令遵守}: 免許不要帯(sXGP)を用いることで、電波法に抵触せず実機検証を実施できる。
	\item \textbf{安全性}: 実験環境を隔離されたネットワークで構築し、商用ネットワークへの影響を排除する。
		\item \textbf{可観測性}: すべての制御/ユーザ面トラフィックをpcap/ログとして取得でき、試験ごとに成果物として保存できること。
		\item \textbf{保守性}: 変換ロジックはメッセージ種別ごとにモジュール化し、追加/無効化が容易であること。
\end{itemize}

\subsection{対象範囲(制御/ユーザ面、認証、セッション管理)}
本提案は、LTE eNBを変更せず5GCへ接続するための\textbf{変換層}(s1n2-converter)である。対象とする範囲と非対象は以下のとおりである。

\paragraph{対象}
\begin{itemize}
	\item 制御面: S1APとNGAPのメッセージ変換およびIEマッピング(Initial UE/Context Setup近傍の手順を中心)
	\item ユーザ面: S1-UとN3間のGTP-Uカプセル化のパススルーおよびTEIDマッピング
	\item コンテキスト: UE識別子(MME-UE-S1AP-ID/ENB-UE-S1AP-ID, AMF-UE-NGAP-ID)の対応管理
\end{itemize}

\paragraph{非対象(現段階)}
\begin{itemize}
	\item 無線(PHY/MAC/RLC/PDCP)の最適化やNR固有機能(本研究はRANにsXGP/TD-LTEを用いる)
	\item EPS/5GS間の移動性最適化(N26インターフェース)\cite{threegpp-23502}
	\item ハンドオーバ、eDRX/PSM等の省電力最適化
\end{itemize}

\section{アーキテクチャ設計}
提案アーキテクチャを図示的に記述すると、\texttt{eNB}—(S1AP,S1-U)→\texttt{Converter}—(NGAP,N3)→\texttt{5GC(AMF/SMF/UPF)} である。Converterは制御面/ユーザ面を分離して処理し、状態を共有する。5GCはOpen5GS\cite{open5gs}を想定するが、NGAP/N3の標準動作に従う実装であれば置換可能である(\cite{threegpp-23501,threegpp-23502})。

\subsection{制御面:S1AP–NGAP変換の設計案}
変換対象とする代表的手順とマッピング方針を示す。
\begin{itemize}
	\item \textbf{Initial UE Message(S1AP)→ Initial UE Message(NGAP)}: UE識別子とTA/PLMN情報、NASコンテナを搬送する。ConverterはeNBからのECGI/TAI等をNGAPのTAI/NR CGI相当へマップし、AMF選定に必要なIEを充足させる。
	\item \textbf{Initial Context Setup(S1AP)↔ Initial Context Setup(NGAP)}: セキュリティ/UE-AMBR等のコンテキストIEを相互に再構成する。E-RAB関連IEは、5GのPDU Session資源手順と分離して取り扱う(後述)。
	\item \textbf{UE Context Release}: 釣り合いの取れた解放手順(S1AP UEContextRelease/NGAP UEContextRelease)を相互に変換する。
\end{itemize}

設計上の原則は、(i) NASは原則透過搬送としConverterで終端しない、(ii) IEの欠落時は安全側(エラー)に倒す、(iii) 5Gで独立した手順(例: PDU Session Resource Setup)はS1APのICSと\textbf{分離}して逐次処理する、である。これにより、AMF/SMF側の前提条件を満たさずに\texttt{unknown-PDU-session-ID}等のエラーとなる事象を回避する。

\subsection{ユーザ面:GTP-U転送・TEID管理}
ユーザ面はパススルーを基本とし、S1-U(eNB⇄Converter)とN3(Converter⇄UPF)で独立したTEID空間を維持する。Converterは方向別に以下の対応表を持つ。
\begin{description}
	\item[UL Map] (ENB-UE-S1AP-ID, eNB\_TEID\_UL) → (UPF\_IP, N3\_TEID\_UL)
	\item[DL Map] (AMF-UE-NGAP-ID, N3\_TEID\_DL) → (eNB\_IP, S1U\_TEID\_DL)
\end{description}
TEIDの学習は、\texttt{E-RAB Setup List}(S1AP)および\texttt{PDUSessionResourceSetupResponseTransfer}(NGAP)に含まれるGTP\_Tunnel情報から行う。学習前パケットはドロップし、ログで検知する。MTUについては、d\_max = min(1500, パス上の最小MTU - GTP/IP/UDPヘッダ) とし、DFビットはクリアする保守的ポリシを採用する。

\subsection{認証・登録(NASメッセージ処理の方針)}
NASは原則透過搬送とし、Converterはカプセル化境界の整合性のみを担保する。具体的には、S1AP/NGAPの\texttt{NAS-PDU}フィールドとして運ばれるバイト列を改変せず、AMF/UEで終端されることを前提とする。学術検証モードとして、事前に加入者・鍵情報を5GCに登録し、認証アルゴリズム不一致による失敗を切り分ける運用を提供する。NASの再暗号化や鍵導出はConverterの責務に含めない(\cite{threegpp-23501})。

\subsection{コンテキスト管理とタイムアウト}
ConverterはUEごとに以下を保持する。
\begin{itemize}
	\item ID束: (ENB-UE-S1AP-ID, MME-UE-S1AP-ID) ↔ (AMF-UE-NGAP-ID)
	\item セッション束: PDU Session ID ↔ E-RAB ID の対応、TEID対応表(UL/DL)
	\item タイマ: T\_ctx(コンテキスト保持), T\_gtp(TEID学習待ち), T\_proc(手順待ち合わせ)
\end{itemize}
T\_ctx満了でUEコンテキストを破棄し、未解放資源をログ出力する。異常系(IE不足/不整合、学習超過)は\texttt{cause}を付して相手側へエラーを返すか、ドロップ+監査ログとする。

\section{sXGP採用の根拠と想定限界}
\subsection{採用の根拠:実装ベース標準化への貢献}

sXGPを採用する最大の理由は、\textbf{免許不要帯での法令遵守により「実装を実際に動かせる」}ことにある。これは、実装ベース標準化の実現において極めて重要である。

\begin{itemize}
	\item \textbf{実機による全スタック検証}: 実UE・実NIC・実時間のプロトコルスタックを動作させることで、シミュレータでは検出できない実装レベルの問題(タイミング、リソース競合、ハードウェア依存の挙動など)を発見できる。これは標準仕様と実装の乖離を早期に検出するために不可欠である。

	\item \textbf{相互運用性問題の早期発見}: 異なるベンダの実装(UE、eNB、コア)を組み合わせた際の相互運用性問題を、実機環境で検証できる。3GPP標準の曖昧な記述や解釈の違いによる非互換性を、標準化プロセスの早い段階でフィードバック可能にする。

	\item \textbf{標準化へのタイムリーなフィードバック}: 実機で問題を再現できることで、再現手順・パケットキャプチャ・ログを含む具体的な報告を標準化団体やOSSプロジェクトに提供できる。これにより、仕様策定 → 実装 → 問題発覚 → 修正のサイクルを数年単位から数ヶ月単位に短縮できる。

	\item \textbf{継続的検証の実現}: 免許不要帯であるため、継続的インテグレーション(CI)環境に組み込んで自動テストを実施できる。コード変更のたびに実機検証を行うことで、回帰テストと品質保証を強化できる。

	\item \textbf{(副次的効果)低コスト・再現性}: 専用周波数ライセンスが不要であり、比較的低コストで実験環境を構築できる。また、Dockerなどの技術と組み合わせることで、再現性の高い検証環境を複数の研究者・組織で共有できる。
\end{itemize}

\subsection{想定限界・外延}
本環境はLTE互換のsXGPをRANとして用いるため、5G NR特有のPHY機能やスケジューリング最適化は対象外となる。しかし、制御面(S1AP/NGAP/NAS)およびユーザ面(GTP-U)のプロトコルレベルでの相互運用性検証は十分に可能であり、標準化フィードバックの主要な価値はこのレイヤーにある。NR固有の無線レイヤーの課題は、今後の拡張として分離して扱う。

\section{実装方針と部品選定}
\subsection{ソフトウェア構成(言語・ライブラリ・依存)}
実装言語はCとし、制御面/ユーザ面は独立プロセスまたはスレッドで分離する。ASN.1の処理は外部ツールで生成したデコーダを用いるか、既存実装(例: Open5GSのエンコーダ/デコーダ参照\cite{open5gs})の表現に合わせてシリアライザを実装する。検証用途として、srsRANのZMQインターフェースを接続し、無線機なしでの反復試験を可能とする(\S\ref{chap:related})。

\subsection{ネットワーク構成(VLAN/VRF/NATの要否)}
最小構成では、ConverterにS1U/N3の2ポート(論理でも可)を割り当て、データプレーンをルーティングで分離する。商用NWとの干渉回避のため、VRFまたは名前空間を用いた論理分離を推奨する。UPFとは静的ルーティングで直結し、アドレス設計はOpen5GSのデフォルト構成に準拠する(\S\ref{chap:experiment})。

\subsection{監視・計測(メトリクス、トレース)}
制御面はメッセージ種別/結果(成功・失敗原因)/処理時間をメトリクス化し、ユーザ面はスループット/RTT/パケット損失率を定期測定する。すべての試験についてpcapと構成スナップショット(コンテナタグ、設定ファイル)を成果物として保存し、回帰の基準とする。

\section{制約と想定される限界}
\subsection{規格差分による制限}
S1APとNGAPはIEの集合と手順が一致しないため、完全な1対1変換は成立しない。特に、5GのPDU Session手順はS1APのE-RAB手順と\textbf{分離}されている。設計では手順の逐次化と、IE欠落時の安全側フェイルにより、相互運用性を最大化しつつ誤動作を抑制する(\cite{threegpp-23501,threegpp-23502})。

\subsection{性能上のボトルネックと回避策}
ユーザ面は二重のGTP終端によるオーバヘッドが生じる可能性がある。最適化として、カーネルオフロード(XDP/TC)やゼロコピーI/Oの導入、TEIDルックアップのハッシュ最適化を段階的に適用する。制御面はASN.1エンコード/デコードのCPU負荷が支配的となるため、ホットパスの事前割付とメモリプールでGC負荷を抑える。
