\chapter{提案手法:sXGP-5Gコンバータ}
% --- 章アウトライン・TODO・参考文献引用例 ---
% この章では、提案するsXGP-5Gコンバータの設計方針・要件・実装案・制約を述べる。
% TODO: docker_open5gs_sXGP-5G の設計思想や工夫点を記述。
% TODO: 参考文献を本文中で引用する(例: \cite{rfc9171})。
% 例: プロトコル変換の設計指針は \cite{rfc5326} などを参照。
% ------------------------------------------
\label{chap:proposal}

\section{要求仕様と設計方針}
\subsection{ユースケース}
提案環境は、実装ベース標準化を支援する以下のユースケースを想定する。

\begin{enumerate}
	\item \textbf{標準仕様の実装検証と相互運用性テスト}: 3GPP仕様に基づくOSS実装を実機RANで動作させ、仕様書上の記述と実装の乖離、異なる実装間の相互運用性問題を早期に発見する。

	\item \textbf{標準化へのタイムリーなフィードバック}: 検出した問題を再現手順・パケットキャプチャとともにドキュメント化し、標準化団体(3GPP等)や関連OSSプロジェクトへ報告・改善提案を行う。

	\item \textbf{継続的インテグレーション・回帰テスト}: 再現性の高い実験環境により、コード変更やプロトコル拡張が既存機能を破壊していないかを自動的に検証する。

	\item \textbf{性能評価とボトルネック分析}: 実機環境での遅延・スループット測定により、プロトコル設計や実装の性能ボトルネックを特定し、標準化・実装の両面から改善を図る。

	\item \textbf{プロトコル拡張の試作・検証}: 新機能や実験的プロトコルを実装し、実機環境で動作検証を行うことで、標準化前の技術検証を加速する。

	\item \textbf{教育・トレーニング用途}: 実機相当の環境で5G/6Gプロトコルスタックの動作を学習し、標準化・実装の実践的スキルを習得する。
\end{enumerate}

\subsection{非機能要件(再現性・法令遵守・安全性)}
\begin{itemize}
	\item \textbf{再現性}: Docker等のコンテナ技術により、環境構築手順を標準化し、異なる環境でも同一の実験を再現可能にする。
	\item \textbf{法令遵守}: 免許不要帯(sXGP)を用いることで、電波法に抵触せず実機検証を実施できる。
	\item \textbf{安全性}: 実験環境を隔離されたネットワークで構築し、商用ネットワークへの影響を排除する。
\end{itemize}

\subsection{対象範囲(制御/ユーザ面、認証、セッション管理)}

\section{アーキテクチャ設計}
\subsection{制御面:S1AP–NGAP変換の設計案}
\subsection{ユーザ面:GTP-U転送・TEID管理}
\subsection{認証・登録(NASメッセージ処理の方針)}
\subsection{コンテキスト管理とタイムアウト}

\section{sXGP採用の根拠と想定限界}
\subsection{採用の根拠:実装ベース標準化への貢献}

sXGPを採用する最大の理由は、\textbf{免許不要帯での法令遵守により「実装を実際に動かせる」}ことにある。これは、実装ベース標準化の実現において極めて重要である。

\begin{itemize}
	\item \textbf{実機による全スタック検証}: 実UE・実NIC・実時間のプロトコルスタックを動作させることで、シミュレータでは検出できない実装レベルの問題(タイミング、リソース競合、ハードウェア依存の挙動など)を発見できる。これは標準仕様と実装の乖離を早期に検出するために不可欠である。

	\item \textbf{相互運用性問題の早期発見}: 異なるベンダの実装(UE、eNB、コア)を組み合わせた際の相互運用性問題を、実機環境で検証できる。3GPP標準の曖昧な記述や解釈の違いによる非互換性を、標準化プロセスの早い段階でフィードバック可能にする。

	\item \textbf{標準化へのタイムリーなフィードバック}: 実機で問題を再現できることで、再現手順・パケットキャプチャ・ログを含む具体的な報告を標準化団体やOSSプロジェクトに提供できる。これにより、仕様策定 → 実装 → 問題発覚 → 修正のサイクルを数年単位から数ヶ月単位に短縮できる。

	\item \textbf{継続的検証の実現}: 免許不要帯であるため、継続的インテグレーション(CI)環境に組み込んで自動テストを実施できる。コード変更のたびに実機検証を行うことで、回帰テストと品質保証を強化できる。

	\item \textbf{(副次的効果)低コスト・再現性}: 専用周波数ライセンスが不要であり、比較的低コストで実験環境を構築できる。また、Dockerなどの技術と組み合わせることで、再現性の高い検証環境を複数の研究者・組織で共有できる。
\end{itemize}

\subsection{想定限界・外延}
本環境はLTE互換のsXGPをRANとして用いるため、5G NR特有のPHY機能やスケジューリング最適化は対象外となる。しかし、制御面(S1AP/NGAP/NAS)およびユーザ面(GTP-U)のプロトコルレベルでの相互運用性検証は十分に可能であり、標準化フィードバックの主要な価値はこのレイヤーにある。NR固有の無線レイヤーの課題は、今後の拡張として分離して扱う。

\section{実装方針と部品選定}
\subsection{ソフトウェア構成(言語・ライブラリ・依存)}
\subsection{ネットワーク構成(VLAN/VRF/NATの要否)}
\subsection{監視・計測(メトリクス、トレース)}

\section{制約と想定される限界}
\subsection{規格差分による制限}
\subsection{性能上のボトルネックと回避策}
