\chapter{Contact Planの臨時更新の天体内への限定的な拡散の提案}
\label{chap:suggestion}
\ref{section:ContactPlanの臨時更新の課題}節で述べたとおり, 
Contact Planの臨時更新には拡散の有効性とコストによって対象を選定することが必要である. 
Bezirgiannidisらの提案した全てのノードにおいて臨時更新を行う既存手法に対して, 
本論文ではこの課題に対して, Contactの失敗が起きた際に, 
その通知を行いContact Planを更新するノードを
天体内のノードにのみ拡散し, 他天体のノードには拡散しない手法を提案する. 
本章ではこの提案が満たすべき要件について整理し, \ref{chap:implementation_and_experimentation}する. 

\section{要件1:臨時更新による配送遅延の低下}
\label{section:要件1}
\ref{section:ContactPlanの臨時更新}で述べたように, 
Contact Planの臨時更新の目的は, 想定されたContactに失敗した場合にその
情報をDTNの他のノードに拡散しContact Planを更新することで, 
DTNの各ノードが最新のトポロジーを認識し配送遅延の増加を抑えることである. 
そのため臨時更新を行わない場合と比較した効果について, 既存手法と提案手法について比較を行う. 

\section{要件2:臨時更新による運用面でのデメリット}
\label{section:要件2}
\ref{section:要件1}で述べたように, Contact Planの臨時更新は
その情報を受け取ったノードでの再計算を伴うものであり, 
もしも遠隔天体の更新情報を全て受け付ける場合, 
非常に多くの回数の再計算が必要になる. 当然ノードによって処理能力は異なり, 
再計算に問題のないノードも存在するが, 近年の宇宙機の小型化などにより, 
処理能力が限られているノードも存在するため, 処理負荷は上げるべきではない. 
またその情報によってトラフィックが一部のノードに集中することにもなり得るため, 
本研究では取り扱わないものの, なんらかの形でこのContactの障害情報が
正しいものかの検証が必要となると考えられる. 提案手法はこの点において, 
既存手法と違い自天体内のノードにのみ拡散するため, 
受け取ったノードが天体間の通信遅延なしに何らかの形で検証を行うことも比較的容易である. 
この点において提案手法は既存手法よりも運用面でのデメリットが少ないと言える. 