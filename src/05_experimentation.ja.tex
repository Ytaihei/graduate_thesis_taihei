\chapter{sXGP-5G環境での実機検証手法}
% --- 章アウトライン・TODO・参考文献引用例 ---
% この章では、実験環境・シナリオ・計測方法・指標を記述する。
% TODO: docker_open5gs_sXGP-5G の実験構成や計測手順を具体的に記述。
% TODO: 参考文献を本文中で引用する(例: \cite{dtn_implementations})。
% 例: DTNの評価手法は \cite{Fraire2021} などを参照。
% ------------------------------------------
\label{chap:experiment}

\section{実験の目的}
本章では、提案したsXGP-5Gコンバータ環境を用いて、以下を実証する:

\begin{itemize}
	\item \textbf{基本的な相互接続動作の確認}: 4G RAN(sXGP eNB + 実UE)と5G Core(Open5GS)間で、登録・PDUセッション確立・データ転送が正常に動作することを検証する。

	\item \textbf{S1AP/NGAP/NAS信令変換の正当性}: コンバータが4G(S1AP)と5G(NGAP)のプロトコル変換を正しく行い、両側のプロトコルスタックが期待通り動作することを確認する。

	\item \textbf{GTP-Uトンネル中継の動作確認}: ユーザ面トラフィックがS1-U(eNB側)とN3(UPF側)間で正しくトンネリングされ、TEID管理が適切に機能することを検証する。

	\item \textbf{実機特有の相互運用性問題の検出能力}: シミュレータでは再現困難な実機依存の挙動(タイミング、NAS再送、端末OS依存の処理など)を検出できることを示す。

	\item \textbf{再現性の確保}: Docker化された環境により、同一の実験を異なる環境で再現可能であることを確認する。
\end{itemize}

\section{実験環境}
\subsection{ハードウェア構成(UE/eNB/sXGP基地局/サーバ)}
本研究の実機検証は、以下の最小構成で実施する。
\begin{itemize}
	\item \textbf{UE}: 市販スマートフォン(4G LTE対応、sXGP対応バンド)またはLTEモデム端末。
	\item \textbf{sXGP基地局(eNB)}: TD-LTE互換のsXGP小型基地局(室内用)。
	\item \textbf{サーバ}: x86\_64または同等性能のホスト1台。ConverterおよびOpen5GSのコンテナを実行する。NICは1~2ポート(管理/データ分離は論理でも可)。
	\item \textbf{スイッチ/ルータ}: 管理ネットワークと実験ネットワークのL2/L3分離に用いる。VRFまたはLinuxネットワーク名前空間で代替可能である。
\end{itemize}

\subsection{ソフトウェア構成(5GC, コンバータ, OS)}
\begin{itemize}
	\item \textbf{5GC}: Open5GS(AMF/SMF/UPF/UDM/AUSF/NRF, WebUI)。Dockerコンテナで起動し、加入者情報をWebUIまたはAPIで登録する\cite{open5gs}。
	\item \textbf{Converter}: s1n2-converter(制御面: S1AP↔NGAP, ユーザ面: S1-U↔N3)。制御/ユーザ面は別プロセス(またはスレッド)として動作し、TEID対応表を共有する(第\ref{chap:proposal}章)。
	\item \textbf{OS/ツール}: Linux(Docker/Compose, tcpdump, tshark, iproute2)。時間同期はNTPで十分である(PPS等の高精度は本検証の範囲外)。
\end{itemize}

\subsection{ネットワークトポロジとIPアドレッシング}
トポロジは、\texttt{UE}—\texttt{eNB(sXGP)}—(S1AP,S1-U)→\texttt{Converter}—(NGAP,N3)→\texttt{5GC} の直列構成である。Converterは制御面(S1AP/NGAP)とユーザ面(S1-U/N3)で論理IFを分ける。5GC側は、AMF/SMF/UPFを同一ホスト内に配置する。UEアドレスプールやAPN等はOpen5GSのデフォルト方針に準拠し、必要に応じてIPv6-only + NAT64/464XLAT, DNS64 構成を併用する(第\ref{chap:related}章)。

\section{実験シナリオ}
\subsection{基本接続(登録・PDUセッション確立)}
\paragraph{目的} ConverterのS1AP↔NGAP変換と5GC連携が成立することを確認する。
\paragraph{前提} 5GC起動済、加入者登録済、eNBのS1接続先がConverterに設定済。
\paragraph{手順}
\begin{enumerate}
	\item UEをsXGPセルに接続し、Attach/Registrationを開始する。
	\item ConverterでInitial UE Message/Initial Context Setupの相互変換が成功し、AMFでUE Contextが生成されることを確認する(ログ/pcap)。
	\item PDU Session(5G)資源手順がSMF/UPF側で成功し、ConverterにTEID情報が学習されることを確認する。
\end{enumerate}
\paragraph{成功基準} AMF/SMF/UPFログにエラーがなく、UEがPDUセッション確立状態となる。Converterの対応表にTEID(UL/DL)が登録される。

\subsection{データ転送(スループット・遅延)}
\paragraph{目的} S1-U/N3間のGTP-U中継とTEIDマッピングが正しく機能することを確認する。
\paragraph{手順}
\begin{enumerate}
	\item UEからコアネットワーク外部(またはUPFデータネット)へのICMP/TCP/UDP通信を発生させる。
	\item ConverterのS1-U/N3でpcapを取得し、TEID・5タプル対応を突合する。
	\item スループット(iperf等)とRTT(ping)を測定し、パスの安定性を確認する。
\end{enumerate}
\paragraph{成功基準} UL/DLともに期待するTEIDでカプセル化され、フローが継続する。損失/再送が異常に多い場合はMTU/MSSやオフロード設定を点検する。

\subsection{ハンドオーバ相当の扱い(必要に応じて)}
本研究はN26等の移動性最適化を対象外とするが、\textbf{再登録/再確立}シナリオにより実運用で近似するイベントを評価する。具体的には、電波減衰やセル再選相当のイベントを与え、RegistrationやPDU Sessionの再確立が円滑に行われるかを、NAS再送やタイマ(T3xx)挙動とあわせて確認する(第\ref{chap:proposal}章)。

\subsection{Androidエミュレータの限界と実UE検証の必要性}
通信機能を有するAndroidアプリケーションの検証において、\textbf{PC上のエミュレータのみでは再現困難な事象}が多い。特に、セルラースタックや事業者設定、端末OSの省電力・バックグラウンド制御など、\textbf{実機依存の挙動}がアプリの通信体験を大きく左右する。本研究では、sXGP+実UE+5GC構成により、以下の観点で\textbf{実機ならではの検証}を行う。

\paragraph{エミュレータで不足しがちな項目}
\begin{itemize}
	\item \textbf{セルラー制御プレーンの実挙動}: 実エミュレータはベースバンドを持たず、Registration/TAU、NAS再送、タイマ境界での挙動差(T3xx/Back-off等)の忠実再現が困難。
	\item \textbf{IMS/音声・SMS連携}: VoLTE/VoNRやSMS over NASなどは端末・キャリア・IMSの三者連携が必要で、エミュレータでは未実装/限定的であることが多い。
	\item \textbf{ネットワーク条件とスタック差分}: IPv6-only + NAT64/464XLAT、DNS64、PCO/APNパラメータ、MTU/MSS、NICオフロード(TSO/GRO)等の影響はエミュレータで再現しづらい。
	\item \textbf{OSによるバックグラウンド制御}: Doze/App Standby、JobScheduler/WorkManagerのネットワーク可用性制御は実機の省電力・無線レイヤと結合しており、エミュレータでは挙動が異なる。
	\item \textbf{移動性・無線イベント}: RRC Idle/Connectedの遷移、電波強度・セルリセレクション、ハンドオーバ相当の切替など、時間軸のイベントは実無線でなければ発現しにくい。
	\item \textbf{ポリシ・スライス連携}: URSPやスライシングは端末/OS/事業者依存が強く、汎用エミュレータでは検証対象外となることが多い(本環境では制御/ユーザ面の相互接続検証を主対象とする)。
\end{itemize}

\paragraph{実UE+sXGP+5GCでの検証手順(例)}
\begin{enumerate}
	\item \textbf{端末側ログ取得}: logcat(特にradioバッファ)とネットワークログを収集し、アプリのAPI呼び出しと無線/NASイベントを時系列で突合する。
	\item \textbf{ネットワーク側トレース}: コンバータ/5GC側でS1AP/NGAP/NAS、GTP-Uのpcapを取得し、再送、原因コード、TEID/フローの対応を解析する。
	\item \textbf{テストシナリオ}:
		\begin{itemize}
			\item IPv6-only + NAT64/464XLAT 下でのアプリ接続(DNS64解決、QUIC/TCPの挙動差)。
			\item RRC Idle復帰や電波減衰を伴う再接続でのセッション継続性(ソケット再確立、タイムアウト)。
			\item 大きなパケットや損失率注入時の再送・バックオフ挙動(アプリ層リトライ設計の妥当性)。
		\end{itemize}
	\item \textbf{再現性確保}: Docker化した5GC/コンバータ構成と固定シナリオにより、回帰テストとして継続的に実行可能にする。
\end{enumerate}

\paragraph{本環境で得られる価値}
実UEを用いることで、\textbf{シミュレータでは顕在化しにくい相互運用性問題や端末OS依存の振る舞い}(例:NAS再送シーケンスのズレ、IPv6-only下でのAPI失敗、バックグラウンド時の接続断)を早期に発見し、\textbf{パケットキャプチャと再現手順}を添えて標準化・OSS実装へフィードバックできる。

\section{計測方法と指標}
\subsection{A/B比較(シミュレータ vs 実機)}
\begin{itemize}
	\item A: RANシミュレータ+5GC(OSS)\quad B: sXGP eNB+実UE+5GC(OSS)
	\item 比較対象: 登録/PDU確立成功率、タイマ境界での失敗率、再送回数、エラーコード、スループット、遅延、リソース使用率
	\item ログ・トレース: S1AP/NGAP/NAS、GTP-U、カーネル/DPDK統計、pcap
\end{itemize}
\subsection{成功基準}
基本機能の安定成立(>99%)、性能劣化の要因特定、実機特有の不具合検出(再現手順付き)を満たすこと。
\subsection{成功判定基準と再現手順}
\paragraph{判定基準}
\begin{itemize}
	\item 制御面: Initial UE/Context SetupおよびPDU Session資源手順がエラーなく完了(AMF/SMFログ, NGAPトレース)
	\item ユーザ面: UE↔UPF間で期待するTEIDによりカプセル化され、連続したシーケンスで転送
	\item 安定性: 連続N回(例: 30回)の接続/切断試験で成功率>99%
\end{itemize}
\paragraph{再現手順(成果物)}
試験ごとに、(i) コンテナタグ/設定ファイル、(ii) Converter/AMF/UPFのログ、(iii) S1AP/NGAP/GTP-Uのpcap、(iv) 計測スクリプトの出力(CSV/JSON)を保存し、成果物として管理する。これにより、実装の回帰有無を定量的に判定できる\cite{dtn_implementations}。

\subsection{メトリクス(接続成功率、遅延、スループット、CPU/メモリ)}
\begin{itemize}
	\item 機能: 登録成功率、PDUセッション確立成功率、再登録時の成功率
	\item 性能: RTT(ICMP/TCP SYN-ACK)、スループット(iperf3)、再送率、フロー継続時間
	\item 資源: Converter/5GCのCPU利用率、メモリ、コンテキスト数、TEIDテーブルサイズ
\end{itemize}

\subsection{ロギング・パケットキャプチャの取得方法}
\begin{itemize}
	\item ConverterのS1AP/NGAP IFでpcap取得(制御面、フィルタ: SCTP, NGAP/S1APポート)
	\item ConverterのS1-U/N3 IFでpcap取得(ユーザ面、フィルタ: UDP dst port 2152)
	\item AMF/SMF/UPFのアプリケーションログ(JSON/テキスト)を保存し、タイムスタンプで突合
\end{itemize}
NAS/NGAP/S1APのメッセージシーケンスは、列挙型(IE)と原因コードで比較する。必要に応じ、メッセージ時系列の可視化にシーケンス図生成ツールを用いる\cite{Fraire2021}。
