\chapter{実験環境と方法}
% --- 章アウトライン・TODO・参考文献引用例 ---
% この章では、実験環境・シナリオ・計測方法・指標を記述する。
% TODO: docker_open5gs_sXGP-5G の実験構成や計測手順を具体的に記述。
% TODO: 参考文献を本文中で引用する(例: \cite{dtn_implementations})。
% 例: DTNの評価手法は \cite{Fraire2021} などを参照。
% ------------------------------------------
\label{chap:experiment}

\section{実験環境}
\subsection{ハードウェア構成(UE/eNB/sXGP基地局/サーバ)}
\subsection{ソフトウェア構成(5GC, コンバータ, OS)}
\subsection{ネットワークトポロジとIPアドレッシング}

\section{実験シナリオ}
\subsection{基本接続(登録・PDUセッション確立)}
\subsection{データ転送(スループット・遅延)}
\subsection{ハンドオーバ相当の扱い(必要に応じて)}

\subsection{Androidエミュレータの限界と実UE検証の必要性}
通信機能を有するAndroidアプリケーションの検証において、\textbf{PC上のエミュレータのみでは再現困難な事象}が多い。特に、セルラースタックや事業者設定、端末OSの省電力・バックグラウンド制御など、\textbf{実機依存の挙動}がアプリの通信体験を大きく左右する。本研究では、sXGP+実UE+5GC構成により、以下の観点で\textbf{実機ならではの検証}を行う。

\paragraph{エミュレータで不足しがちな項目}
\begin{itemize}
	\item \textbf{セルラー制御プレーンの実挙動}: 実エミュレータはベースバンドを持たず、Registration/TAU、NAS再送、タイマ境界での挙動差(T3xx/Back-off等)の忠実再現が困難。
	\item \textbf{IMS/音声・SMS連携}: VoLTE/VoNRやSMS over NASなどは端末・キャリア・IMSの三者連携が必要で、エミュレータでは未実装/限定的であることが多い。
	\item \textbf{ネットワーク条件とスタック差分}: IPv6-only + NAT64/464XLAT、DNS64、PCO/APNパラメータ、MTU/MSS、NICオフロード(TSO/GRO)等の影響はエミュレータで再現しづらい。
	\item \textbf{OSによるバックグラウンド制御}: Doze/App Standby、JobScheduler/WorkManagerのネットワーク可用性制御は実機の省電力・無線レイヤと結合しており、エミュレータでは挙動が異なる。
	\item \textbf{移動性・無線イベント}: RRC Idle/Connectedの遷移、電波強度・セルリセレクション、ハンドオーバ相当の切替など、時間軸のイベントは実無線でなければ発現しにくい。
	\item \textbf{ポリシ・スライス連携}: URSPやスライシングは端末/OS/事業者依存が強く、汎用エミュレータでは検証対象外となることが多い(本環境では制御/ユーザ面の相互接続検証を主対象とする)。
\end{itemize}

\paragraph{実UE+sXGP+5GCでの検証手順(例)}
\begin{enumerate}
	\item \textbf{端末側ログ取得}: logcat(特にradioバッファ)とネットワークログを収集し、アプリのAPI呼び出しと無線/NASイベントを時系列で突合する。
	\item \textbf{ネットワーク側トレース}: コンバータ/5GC側でS1AP/NGAP/NAS、GTP-Uのpcapを取得し、再送、原因コード、TEID/フローの対応を解析する。
	\item \textbf{テストシナリオ}:
		\begin{itemize}
			\item IPv6-only + NAT64/464XLAT 下でのアプリ接続(DNS64解決、QUIC/TCPの挙動差)。
			\item RRC Idle復帰や電波減衰を伴う再接続でのセッション継続性(ソケット再確立、タイムアウト)。
			\item 大きなパケットや損失率注入時の再送・バックオフ挙動(アプリ層リトライ設計の妥当性)。
		\end{itemize}
	\item \textbf{再現性確保}: Docker化した5GC/コンバータ構成と固定シナリオにより、回帰テストとして継続的に実行可能にする。
\end{enumerate}

\paragraph{本環境で得られる価値}
実UEを用いることで、\textbf{シミュレータでは顕在化しにくい相互運用性問題や端末OS依存の振る舞い}(例:NAS再送シーケンスのズレ、IPv6-only下でのAPI失敗、バックグラウンド時の接続断)を早期に発見し、\textbf{パケットキャプチャと再現手順}を添えて標準化・OSS実装へフィードバックできる。

\section{計測方法と指標}
\subsection{A/B比較(シミュレータ vs 実機)}
\begin{itemize}
	\item A: RANシミュレータ+5GC(OSS)\quad B: sXGP eNB+実UE+5GC(OSS)
	\item 比較対象: 登録/PDU確立成功率、タイマ境界での失敗率、再送回数、エラーコード、スループット、遅延、リソース使用率
	\item ログ・トレース: S1AP/NGAP/NAS、GTP-U、カーネル/DPDK統計、pcap
\end{itemize}
\subsection{成功基準}
基本機能の安定成立(>99%)、性能劣化の要因特定、実機特有の不具合検出(再現手順付き)を満たすこと。
\subsection{成功判定基準と再現手順}
\subsection{メトリクス(接続成功率、遅延、スループット、CPU/メモリ)}
\subsection{ロギング・パケットキャプチャの取得方法}
