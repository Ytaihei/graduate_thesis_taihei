卒業論文要旨 - 2025年度(令和7年度)
\begin{center}
\begin{large}
\begin{tabular}{|M{0.97\linewidth}|}
    \hline
      \title \\
    \hline
\end{tabular}
\end{large}
\end{center}
\begin{spacing}{1.2}
\small
~ \\

モバイルネットワークは,今やインターネットトラフィックのうちの

エッジコンピューティングは,低遅延・高帯域・セキュリティを確保しながら,ユーザ
の位置に応じた最適なサービスを提供する技術である.ネットワーク上の各ポイントでリ
アルタイムかつ高度なサービスを提供するために極めて重要であり,単なるクラウドの延
長ではない.エッジコンピューティングは,中央集約型のクラウドでは対応しきれない,
分散型の処理能力とデータアクセスのニーズを満たすものである.アプリケーションご
とに遅延許容範囲,帯域幅,セキュリティ要件,ユーザ属性,コンピューティングリソー
スといったサービスポリシーに基づいた適切な展開が求められ,これによりネットワー
クとコンピューティングの統合が効果的に進む.こうした特性により,エッジコンピュー
ティングは,モバイルネットワークや IoT(Internet of Things),自動運転車,スマート
シティなどの幅広い分野で応用が進められている.
一方で,MEC(Multi-access Edge Computing)環境では,特有の課題が存在する.MEC
アプリケーション基盤は通常,UPF(User Plane Function)の外側に配置されるため,5GC
(5G コアネットワーク)によるポリシー適用が UE(User Equipment)から UPF までに
限定される.これ以降の領域ではポリシー適用が途切れるため,5GC 側と MEC 基盤側
の間でポリシーの一貫性が失われるリスクがある.このポリシー適用のギャップにより,
MEC アプリケーション基盤と 5GC の間でサービス管理が分断され,結果として管理負担
の増加や,サービスの品質や可用性の低下が懸念される.特に,異なるアプリケーション
がそれぞれ独自のポリシー要件を持つ場合,この分断が顕著になる.
さらに,この問題を解消するためにポリシー変換や共通化を導入した場合,システム全
体の構造が複雑化する.ポリシーを変換する仕組みや,5GC と MEC 基盤の双方で理解可
能な共通表現を導入することは,システム設計や運用における負担を増加させるだけでな
く,障害リスクを高める要因となる.例えば,ポリシー変換を担う新たなコントローラー
を追加する場合,それがボトルネックや単一障害点となる可能性がある.また,大規模な
ネットワーク障害が発生した際,影響範囲が広がり,復旧プロセスが不透明になるなど,
管理の複雑性がさらなる問題を引き起こすことも想定される.
これらの課題を解決するために,本研究では 5GC のコントロールプレーンを活用し,
MEC 基盤を仮想的なユーザ端末(仮想 UE,vUE)として扱う新しいアーキテクチャを
提案する.このアプローチにより,MEC アプリケーションは仮想 UE 上で動作し,5GC
の既存のポリシー制御フレームワークを利用して一貫性のあるポリシー管理が可能とな
る.5GC の既存機能を最大限に活用することで,追加の変換や共通化の仕組みを不要と
し,システム全体を簡素化することができる.これにより,運用の効率化が図られるだけ
でなく,動的な環境においても柔軟で迅速なサービス提供が実現される.また,仮想 UE
を用いることで,サービス要件に応じたきめ細やかなポリシー適用が可能となり,エッジ
コンピューティングの効果をさらに高めることが期待される.
提案したアーキテクチャをシミュレータ上で実装し,評価を行った.その結果,従来の
方法と比較して,サービスポリシーの一貫性が確保されるとともに,管理の効率化が達成
されることを確認した.特に,動的な環境においても,柔軟かつ迅速なサービス提供が可
能であることが示された.さらに,提案アーキテクチャにより,ネットワークスライシン
グやトラフィックエンジニアリングといった高度なネットワーク機能との統合が容易にな
ることが分かった.この成果は,エッジコンピューティングが抱える課題を解決するだけ
でなく,その利便性をさらに高める新たな可能性を示している.
本研究が提案するアーキテクチャは,MEC 基盤とモバイルネットワーク間の統合を促
進し,エッジコンピューティングの実現における重要な一歩となる.効率的かつ安定した
サービス提供を可能にすることで,将来的なネットワーク設計や運用に対する指針を提
供するものである.また,エッジコンピューティングが拡張されることで,自動運転やス
マートシティなどの高度なユースケースにも対応できる柔軟な基盤となる可能性を秘め
ている.この研究は,エッジコンピューティングの進化を加速させるだけでなく,将来の
ネットワークの設計思想における新たな基準を確立するものである.
~ \\

\end{spacing}

キーワード:\\
\underline{1. モバイルシステム} 
\underline{2. sXGP}
\begin{flushright}
\dept \\
\author
\end{flushright}
