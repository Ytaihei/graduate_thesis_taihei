卒業論文要旨 - 2024年度(令和6年度)
\begin{center}
\begin{large}
\begin{tabular}{|M{0.97\linewidth}|}
    \hline
      \title \\
    \hline
\end{tabular}
\end{large}
\end{center}
\begin{spacing}{1.2}
\small
~ \\
本論文では, 宇宙における遅延・途絶耐性ネットワーク(Delay/Disruption Tolerant Network : DTN)において, 
ルーティングの基本概念となっているContact Graph Routing(CGR)に必要なContact Planの更新に関し, 
Contactの失敗が起きた場合にその情報伝播を当該天体内に限定することを提案する.

近年宇宙開発が大きく進展しており, 特にNASA中心のアルテミス計画では2020年代後半以降, 
月・火星において有人基地や周回軌道のステーションの建設なども予定され, 
2030年代以降, 月・火星に順次通信を行うノードの数が増加することが見込まれる.

しかし現状の地球外のノードとの通信は地上アンテナとの1対1の直接的な通信であり, 
今後月や火星にノードが増加した際には対応できない.そのため直接的な通信のみに頼らず, 
衛星などの宇宙のノードによる通信網, すなわち宇宙インターネットの必要性が認識されており, 
その通信網における技術として既存のインターネット技術を適用することが検討されている.
ただし通信の視点において宇宙環境は, 大きな遅延・頻繁な断絶などが存在し, 
地球の既存のインターネットは前提が大きく異なるため, 
上記のDTNというコンセプトが構想されている.

宇宙のDTNにおいて, そのネットワークを構成するノードの多くは宇宙機であり, 
通信可能なリンクは時間によって変化するものの, そのタイミング等は軌道計算により予測可能である.
このためDTNにおけるルーティングは, 上記のCGRを用いることが計画されている.

CGRでは, 各ノードは自らを含めた全ての衛星に関して, 
特定の2ノードの通信可能な機会(Contact)のリストであるContact Planを事前に保持しており, 
これを元に経路の計算を行い, Next hopとなるDTNノードを決定し転送を行う.
そのためContact Planがノードに配布され適宜更新されていることが必要になるが, 
その配布手法について標準化はなされていない. 

本研究ではContact Planの更新を, 
更新が必要となるタイミングに基づいて定期的・継続的な配布と臨時更新に分類しており, 
臨時更新によるDTNのルーティングにおけるメリットは既に先行研究で明らかになっている.

しかし先行研究では異なる天体間ネットワーク間でもこの臨時更新の情報伝搬を行なっているが, 
これは天体間の貴重なリンクを消費する手法であり, 本研究ではこれを天体内に留めることを提案する. 

2030年代の地球-月間にまたがる宇宙のDTN, 2040年代の地球-月-火星間にまたがる宇宙のDTNを想定した 
シミュレーションの結果, 本研究での提案手法でも, 
先行研究での手法と同等程度の配送能力の向上が確認され, 情報拡散を天体間でも行うことによる
貴重な天体間のリンクの消費を避けつつ, 十分効果的な臨時更新を行うことが可能であることが示された.
~ \\

\end{spacing}

キーワード:\\
\underline{1. Delay/Disruption Tolerant Network} 
\underline{2. Contact Graph Routing} 
\begin{flushright}
\dept \\
\author
\end{flushright}
