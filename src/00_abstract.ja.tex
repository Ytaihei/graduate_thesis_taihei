卒業論文要旨 - 2024年度(令和6年度)
\begin{center}
\begin{large}
\begin{tabular}{|M{0.97\linewidth}|}
    \hline
      \title \\
    \hline
\end{tabular}
\end{large}
\end{center}

~ \\
本論文では、Delay/Disruption Tolerant Network(DTN)において、
ルーティングの基本概念となっているContact Graph Routingに必要なContact Planの更新に関し、
リンク障害が起きた場合にその天体内に情報伝播を限定することを提案する。
近年宇宙開発が大きく進展しており、特にNASA中心のアルテミス計画では2020年代後半以降、
月・火星において有人基地や周回軌道のステーションの建設なども予定され、
2030年代以降、月・火星に順次通信を行うノードの数が増加することが見込まれる。
しかし現状の地球外のノードとの通信は地上アンテナとの1対1の直接的な通信であり、
今後月や火星にノードが増加した際には、
~ \\


キーワード:\\
\underline{1. Delay/Disruption Tolerant Network} 
\underline{2. Contact Graph Routing} 
\begin{flushright}
\dept \\
\author
\end{flushright}
