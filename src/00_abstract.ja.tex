卒業論文要旨 - 2025年度(令和7年度)
\begin{center}
\begin{large}
\begin{tabular}{|M{0.97\linewidth}|}
    \hline
      \title \\
    \hline
\end{tabular}
\end{large}
\end{center}
\begin{spacing}{1.2}
\small
モバイル通信の標準化において、3GPPは仕様書ベースのプロセスを採用しているため、実装・検証が後回しになり、「仕様上は規定されているが実際には動作しない機能」が多数生じている。次世代(6G)に向けては、実装ベースの標準化(implementation-driven standardization)による早期の相互運用性検証と標準化へのフィードバックサイクルの確立が不可欠である。しかし、実機RANを用いた検証は電波法上の制約から困難であり、標準化プロセスの検証基盤が不足している。

本研究は、免許不要帯で運用可能なsXGP(TD-LTE互換)をeNBとして活用し、4GのRAN(UE・eNB)と5G Core(5GC)を接続するコンバータを実装することで、実装ベース標準化を支援する実機検証環境を提案する。本研究では、(1) sXGPを用いた法令遵守型RANの構築、(2) S1AP/NGAP/NASの信令処理とGTP-U中継を行うコンバータの設計・実装、(3) 相互運用性検証とフィードバックのための計測・再現手順の整備、を行った。

評価として、登録・PDUセッション確立などの基本機能を確認し、実機特有の相互運用性問題の検出能力を検証した。提案環境は、(a) 標準仕様の実装検証と相互運用性テスト、(b) 標準化へのタイムリーなフィードバック、(c) 継続的インテグレーション・回帰テスト、(d) 性能評価とボトルネック分析、(e) 教育・トレーニング用途、(f) プロトコル拡張の試作・検証、といったユースケースに適用可能である。本環境により、実装を動かしながら標準化を進めるサイクルを確立し、6G時代のアジャイルな標準化プロセスに貢献する。

\end{spacing}

キーワード:\\
\underline{1. モバイルシステム}
\underline{2. sXGP}
\underline{3. 5G Core}
\underline{4. Open5GS}
\underline{5. GTP-U}
\begin{flushright}
\dept \\
\author
\end{flushright}
