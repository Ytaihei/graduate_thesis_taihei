卒業論文要旨 - 2025年度(令和7年度)
\begin{center}
\begin{large}
\begin{tabular}{|M{0.97\linewidth}|}
    \hline
      \title \\
    \hline
\end{tabular}
\end{large}
\end{center}
\begin{spacing}{1.2}
\small
モバイル通信システムの標準化は、技術革新と商用展開を支える重要なプロセスである。3GPPは仕様を先に策定し、その後実装・検証を行うアプローチを採用しており、これは大規模インフラの互換性確保に寄与してきた。一方で、コアネットワークがソフトウェア中心へ移行した今日、実装の進展が仕様策定に追いつかず、標準仕様と実装の乖離や実装間の相互運用性問題が生じやすい状況が指摘されている。IETFの"rough consensus and running code"原則のように、実装を動かしながら標準化を進める手法は、6G時代のアジャイルな開発サイクルに適している。しかし、モバイル通信では実機RANを用いた検証が電波法上の制約から困難であり、実装ベース標準化を支援する検証基盤が限られている。

本研究では、免許不要帯で運用可能なsXGP(TD-LTE互換)をeNBとして活用し、4G RAN(UE・eNB)と5G Core(5GC)を接続するコンバータを実装することで、法令遵守の範囲で実機検証を可能にする環境を構築した。具体的には、(1) S1AP/NGAP/NASプロトコルの信令変換、(2) GTP-Uトンネル中継によるユーザ面接続、(3) Docker化による再現可能な実験環境の整備、を実現した。登録・PDUセッション確立などの基本機能を検証し、さらにPCOパラメータ処理、QoS/5QI分離、SSCモード設定など、コンバータレベルで実装した相互運用性検証項目を確認した。加えて、IPv6-only + NAT64/464XLAT、UL-CLローカルブレークアウト、GTP-U PMTUD、SMS over NAS、Paging到達性、複数DNN選択など、標準仕様は存在するが実装の成熟度や相互運用性に課題がある項目について、実機を用いた検証が可能な基盤を提供する。

本研究により、実機UEとコアネットワーク間の相互運用性問題を早期検出し、標準化団体へタイムリーにフィードバックする基盤が確立された。提案環境は、標準化サイクルの短縮、継続的インテグレーション・回帰テスト、プロトコル拡張の試作検証に適用可能である。本手法は、ソフトウェア中心の5GC実装が普及する中で、既存4G RAN資産を活用しつつ次世代コアへの移行を支援する実践的アプローチとなる。さらに、免許不要帯を活用した本アプローチは、電波法制約下でも実装検証が可能な研究・教育基盤として、6G標準化における実装ベース検証の促進に貢献することが期待される。

\end{spacing}

キーワード:\\
\underline{1. モバイルシステム}
\underline{2. sXGP}
\underline{3. 5G Core}
\underline{4. Open5GS}
\underline{5. GTP-U}
\begin{flushright}
\dept \\
\author
\end{flushright}
