\chapter{Contact Planの臨時更新の惑星内への限定的な伝播の提案}
\label{chap:suggestion}
本章では、前章でContact Planの臨時更新時に述べた課題に対して、
その解決策が満たすべき要件について整理し、
Contact Planの臨時更新の際の情報拡散を惑星内に限定することを提案する。
Contact Planの臨時更新の目的は、DTNの各ノードがその時点における
ネットワークのより正確なトポロジー情報を得て、最適な経路を選択できるようにすることである。
DTNのどこかでリンク障害が起きた場合、Contact Planを



\section{Contact Planの臨時更新における要件}
ここでは、Contact Planの臨時更新において満たされるべき要件を整理する。

\subsection{要件1: 臨時更新による経路収束までに要する時間の短縮}
\label{subsection:要件1}
\ref{sec:ContactPlanの臨時更新の課題}で述べたように、
リンク障害の発生を検知したノードから、順次他のノードにその情報を通知し
Contact Planを更新していく場合、その情報の到達時間の差によって、
保持しているContact Plan、ひいてはそれから計算しうる最適な経路に、
一時的に不整合が生じ得る。この場合、最適な経路で計算できていないため、
Bundleの到達遅延が一時的に増加すると考えられる。
これが収束するまでの時間は短ければ短いほどよい。

\subsection{要件2: 経路収束後の到達遅延の短縮}
\label{subsection:要件2}
\ref{subsection:要件1}同様、
\ref{sec:ContactPlanの臨時更新の課題}で述べたように、
臨時更新を行うと一時的に到達遅延が増加することが予想されるが、
経路が収束した後は、基本的に到達遅延は一定の値に安定すると考えられる。
この到達遅延はできるだけ短いほど良い。

\section{要件に対する先行手法と本研究の提案手法との比較}
定性的に考えると、上記の要件1・2に対し、既存手法と本研究の提案手法では、
以下のようになることが考えられる。

\begin{itemize}
    \item 要件1について

    既存手法では天体間での情報拡散による臨時更新を行うが、
    当然これは大きな遅延を超えた上でせねばならず、経路収束までの時間が非常に大きくなることが予想される。
    
    \item 要件2について
    
    提案手法ではリンク障害が発生した当該天体以外には
    その情報を拡散しないため、他の天体には現状により即したContact Planは
    提供されておらず、そのため既存手法と比較した場合、経路収束後の到達遅延は
    より大きくなることが予想される。
    
\end{itemize}
