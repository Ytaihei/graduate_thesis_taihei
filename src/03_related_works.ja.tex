\chapter{宇宙におけるコンタクト情報に関連した研究と標準化動向}

\label{chap:related_works}
\section{宇宙インターネットにおけるルーティングのアルゴリズム}
\subsection{CGRのアルゴリズム}
先行研究1: Routing in the Space Internet: A contact graph routing tutorial
CGRについて定義した論文
DTNでのrouting schemeの管理方法については、Centralized,Distributed,Source routingがあるとした上で、「in-depth quantitative comparison of the centralized, source routing and distributed routing approach remains a topic for future research.」として、配送方法についての検討の必要性が提唱されている
\subsection{Yen routing algorithm}

\section{宇宙ネットワークのシミュレーション}
宇宙ネットワークの立ち位置
既存研究・関連研究

先行研究2: Tracking Lunar Ring Road Communication
先行研究1で述べられた、DTNにおけるrouting schemeのうち、Source routingによって配送を行う、月のcubesat コンステレーションを想定。
経路を多く把握するノードを決めておき、各ノードは最低限の経路情報のみを把握
評価には配送成功率と平均配送時間のみを使用し、
上記で述べた「宇宙環境下で最も適した」かどうかについては定性的に判断しているのみ
提案(Approach)
先行研究では、Centralized,Distributedの手法についての定量的な比較が存在しない。
Centralized,Distributedの手法のどちらがより想定する宇宙環境におけるコンタクト情報配布方法として適切か、またそのための適切な条件を探索する

\section{宇宙インターネットにおける経路情報管理手法の確立の必要性}
\section{コンタクト情報から経路計算への処理を行うノード}
\subsection{Distributed Method}
コンタクトグラフそのもの
CCSDSの見本(単純な表形式)に沿って作成。

\subsection{Centralized Method}
コンタクトグラフから計算された経路情報
