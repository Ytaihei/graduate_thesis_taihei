\chapter{DTNのルーティングとContact Planのアップデートにおける課題}

\label{chap:03}
\ref{chap:宇宙インターネットにおけるDTN技術とそのルーティング}では、宇宙開発の進展に伴う宇宙のインターネットの必要性と、
そこで用いられることが構想されているDTN技術について説明した。
しかしDTN技術は今まで実際に宇宙空間で使用された実績・及び使用される計画が少なかったこともあり、
実際の運用においては\ref{chap:02}で述べた通り、未だ多くの課題が残されている。CGRはそれらの課題の多い分野の一つであり、
大きく分けて、Contact Planから経路計算を行う際のアルゴリズムと、それに必要なContact Planの配布手法についてが課題となる。
本章ではこの2つの課題についてそれぞれ先行研究をまとめるとともに、本研究のテーマとしているContact Planのアップデートについての課題について説明する。

\section{宇宙インターネットにおけるルーティングのアルゴリズム}
現状DTNでのルーティングは基本的にCGRが用いられており、CGRでは経路計算の際には基本的にダイクストラが用いられている。しかし他のアルゴリズムの使用や、パラメータを変更することによる改善方法が複数研究されており、本セクションではその内容についてまとめる。
\subsection{Yen routing algorithm}
Fraireらはルーティングテーブルの管理手法について研究を行い、これに基づき現状のCGRは基本的にYen Routing Algorithmを用いている。\cite{FRAIRE2018}

\subsection{CGRに代替するアルゴリズムの研究}
Efficient Contact Graph Routing Algorithms for Unicast and Multicast Bundles\cite{DeJonckere2019}
\section{Contact Planのアップデート}
既に述べた通り、CGRのアルゴリズム・パラメータについての研究は多く行われているが、これらはContact Planに対しての計算手法であり、
経路計算を行いたいノードにおいてそもそもContact Planが存在していることが前提となる。
\subsection{Contact Plan の Planned Update}
ここでは定時的なContact Planのアップデートについての述べる。
\subsection{Contact Plan の Unplanned Update}
ここでは故障時などのContact Planのアップデートについて述べる。
先行研究として\cite{Bezirgiannidis2013}を用いる。
\section{Contact PlanのUnplanned Updateの問題点}
