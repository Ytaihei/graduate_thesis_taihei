\chapter{関連研究と事例}
% --- 章アウトライン・TODO・参考文献引用例 ---
% この章では、既存の5Gコア・RAN実装、標準化、プロトコル変換事例、sXGPの研究利用事例を整理する。
% TODO: docker_open5gs_sXGP-5G 実装と関連する既存研究・事例を明記。
% TODO: 参考文献を本文中で引用する(例: \cite{bundle_protocol_architecture})。
% 例: Open5GSの実装事例は \cite{Fraire2021} などで報告されている。
% ------------------------------------------
\label{chap:related}

\section{オープンソース5GCとテストベッド}

\subsection{Open5GS / free5GC / srsRAN 5G}

5G CoreのOSS実装としては、Open5GS、free5GC、OpenAirInterface(OAI)などが広く用いられている。Open5GSは4G EPC/5G Coreを統合的に実装し、Dockerコンテナでの展開が容易である。WebUIによる加入者管理や、AMF/SMF/UPF等の機能分離が明確で、研究用途に適する。本研究では、Open5GSを基盤として5GCを構築し、sXGP-5Gコンバータとの相互接続検証を行った(\S\ref{chap:experiment})。

free5GCはGo言語で実装された5GCであり、SBAのサービス間通信やNFs間のHTTP/2ベースのやり取りを可視化しやすい利点がある。OAIはRAN/COREの双方を包含する大規模プロジェクトであり、物理層から上位層までの包括的な研究に用いられている。一方で、ビルド依存や環境構築が比較的複雑で、再現性確保の観点ではDocker化・固定化が前提となる。

RAN側のOSSとしては、srsRAN(旧srsLTE)とUERANSIMが代表的である。srsRANは4G eNB/UEに加えて5G gNB/UE(Project系)も提供し、ZMQベースのRFシミュレーションによって無線機器なしにプロトコル検証が可能である。UERANSIMは5GのgNB/UEシミュレータであり、NGAP/NASの制御面検証に適する。本研究では、srsRANのZMQ構成を用い、4G/5G双方の接続性とデータ疎通を確認している(TESTING\_RESULTS参照)。

\subsection{商用テストベッド・教育用環境}

教育・研究のためのテストベッドとして、米国のPOWDERや、欧州のONELab等が提供されている。これらは、遠隔からRAN/CORE資源にアクセスして実験を実施できる環境であるが、アカウント申請や利用審査、スケジューリングが必要であり、迅速な反復実験には必ずしも向かない。一方、OSSとDockerを組み合わせたローカルテストベッドは、構築コストと運用負荷が低く、再現性の高い実験を短時間で繰り返せる利点がある。本研究の環境設計は後者に属する。

\section{OSSエコシステムと適合/相互接続テスト}

\subsection{仕様準拠テスト、相互接続試験の枠組み}

3GPPは適合性試験のための参照フレームワークを整備しているが、実際の実装間相互接続性(IOT)はベンダ横断のイベントや二者間検証で担保されることが多い。TTCN-3を用いた適合性試験スイートは存在するものの、導入・運用コストが高い。研究用途では、pcapのメッセージシーケンス照合、ログ解析、期待結果との比較による軽量な検証が現実的である。

\subsection{回帰テストとCI/CD連携の事例}

OSSコミュニティでは、CI(Continuous Integration)により回帰の早期検出を図る実践が一般化している。Open5GSでもPull Requestに対して自動ビルド・一部ユニットテストが実行される。ネットワークシステムに固有の相互接続テストはCI上での完全再現が難しいが、Docker Composeで最小構成を起動し、UE模擬・RANシミュレータで手順を自動化するアプローチが有効である。本研究でも、Attach/RegistrationとPing疎通を最小回帰セットとして設計している。

\section{LTE eNBと5GCの相互接続に関する標準化}

\subsection{ng-eNBの位置づけ}

3GPP TS~\cite{threegpp-23501}では、E-UTRA無線を用いながらも5GCに接続する基地局としてng-eNBが定義される。ng-eNBはNG-APでAMFと接続し、SCTP/NGAPを用いて制御信令を交換する。一方、従来のeNBはMMEとS1-APで接続し、EPCに属する。両者は同一の無線(E-UTRA)であっても、制御面・ユーザ面の接続先と手順が異なるため、単純な置換では相互接続できない。本研究のs1n2-converterは、既存eNBを変更せずに5GCへ接続するための変換層として機能する。

\subsection{EPS/5GS間インターワーキング(N26, 移動性)}

EPS(EPCを用いる4G)と5GS(5GCを用いる5G)間の相互運用性は、選択的に規定されている。特に、UEの移動性管理やコンテキスト引継ぎに関して、N26インターフェース(MMEとAMF間)が定義されるが、実装と運用は必須ではない。N26が未実装の環境では、Attach/Registrationのやり直しが発生し、遷移時に遅延が増大する。本研究の対象は、ng-eNB化ではなく、既存eNBのまま5GCへの接続性を実現する変換であり、N26相当の移動性は扱わない。

\section{プロトコル変換・ゲートウェイの先行事例}

\subsection{S1AP–NGAP変換のアプローチ}

S1APとNGAPはASN.1ベースの設計思想を共有するが、IEと手順の多くが非互換である。先行事例では、S1AP InitialContextSetup(ICS)に含まれるE-RAB情報から、NGAP InitialContextSetup/PDUSessionResourceSetupに必要な情報を抽出・再構成する手法が示されている。変換層は、UE識別子のマッピング(MME-UE-S1AP-ID/AMF-UE-NGAP-ID等)、E-RAB IDとPDU Session IDの対応付け、NASメッセージの封入形式(4G NAS/5G NAS)を管理する必要がある。

本研究の実装では、ICS Response中のE-RAB Setup ListからGTPトンネル情報(TEID/IP)を抽出し、PDUSessionResourceSetupResponseTransferに対応するTNL情報を生成する試みを行った。しかし、AMF側でPDU Sessionが事前に認識されていない場合に"unknown-PDU-session-ID"が返るなど、5G手順の前提に起因する課題が確認された。対策として、手順の分離(ICSとPDU Session Setupの段階的処理)や、AMF/SMFとの連携順序の見直しが必要であることを示した(開発日誌参照)。

\subsection{GTP-UのTEID管理とパススルー}

ユーザ面におけるGTP-Uは4G/5Gで基本互換であるため、TEIDのマッピングによりパススルーが可能である。変換層は、S1-U側とN3側で独立したTEID空間を持ち、方向ごとにTEIDを管理する。先行事例では、トラフィックの負荷分散やQoS方針に応じてTEID割当を動的に変更するアプローチも報告されている。本研究では、最小実装として静的/逐次割当を採用し、ping疎通までのEnd-to-End動作を確認している。

\section{sXGPの研究利用事例}

\subsection{学術・企業での実験報告}

sXGPは免許不要帯での運用が可能であることから、学術機関や企業において実機に近い環境でのLTE検証に用いられている。屋内小規模のカバレッジで、無線条件や端末挙動を制御しやすく、プロトコル解析や相互運用性評価に適している。OSSコア(Open5GS)との組み合わせにより、加入者管理からPDUセッション確立までの一連の手順を実機で再現できる。

\subsection{運用制約と利点}

sXGPの利点は、法令遵守の範囲で迅速な反復実験が可能である点にある。一方、チャネル幅10MHz・送信電力上限といった物理的制約が存在し、高スループット評価や屋外広域での検証には適さない。研究計画においては、sXGPを基盤とした実機検証と、ZMQ/RFシミュレーションによる反復試験を適切に使い分けることが重要である。本研究は、両者を組み合わせることで、実装ベース標準化に資する検証サイクルを構築した点に特徴がある。
