\chapter{序論}
\section{近年の宇宙開発の進展}
宇宙開発は近年大きく進展している。1970年代に米ソによって月探査が進展した後、その後月面、特に有人による探査は中断されていたが、
2004年にブッシュ大統領は米国の新宇宙政策を発表し、2020年までに米国が再び宇宙飛行士を月面に送り、有人滞在施設の建設することを提唱した。\cite{久保田2009}
この計画は実際には中断されたものの、2017年にトランプ大統領が有人月探査・火星探査を進める大統領令に署名し、2019年にアルテミス計画として発表された。\cite{nasa2020}
2020年には「アルテミス計画を含む広範な宇宙空間の民生探査・利用の諸原則について、関係各国の共通認識を示すこと」を目的にアルテミス合意\cite{artemis_agreement1}も成立し、
当初日本・アメリカ・カナダ・イギリス・イタリア・オーストラリア・ルクセンブルク・アラブ首長国連邦の8カ国が参加した。\cite{artemis_agreement2}
加盟国はその後増加し、2024年時点で40カ国である。\cite{artemis_agreement3}
このように近年宇宙開発は急激に進展しており、その進展について以下の\ref{月・火星の探査計画}、
\ref{民間事業者の宇宙事業への参画}、\ref{深宇宙の探査計画}の三つの視点からの述べる。
\subsection{月・火星の探査計画}
\label{月・火星の探査計画}
このセクションでは、西側のアルテミス計画、及び中露の月・火星探査計画とそのタイムラインについて詳述する。
\subsection{深宇宙の探査計画}
\label{深宇宙の探査計画}
このセクションでは、火星以遠の探査計画、特に小惑星探査やそのほか木星土星の衛星探査についても詳述する。

\subsection{民間事業者の宇宙事業への参画}
\label{民間事業者の宇宙事業への参画}

\section{宇宙通信におけるインターネット技術の適用性}
これらの宇宙開発計画に伴い、 月・火星の地表及びその近傍の空間に多くの人や宇宙機、その他機材が存在するようになり、
天体内・天体間での通信需要が大きくなることが予想される。 
従来までの宇宙ミッションにおいて宇宙のノードと地球との通信は、 地球上にある各国の大型アンテナを利用し、 一対一の通信を行っていた。
しかしこのような計画でノードの数が増加する場合、通信ニーズに対応するためには宇宙にも多対多のノードで通信が可能な宇宙インターネットが必要となる。 
これに向け、既存のインターネットの技術を宇宙インターネットに向け改良し活用することが検討されているが、
当然ながら宇宙環境は地球とは環境が大きく異なり、特に以下の部分に関して考慮が必要となる

\section{通信における宇宙の環境}
宇宙における通信やネットワークに関して、地球とは次のような大きな違いが存在する。
\subsection{大きな遅延のある通信環境}
宇宙での通信は既存のインターネットにおける通信の遅延に比較して非常に大きい。
東京-ニューヨーク間であれば、伝搬遅延のみを考慮した場合、片道50ms以内で通信が可能である一方、宇宙における通信の際には地球月間でも片道1。3秒、
地球火星間では太陽に対する2天体の公転の状況によって変動するが最大20分程度の遅延が想定されている。\cite{doi:10.2514/6.2022-4239}
End-to-EndでTCPを用いた通信を行う際には、 3-way-handshakeなどを含めこれらの天体間を複数回往復する通信を行う必要があり、 
遅延はさらに大きな時間になる。 

\section{断絶のある通信環境}    
そのため宇宙のインターネットにはDelay and Disruption Tolerant Networking(DTN)の技術を利用することが考えられている。 
DTNの技術の一つにBundle Protocol(BP)があり、 BPでは通信されるデータはバンドルという可変長のデータとして転送される。 
中間ノードでは経路上の次のノードへ転送可能なタイミングまでバンドルを蓄積することが可能になっているため、 
End-to-Endの通信疎通性が確保できていない場合でも、 この蓄積による転送を行うことにより断絶に強い通信ができる。
 またトランスポートレイヤにUDPなどのプロトコルを用いることで、 
 比較的遅延を抑えて通信することもできる(図)。

\subsection{ネットワークトポロジーの変動}
中継ノードとなる様々な宇宙機は宇宙空間での位置が常に変化しており、
 天体の影に入るなどで断絶が頻繁に起こる。 
あああああああ
\section{本研究の目的と構成}