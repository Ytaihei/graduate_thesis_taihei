\chapter{序論}
% --- 章アウトライン・TODO・参考文献引用例 ---
% この章では、研究の背景・課題・目的・本論文の貢献・構成を述べる。
% TODO: 具体的な背景事例や課題を Open5GS/sXGP-5G 実装と関連付けて記述。
% TODO: 参考文献を本文中で引用する(例: \cite{rfc5326})。
% 例: 5Gの研究環境構築に関する課題は \cite{McBrayer2022} などで議論されている。
% ------------------------------------------

\section{背景}
モバイル通信の標準化において、3GPPは仕様書を先に策定し、実装・相互運用性検証は後工程となる「仕様先行型」のプロセスを採用している。これは大規模な通信インフラの標準化として必要な側面もあるが、実装が追いつかず、「仕様上は存在するが実際には動作しない機能」や「異なる実装間での相互運用性問題」が多数生じている。一方、IETFなどインターネット技術の標準化では"rough consensus and running code"の原則に基づき、実装を動かしながら標準化を進める「実装ベース標準化」が実践されており、標準と実装の乖離が少ない。

6G時代に向けては、標準化サイクルの高速化と実装・検証の早期化が不可欠である。しかし、モバイル通信では実機RANを用いた検証が電波法上の制約から困難であり、実装ベース標準化を支援する検証基盤が不足している。本研究は、免許不要帯で運用可能なsXGP(TD-LTE互換)をeNBとして活用し、4G RAN(UE・eNB)と5G Core(5GC)を接続するコンバータを提案することで、法令遵守の範囲で実機検証を可能にする。5GCの基本的なアーキテクチャと手順は3GPP TS~\cite{threegpp-23501,threegpp-23502}に規定されており、OSS実装としてはOpen5GS~\cite{open5gs}が広く用いられている。

\section{問題意識と課題}
本研究の背景には、以下の三点に関する強い問題意識がある。

\begin{itemize}
	\item \textbf{実装ベース標準化の不在}: 3GPPの標準化プロセスは仕様書ベースであり、実装・相互運用性検証が後回しになる。結果として「仕様上は存在するが実際には動作しない機能」「実装間での非互換性」が多数生じ、標準化へのフィードバックループが極めて遅い。IETFの"running code"原則とは対照的に、実装を動かしながら標準化を進める文化が欠如している。6G時代のアジャイルな開発・標準化サイクルを確立するには、実装ベースの検証基盤が不可欠である。

	\item \textbf{実装検証環境の不足}: モバイルコアはソフトウェア化が進み、Open Source Software(OSS)も充実してきた。しかし、実機RANを用いた検証環境は電波法上の制約から構築困難であり、OSSの相互運用性や実機特有の問題を検証する手段が限られている。シミュレータでは発見できない実装レベルの問題(タイミング、リソース競合、NIC/ドライバ依存の挙動など)を早期に検出できないことが、標準化と実装のギャップを拡大させている。

	\item \textbf{標準化フィードバックサイクルの遅延}: 現状では、仕様策定 → 実装 → 商用展開 → 問題発覚 → 次期仕様での修正、というサイクルに数年を要する。この遅延により、不具合のある仕様が長期間放置され、互換性問題が蓄積する。実機検証環境を用いた早期の相互運用性テストと標準化団体へのタイムリーなフィードバックが可能になれば、このサイクルを大幅に短縮できる。免許不要帯で運用可能なsXGPは、この課題に対する実践的な解決策となる。
\end{itemize}

これらに加えて、相互接続性、運用・再現性、計測基盤の不足といったモバイルシステム全体の横断的課題を踏まえ、とりわけRANとコア間のインターワーキングが標準化・実装検証のボトルネックである点を指摘する。

\section{研究目的}
\begin{itemize}
	\item 実装ベース標準化を支援する実機検証環境の構築方法を示す。
	\item sXGPベースの4G RANと5GCを接続するコンバータの設計・実装を通じて、標準仕様の相互運用性検証を実現する。
	\item 実機特有の問題を早期検出し、標準化へのフィードバックサイクルを短縮する手法を提案する。
	\item 再現可能な検証環境により、継続的インテグレーション・回帰テストを可能にする。
\end{itemize}

\section{本論文の貢献}
\begin{itemize}
	\item 実装ベース標準化を可能にする実機検証環境(sXGP + 5GC)の実証。標準仕様と実装の乖離を早期発見し、標準化団体へタイムリーにフィードバックする基盤を提供。
	\item 4G RANと5GCの信令・ユーザ面の相互接続に関する設計指針と、実装レベルでの相互運用性問題の分類・解決手法の整理。
	\item 再現性の高い検証プロファイル(トポロジ、計測項目、実験手順)の提示により、継続的な相互運用性テストと回帰テストを支援。
	\item 免許不要帯を活用した法令遵守型の実機検証手法の確立。電波法制約下でも実装を動かしながら標準化を進めるアプローチの実現可能性を示す。
\end{itemize}

\section{論文構成}
本論文は以下の構成である。第\ref{chap:background}章で基礎知識と課題整理、第\ref{chap:related}章で関連研究と事例、第\ref{chap:proposal}章で提案手法、第\ref{chap:experiment}章で実験環境と方法、第\ref{chap:evaluation}章で評価、第\ref{chap:conclusion}章で結論と今後の課題を述べる。
