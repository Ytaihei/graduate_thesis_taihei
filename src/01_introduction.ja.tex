\chapter{序論}

\section{背景}
5Gの研究・実験には専用周波数や高価なRAN機器が必要になることが多く、参入障壁が高い。本研究は、免許不要帯で運用可能なsXGP(TD-LTE互換)をeNBとして活用し、4G RAN(UE・eNB)と5G Core(5GC)を接続するコンバータを提案することで、手軽な5G実験環境を実現する。

\section{問題意識と課題}
モバイルシステム全体での課題(装置コスト、無線免許、相互接続性、運用・再現性、セキュリティ・計測基盤の不足)を整理し、特にRANとコア間のインターワーキングが研究のボトルネックである点を指摘する。

\section{研究目的}
\begin{itemize}
	\item sXGPベースの4G RANと5GCを接続するコンバータの設計・実装方針を示す。
	\item 小規模・低コストで再現可能な5G実験環境の構築方法を提示する。
	\item 基本機能・性能の評価を通じて有用性と限界を明らかにする。
\end{itemize}

\section{本論文の貢献}
\begin{itemize}
	\item sXGPを用いた免許不要・低コストな5G研究環境の実現可能性を実証。
	\item 4G RANと5GCの信令・ユーザ面の相互接続に関する設計指針を整理。
	\item 実験プロファイル(トポロジ、計測項目、再現手順)の提示。
\end{itemize}

\section{論文構成}
本論文は以下の構成である。第\ref{chap:background}章で基礎知識と課題整理、第\ref{chap:related}章で関連研究と事例、第\ref{chap:proposal}章で提案手法、第\ref{chap:experiment}章で実験環境と方法、第\ref{chap:evaluation}章で評価、第\ref{chap:conclusion}章で結論と今後の課題を述べる。
