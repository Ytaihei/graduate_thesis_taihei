\chapter{序論}

\section{本研究の概要}
本論文では, 宇宙における遅延・途絶耐性ネットワーク(Delay/Disruption Tolerant Network : DTN)において, 
ルーティングの基本概念となっているContact Graph Routing(CGR)に必要なContact Planの更新に関し, 
リンク障害が起きた場合にその天体内に情報伝播を限定することを提案する. 
近年宇宙開発が大きく進展しており, 特にNASA中心のアルテミス計画では2020年代後半以降, 
月・火星において有人基地や周回軌道のステーションの建設なども予定され, 
2030年代以降, 月・火星に順次通信を行うノードの数が増加することが見込まれる. 
しかし現状の地球外のノードとの通信は地上アンテナとの1対1の直接的な通信であり, 
今後月や火星にノードが増加した際には対応できない. そのため直接的な通信のみに頼らず, 
衛星などの宇宙のノードによる通信網, すなわち宇宙インターネットの必要性が認識されており, 
その通信網における技術として既存のインターネット技術を適用することが検討されている. 
ただし通信の視点において宇宙環境は, 大きな遅延・頻繁な断絶などが存在し, 
地球の既存のインターネットは前提が大きく異なるため, 
上記のDTNというコンセプトが構想されている. 

また宇宙インターネットにおいて, そのネットワークを構成するノードの多くは宇宙機であり, 
通信可能なリンクは時間によって変化するものの, そのタイミング等は軌道計算により予測可能であるという特徴をもつ. 
このためDTNにおけるルーティングは, 基本的に上記でも述べたCGRを用いることが計画されている. 
CGRでは, 各ノードは自らを含めた全ての衛星に関して, 
特定の2ノードの通信可能な機会(Contact)のリストであるContact Planを事前に保持しており, 
これを元にアルゴリズムで経路の計算を行い, Next hopとなるDTNノードを決定し転送を行う. 
そのためContact Planが適切にノードに配布され適宜更新されていることが必要になるが, 
その配布手法について標準化はなされていない. 本研究では, Contact Planの更新を, 
更新が必要となるタイミングに基づいて定期更新と臨時更新に分類しており, 
臨時更新によるDTNのルーティング上のメリットは既に先行研究で明らかになっている. 
しかし先行研究では異なる天体間ネットワーク間でもこの臨時更新の情報伝搬を行なっている. 
本研究ではこれを天体内に留めることを提案する. 

\section{本研究の目的と構成}
本研究では
Contact Planの臨時更新を行うことで該当ネットワーク内でのBundleの配送遅延が増加することを防ぎつつ, 
かつContact Planの臨時更新の範囲を,リンク障害発生の該当惑星内に限定することにより, 
天体間のリンク消費を抑えている. 
本論文においては, これ以降, 以下のような構成をとる. 

1章では, 本論文の概要, 目的及び構成を述べる. 
2章では, 本論文を読むにあたって必要となる宇宙開発の動向, 通信に関わる宇宙環境, DTNの諸知識について説明する. 
3章では, 本論文の背景となるDTNにおけるルーティングの中心的なコンセプトであるCGRと, その課題について説明する.
4章では, 課題に対する解決策が満たすべき要件と, これらの要件に対する既存手法と提案手法の差異について説明する. 
5章では, 本論文の提案する手法を評価のためのシミュレーション環境について説明する. 
6章では, 提案手法が4章で述べた要件を満たすかについて, それぞれどのような評価を行うかについて説明し, その結果を示すとともに考察を行う. 
7章では, 本論文における結論と今後の展望について述べる.