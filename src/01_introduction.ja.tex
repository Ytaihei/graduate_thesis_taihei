\chapter{序論}
% --- 章アウトライン・TODO・参考文献引用例 ---
% この章では、研究の背景・課題・目的・本論文の貢献・構成を述べる。
% TODO: 具体的な背景事例や課題を Open5GS/sXGP-5G 実装と関連付けて記述。
% TODO: 参考文献を本文中で引用する(例: \cite{rfc5326})。
% 例: 5Gの研究環境構築に関する課題は \cite{McBrayer2022} などで議論されている。
% ------------------------------------------

\section{背景}
\subsection{専用装置からソフトウェア中心への移行}
モバイル通信システムの実装形態は、歴史的に大きな変遷を遂げている。専用装置が主流であった時代には、実装コストが高く、特に物理的な制約を伴うRANシステムの装置開発には長期間と高額な投資が必要であった。このような背景において、仕様を先に固めてから実装を進める「仕様ファースト」のアプローチは合理的であり、装置間の互換性確保や投資リスクの低減に寄与してきた。

一方、コアネットワークは4G初期には専用装置として実装されていたが、その後のNFV(Network Functions Virtualization)の普及を経て、今日では汎用ハードウェア上でコンテナ実装された5GCなど、ソフトウェアが中心となっている。ソフトウェアは物理的な制約が少なく、実装の柔軟性が高い。実装を先行させながら標準化を進めるアプローチは、実際に使われない仕様の発生を抑制したり、製品導入までのリードタイムを短縮できたりするなど、効率的な面も大きい。

\subsection{標準化プロセスと実装ベース検証の重要性}
モバイル通信の標準化において、3GPPは仕様書を先に策定し、実装・相互運用性検証は後工程となるプロセスを採用している。これは大規模な通信インフラの標準化として必要な側面もある一方で、実装の進展が仕様策定に追いつかない場合があり、標準仕様と実装の乖離や異なる実装間での相互運用性の課題が生じやすい。一方、IETFなどインターネット技術の標準化では"rough consensus and running code"の原則に基づき、実装を動かしながら標準化を進める「実装ベース標準化」が実践されており、標準と実装の整合性が保たれやすい。

6G時代に向けては、標準化サイクルの高速化と実装・検証の早期化が重要である。特に、ソフトウェア中心のコアネットワークと既存RANとの間で互換性を確保し、次世代のコア実装へスムーズに移行できる環境を整備することは、標準化と実装の乖離を縮小する上で意義がある。しかし、モバイル通信では実機RANを用いた検証が電波法上の制約から困難であり、実装ベース標準化を支援する検証基盤が限られている。本研究は、免許不要帯で運用可能なsXGP(TD-LTE互換)をeNBとして活用し、4G RAN(UE・eNB)と5G Core(5GC)を接続するコンバータを提案することで、法令遵守の範囲で実機検証を可能にする。5GCの基本的なアーキテクチャと手順は3GPP TS~\cite{threegpp-23501,threegpp-23502}に規定されており、OSS実装としてはOpen5GS~\cite{open5gs}が広く用いられている。

\section{問題意識と課題}
本研究の背景には、以下の三点に関する強い問題意識がある。

\begin{itemize}
	\item \textbf{実装ベース標準化の機会}: 3GPPの標準化プロセスは仕様書を先に策定するアプローチであり、実装・相互運用性検証は後工程となる。これにより、標準仕様と実装の乖離や実装間の非互換性が生じる場合があり、標準化へのフィードバックに時間を要することがある。IETFの"running code"原則のように、実装を動かしながら標準化を進めるアプローチを補完することで、6G時代のアジャイルな開発・標準化サイクルに貢献できる可能性がある。実装ベースの検証基盤はこの方向性を支援する有効な手段である。

	\item \textbf{実装検証環境の制約}: モバイルコアはソフトウェア化が進み、Open Source Software(OSS)も充実してきた。しかし、実機RANを用いた検証環境は電波法上の制約から構築が難しく、OSSの相互運用性や実機特有の問題を検証する手段が限られている。シミュレータでは発見しにくい実装レベルの課題(タイミング、リソース競合、NIC/ドライバ依存の挙動など)を早期に検出できる環境が求められている。

	\item \textbf{標準化フィードバックサイクルの改善余地}: 現状では、仕様策定 → 実装 → 商用展開 → 問題発覚 → 次期仕様での修正、というサイクルに時間を要する場合がある。実機検証環境を用いた早期の相互運用性テストと標準化団体へのタイムリーなフィードバックが可能になれば、このサイクルを短縮し、互換性問題の早期解決に貢献できる。免許不要帯で運用可能なsXGPは、この課題に対する実践的なアプローチとなる。
\end{itemize}これらに加えて、相互接続性、運用・再現性、計測基盤の不足といったモバイルシステム全体の横断的課題を踏まえ、とりわけRANとコア間のインターワーキングが標準化・実装検証のボトルネックである点を指摘する。

\section{研究目的}
\begin{itemize}
	\item 実装ベース標準化を支援する実機検証環境の構築方法を示す。
	\item sXGPベースの4G RANと5GCを接続するコンバータの設計・実装を通じて、標準仕様の相互運用性検証を実現する。
	\item 実機特有の問題を早期検出し、標準化へのフィードバックサイクルを短縮する手法を提案する。
	\item 再現可能な検証環境により、継続的インテグレーション・回帰テストを可能にする。
\end{itemize}

\section{本論文の貢献}
\begin{itemize}
	\item 実装ベース標準化を可能にする実機検証環境(sXGP + 5GC)の実証。標準仕様と実装の乖離を早期発見し、標準化団体へタイムリーにフィードバックする基盤を提供。
	\item 4G RANと5GCの信令・ユーザ面の相互接続に関する設計指針と、実装レベルでの相互運用性問題の分類・解決手法の整理。
	\item 再現性の高い検証プロファイル(トポロジ、計測項目、実験手順)の提示により、継続的な相互運用性テストと回帰テストを支援。
	\item 免許不要帯を活用した法令遵守型の実機検証手法の確立。電波法制約下でも実装を動かしながら標準化を進めるアプローチの実現可能性を示す。
\end{itemize}

\section{論文構成}
本論文は以下の構成である。第\ref{chap:background}章でsXGP・4G/5Gの技術概要、3GPP標準化プロセスと実装ギャップ、RAN-コア間インターワーキング、および法規制上の課題を整理する。第\ref{chap:related}章で関連研究と事例、第\ref{chap:proposal}章で提案手法、第\ref{chap:experiment}章で実験環境と方法、第\ref{chap:evaluation}章で評価、第\ref{chap:conclusion}章で結論と今後の課題を述べる。
