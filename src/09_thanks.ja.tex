\chapter*{謝辞}\markboth{謝辞}{謝辞}
論文の執筆にあたり, ご指導賜りました慶應義塾大学教授 村井純博士, 環境情報学部
教授中村修博士, 同学部教授楠本博之博士, 同学部教授高汐一紀博士, 同学部教授Rodney
D Van Meter 博士, 同学部教授 三次仁博士, 同学部教授 中澤仁博士, 同学部准教授 大越
匡博士, またここには列挙しきれませんでしたがその他合同研究室の教員の皆様に感謝致します. 

特にほぼ全てが未経験の状態で学部3年時にicarに飛び込んでからの2年間, 
長期目標も二転三転し基礎知識もあやふや, あげく手も中々動かせずにいた私に, 
研究のすすめ方から長期的な行動指針に至るまで, 叱咤激励しながらご指導頂いた
環境情報学部教授 植原啓介博士と
東海大学観光学部,政策・メディア研究科特任准教授, 佐藤雅明博士のお二人に感謝申し上げます. 
WIDEプロジェクトSpaceWGの活動を通じご指導を賜りました, 
東京大学大学院総合文化研究科 准教授 石原知洋博士にも感謝申し上げます. 
またAPIEプログラムへのSAとしての参加や, WIDEプロジェクトでの活動範囲を
広げる機会を何度もくださった工藤紀篤博士に感謝申し上げます. 

SpaceWGの活動で苦楽を共にした内田祥喜氏, Tony Chen氏をはじめとする関係者の皆様, 
インターネット自動車研究グループICARでお世話になりました, 
石原匠氏, 古本裕一氏, 赤間滉星氏, 山口泰平氏をはじめとする皆様, 
またデルタ館において多くの時間を共にした
Korry Luke氏, 望月理来氏, 竹村太希氏, 大﨑敦也氏, 斉藤楽氏をはじめとする皆様に感謝申し上げます. 
上記の皆様には特に本研究においても多大なる助言・助力を賜りました. 
皆様と学び, 議論し, 苦境の中でも共にもがき, ラーメンを食べる日々は, 刺激的で大変楽しいものでした. 

Fraire氏に感謝します. 本論文を執筆するにあたり引用させていただいたCGR関連の多くの論文, 実験データ取得に用いた
DTNsimなどは氏が整備してくださったものであり, 氏のDTNとCGRに対する多大なる貢献なしには本論文は完成し得ません
でした. 

私は政策・メディア研究科でもう2年宇宙インターネットの実現に向け
努力を続ける所存です. 主査としてご指導いただく植原教授, より一層の苦境を共にすることであろう
SpaceWGや合同研究会の皆様, 今後ともよろしくお願い申し上げます. 

最後に, 鈴木信章氏, 入江智美氏に深く感謝申し上げます. 色々な制約がある中, 
私が学業を続けられるよう, 大学入学時から今に至るまで, 物心両面で常に応援してくださいました. あと2年は学生期間が続きますが, 
その後の人生に関しては安心してもらえるよう, これからも精進致しますのでもう少々お付き合い願います. 

以上をもって本論文の謝辞とさせていただきます. 研究者としても人間としても未熟者ではありますが, 
ここまでの皆様のご支援に感謝するとともに, 改めまして今後ともよろしくお願い申し上げます. 
\addcontentsline{toc}{chapter}{謝辞}


