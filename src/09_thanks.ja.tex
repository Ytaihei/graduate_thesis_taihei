\chapter*{謝辞}\markboth{謝辞}{謝辞}
\addcontentsline{toc}{chapter}{謝辞}
\label{thanks}
本論文を執筆するにあたり,ご指導賜りました慶應義塾大学教授 村井純博士,慶應義塾大学環境情報学部教授 中村修博士,同学部教授 楠本博之博士,同学部教授 高汐一紀博士,同学部教授 Rodney D.Van Meter 博士,同学部教授 植原啓介博士,同学部教授 三次仁博士,
同学部教授 中澤仁博士,同学部教授 手塚悟博士,同学部教授 武田圭史博士,同学部准教授 大越匡博士,同大学政策・メディア研究科特任教授 鈴木茂哉博士,同研究科特任助教 工藤紀篤博士, 同研究科特任講師 松谷健史博士に感謝いたします.

特に植原啓介博士には rgroot のファカルティとして,日頃から研究面や運用面で指導をしていただきました.
また,1 月に参加した私にとって初めての国際会議に同伴していただき,緊張している私をサポートしていただきました.感謝いたします.

私がコンピュータネットワークの分野に進むきっかけを作っていただいた,慶應義塾大学大学院 豊田安信氏,元慶應義塾大学大学院 (現 NTT コミュニケーションズ) 深川祐太氏に感謝いたします.
私は 2020 年秋学期に開講された,インターネットの設計と運用 という講義でネットワーク技術の面白さを知ることができました.
豊田安信氏,深川祐太氏は TA/SA として私にネットワーク技術の面白さを伝えてくださりました.
また,私を rgroot に誘ってくださったのもこのお二人でした.
ありがとうございます.

東京大学准教授 中村遼博士に感謝いたします.
中村遼博士には,研究面で多大な指導をしていただきました.
研究ネタを一緒に考えてくださり,本論文のアイデアも中村遼博士からいただきました.
また,中村遼博士に指導をしていただきながら執筆した論文は ICOIN 国際会議に採択していただくことができました.感謝いたします.

慶應義塾大学修士課程 石原匠氏に感謝いたします.
石原匠氏は友人として私に接してくれながら,ときには先輩としてその背中を見せてくださりました.
コロナ禍に入学した私には大学に友人が少なかったので,先輩でありながら気軽に話せる存在は大変心の支えになりました.

東京大学大学院 伊藤広記氏,元東京大学大学院 (現 LINE ヤフー株式会社) 金谷光一郎氏に感謝いたします.
伊藤広記氏,金谷光一郎氏は,当時の私と同様にネットワーク運用未経験者として WIDE Project の vSIX ワーキンググループに参加し,共に切磋琢磨しあって頂きました.
伊藤広記氏,金谷光一郎氏は他大学の先輩でありながら,友人としても私に接してくださいました.
ネットワークに入門して日が浅く右も左もわからないとき,わからないなりに共に考え,議論したことはとても良い経験になりました.

父の澤田裕司氏,母の澤田由紀氏に感謝いたします.
家では口数の少ない私ですが,部屋に引きこもってパソコン作業を続けることができたのは家族のサポートあってこそでした.感謝いたします.

東京工業大学附属科学技術高等学校 13 期マイコン制御部 OB に感謝いたします.
コロナ禍で大学に通えず,また新たな友人を作る機会が殆どなかった当時,同期の皆さんと毎晩オンラインゲームに励んだことは心の支えでした.
コロナ禍が明けた今でも,たまに飲みに行ったり,変わらずゲームをしたり,Twitter (X) 上で他愛もないコミュニケーションを取れることは大変嬉しいことです.
本論文執筆に関しても,別の大学,別分野の研究でありながら,互いに鼓舞しあうことでモチベーションを高め合い,書き切ることができました.ありがとうございます.

最後に,全員の名前を書くことはできませんが,村井合同研,WIDE プロジェクト関係者全員に感謝いたします.
私がネットワーク分野に興味を持ち,続けられたのは皆様の力あってこそでした.深く感謝申し上げます.