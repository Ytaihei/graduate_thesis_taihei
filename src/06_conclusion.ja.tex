\chapter{結論と展望}
\label{chap:conclusion}
本論文では,Linux netfilter を統合し,BGP などの既存のルーティングプロトコルとの共存を実現する新しい SRv6 End ビヘイビア,End.AN.NF を提案した.
End.AN.NF は,SRv6 の内部のパケットに対して netfilter の 3 つのフックポイント prerouting,forward,postrouting を透過的に適用させることができる.
netfilter のフックポイントを透過する事により,netfilter を実装に利用して作成されたアプリケーションは,その実装を変更せずに SR-aware アプリケーションとして機能させることができる.
また,End.AN.NF はパケットをマークするために SID の \texttt{ARG} フィールドを活用する.
このアプローチにより,netfilter を内部実装に利用した SF アプリケーションは,パケットバッファ上のマークをマッチングさせることによる動的ななルール調整が可能となる.
論文では End.AN.NF を Linux カーネルに実装し,その性能を評価した.
評価の結果,提案手法の実装は,SRv6 インナーパケットに netfilter のルールを適用する方法である End.DT4 と H.Encaps の組み合わせと比較して,27\% 高いスループットと3.0マイクロ秒低いレイテンシを実現した.
さらに,End と End.AN.NF のスループットの差は 6\% 未満であり,End.AN.NF のオーバーヘッドは最も基本的な End の動作と比較して許容範囲内であることを示している.

本論文の課題として,End.AN.NF について本論文で議論したのはデータプレーンの範疇に収まっている点が挙げられる.
データプレーンとは,ネットワーク通信において,実際のユーザのパケットを処理して転送するメカニズムのことを指す.
データプレーンと対になる概念として,コントロールプレーンが存在する.
コントロールプレーンとは,ユーザパケットの通る経路やポリシーなどを制御する概念であり,BGP などのルーティングプロトコルがコントロールプレーンの要素の例である.
End.AN.NF は SRv6 End ビヘイビアとして設計されているため,その SID を経路情報として広告することができる.
ただし,本論文では具体的なコントロールプレーンの設計を提案して議論することはできていない.

例えば,ある netfilter-based アプリケーションを SF 利用するために End.AN.NF と組み合わせるとき,アプリケーションが想定しているパケットバッファのマークを具体的なルールと結びつけて他のノードに伝える手法は現状議論できていない.
1 つのアイデアとしては,SDN 的な仕組みを使って SF アプリケーションの持つルールセットから経路情報を生成し,それをルートリフレクタを利用して iBGP で広告する手法が挙げられる.
しかし,この手法では SF アプリケーションごとに別な SDN コントローラの実装が必要であり,End.AN.NF の提供する netfilter-based アプリケーションの実装を変更することなく利用可能,という利点を活かすことができない.
今後は本研究を発展させてコントロールプレーンについても議論を行い,Linux と SRv6 による SFC の有用性について更に模索して行きたい.