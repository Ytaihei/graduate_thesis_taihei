\documentclass[12pt]{ujreport}
\usepackage{comment}
\usepackage{./sty/eclepsf}
\usepackage{tascmac}
\usepackage{tabularx}
\usepackage{listliketab}
\usepackage[numbers,sort&compress]{natbib}
\usepackage[dvipdfmx]{graphics}
\usepackage[dvipdfmx]{graphicx}
\usepackage[dvipdfmx]{color}
\usepackage{subfigure}
\usepackage{alltt}
\usepackage{here}
\usepackage{afterpage}
\usepackage{./sty/ncodeline}
\usepackage{url}
\usepackage{amsmath}
\usepackage{setspace}
%\usepackage{glossaries}  % Overleafでのコンパイルエラーを避けるためコメントアウト

%\usepackage[dvipdfmx, colorlinks, breaklinks,%
\usepackage[dvipdfmx, breaklinks,%
bookmarks=true, bookmarksnumbered=true,%
bookmarkstype=toc, bookmarksopen=true,bookmarksopenlevel=3,%
pdftitle={RG},%
]{hyperref}
\usepackage{bookmark}

\AtBeginDvi{\special{pdf:tounicode EUC-UCS2}}

\usepackage{fancyhdr}

\usepackage{./sty/doxygenorig}

\usepackage{indentfirst}
\usepackage{listings,./sty/jlisting}
\usepackage{algorithm}
\usepackage{algpseudocode}
\usepackage{multicol}
\usepackage{caption}
\captionsetup[table]{justification=centering}


\def\lstlistingname{付録}

\lstset{%
 language={C++},
 %backgroundcolor={\color[gray]{.85}},%
 basicstyle={\small\ttfamily},%
 identifierstyle={\small},%
 commentstyle={\small\itshape},%
 keywordstyle={\small\bfseries},%
 ndkeywordstyle={\small\ttfamily},%
 stringstyle={\small\ttfamily},
 frame={tb},
 framesep=1zw,
 breaklines=true,
 numbers=left,%
 xrightmargin=0zw,%
 xleftmargin=1.5zw,%
 numberstyle={\scriptsize},%
 stepnumber=1,
 numbersep=1zw,%
 lineskip=-0.5ex%
}

\usepackage{amssymb}
%\usepackage{supertabular,multirow}

\usepackage{array}
\newcolumntype{M}[1]{>{\centering\arraybackslash}m{#1}}

% A4  size: 297mm*210mm %1pt = 0.35mm
\setlength{\topmargin}{-3.4mm} % 10pt 25.4mm - 3.4mm = 22mm
\setlength{\oddsidemargin}{-0.4mm} % 25.4mm - 0.4mm = 25mm
\setlength{\evensidemargin}{-0.4mm} % 25.4mm - 0.4mm = 25mm
\setlength{\textheight}{231mm} % 660pt % original is 225.75mm 645pt
\setlength{\textwidth}{160mm} % 457pt

\renewcommand{\topfraction}{.99}
\renewcommand{\textfraction}{.0}
\renewcommand{\floatpagefraction}{.99}
\renewcommand{\bibname}{参考文献}
\renewcommand{\baselinestretch}{1.2}

\usepackage{tikz}
\newcommand*\circled[1]{\tikz[baseline=(char.base)]{
            \node[shape=circle,draw,inner sep=1pt] (char) {#1};}}
\pagestyle{fancy}
\lhead[]{}

\makeatletter
\def\chaptermark#1{\markboth {\ifnum \c@secnumdepth>\m@ne
\@chapapp\ \thechapter \@chappos\ \fi #1}{}}
\makeatother

% タイトル
\def\title{sXGP-5G:次世代モバイルコア研究への参入障壁低減のための実験環境の提案}
% 英語タイトル
\def\etitle{sXGP-5G:Proposal of flexible 5G network construction method for next-generation mobile network development}
% 著者(日本語)
\def\author{山口 泰平}
% 著者(英語)
\def\eauthor{Taihei Yamaguchi}
% 学部・研究科
\def\dept{慶應義塾大学 総合政策学部}
% 学部・研究科(英語)
\def\edept{Keio University Bachelor of Arts in Policy Management}


\usepackage{hyperref}
\begin{document}

\pagenumbering{roman}
\begin{titlepage}
  \begin{center}
    \begin{large}
      卒業論文   2024年度(令和6年度)\\
      \vspace{24pt}
      {\Huge \title}
    \end{large}
  \end{center}
  \vspace{40em}
  \begin{flushright}
    \large \dept\\
    \author
  \end{flushright}
\end{titlepage}

\thispagestyle{empty}


卒業論文要旨 - 2025年度(令和7年度)
\begin{center}
\begin{large}
\begin{tabular}{|M{0.97\linewidth}|}
    \hline
      \title \\
    \hline
\end{tabular}
\end{large}
\end{center}
\begin{spacing}{1.2}
\small
~ \\

モバイルネットワークは,今やインターネットトラフィックのうちの

エッジコンピューティングは,低遅延・高帯域・セキュリティを確保しながら,ユーザ
の位置に応じた最適なサービスを提供する技術である.ネットワーク上の各ポイントでリ
アルタイムかつ高度なサービスを提供するために極めて重要であり,単なるクラウドの延
長ではない.エッジコンピューティングは,中央集約型のクラウドでは対応しきれない,
分散型の処理能力とデータアクセスのニーズを満たすものである.アプリケーションご
とに遅延許容範囲,帯域幅,セキュリティ要件,ユーザ属性,コンピューティングリソー
スといったサービスポリシーに基づいた適切な展開が求められ,これによりネットワー
クとコンピューティングの統合が効果的に進む.こうした特性により,エッジコンピュー
ティングは,モバイルネットワークや IoT(Internet of Things),自動運転車,スマート
シティなどの幅広い分野で応用が進められている.
一方で,MEC(Multi-access Edge Computing)環境では,特有の課題が存在する.MEC
アプリケーション基盤は通常,UPF(User Plane Function)の外側に配置されるため,5GC
(5G コアネットワーク)によるポリシー適用が UE(User Equipment)から UPF までに
限定される.これ以降の領域ではポリシー適用が途切れるため,5GC 側と MEC 基盤側
の間でポリシーの一貫性が失われるリスクがある.このポリシー適用のギャップにより,
MEC アプリケーション基盤と 5GC の間でサービス管理が分断され,結果として管理負担
の増加や,サービスの品質や可用性の低下が懸念される.特に,異なるアプリケーション
がそれぞれ独自のポリシー要件を持つ場合,この分断が顕著になる.
さらに,この問題を解消するためにポリシー変換や共通化を導入した場合,システム全
体の構造が複雑化する.ポリシーを変換する仕組みや,5GC と MEC 基盤の双方で理解可
能な共通表現を導入することは,システム設計や運用における負担を増加させるだけでな
く,障害リスクを高める要因となる.例えば,ポリシー変換を担う新たなコントローラー
を追加する場合,それがボトルネックや単一障害点となる可能性がある.また,大規模な
ネットワーク障害が発生した際,影響範囲が広がり,復旧プロセスが不透明になるなど,
管理の複雑性がさらなる問題を引き起こすことも想定される.
これらの課題を解決するために,本研究では 5GC のコントロールプレーンを活用し,
MEC 基盤を仮想的なユーザ端末(仮想 UE,vUE)として扱う新しいアーキテクチャを
提案する.このアプローチにより,MEC アプリケーションは仮想 UE 上で動作し,5GC
の既存のポリシー制御フレームワークを利用して一貫性のあるポリシー管理が可能とな
る.5GC の既存機能を最大限に活用することで,追加の変換や共通化の仕組みを不要と
し,システム全体を簡素化することができる.これにより,運用の効率化が図られるだけ
でなく,動的な環境においても柔軟で迅速なサービス提供が実現される.また,仮想 UE
を用いることで,サービス要件に応じたきめ細やかなポリシー適用が可能となり,エッジ
コンピューティングの効果をさらに高めることが期待される.
提案したアーキテクチャをシミュレータ上で実装し,評価を行った.その結果,従来の
方法と比較して,サービスポリシーの一貫性が確保されるとともに,管理の効率化が達成
されることを確認した.特に,動的な環境においても,柔軟かつ迅速なサービス提供が可
能であることが示された.さらに,提案アーキテクチャにより,ネットワークスライシン
グやトラフィックエンジニアリングといった高度なネットワーク機能との統合が容易にな
ることが分かった.この成果は,エッジコンピューティングが抱える課題を解決するだけ
でなく,その利便性をさらに高める新たな可能性を示している.
本研究が提案するアーキテクチャは,MEC 基盤とモバイルネットワーク間の統合を促
進し,エッジコンピューティングの実現における重要な一歩となる.効率的かつ安定した
サービス提供を可能にすることで,将来的なネットワーク設計や運用に対する指針を提
供するものである.また,エッジコンピューティングが拡張されることで,自動運転やス
マートシティなどの高度なユースケースにも対応できる柔軟な基盤となる可能性を秘め
ている.この研究は,エッジコンピューティングの進化を加速させるだけでなく,将来の
ネットワークの設計思想における新たな基準を確立するものである.
~ \\

\end{spacing}

キーワード:\\
\underline{1. モバイルシステム} 
\underline{2. sXGP}
\begin{flushright}
\dept \\
\author
\end{flushright}

\thispagestyle{plain}
\clearpage

Abstract of Bachelor's Thesis - Academic Year 2024
\begin{center}
\begin{large}
\begin{tabular}{|p{0.97\linewidth}|}
    \hline
      \etitle \\
    \hline
\end{tabular}
\end{large}
\end{center}

~ \\

~ \\
Keywords : \\
\underline{1. Delay/Disruption Tolerant Network} 
\underline{2. Contact Graph Routing} 
\begin{flushright}
\edept \\
\eauthor
\end{flushright}
\thispagestyle{plain}
\clearpage

\tableofcontents\thispagestyle{plain} %目次
\clearpage
\listoffigures\thispagestyle{plain} %図目次
\clearpage
\listoftables\thispagestyle{plain} %表目次
\clearpage

\clearpage

\pagenumbering{arabic}

% 第1章 序論
\chapter{序論}
\section{近年の宇宙開発の進展}
宇宙開発は近年大きく進展している。1970年代に米ソによって月探査が進展した後、その後月面、特に有人による探査は中断されていたが、
2004年にブッシュ大統領は米国の新宇宙政策を発表し、2020年までに米国が再び宇宙飛行士を月面に送り、有人滞在施設の建設することを提唱した。\cite{久保田2009}
この計画は実際には中断されたものの、2017年にトランプ大統領が有人月探査・火星探査を進める大統領令に署名し、2019年にアルテミス計画として発表された。\cite{nasa2020}
2020年には「アルテミス計画を含む広範な宇宙空間の民生探査・利用の諸原則について、関係各国の共通認識を示すこと」を目的にアルテミス合意\cite{artemis_agreement1}も成立し、
当初日本・アメリカ・カナダ・イギリス・イタリア・オーストラリア・ルクセンブルク・アラブ首長国連邦の8カ国が参加した。\cite{artemis_agreement2}
加盟国はその後増加し、2024年時点で40カ国である。\cite{artemis_agreement3}
このように近年宇宙開発は急激に進展しており、その進展について以下の\ref{月・火星の探査計画}、
\ref{民間事業者の宇宙事業への参画}、\ref{深宇宙の探査計画}の三つの視点からの述べる。
\subsection{月・火星の探査計画}
\label{月・火星の探査計画}
このセクションでは、西側のアルテミス計画、及び中露の月・火星探査計画とそのタイムラインについて詳述する。
\subsection{深宇宙の探査計画}
\label{深宇宙の探査計画}
このセクションでは、火星以遠の探査計画、特に小惑星探査やそのほか木星土星の衛星探査についても詳述する。

\subsection{民間事業者の宇宙事業への参画}
\label{民間事業者の宇宙事業への参画}

\section{宇宙通信におけるインターネット技術の適用性}
これらの宇宙開発計画に伴い、 月・火星の地表及びその近傍の空間に多くの人や宇宙機、その他機材が存在するようになり、
天体内・天体間での通信需要が大きくなることが予想される。 
従来までの宇宙ミッションにおいて宇宙のノードと地球との通信は、 地球上にある各国の大型アンテナを利用し、 一対一の通信を行っていた。
しかしこのような計画でノードの数が増加する場合、通信ニーズに対応するためには宇宙にも多対多のノードで通信が可能な宇宙インターネットが必要となる。 
これに向け、既存のインターネットの技術を宇宙インターネットに向け改良し活用することが検討されているが、
当然ながら宇宙環境は地球とは環境が大きく異なり、特に以下の部分に関して考慮が必要となる

\section{通信における宇宙の環境}
宇宙における通信やネットワークに関して、地球とは次のような大きな違いが存在する。
\subsection{大きな遅延のある通信環境}
宇宙での通信は既存のインターネットにおける通信の遅延に比較して非常に大きい。
東京-ニューヨーク間であれば、伝搬遅延のみを考慮した場合、片道50ms以内で通信が可能である一方、宇宙における通信の際には地球月間でも片道1。3秒、
地球火星間では太陽に対する2天体の公転の状況によって変動するが最大20分程度の遅延が想定されている。\cite{doi:10.2514/6.2022-4239}
End-to-EndでTCPを用いた通信を行う際には、 3-way-handshakeなどを含めこれらの天体間を複数回往復する通信を行う必要があり、 
遅延はさらに大きな時間になる。 

\section{断絶のある通信環境}    
そのため宇宙のインターネットにはDelay and Disruption Tolerant Networking(DTN)の技術を利用することが考えられている。 
DTNの技術の一つにBundle Protocol(BP)があり、 BPでは通信されるデータはバンドルという可変長のデータとして転送される。 
中間ノードでは経路上の次のノードへ転送可能なタイミングまでバンドルを蓄積することが可能になっているため、 
End-to-Endの通信疎通性が確保できていない場合でも、 この蓄積による転送を行うことにより断絶に強い通信ができる。
 またトランスポートレイヤにUDPなどのプロトコルを用いることで、 
 比較的遅延を抑えて通信することもできる(図)。

\subsection{ネットワークトポロジーの変動}
中継ノードとなる様々な宇宙機は宇宙空間での位置が常に変化しており、
 天体の影に入るなどで断絶が頻繁に起こる。 
あああああああ
\section{本研究の目的と構成}

% 第2章 基礎知識と課題整理
\chapter{sXGP・4G/5Gの技術概要と標準化プロセスの課題}
\label{chap:background}

\section{sXGPと4G/5Gの概要}

\subsection{sXGPの位置づけ(免許不要・TD-LTE互換)}

sXGP(shared eXtended Global Platform)は、日本国内において免許不要帯である1.9GHz帯を使用するTD-LTE互換の通信規格である。2015年の電波法改正により、自営等BWAシステムとして制度化され、免許取得なしで基地局の設置・運用が可能となった。チャネル幅は10MHz、送信電力は基地局で1W以下、端末で200mW以下に制限されており、屋内利用を前提とした運用形態が想定されている。

sXGPの最大の特徴は、3GPP標準のLTE(特にTD-LTE)との互換性を維持しながら、免許不要で運用可能な点にある。これにより、一般的なLTE対応端末(スマートフォン等)がそのまま利用でき、研究・教育機関において実機を用いたモバイルネットワーク実験が法令遵守の範囲で実施可能となる。特に、電波暗室や特定実験試験局の許可を必要とする通常のLTE実験と比較して、導入の障壁が大幅に低減される。

一方で、sXGPには周波数帯域幅の制約(10MHz)やカバレッジの制限(屋内利用想定)といった限界も存在する。また、利用可能な基地局機器は限定的であり、研究用途では商用装置の転用やSDR(Software Defined Radio)を用いた実装が主流となる。本研究では、sXGPをeNBとして活用することで、4G RAN(UE・eNB)と5G Core(5GC)を接続する実機検証環境を構築し、実装ベース標準化を支援する検証基盤を提供する。

\subsection{4G RAN(UE・eNB)とEPC/5GCの要素}

4G LTEシステムは、UE(User Equipment)、eNB(evolved NodeB)、EPC(Evolved Packet Core)から構成される。UEとeNB間の無線インタフェースはUuインタフェースと呼ばれ、RRC(Radio Resource Control)プロトコルによって無線リソース管理が行われる。eNBとEPC間は、制御面(Control Plane)がS1-MMEインタフェース、ユーザ面(User Plane)がS1-Uインタフェースで接続される。

EPCの主要構成要素として、MME(Mobility Management Entity)、HSS(Home Subscriber Server)、SGWC/SGWU(Serving Gateway Control/User plane)、PGWC/PGWU(PDN Gateway Control/User plane)が存在する。MMEは加入者の認証・位置管理・セッション管理を担当し、S1-APプロトコルを用いてeNBと通信する。HSSは加入者情報データベースであり、認証鍵(K、OPc)やAMF値を管理する。SGWU/PGWUはGTP-U(GPRS Tunneling Protocol - User plane)トンネルを用いてユーザデータを転送する。

4G LTEにおける基本的な手順は、Attachプロシージャ(初期接続)、Authentication(認証)、Default Bearer確立、PDN接続確立から構成される。NAS(Non-Access Stratum)メッセージは、UEとMME間でエンドツーエンドに交換され、eNBは透過的に中継する。本研究では、4G RANからのS1-APメッセージを5GC向けのNGAPメッセージに変換するコンバータを実装することで、既存4G RAN資産を5GCと接続可能にする。

\subsection{5GCのアーキテクチャ(AMF/SMF/UPF等)}

5G Core(5GC)は、サービスベースアーキテクチャ(SBA: Service Based Architecture)を採用し、各ネットワーク機能(NF: Network Function)がHTTP/2ベースのサービスインタフェースを介して相互に通信する。主要なNFとして、AMF(Access and Mobility Management Function)、SMF(Session Management Function)、UPF(User Plane Function)、UDM(Unified Data Management)、AUSF(Authentication Server Function)、NRF(Network Repository Function)などが存在する。

AMFは、UEの登録管理・接続管理・モビリティ管理を担当し、NG-APプロトコルを用いてgNB(5G基地局)または本研究のs1n2-converterと通信する。SMFは、PDUセッション管理・QoS制御・UPF選択を担当し、AMFおよびUPFと連携してデータパスを確立する。UPFは、ユーザプレーンのトラフィック転送・QoS適用・課金情報収集を担当し、GTP-Uトンネルを終端する。

5GCにおける基本的な手順は、Registration(登録)、Authentication(認証)、PDU Session Establishment(セッション確立)から構成される。4G LTEと比較して、制御面とユーザ面の分離(CUPS: Control and User Plane Separation)が徹底されており、SMFとUPFの独立性が高い。また、NASメッセージは5G NAS形式となり、セキュリティコンテキストの導出方法も変更されている(KAMFからKNASint/KNASencを導出)。

本研究で利用する5GC機能範囲は、基本的な登録・認証・PDUセッション確立に限定し、ネットワークスライシング・エッジコンピューティング・ローカルブレイクアウトなどの高度な機能は対象外とする。実装にはOpen5GSを採用し、AMF、SMF、UPF、UDM、AUSF、NRF、PCF(Policy Control Function)などのNFをDockerコンテナとして展開する。

\section{3GPP標準化と標準-実装ギャップ}

\subsection{仕様の包含範囲とリファレンス実装の不足}

3GPP(3rd Generation Partnership Project)は、モバイル通信システムの標準化を行う国際的な組織であり、SA(Service and System Aspects)、CT(Core network and Terminals)、RAN(Radio Access Network)の3つの技術仕様グループ(TSG: Technical Specification Group)に分かれている。各TSGは複数のワーキンググループ(WG)を持ち、それぞれがTS(Technical Specification)やTR(Technical Report)を策定する。

例えば、5G Coreの基本アーキテクチャはTS 23.501、プロシージャはTS 23.502、NGAPはTS 38.413、5G NASはTS 24.501に規定されている。これらの仕様書は相互に参照関係を持ち、全体として数千ページに及ぶ膨大な文書群を形成している。仕様には必須(Mandatory)要件と任意(Optional)要件が混在し、実装者は必須要件のみを実装することが許容される。

しかし、3GPP仕様にはリファレンス実装が存在せず、仕様書の記述が実装レベルで曖昧な場合がある。例えば、エラー処理の詳細、タイマー値の推奨範囲、再送ポリシー、並行処理時の順序保証などは、仕様書に明示的に記載されていないことが多い。このため、ベンダ各社が独自に解釈・実装を行い、結果として実装間の相互運用性に課題が生じる。

研究においては、このような仕様の曖昧性が実験の再現性や比較可能性に影響を与える。特に、OSS実装(Open5GS、srsRAN等)は仕様の一部のみを実装しており、商用実装との機能差が存在する。本研究では、Open5GSを基盤として使用するが、実装上の制約を明示的に文書化し、検証範囲を明確にすることで、研究の内的妥当性を確保する。

\subsection{ベンダ実装差と相互接続性の課題}

モバイル通信システムの実装には、ベンダ固有の解釈や拡張が含まれることが一般的である。代表的なベンダ実装差として、以下が挙げられる:

\begin{itemize}
\item \textbf{Information Element(IE)の扱い}:任意IEの実装有無、未知IEの無視/拒否ポリシー、IE順序の依存性
\item \textbf{タイマー値とリトライ回数}:デフォルト値の差異、ネットワーク条件に応じた動的調整の有無
\item \textbf{再送とフォールバック}:メッセージ再送時の挙動、暗号化アルゴリズムのネゴシエーション失敗時の処理
\item \textbf{ベンダ拡張IE}:3GPP標準外の独自IEの追加、相互運用性試験での扱い
\end{itemize}

これらの実装差は、IOT(InterOperability Test)において顕在化する。特に、異なるベンダのUE/RAN/Coreを組み合わせた場合、標準仕様に準拠していても相互接続に失敗するケースが報告されている。既知の回避策として、IOTイベントでの事前検証、ベンダ間での実装ノートの共有、3GPP Change Request(CR)による仕様明確化などが行われている。

OSS実装においても、実装の成熟度や機能範囲の差が存在する。例えば、Open5GSはデータ通信機能を中心に実装されており、VoLTE/IMS、SMS over NAS、複数DNN選択などの機能は未実装または部分実装である。本研究では、基本的なデータ通信機能を中心に検証を行い、高度な機能については今後の拡張課題として位置づける。

\subsection{研究・評価におけるギャップの影響}

標準仕様と実装の乖離は、研究の再現性および比較可能性に重大な影響を与える。特に、以下の点が課題となる:

\begin{itemize}
\item \textbf{内的妥当性}:実験条件の統制が困難(実装依存の挙動、バージョン差異、設定パラメータの影響)
\item \textbf{外的妥当性}:OSS実装での検証結果が商用環境に適用可能か不明(機能範囲の差、性能特性の違い)
\item \textbf{再現性}:実装のバージョン、設定ファイル、依存ライブラリのバージョンが論文に明記されないことが多い
\end{itemize}

これらの課題に対する対策として、本研究ではOSS実装を用いた検証基盤の構築を提案する。具体的には、Docker化による環境の固定化、設定ファイルやスクリプトの公開、パケットキャプチャの保存により、第三者による検証の再現を可能にする。また、実装の制約を明示的に文書化することで、検証結果の適用範囲を明確にする。

さらに、IETFの"rough consensus and running code"原則に倣い、実装を動かしながら標準化にフィードバックするアプローチを採用する。これにより、仕様の曖昧性や実装上の問題点を早期に発見し、標準化サイクルの短縮に貢献することを目指す。

\section{RAN–コア間インターワーキングの基礎}

\subsection{制御面(S1-AP vs. NG-AP)}

4G LTEにおけるeNBとMME間の制御面プロトコルはS1-AP(S1 Application Protocol)であり、5GにおけるgNBとAMF間の制御面プロトコルはNG-AP(NG Application Protocol)である。両プロトコルはASN.1(Abstract Syntax Notation One)で定義され、SCTP(Stream Control Transmission Protocol)上で動作する点は共通であるが、メッセージ構造やInformation Element(IE)の定義が異なる。

表\ref{tab:s1ap_ngap_mapping}に、主要なS1-APとNG-APメッセージの対応関係を示す。

\begin{table}[htbp]
\centering
\caption{S1-APとNG-APの主要メッセージ対応}
\label{tab:s1ap_ngap_mapping}
\begin{tabular}{|l|l|l|}
\hline
\textbf{手順} & \textbf{S1-AP} & \textbf{NG-AP} \\
\hline
初期接続 & InitialUEMessage & InitialUEMessage \\
\hline
初期コンテキスト設定 & InitialContextSetupRequest & InitialContextSetupRequest \\
\hline
初期コンテキスト設定応答 & InitialContextSetupResponse & InitialContextSetupResponse \\
\hline
NAS下り転送 & DownlinkNASTransport & DownlinkNASTransport \\
\hline
NAS上り転送 & UplinkNASTransport & UplinkNASTransport \\
\hline
UEコンテキスト解放要求 & UEContextReleaseRequest & UEContextReleaseRequest \\
\hline
\end{tabular}
\end{table}

メッセージ名は類似しているが、含まれるIEの内容は大きく異なる。特に、セッション管理に関連するIEは、4GのBearer概念から5GのPDU Session概念への変更に伴い、大幅に再設計されている。例えば、4GのE-RAB(E-UTRAN Radio Access Bearer)は5GのPDU Session Resourceに対応するが、QoS制御の粒度が異なる(4GはBearer単位、5GはQoS Flow単位)。

本研究で実装するs1n2-converterは、S1-APメッセージを受信してNG-APメッセージに変換する。変換時の主要なポイントとして、以下が挙げられる:

\begin{itemize}
\item UE識別子の変換(MME-UE-S1AP-ID ↔ AMF-UE-NGAP-ID、eNB-UE-S1AP-ID ↔ RAN-UE-NGAP-ID)
\item E-RAB情報からPDU Session情報への変換(E-RAB ID → PDU Session ID、QCI → 5QI)
\item GTP-Uトンネル情報の抽出と再マッピング(TEID、Transport Layer Address)
\item NASメッセージのカプセル化(4G NAS → 5G NAS変換は別途実施)
\end{itemize}

エラー処理については、変換不可能なIEが含まれる場合や、必須IEが欠落している場合の処理方針を定義する必要がある。本研究では、基本的なデータ通信に必要なIEのみを変換し、未対応のIEについてはログに記録した上でスキップする方針を採用する。

\subsection{ユーザ面(GTP-U互換性とパススルー/変換)}

ユーザ面プロトコルであるGTP-U(GPRS Tunneling Protocol - User plane)は、4Gと5Gで基本的に互換性が維持されている。GTP-Uは、IPパケットをカプセル化してトンネル転送するプロトコルであり、TEID(Tunnel Endpoint Identifier)を用いて各トンネルを識別する。

GTP-Uトンネルの確立は、制御面メッセージ(S1-AP/NG-AP)によって指示される。4Gでは、InitialContextSetupRequestメッセージにE-RAB Setup ListとしてTEIDとTransport Layer Address(IPアドレス)が含まれる。5Gでは、PDUSessionResourceSetupRequestメッセージにPDU Session Resource Setup Request Transferとして同様の情報が含まれる。

本研究で実装するs1n2-converterは、GTP-Uパケットの中継を行う。具体的には、以下の処理を実施する:

\begin{itemize}
\item \textbf{S1-U → N3方向}:4G eNBからのGTP-Uパケットを受信し、TEID変換テーブルを参照して5G UPF向けのTEIDに書き換え、N3インタフェース(5G UPF)に転送
\item \textbf{N3 → S1-U方向}:5G UPFからのGTP-Uパケットを受信し、TEID変換テーブルを参照して4G eNB向けのTEIDに書き換え、S1-Uインタフェース(4G eNB)に転送
\end{itemize}

TEID割り当てについては、s1n2-converter自身が上り方向と下り方向のTEIDを独立に管理し、eNB側とUPF側に対してそれぞれ異なるTEIDを割り当てる。これにより、eNBとUPFが同じTEID空間を共有する必要がなくなり、TEID衝突のリスクを回避できる。

QoS制御については、4GのQCI(QoS Class Identifier)と5Gの5QI(5G QoS Identifier)の対応関係を定義し、変換を行う。標準化されたQCI/5QIの対応関係は3GPP TS 23.501に規定されているが、本研究では基本的なデータ通信(QCI=9 → 5QI=9)のみを対象とする。

MTU(Maximum Transmission Unit)とフラグメント処理については、GTP-Uヘッダのオーバーヘッド(8バイト + UDPヘッダ8バイト + IPヘッダ20バイト = 36バイト)を考慮し、必要に応じてPath MTU Discoveryを実施する。ただし、本研究の実験環境はDocker内部ネットワークを使用するため、MTU問題は発生しにくい。

\subsection{ID/コンテキスト管理(NAS, IMSI/SUPI 等)}

モバイルネットワークにおけるUE識別子は、永続的識別子と一時的識別子に分類される。4Gでは、永続的識別子としてIMSI(International Mobile Subscriber Identity)、一時的識別子としてGUTI(Globally Unique Temporary Identifier)が使用される。5Gでは、永続的識別子としてSUPI(Subscription Permanent Identifier)、一時的識別子として5G-GUTIが使用される。

IMSI/SUPIの形式は類似しており、MCC(Mobile Country Code)、MNC(Mobile Network Code)、MSIN(Mobile Subscription Identification Number)から構成される。本研究では、4GのIMSIを5GのSUPIとして扱い、識別子の変換を行わない方針を採用する。これは、Open5GS HSSとUDMが同一の加入者データベース(MongoDB)を共有しており、IMSIベースでの管理が可能なためである。

NASセキュリティコンテキストは、UEとMME/AMF間のNASメッセージを保護するために使用される。4Gでは、認証成功後にKASME(Key Access Security Management Entity)が導出され、そこからKNASint(NAS Integrity Key)とKNASenc(NAS Encryption Key)が導出される。5Gでは、認証成功後にKAMF(Key Access and Mobility Management Function)が導出され、そこからKNASint/KNASencが導出される。

鍵導出のアルゴリズムは4Gと5Gで異なり、単純な変換は不可能である。本研究では、s1n2-converterがNASメッセージの変換を行う際に、NAS Integrity Protectionの検証を行わない方針を採用する。これは、4G eNBと5G AMF間でNASセキュリティコンテキストが共有されないためである。ただし、この方針はセキュリティ上のリスクを伴うため、実験環境に限定した運用とし、本番環境での使用は推奨しない。

コンバータ内でのUEコンテキスト管理については、S1-AP UE IDとNG-AP UE IDのマッピングテーブルを維持し、メッセージ変換時に参照する。また、GTP-U TEIDマッピングテーブルも同様に維持し、ユーザ面パケットの中継時に使用する。これらのテーブルは、UEContextReleaseRequestメッセージ受信時にクリーンアップされる。

\section{OSSと6Gに向けた開発・検証サイクル}

\subsection{OSSの役割(Open5GS等)と試作の加速}

Open Source Software(OSS)は、モバイルネットワーク研究における重要な基盤となっている。特に、Open5GSは4G EPC/5G Coreの包括的な実装を提供し、世界中の研究機関や企業で広く利用されている。Open5GSの主な特徴として、以下が挙げられる:

\begin{itemize}
\item \textbf{モジュール構成}:各ネットワーク機能(MME、AMF、SMF、UPF等)が独立したプロセスとして実装され、マイクロサービスアーキテクチャを採用
\item \textbf{拡張性}:C言語で実装され、ソースコードが公開されているため、独自機能の追加や動作のカスタマイズが容易
\item \textbf{活発なコミュニティ}:GitHub上で継続的に開発が行われており、バグ修正や新機能追加が頻繁にリリースされる
\item \textbf{標準準拠}:3GPP仕様に基づいた実装が行われており、商用実装との相互運用性が考慮されている
\end{itemize}

OSSを活用することで、研究プロトタイプの実装速度が大幅に向上する。商用装置を用いた実験では、ベンダとの契約や機器調達に数ヶ月を要するのに対し、OSSを用いた実験環境は数日で構築可能である。また、学習コストも低減される。Open5GSはドキュメントが充実しており、Webベースの管理画面(WebUI)を提供するため、初学者でも容易に環境構築が可能である。

本研究では、Open5GSを基盤として使用し、4G EPCおよび5G Coreを構築する。さらに、srsRAN(旧srsLTE)をRANシミュレータとして使用し、ZMQ(ZeroMQ)ベースのRF simulationによってeNB/gNBとUEを接続する。これにより、物理的な無線機器を必要とせず、ソフトウェアのみで完全なEnd-to-End環境を構築できる。

\subsection{適合・相互接続・回帰テストの自動化}

モバイルネットワークの品質保証には、適合性試験(Conformance Test)、相互運用性試験(InterOperability Test)、回帰試験(Regression Test)が不可欠である。これらの試験を手動で実施することは時間とコストがかかるため、自動化の仕組みが重要となる。

自動試験の枠組みとして、以下のアプローチが考えられる:

\begin{itemize}
\item \textbf{シナリオ駆動試験}:試験シナリオをスクリプトまたはYAML/JSON形式で記述し、試験フレームワークが自動的に実行
\item \textbf{パケットキャプチャ照合}:Wireshark/tsharkを用いてパケットキャプチャを取得し、期待されるメッセージシーケンスと照合
\item \textbf{ログ解析}:各ネットワーク機能のログ出力を解析し、エラーメッセージや警告の有無を検証
\item \textbf{差分検出}:基準となる「正常動作時」のパケットキャプチャ/ログと、変更後の結果を比較し、差分を自動検出
\end{itemize}

既存のテスト資産として、TTCN-3(Testing and Test Control Notation version 3)を用いた適合性試験ツールが存在する。TTCN-3は、通信プロトコルのテストケースを形式的に記述するための言語であり、3GPP標準化においても参照されている。ただし、TTCN-3の学習コストは高く、研究用途では軽量なスクリプトベースのアプローチが好まれる。

本研究では、基本的なEnd-to-End接続性を確認するための最小限の回帰テストセットを設計する。具体的には、以下のシナリオを自動実行し、成功/失敗を判定する:

\begin{enumerate}
\item 4G Attach + Default Bearer確立 + Ping疎通
\item 5G Registration + PDU Session確立 + Ping疎通
\item 4G → 5G変換環境でのAttach + Session確立 + Ping疎通
\end{enumerate}

これらのテストは、Docker Composeを用いた環境起動スクリプトと組み合わせることで、完全に自動化可能である。将来的には、CI/CD(Continuous Integration / Continuous Deployment)パイプラインに統合し、コード変更のたびに自動的に実行することを想定している。

\subsection{再現性(Docker等)と研究公開の促進}

研究の再現性を確保するためには、実験環境の詳細な記録と共有が不可欠である。しかし、モバイルネットワーク実験は多数のコンポーネント(UE、RAN、Core)と複雑な設定を伴うため、環境再現が困難であることが多い。

Dockerコンテナ技術を活用することで、この課題を大幅に軽減できる。Dockerは、アプリケーションとその依存関係を単一のコンテナイメージとしてパッケージ化し、異なる環境で同一の動作を保証する技術である。本研究では、以下の方針でDocker化を実施する:

\begin{itemize}
\item \textbf{固定バージョン}:Open5GS、srsRAN、依存ライブラリのバージョンをDockerfileに明示的に記述
\item \textbf{設定ファイル共有}:各ネットワーク機能の設定ファイル(YAML形式)をGitリポジトリで管理
\item \textbf{ネットワーク構成の明示}:Docker Composeを用いて、コンテナ間のネットワーク構成(IPアドレス、サブネット、ブリッジ)を定義
\item \textbf{ログとパケットキャプチャの保存}:実験結果として、各コンポーネントのログファイルとWiresharkパケットキャプチャを保存
\end{itemize}

研究成果を公開する際には、以下の留意点がある:

\begin{itemize}
\item \textbf{認証鍵の除去}:HSS/UDMに登録された加入者の認証鍵(K、OPc)はダミー値に置き換える
\item \textbf{個人情報の除去}:IMSI、MSISDN、IPアドレスなどの個人情報は匿名化またはダミー値に置き換える
\item \textbf{ライセンス確認}:使用したOSSのライセンス(Apache、GPL等)を確認し、派生物の公開条件を遵守
\end{itemize}

本研究では、GitHub上でソースコード、設定ファイル、実験手順書を公開することを計画している。これにより、第三者が同一の実験環境を再現し、検証結果を確認できるようにする。また、Docker Hubにコンテナイメージを公開することで、環境構築の手間をさらに削減することも検討している。

\section{モバイルシステム全体での課題感}

\subsection{研究環境のコストとライセンス(免許・機器費用)}

モバイルネットワーク研究における最大の障壁の一つは、実機環境構築のコストと法的手続きである。典型的な4G/5G研究環境を構築する場合、以下の費用が発生する:

\begin{itemize}
\item \textbf{端末/モデム}:LTE/5G対応スマートフォンまたはUSBモデムで、数万円〜数十万円(研究用途では複数台必要)
\item \textbf{基地局(eNB/gNB)}:商用装置は数百万円〜数千万円、研究用小型基地局でも数十万円〜数百万円
\item \textbf{SDR(Software Defined Radio)}:USRP B210(約50万円)、LimeSDR(約5万円)など
\item \textbf{アンテナ・RF機器}:周波数帯に応じた適切なアンテナ、増幅器、フィルタなど
\item \textbf{測定機器}:スペクトラムアナライザ、シグナルジェネレータ、ネットワークアナライザなど(数百万円〜)
\end{itemize}

さらに、電波法に基づく免許取得または特定実験試験局の申請が必要となる。申請には技術基準適合証明(技適)の有無、使用周波数帯、出力電力、設置場所の詳細な情報が求められ、承認までに数週間〜数ヶ月を要する。また、屋外での実験は原則として禁止されており、電波暗室の使用が推奨されるが、電波暗室の建設・維持コストも高額である。

sXGP採用によるコスト/手続き削減効果は顕著である。sXGPは免許不要帯を使用するため、免許申請が不要であり、屋内であれば比較的自由に実験が可能である。基地局機器としては、LimeSDRなどの低コストSDRを用いることで、数万円程度で環境構築が可能となる。本研究では、ZMQベースのRF simulationを使用することで、さらにコストを削減し、ソフトウェアのみで完全な実験環境を構築している。

\subsection{相互接続性・ベンダ依存の壁}

モバイルネットワーク実験における相互接続性の課題は、複数のレイヤーで発生する:

\begin{itemize}
\item \textbf{UE/モデム差}:チップセットベンダ(Qualcomm、MediaTek、Exynos等)による実装差、OSバージョン(Android、iOS)の影響
\item \textbf{カーネル/ドライバ差}:LinuxカーネルのNetfilter/iptables設定、NICドライバのTCO(TCP Checksum Offload)やGRO(Generic Receive Offload)の挙動
\item \textbf{RAN実装差}:商用基地局、研究用小型基地局、SDRベース実装の機能範囲とタイミング精度の違い
\item \textbf{Core実装差}:Open5GS、商用EPC/5GC、他のOSS実装(free5GC、OAI等)の機能カバレッジと相互運用性
\end{itemize}

これらの差異は、実験結果の再現性と汎用性に影響を与える。例えば、特定のUEでは成功する手順が、別のUEでは失敗するケースが報告されている。また、Linux環境でのパケット転送性能は、NICオフロード機能の有効/無効によって大きく変動する。

評価設計での緩和策として、以下のアプローチが考えられる:

\begin{itemize}
\item \textbf{多様な端末での検証}:複数のベンダ/OSの端末を用いて、実装依存性を明らかにする
\item \textbf{OSS実装の組み合わせ}:Open5GS、free5GC、srsRANなど、複数のOSS実装を試験することで、標準準拠度を評価
\item \textbf{設定の明示的な文書化}:カーネルバージョン、NIC設定、オフロード機能の有効/無効を明記
\end{itemize}

本研究では、Open5GSとsrsRANの組み合わせを基本構成とし、実装の制約を明示的に文書化する。将来的には、他のOSS実装との相互運用性試験を実施することを計画している。

\subsection{運用・再現性とオープンテストベッドの不足}

モバイルネットワーク研究において、共有可能なテストベッドの不足は長年の課題である。研究機関ごとに独自の環境を構築しているが、環境の詳細が論文に記載されないことが多く、第三者による再現が困難である。また、商用ネットワークを用いた実験は、ベンダとの契約やセキュリティ上の制約から、詳細な実験条件を公開できないことが多い。

オープンテストベッドの事例として、米国のPOWDER(Platform for Open Wireless Data-driven Experimental Research)やヨーロッパのONELab(Open Network Laboratory)が存在する。これらは、研究者が遠隔からアクセス可能なテストベッドを提供し、実機を用いた実験を支援している。しかし、利用にはアカウント登録や審査が必要であり、自由度が制限される場合がある。

本研究では、Docker化された実験環境を提供することで、第三者が自身の環境で実験を再現できるようにする。提供するアーティファクトとして、以下を計画している:

\begin{itemize}
\item \textbf{Dockerコンテナイメージ}:Open5GS、srsRAN、s1n2-converterを含む完全なシステム
\item \textbf{設定ファイル}:各コンポーネントのYAML設定ファイル、ネットワーク構成
\item \textbf{実験スクリプト}:環境起動、UE接続、データ通信確認、ログ収集の自動化スクリプト
\item \textbf{パケットキャプチャ}:Wiresharkで解析可能なpcapファイル(S1-AP、NG-AP、GTP-U、NAS)
\item \textbf{実験手順書}:セットアップから実験実施、結果確認までの詳細な手順
\end{itemize}

これにより、研究成果の再現性を確保し、他の研究者が本研究を基盤として発展させることを可能にする。

\subsection{セキュリティ・計測基盤の一般化の難しさ}

モバイルネットワーク実験におけるセキュリティ上の課題は、以下の通りである:

\begin{itemize}
\item \textbf{認証鍵の取り扱い}:HSS/UDMに登録された加入者の認証鍵(K、OPc)は、SIMカード内に格納されており、外部からの読み取りは困難。実験用途では、ダミー鍵を使用するか、eUICC(embedded Universal Integrated Circuit Card)を用いたプログラマブルSIMを使用する必要がある
\item \textbf{プライバシ/PII(Personally Identifiable Information)}:IMSI、MSISDN、位置情報などの個人識別可能な情報は、研究成果公開時に匿名化またはダミー値に置き換える必要がある
\item \textbf{攻撃面の考慮}:実験環境が外部ネットワークに接続されている場合、セキュリティホール(例:認証バイパス、暗号化無効化)が攻撃経路となるリスクがある
\item \textbf{計測オーバーヘッド}:パケットキャプチャ、ログ記録、性能計測ツールの実行によるCPU/メモリオーバーヘッドが、実験結果に影響を与える可能性がある
\end{itemize}

本研究では、実験環境を外部ネットワークから隔離されたDockerネットワーク内に構築することで、セキュリティリスクを最小化する。また、認証鍵やIMSIはダミー値を使用し、研究成果公開時には個人情報を含まないように配慮する。計測オーバーヘッドについては、本格的な性能評価を行う際には計測ツールの影響を定量化し、補正を行う方針とする。

\section{法規制とsXGPの位置づけ}

\subsection{電波法の制約と実機検証の困難}

日本国内における無線通信実験は、電波法によって厳格に規制されている。主な制約として、以下が挙げられる:

\begin{itemize}
\item \textbf{屋内外の制約}:屋外での電波発射には原則として無線局免許が必要。屋内実験でも、電波漏洩を防ぐための電波暗室の使用が推奨される
\item \textbf{出力制限}:免許不要帯以外の周波数帯を使用する場合、特定実験試験局として申請が必要。出力電力は用途に応じて制限される
\item \textbf{使用帯域}:LTE Band 1(2.1GHz)、Band 3(1.8GHz)、5G n77(3.7GHz)、n79(4.5GHz)などの商用周波数帯は、免許取得または実験試験局申請が必須
\item \textbf{干渉リスク}:実験電波が商用ネットワークや他の無線システムに干渉を与えるリスクがあり、周波数調整や出力制限が求められる
\end{itemize}

学術機関での典型的な許認可フローは、以下の通りである:

\begin{enumerate}
\item 実験計画書の作成(使用周波数、出力電力、実験期間、設置場所の詳細)
\item 技術基準適合証明(技適)の確認、またはシールドルーム使用計画
\item 総務省総合通信局への特定実験試験局の申請
\item 審査期間(通常2週間〜1ヶ月)
\item 免許交付後、実験開始
\item 実験終了後、結果報告書の提出
\end{enumerate}

このプロセスは、迅速なプロトタイピングや反復実験を前提とする研究開発には不向きである。特に、学生や若手研究者が短期間で実験環境を構築し、試行錯誤を繰り返すことが困難となる。

\subsection{免許不要帯を用いた代替アプローチとしてのsXGPの意義}

sXGPは、免許不要帯(1.9GHz帯)を使用することで、上記の法的制約を大幅に緩和する。sXGPの法的位置づけは、電波法施行規則第6条第4項第3号の6に規定される「小電力データ通信システムの無線局」に該当し、技術基準適合証明(技適)を取得した機器であれば、免許取得なしで運用可能である。

sXGPを用いた実機検証の意義は、以下の通りである:

\begin{itemize}
\item \textbf{法令遵守}:免許申請不要で、合法的に実機相当のRAN実験が可能
\item \textbf{迅速な環境構築}:機器の調達から実験開始までの期間が大幅に短縮(数日〜数週間)
\item \textbf{反復実験の容易化}:設定変更や機能追加を繰り返し、試行錯誤が可能
\item \textbf{教育利用}:学生が自主的にモバイルネットワーク実験を行える環境を提供
\end{itemize}

一方で、sXGPには以下の限界も存在する:

\begin{itemize}
\item \textbf{周波数帯域幅}:10MHzに制限されており、高スループット実験(例:100Mbps以上)には不向き
\item \textbf{設備可用性}:sXGP対応基地局機器は限定的であり、商用装置の入手は困難。研究用途ではSDRを用いた自作が主流
\item \textbf{カバレッジ}:送信電力が制限されており(基地局1W、端末200mW)、広範囲のカバレッジ実験には不向き
\item \textbf{商用環境との差異}:商用LTE/5Gネットワークとは周波数帯・出力・設定が異なるため、結果の外的妥当性に留意が必要
\end{itemize}

本研究では、sXGPの利点を活かしつつ、限界を認識した上で実験設計を行う。特に、基本的なプロトコル変換機能と相互運用性の検証に焦点を当て、高スループットや広域カバレッジは今後の課題として位置づける。sXGPを用いることで、実装ベース標準化を支援する検証基盤を低コストかつ法令遵守の範囲で提供し、モバイルネットワーク研究の民主化に貢献することを目指す。

\section{シミュレータの対象範囲と限界}
\subsection{シミュレーションで十分な領域}
新NFのアルゴリズム検討、プロトコル状態機械の基本遷移の正当性確認、ラフなスケール試験、CIでの高速回帰などはシミュレータで効率的に実施できる。
\subsection{シミュレーションの限界}
実UE固有の実装差(RRC/NASのタイマ・再送・Capability差)、RRC/PHY/MACの時間ゆらぎ由来の境界条件、OS/NICのオフロードやキューイング、パケット順序入替・フラグメンテーション、S1AP$\leftrightarrow$NGAP変換やTEID管理の例外パスなどは、シミュレータでは再現が難しい。

\section{実機が必要となる検証項目}
\begin{itemize}
	\item 厳密なタイマ境界・再送・例外遷移(登録/PDU確立、NAS/RRCの異常系)
	\item セキュリティと鍵管理(AKA/EAP-AKA'、KDF、再同期、アルゴ協定の実装差)
	\item ユーザ面の実性能(GTP-Uオフロード、カーネル/DPDK、NUMA/CPUスケジューリング)
	\item 相互接続・回帰(複数UE/モデム/OSでの互換性検証)
	\item フェイルシナリオ(部分ロス、遅延、再順序化、NICドロップ、CPUスタベーション)
\end{itemize}

\section{本環境がもたらす価値(ユースケース)}
\label{sec:values}
\subsection{教育・トレーニング}
免許不要帯で運用可能なsXGPをeNBとして活用し、4G RANと5GCを接続できるため、学部・大学院・企業研修において、法令遵守のもとで実機に近い無線・コア連携の学習が可能となる。設定の再現手順を公開することで、実験レポートの客観性と再現性も高められる。

\subsection{プロトタイピングと検証の高速化}
コントロールプレーン(S1AP/NGAP/NAS)およびユーザプレーン(GTP-U)の変換・中継部を差し替え可能にすることで、新しいアルゴリズムやポリシー(例:識別子管理、TEID割当、タイムアウト制御)の試作・検証を低コストで繰り返し実施できる。

\subsection{相互接続・回帰テスト}
Open5GSなどの5GC実装や各種eNB/UEに対して、信令互換性や基本機能(登録、PDUセッション確立)の回帰テストを自動化できる。コンバータを介した差分吸収により、ベンダ差や仕様差を可視化しつつ評価できる。

\subsection{性能評価とボトルネック分析}
遅延・スループット・リソース使用率(CPU/メモリ)の計測基盤を同梱することで、コンバータの処理遅延やカプセル化オーバヘッド、TEID管理戦略の影響を定量化でき、設計改善に資する。

\subsection{再現性の高い研究公開と共同研究}
Dockerを用いた構成管理とビルド手順の統一により、第三者が同一環境で追試できる。研究成果の普及や共同研究の立ち上げコストを下げる。

\subsection{運用研究・障害シナリオの再現}
認証失敗やセキュリティモードエラー、ハンドオーバ相当の遷移など、現場で発生しうる事象の再現とトラブルシュート手順の確立に役立つ。

\section{本研究が対象とする課題の定義}
\subsection{研究者・学生が手軽に再現できる環境の要件}
必要構成要素・費用・所要時間・依存関係・再現手順の粒度を定義し、到達すべきユーザ体験(セットアップから成功確認まで)を示す。
\subsection{sXGP×5GC接続の技術的論点の範囲}
本研究で扱う範囲(制御/ユーザ面の変換、識別子/セキュリティ/状態管理)と扱わない範囲(RRC/PHY最適化、商用品質の最適化など)を明確化する。
\subsection{評価指標と成功基準の設定}
機能(登録/セッション確立率)、性能(遅延/スループット/CPU等)、堅牢性(異常時の復帰)、再現性(環境差耐性)の指標と合格基準を定める。


% (注)宇宙通信に関する章は本研究スコープ外のため削除。

% (注)該当節を削除。


% (注)該当節を削除。


% (注)該当節を削除。

% (注)DTN記述は別研究のため削除。

% \subsection{通信機会の非対称性}
% \label{subsection:通信機会の非対称性}

% \begin{figure}[tbh]
%     \centering
%     \includegraphics[width=0.7\textheight]{img/dtnprotocolstack.pdf}
%     \caption{DTNを搭載したノード間のみでの通信}
%     \label{fig:dtnprotocolstack}
%     \begin{minipage}{\textwidth}
%         \raggedright
%         \vspace{3mm}
%         \fontsize{10.5pt}{12pt}\selectfont
%         参考文献\cite{bundle_protocol_architecture}Figure1をもとに作成.
%         図中のConvergence Layer(CL)については
%         \ref{section:Convergence LayerとLTP}項で説明する.
%     \end{minipage}
% \end{figure}

% (注)該当節を削除。

% (注)該当節を削除。

% (注)該当節を削除。

% (注)該当節を削除。

% (注)該当節を削除。

% \begin{table}[tbh]
%     \centering
%     \caption{DTN実装とその機能の比較}
%     \begin{minipage}[t]{\textwidth}
%         \centering
%         \fontsize{10.5pt}{12pt}\selectfont
%         参考文献\cite{dtn_implementations}figure1より抜粋
%         \vspace{3mm}
%     \end{minipage}
%     \includegraphics[width=0.7\textheight]{img/chart_dtn_implementations.pdf}
%     \label{table:chart_dtn_implementations}
% \end{table}


% 第3章 関連研究と事例
\chapter{DTNの運用におけるルーティングと研究}
\label{chap:DTNにおけるルーティングの研究と課題}

\ref{subsection:ネットワークトポロジーの変動と間欠的接続}項で述べた通り、
宇宙のネットワークにおけるノードの多くは衛星であり、その位置は常に変動する。
そのためノード間の通信は特定の時間にのみ可能なものであり、
この間欠的なリンクを順次利用して転送しEnd-to-Endのデータグラムの転送を行う必要がある。
DTNでは、特定の2つのノード間の通信が可能なこの時間やタイミングをContactと呼び、
軌道計算などにより事前に計画されたContactを次々と利用して転送を行う
Contact Graph Routing(CGR)\cite{Fraire2021}というコンセプトが構想されている。
既に述べた通り既存のDTN実装は複数あるが、これらのDTN実装における
ルーティング手法でも主にCGRが用いられ、宇宙データ通信システムに関わる国際標準化検討委員会である
宇宙データシステム諮問委員会(CCSDS : Consultative Committee 
for Space Data System)ではSCHEDULE-AWARE BUNDLE ROUTING
\cite{schedule_aware_bundle_routing}として標準化されている。
DTNでCGRを用いたルーティングを運用する場合、その運用プロセスは以下の3つの段階に大別できる。
1つ目の段階は運用計画の決定、2つ目はルート決定、3つ目は実際のバンドルの転送である。
本章ではそれぞれの段階ごとにDTNにおけるCGRとその研究について分類し、現状の課題について説明する。
\section{運用計画の決定}
\label{section:運用計画の決定}
1つ目の段階では、ミッションコントロールなどを担う地上局など(以後、マスターノードとする)が、
各ノードの軌道計算やその他の情報に基づいてContact Planを作成する。
宇宙におけるノードの物理的な軌道は計算により予測可能であり、
2ノード間のContactも事前に計算することが可能である。 
CGRの例として、図\ref{fig:contact_example_topology}のようなAからDの4つのノードからなるトポロジーのDTNを考える。
マスターノードは軌道計算によるこれらのノードの位置や、搭載する機材の性能等をもとにContact Planを作成する。
Contact Planには、Contactについての記述とRangeについての記述が含まれ、
Contactについての記述では、特定の2ノードの通信機会についての通信開始・終了時間、データレートなどが記載され
(図\ref{fig:contact_example_contactplan})、Rangeについての記述では特定の2ノードの物理的な距離について記載される
(図\ref{fig:contact_example_contactrange})。
ただし\ref{subsection:通信機会の非対称性}項で述べた通り、宇宙における特定の2ノード間の通信機会は非対称であるため、
Contact Planに記載される通信機会は、特定の2ノードについての両方のリンクの通信可能機会ではなく、
片方向のリンクについての通信機会である。

\begin{figure}[tbh]
    \centering
    \includegraphics[width=0.5\textheight]{img/contact_example_topology.pdf}
    \caption{4つのノードからなるDTNの例}
    \label{fig:contact_example_topology}
    \begin{minipage}{\textwidth}
        \centering
        \vspace{3mm}
        参考文献\cite{schedule_aware_bundle_routing}figure3-1より引用
    \end{minipage}
\end{figure}
\begin{figure}[tbh]
    \centering
    \includegraphics[width=0.5\textheight]{img/contact_example_contactplan.pdf}
    \caption{図\ref{fig:contact_example_topology}のトポロジーにおけるContact Planの例(Contactに関する表記)}
    \label{fig:contact_example_contactplan}
    \begin{minipage}{\textwidth}
        \raggedright
        \vspace{3mm}
        参考文献\cite{schedule_aware_bundle_routing}figure3-2より引用。Senderは送信元のノードの識別子、Receiverは受信元のノードの識別子、FromはContactの開始時刻、Untilは終了時刻、Rateは転送速度を示す。
    \end{minipage}
\end{figure}
\begin{figure}[tbh]
    \centering
    \includegraphics[width=0.5\textheight]{img/contact_example_contactrange.pdf}
    \caption{図\ref{fig:contact_example_topology}のトポロジーにおけるContact Planの例(Rangeに関する表記)}
    \label{fig:contact_example_contactrange}
    \begin{minipage}{\textwidth}
        \centering
        \vspace{3mm}
        参考文献\cite{schedule_aware_bundle_routing}figure3-3より引用。
    \end{minipage}
\end{figure}
ノードAからノードDに向けたBundleを配送する場合、
これらのContact PlanからBundleの配送の状態遷移について図\ref{fig:contact_example_contactgraph}を得られる。
CGRではDTNの経路途中のノードはContact Planを保持しており、この一連の計算によって各Bundleの転送先を決定する。

\section{経路決定}
\label{subsection:経路決定}

\ref{section:運用計画の決定}節で決定されたContact Planは、
衛星どうしのContactを記載したものであり、実際のDTNの運用においては
Contact Planに基づいて転送すべき経路を決定する必要がある。
この経路の決定においては、図\ref{fig:contact_example_contactplan}
及び図\ref{fig:contact_example_contactrange}からなるContact Planに対し、
\ref{subsection:宇宙インターネットにおけるルーティングのアルゴリズム}項で述べるアルゴリズムを用いることにより
図\ref{fig:contact_example_contactgraph}のようなContact Graphを得ることができる。

\begin{figure}[tbh]
    \centering
    \includegraphics[width=0.5\textheight]{img/contact_example_contactgraph.pdf}
    \caption{図\ref{fig:contact_example_contactplan}及び
    図\ref{fig:contact_example_contactrange}のContact Planから計算されるContact Graphの例}
    \label{fig:contact_example_contactgraph}
    \begin{minipage}{\textwidth}
        \centering
        \vspace{3mm}
        参考文献\cite{schedule_aware_bundle_routing}figure3-4より引用。
    \end{minipage}
\end{figure}
\label{chap:related_works}


\subsection{CGRにおけるルーティングのアルゴリズム}
\label{subsection:宇宙インターネットにおけるルーティングのアルゴリズム}    
現状DTNでのルーティングは基本的にCGRが用いられており、
CGRでは経路計算の際には基本的にダイクストラが用いられている。
しかし他のアルゴリズムの使用や、パラメータを変更することによる改善方法が複数研究されており、
本セクションではその内容についてまとめる。
\subsection{Yen routing algorithm}
\label{subsection:Yen routing algorithm}
Fraireらはルーティングテーブルの管理手法について研究を行い、これに基づき現状のCGRは基本的にYen Routing Algorithmを用いている。\cite{FRAIRE2018}

\subsection{CGRに代替するアルゴリズム}
\label{subsection:CGRに代替するアルゴリズムの研究}
Efficient Contact Graph Routing Algorithms for Unicast and Multicast Bundles
\cite{DeJonckere2019}

\section{バンドルの転送}
\label{section:バンドルの転送}


\section{Contact Planの更新}
\label{sec:ContactPlanの更新}
このようにCGRのアルゴリズム・パラメータについての研究は多く行われているが、
これらはContact Planをもとに経路を計算手法の研究であり、
経路計算を行いたいノードにおいてそもそもContact Planが存在していることが前提となる。
そのため、Contact Planをノードに配送する手法が必要となるが、
Contact Planはその性質上、更新が必要であり、定期更新と臨時更新に分けて考えることができる。

\subsection{Contact Planの定期更新}
\label{sec:ContactPlanの定期更新}
Contact Planは有限時間内でのContactについて記述したものであり、
その時間以降のContact Planに関しては定期的に更新する必要がある。
ただし、Contact Planに記載されるノードの全てが周期的な運動のみを行う場合には
Contact Planを繰り返し使える可能性がある。

\subsection{Contact Plan の臨時更新}
\label{sec:ContactPlanの臨時更新}
ここでは故障時などのContact Planの臨時更新について述べる。
先行研究として\cite{Bezirgiannidis2013}を用いる。

\section{問題提起}
\label{sec:ContactPlanの臨時更新の課題}
\ref{sec:ContactPlanの臨時更新}で述べたように、想定されたContactに失敗が発生した場合、
その情報をDTNの他のノードに拡散しContact Planを更新することで、
DTNの各ノードはその時点におけるネットワークの最新のトポロジーと一致するContact Planを
保持することができ、これによりより適する経路がある場合これを選択することが可能になる。
しかし実際にリンク障害などが発生した場合、DTNの各ノードにその情報の拡散が完了する時間は、
拡散を開始するノードからCGRによってその情報を格納したバンドルが到達する時間によって決まる。
そのため天体間にまたがるDTNを運用しており、それらの全てノードに通知を行うことを想定した場合、
\ref{subsection:大きな遅延のある通信環境}で述べた天体間の遅延と、
さらにその天体内でのContact Planに応じた時間分、障害情報の拡散の完了までには大きな時間を要する。

そのためContact Planの臨時更新においては、
必要なノードにのみ効率よくその情報を拡散し遅延を抑えることを達成することが求められる。

% 第4章 提案手法:sXGP-5Gコンバータ
\chapter{提案手法:sXGP-5Gコンバータ}
% --- 章アウトライン・TODO・参考文献引用例 ---
% この章では、提案するsXGP-5Gコンバータの設計方針・要件・実装案・制約を述べる。
% TODO: docker_open5gs_sXGP-5G の設計思想や工夫点を記述。
% TODO: 参考文献を本文中で引用する(例: \cite{rfc9171})。
% 例: プロトコル変換の設計指針は \cite{rfc5326} などを参照。
% ------------------------------------------
\label{chap:proposal}

\section{要求仕様と設計方針}
\subsection{ユースケース}
提案環境は、実装ベース標準化を支援する以下のユースケースを想定する。

\begin{enumerate}
	\item \textbf{標準仕様の実装検証と相互運用性テスト}: 3GPP仕様に基づくOSS実装を実機RANで動作させ、仕様書上の記述と実装の乖離、異なる実装間の相互運用性問題を早期に発見する。

	\item \textbf{標準化へのタイムリーなフィードバック}: 検出した問題を再現手順・パケットキャプチャとともにドキュメント化し、標準化団体(3GPP等)や関連OSSプロジェクトへ報告・改善提案を行う。

	\item \textbf{継続的インテグレーション・回帰テスト}: 再現性の高い実験環境により、コード変更やプロトコル拡張が既存機能を破壊していないかを自動的に検証する。

	\item \textbf{性能評価とボトルネック分析}: 実機環境での遅延・スループット測定により、プロトコル設計や実装の性能ボトルネックを特定し、標準化・実装の両面から改善を図る。

	\item \textbf{プロトコル拡張の試作・検証}: 新機能や実験的プロトコルを実装し、実機環境で動作検証を行うことで、標準化前の技術検証を加速する。

	\item \textbf{教育・トレーニング用途}: 実機相当の環境で5G/6Gプロトコルスタックの動作を学習し、標準化・実装の実践的スキルを習得する。
\end{enumerate}

\subsection{非機能要件(再現性・法令遵守・安全性)}
\begin{itemize}
	\item \textbf{再現性}: Docker等のコンテナ技術により、環境構築手順を標準化し、異なる環境でも同一の実験を再現可能にする。
	\item \textbf{法令遵守}: 免許不要帯(sXGP)を用いることで、電波法に抵触せず実機検証を実施できる。
	\item \textbf{安全性}: 実験環境を隔離されたネットワークで構築し、商用ネットワークへの影響を排除する。
\end{itemize}

\subsection{対象範囲(制御/ユーザ面、認証、セッション管理)}

\section{アーキテクチャ設計}
\subsection{制御面:S1AP–NGAP変換の設計案}
\subsection{ユーザ面:GTP-U転送・TEID管理}
\subsection{認証・登録(NASメッセージ処理の方針)}
\subsection{コンテキスト管理とタイムアウト}

\section{sXGP採用の根拠と想定限界}
\subsection{採用の根拠:実装ベース標準化への貢献}

sXGPを採用する最大の理由は、\textbf{免許不要帯での法令遵守により「実装を実際に動かせる」}ことにある。これは、実装ベース標準化の実現において極めて重要である。

\begin{itemize}
	\item \textbf{実機による全スタック検証}: 実UE・実NIC・実時間のプロトコルスタックを動作させることで、シミュレータでは検出できない実装レベルの問題(タイミング、リソース競合、ハードウェア依存の挙動など)を発見できる。これは標準仕様と実装の乖離を早期に検出するために不可欠である。

	\item \textbf{相互運用性問題の早期発見}: 異なるベンダの実装(UE、eNB、コア)を組み合わせた際の相互運用性問題を、実機環境で検証できる。3GPP標準の曖昧な記述や解釈の違いによる非互換性を、標準化プロセスの早い段階でフィードバック可能にする。

	\item \textbf{標準化へのタイムリーなフィードバック}: 実機で問題を再現できることで、再現手順・パケットキャプチャ・ログを含む具体的な報告を標準化団体やOSSプロジェクトに提供できる。これにより、仕様策定 → 実装 → 問題発覚 → 修正のサイクルを数年単位から数ヶ月単位に短縮できる。

	\item \textbf{継続的検証の実現}: 免許不要帯であるため、継続的インテグレーション(CI)環境に組み込んで自動テストを実施できる。コード変更のたびに実機検証を行うことで、回帰テストと品質保証を強化できる。

	\item \textbf{(副次的効果)低コスト・再現性}: 専用周波数ライセンスが不要であり、比較的低コストで実験環境を構築できる。また、Dockerなどの技術と組み合わせることで、再現性の高い検証環境を複数の研究者・組織で共有できる。
\end{itemize}

\subsection{想定限界・外延}
本環境はLTE互換のsXGPをRANとして用いるため、5G NR特有のPHY機能やスケジューリング最適化は対象外となる。しかし、制御面(S1AP/NGAP/NAS)およびユーザ面(GTP-U)のプロトコルレベルでの相互運用性検証は十分に可能であり、標準化フィードバックの主要な価値はこのレイヤーにある。NR固有の無線レイヤーの課題は、今後の拡張として分離して扱う。

\section{実装方針と部品選定}
\subsection{ソフトウェア構成(言語・ライブラリ・依存)}
\subsection{ネットワーク構成(VLAN/VRF/NATの要否)}
\subsection{監視・計測(メトリクス、トレース)}

\section{制約と想定される限界}
\subsection{規格差分による制限}
\subsection{性能上のボトルネックと回避策}


% 第5章 実験環境と方法
\chapter{提案手法の実装とシミュレーションの環境}
\label{chap:implementation_and_experimentation}

\section{Omnet++とDTNsimを用いた評価環境}
\label{section:Omnet++とDTNsimを用いた評価環境}
Omnet++はネットワークシミュレータの構築を目的としたオープンソースのプラットフォームであり, 
既存の通信プロトコルや物理層のシミュレーションを実装しているほか, 
各ノードにC++でのアプリケーションを実装し拡張することが可能である. 
DTNsimはOmnet++のフレームワーク上で動作するDTNのシミュレーターであり, 
ION-DTN, HDTNなどの種々のDTN実装が動作するネットワークをシミュレートすることができ, 
各種CGRのバリエーションにも対応する. そのため本研究ではこのOmnet++とDTNsimを用いて
宇宙で運用されているDTNをシミュレーションを行う. 

\section{2030・2040年代のDTNを想定したシミュレーション}
\label{section:2030・2040年代のDTNを想定したシミュレーション}
本実験においては, 2030年代に地球・月間, 2040年代に地球・月・火星間にまたがるDTNを運用していることを想定し, 
そのうち任意の2天体間について既存手法と本研究の提案手法の実装とシミュレーションを行い, 
本研究の提案の有効性について検証する. そのため任意の2天体間のDTNネットワークとして, 
図\ref{fig:experimentation_topology}と同様のトポロジーを想定する. 

\ref{section:2030年代の地球・月間のDTNを想定したシミュレーションのシナリオとパラメータ}項では
2030年代の地球・月間のDTNにおけるシナリオ, 
\ref{section:2040年代の地球・月・火星間のDTNを想定したシミュレーションのシナリオとパラメータ}項では
2040年代の地球・月・火星間のDTNにおけるシナリオを説明する. 

\subsection{2030年代の地球・月間のDTNを想定したシミュレーションのシナリオとパラメータ}
\label{section:2030年代の地球・月間のDTNを想定したシミュレーションのシナリオとパラメータ}
2030年代の人類の月における活動区域は当面月の南極域になることが予想され, 
図\ref{fig:artemis_moon_station_orbit}に示すように, 
月南極域との通信可能時間の長いNRHO軌道の利用が検討されている. 
本実験ではこのNRHO軌道を利用した地球・月間のDTNネットワークを想定し, 
図\ref{fig:experimentation_topology}のトポロジーにおいて, 
それぞれ以下の構成を用いる. 

\begin{table}[htbp]
    \centering
    \caption{地球・月間シナリオのトポロジーと図\ref{fig:experimentation_topology}での表示の対応関係}
    \begin{tabular}{cc}  \hline
        図\ref{fig:experimentation_topology}での表示 & 本シナリオにおける対応 \\ \hline
        天体A & 地球 \\
        天体B & 月 \\
        ノード1 & 地球の地上DTNノード \\
        ノード2 & 地球の静止軌道上に存在するDTNノード \\
        ノード3 & 地球の静止軌道上に存在するDTNノード \\
        ノード4 & 月のNRHO軌道上に存在するDTNノード \\
        ノード5 & 月のNRHO軌道上に存在するDTNノード \\
        ノード6 & 月の地表DTNノード \\ \hline
    \end{tabular}
    \label{table:earth_moon_scenario_topology}
\end{table}

地球の静止軌道上に存在するDTNノードは地球に対してほぼ円軌道を描き, 
地表からおよそ36,000km(およそ0.12光秒)の高度を飛行する. 
また月のNRHO軌道上に存在するDTNノードは,  
月面から4,000km(およそ0.01光秒)$\sim$75,000km(およそ0.25光秒)程度を極端な楕円軌道を描いて飛行する. 
そのため地球の静止軌道上に存在するDTNノードと月のNRHO軌道上に存在するDTNノードの距離について
図\ref{fig:distance_earth_moon}のように考えられる. 
これらを考慮し, シミュレーションにおけるノード間の距離は表\ref{table:earth_moon_scenario_distance}のように設定した. 
\begin{figure}[tbh]
    \centering
    \includegraphics[width=0.7\textheight]{img/simulation_params_earth_moon.pdf}
    \caption{地球・月間シナリオにおけるノード間の距離の概要}
    \label{fig:distance_earth_moon}
\end{figure}

\begin{table}[htbp]
    \centering
    \caption{地球・月間シナリオのシミュレーションに用いる各ノード間の距離}
    \begin{tabular}{cc}  \hline
        図\ref{fig:experimentation_topology}でのノードペア & Contact Planに表記する距離 \\ \hline
        ノード1 - ノード2またはノード3 間 & 0.12光秒 \\
        ノード2 - ノード3 間 & 0.37光秒 \\
        ノード2またはノード3 - ノード4またはノード5  間 & 1.29光秒 \pm15\% \\
        ノード4 - ノード5  間 & 0.14光秒 \pm 50\% \\
        ノード4またはノード5 - ノード6  間 & 0.13光秒 \pm 92\% \\ \hline
    \end{tabular}
    \label{table:earth_moon_scenario_distance}
\end{table}

\subsection{2040年代の地球・月・火星間のDTNを想定したシミュレーションのシナリオとパラメータ}
\label{section:2040年代の地球・月・火星間のDTNを想定したシミュレーションのシナリオとパラメータ}
2040年代には活動区域は火星にまで広がっていることが予想される. 
火星における運用計画等は現時点では未定ではあるものの, 
PhobosとDeimosを利用した軌道上の基地の設置なども構想されている.
そのため本実験では表\ref{table:earth_mars_scenario_topology}のように
地球・火星間のDTNネットワークを想定する. 

\begin{table}[htbp]
    \centering
    \caption{地球・火星間シナリオのトポロジーと図\ref{fig:experimentation_topology}での表示の対応関係}
    \begin{tabular}{cc}  \hline
        図\ref{fig:experimentation_topology}での表示 & 本シナリオにおける対応 \\ \hline
        天体A & 地球 \\
        天体B & 月 \\
        ノード1 & 地球の地上DTNノード \\
        ノード2 & 地球の静止軌道上に存在するDTNノード \\
        ノード3 & 地球の静止軌道上に存在するDTNノード \\
        ノード4 & Phobosの近傍に存在するDTNノード \\
        ノード5 & Deimosの近傍に存在するDTNノード \\
        ノード6 & 火星の地表DTNノード \\ \hline
    \end{tabular}
    \label{table:earth_mars_scenario_topology}
\end{table}

この時, 地球・火星・Phobos・Deimos間の距離は図\ref{fig:distance_earth_mars}のような値を用いることでおおよその値が推定ができる. 
また地球と火星間の距離の変動周期が非常に長いため, 200光秒・750光秒・1300光秒の3つの値を用いてそれぞれ別にシミュレーションを行った.
地球と火星間の距離が200光秒の時のシミュレーションにおけるノード間の距離を表\ref{table:earth_mars_scenario_distance_200}のように, 
750光秒の時のシミュレーションにおけるノード間の距離を表\ref{table:earth_mars_scenario_distance_750}のように, 
1300光秒の時のシミュレーションにおけるノード間の距離を表\ref{table:earth_mars_scenario_distance_1300}のように設定した.


\begin{figure}[tbh]
    \centering
    \includegraphics[width=0.7\textheight]{img/simulation_params_earth_mars.pdf}
    \caption{地球・月・火星間シナリオにおけるノード間の距離の概要}
    \label{fig:distance_earth_mars}
    \begin{minipage}{\textwidth}
        \raggedright
        \fontsize{10.5pt}\selectfont
    \end{minipage}
\end{figure}

\begin{table}[htbp]
    \centering
        \caption{地球・月・火星間シナリオのシミュレーションに用いる各ノード間の距離 \\(地球・火星間の距離が200光秒のシナリオ)}
    \begin{tabular}{cc}  \hline
        図\ref{fig:experimentation_topology}でのノードペア & Contact Planに表記する距離 \\ \hline
        ノード1 - ノード2またはノード3 間 & 0.12光秒 \\
        ノード2 - ノード3 間 & 0.37光秒 \\
        ノード2またはノード3 - ノード4  間 & 200光秒 \pm0.050\% \\
        ノード2またはノード3 - ノード5  間 & 200光秒 \pm0.026\% \\
        ノード4 - ノード5  間 & 0.078光秒 \pm40\% \\
        ノード4 - ノード6  間 & 0.031光秒 \\ 
        ノード5 - ノード6  間 & 0.078光秒 \\ \hline
    \end{tabular}
    \label{table:earth_mars_scenario_distance_200}
\end{table}
\begin{table}[htbp]
    \centering
        \caption{地球・月・火星間シナリオのシミュレーションに用いる各ノード間の距離 \\(地球・火星間の距離が750光秒のシナリオ)}
    \begin{tabular}{cc}  \hline
        図\ref{fig:experimentation_topology}でのノードペア & Contact Planに表記する距離 \\ \hline
        ノード1 - ノード2またはノード3 間 & 0.12光秒 \\
        ノード2 - ノード3 間 & 0.37光秒 \\
        ノード2またはノード3 - ノード4  間 & 750光秒 \pm0.013\% \\
        ノード2またはノード3 - ノード5  間 & 750光秒 \pm0.007\% \\
        ノード4 - ノード5  間 & 0.078光秒 \pm40\% \\
        ノード4 - ノード6  間 & 0.031光秒 \\ 
        ノード5 - ノード6  間 & 0.078光秒 \\ \hline
    \end{tabular}
    \label{table:earth_mars_scenario_distance_750}
\end{table}
\begin{table}[htbp]
    \centering
        \caption{地球・月・火星間シナリオのシミュレーションに用いる各ノード間の距離 \\(地球・火星間の距離が1300光秒のシナリオ)}
    \begin{tabular}{cc}  \hline
        図\ref{fig:experimentation_topology}でのノードペア & Contact Planに表記する距離 \\ \hline
        ノード1 - ノード2またはノード3 間 & 0.12光秒 \\
        ノード2 - ノード3 間 & 0.37光秒 \\
        ノード2またはノード3 - ノード4  間 & 1300光秒 \pm0.007\% \\
        ノード2またはノード3 - ノード5  間 & 1300光秒 \pm0.004\% \\
        ノード4 - ノード5  間 & 0.078光秒 \pm40\% \\
        ノード4 - ノード6  間 & 0.031光秒 \\ 
        ノード5 - ノード6  間 & 0.078光秒 \\ \hline
    \end{tabular}
    \label{table:earth_mars_scenario_distance_1300}
\end{table}
\subsection{シミュレーションで用いるBundleトラフィック}
\label{section:シミュレーションで用いるBundleトラフィック}
本実験では図\ref{fig:experimentation_topology}のトポロジーにおいて, 
ノード1からノード6に向かう, すなわち天体Aの地上DTNネットワークから天体Bの地上DTNネットワークに
向けてのトラフィックを想定する. DTNsimではトラフィックとして生成するBundleについて様々な
パラメータを設定することが可能である. 本論文におけるシミュレーションではそれぞれ
表\ref{table:traffic_earth_moon}と表\ref{table:traffic_earth_mars}の値を用いた. 
また実際の宇宙インターネットにおけるトラフィックの発生状況は当然時刻変動するものであるが, 
本論文におけるシミュレーションでは単位時間あたりのトラフィック量は概ね一定であると仮定した.

\begin{table}[htbp]
    \centering
    \caption{地球・月間シナリオのトポロジーと図\ref{fig:experimentation_topology}での表示の対応関係}
    \begin{tabular}{ccc}  \hline
        図\ref{fig:experimentation_topology}での表示 & パラメータ名 & 値 \\ \hline
        シミュレーション全体におけるBundleの総生成数 & bundlesNumber & 1800 \\
        生成したBundleの送信元ノードのendpoint ID & sourceEid & 1 \\
        生成したBundleの宛先ノードのendpoint ID & destinationEid & 6 \\
        Bundle生成イベントごとのBundleのサイズ(bytes)& size & 1500\\
        Bundle生成イベントの間隔(秒)& interval & 1 \\ \hline
    \end{tabular}
    \label{table:traffic_earth_moon}
\end{table}

\begin{table}[htbp]
    \centering
    \caption{地球・火星間シナリオのトポロジーと図\ref{fig:experimentation_topology}での表示の対応関係}
    \begin{tabular}{ccc}  \hline
        図\ref{fig:experimentation_topology}での表示 & パラメータ名 & 値 \\ \hline
        シミュレーション全体におけるBundleの総生成数 & bundlesNumber & 8000 \\
        生成したBundleの送信元ノードのendpoint ID & sourceEid & 1 \\
        生成したBundleの宛先ノードのendpoint ID & destinationEid & 6 \\
        Bundle生成イベントごとのBundleのサイズ(bytes)& size & 1500\\
        Bundle生成イベントの間隔(秒)& interval & 10 \\ \hline
    \end{tabular}
    \label{table:traffic_earth_mars}
\end{table}


% 第6章 評価
\chapter{評価}
\label{chap:evaluation}

\section{結果}
\subsection{機能検証の結果(登録・PDUセッション)}
\subsection{性能測定の結果(遅延・スループット)}
\subsection{リソース使用率・スケーラビリティ}

\section{考察}
\subsection{提案手法の有効性と限界}
\subsection{関連研究との比較と位置づけ}
\subsection{実運用への適用可能性}


% 第7章 結論と展望
\chapter{結論と展望}
\label{chap:conclusion}
\section{本研究のまとめ}
ここでは本研究全体を総括する. 先行研究に対する本研究の立ち位置を再度明確にするとともに, 
結果から明らかになった優位性について再度まとめる. 
\section{今後の課題と展望}
今後の課題と展望について述べる. 主に, 臨時更新のみならず定期更新の際から
天体間をASとして切り分ける必要性について述べる. 

% 謝辞
\chapter*{謝辞}\markboth{謝辞}{謝辞}
\addcontentsline{toc}{chapter}{謝辞}
\label{thanks}


\renewcommand{\thechapter}{\Alph{chapter}}
\setcounter{chapter}{0}

\vspace{-5mm}
\bibliographystyle{unsrt}
\bibliography{bib/thesis}

\chapter*{付録}\markboth{付録}{付録}
\addcontentsline{toc}{chapter}{付録}

\label{appendix}
\input{src/11_kconfig.tex}



%\input{bib/biblio}\thispagestyle{plain}%bibtex

% \chapter*{付録}\markboth{付録}{付録}
\addcontentsline{toc}{chapter}{付録}

\label{appendix}
\input{src/11_kconfig.tex}

\end{document}

%%% Local Variables:
%%% mode: japanese-latex
%%% TeX-master: t
%%% End:
