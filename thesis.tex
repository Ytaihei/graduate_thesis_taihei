\documentclass[12pt]{ujreport}
\usepackage{comment}
\usepackage{./sty/eclepsf}
\usepackage{tascmac}
\usepackage{tabularx}
\usepackage{listliketab}
\usepackage[longnamesfirst]{natbib}
\usepackage[dvipdfmx]{graphics}
\usepackage[dvipdfmx]{graphicx}
\usepackage[dvipdfmx]{color}
\usepackage{subfigure}
\usepackage{alltt}
\usepackage{here}
\usepackage{afterpage}
\usepackage{./sty/ncodeline}
\usepackage{url}
\usepackage{amsmath}
\usepackage{setspace}
%\usepackage{glossaries}  % Overleafでのコンパイルエラーを避けるためコメントアウト

%\usepackage[dvipdfmx, colorlinks, breaklinks,%
\usepackage[dvipdfmx, breaklinks,%
bookmarks=true, bookmarksnumbered=true,%
bookmarkstype=toc, bookmarksopen=true,bookmarksopenlevel=3,%
pdftitle={RG},%
]{hyperref}
\usepackage{bookmark}

\AtBeginDvi{\special{pdf:tounicode EUC-UCS2}}

\usepackage{fancyhdr}

\usepackage{./sty/doxygenorig}

\usepackage{indentfirst}
\usepackage{listings,./sty/jlisting}
\usepackage{algorithm}
\usepackage{algpseudocode}
\usepackage{multicol}
\usepackage{caption}
\captionsetup[table]{justification=centering}


\def\lstlistingname{付録}

\lstset{%
 language={C++},
 %backgroundcolor={\color[gray]{.85}},%
 basicstyle={\small\ttfamily},%
 identifierstyle={\small},%
 commentstyle={\small\itshape},%
 keywordstyle={\small\bfseries},%
 ndkeywordstyle={\small\ttfamily},%
 stringstyle={\small\ttfamily},
 frame={tb},
 framesep=1zw,
 breaklines=true,
 numbers=left,%
 xrightmargin=0zw,%
 xleftmargin=1.5zw,%
 numberstyle={\scriptsize},%
 stepnumber=1,
 numbersep=1zw,%
 lineskip=-0.5ex%
}

\usepackage{amssymb}
%\usepackage{supertabular,multirow}

\usepackage{array}
\newcolumntype{M}[1]{>{\centering\arraybackslash}m{#1}}

% A4  size: 297mm*210mm %1pt = 0.35mm
\setlength{\topmargin}{-3.4mm} % 10pt 25.4mm - 3.4mm = 22mm
\setlength{\oddsidemargin}{-0.4mm} % 25.4mm - 0.4mm = 25mm
\setlength{\evensidemargin}{-0.4mm} % 25.4mm - 0.4mm = 25mm
\setlength{\textheight}{231mm} % 660pt % original is 225.75mm 645pt
\setlength{\textwidth}{160mm} % 457pt

\renewcommand{\topfraction}{.99}
\renewcommand{\textfraction}{.0}
\renewcommand{\floatpagefraction}{.99}
\renewcommand{\bibname}{参考文献}
\renewcommand{\baselinestretch}{1.2}

\usepackage{tikz}
\newcommand*\circled[1]{\tikz[baseline=(char.base)]{
            \node[shape=circle,draw,inner sep=1pt] (char) {#1};}}
\pagestyle{fancy}
\lhead[]{}

\makeatletter
\def\chaptermark#1{\markboth {\ifnum \c@secnumdepth>\m@ne
\@chapapp\ \thechapter \@chappos\ \fi #1}{}}
\makeatother

% タイトル
\def\title{sXGP-5G:次世代モバイルコア研究への参入障壁低減のための実験環境の提案}
% 英語タイトル
\def\etitle{sXGP-5G:Proposal of flexible 5G network construction method for next-generation mobile network development}
% 著者(日本語)
\def\author{山口 泰平}
% 著者(英語)
\def\eauthor{Taihei Yamaguchi}
% 学部・研究科
\def\dept{慶應義塾大学 総合政策学部}
% 学部・研究科(英語)
\def\edept{Keio University Bachelor of Arts in Policy Management}


\usepackage{hyperref}
\begin{document}

\pagenumbering{roman}
\begin{titlepage}
  \begin{center}
    \begin{large}
      卒業論文   2023年度(令和5年度)\\
      \vspace{24pt}
      {\Huge \title}
    \end{large}
  \end{center}
  \vspace{40em}
  \begin{flushright}
    \large \dept\\
    \author
  \end{flushright}
\end{titlepage}

\thispagestyle{empty}


卒業論文要旨 - 2025年度(令和7年度)
\begin{center}
\begin{large}
\begin{tabular}{|M{0.97\linewidth}|}
    \hline
      \title \\
    \hline
\end{tabular}
\end{large}
\end{center}
\begin{spacing}{1.2}
\small
モバイル通信の標準化において、3GPPは仕様書ベースのプロセスを採用しているため、実装・検証が後回しになり、「仕様上は規定されているが実際には動作しない機能」が多数生じている。次世代(6G)に向けては、実装ベースの標準化(implementation-driven standardization)による早期の相互運用性検証と標準化へのフィードバックサイクルの確立が不可欠である。しかし、実機RANを用いた検証は電波法上の制約から困難であり、標準化プロセスの検証基盤が不足している。

本研究は、免許不要帯で運用可能なsXGP(TD-LTE互換)をeNBとして活用し、4GのRAN(UE・eNB)と5G Core(5GC)を接続するコンバータを実装することで、実装ベース標準化を支援する実機検証環境を提案する。本研究では、(1) sXGPを用いた法令遵守型RANの構築、(2) S1AP/NGAP/NASの信令処理とGTP-U中継を行うコンバータの設計・実装、(3) 相互運用性検証とフィードバックのための計測・再現手順の整備、を行った。

評価として、登録・PDUセッション確立などの基本機能を確認し、実機特有の相互運用性問題の検出能力を検証した。提案環境は、(a) 標準仕様の実装検証と相互運用性テスト、(b) 標準化へのタイムリーなフィードバック、(c) 継続的インテグレーション・回帰テスト、(d) 性能評価とボトルネック分析、(e) 教育・トレーニング用途、(f) プロトコル拡張の試作・検証、といったユースケースに適用可能である。本環境により、実装を動かしながら標準化を進めるサイクルを確立し、6G時代のアジャイルな標準化プロセスに貢献する。

\end{spacing}

キーワード:\\
\underline{1. モバイルシステム}
\underline{2. sXGP}
\underline{3. 5G Core}
\underline{4. Open5GS}
\underline{5. GTP-U}
\begin{flushright}
\dept \\
\author
\end{flushright}

\thispagestyle{plain}
\clearpage

Abstract of Bachelor's Thesis - Academic Year 2024
\begin{center}
\begin{large}
\begin{tabular}{|p{0.97\linewidth}|}
    \hline
      \etitle \\
    \hline
\end{tabular}
\end{large}
\end{center}

~ \\

~ \\
Keywords : \\
\underline{1. Delay/Disruption Tolerant Network} 
\underline{2. Contact Graph Routing} 
\begin{flushright}
\edept \\
\eauthor
\end{flushright}
\thispagestyle{plain}
\clearpage

\tableofcontents\thispagestyle{plain} %目次
\clearpage
\listoffigures\thispagestyle{plain} %図目次
\clearpage
\listoftables\thispagestyle{plain} %表目次
\clearpage

\clearpage

\pagenumbering{arabic}

% 第1章 序論
\chapter{序論}
% --- 章アウトライン・TODO・参考文献引用例 ---
% この章では、研究の背景・課題・目的・本論文の貢献・構成を述べる。
% TODO: 具体的な背景事例や課題を Open5GS/sXGP-5G 実装と関連付けて記述。
% TODO: 参考文献を本文中で引用する(例: \cite{rfc5326})。
% 例: 5Gの研究環境構築に関する課題は \cite{McBrayer2022} などで議論されている。
% ------------------------------------------

\section{背景}
モバイル通信の標準化において、3GPPは仕様書を先に策定し、実装・相互運用性検証は後工程となる「仕様先行型」のプロセスを採用している。これは大規模な通信インフラの標準化として必要な側面もあるが、実装が追いつかず、「仕様上は存在するが実際には動作しない機能」や「異なる実装間での相互運用性問題」が多数生じている。一方、IETFなどインターネット技術の標準化では"rough consensus and running code"の原則に基づき、実装を動かしながら標準化を進める「実装ベース標準化」が実践されており、標準と実装の乖離が少ない。

6G時代に向けては、標準化サイクルの高速化と実装・検証の早期化が不可欠である。しかし、モバイル通信では実機RANを用いた検証が電波法上の制約から困難であり、実装ベース標準化を支援する検証基盤が不足している。本研究は、免許不要帯で運用可能なsXGP(TD-LTE互換)をeNBとして活用し、4G RAN(UE・eNB)と5G Core(5GC)を接続するコンバータを提案することで、法令遵守の範囲で実機検証を可能にする。5GCの基本的なアーキテクチャと手順は3GPP TS~\cite{threegpp-23501,threegpp-23502}に規定されており、OSS実装としてはOpen5GS~\cite{open5gs}が広く用いられている。

\section{問題意識と課題}
本研究の背景には、以下の三点に関する強い問題意識がある。

\begin{itemize}
	\item \textbf{実装ベース標準化の不在}: 3GPPの標準化プロセスは仕様書ベースであり、実装・相互運用性検証が後回しになる。結果として「仕様上は存在するが実際には動作しない機能」「実装間での非互換性」が多数生じ、標準化へのフィードバックループが極めて遅い。IETFの"running code"原則とは対照的に、実装を動かしながら標準化を進める文化が欠如している。6G時代のアジャイルな開発・標準化サイクルを確立するには、実装ベースの検証基盤が不可欠である。

	\item \textbf{実装検証環境の不足}: モバイルコアはソフトウェア化が進み、Open Source Software(OSS)も充実してきた。しかし、実機RANを用いた検証環境は電波法上の制約から構築困難であり、OSSの相互運用性や実機特有の問題を検証する手段が限られている。シミュレータでは発見できない実装レベルの問題(タイミング、リソース競合、NIC/ドライバ依存の挙動など)を早期に検出できないことが、標準化と実装のギャップを拡大させている。

	\item \textbf{標準化フィードバックサイクルの遅延}: 現状では、仕様策定 → 実装 → 商用展開 → 問題発覚 → 次期仕様での修正、というサイクルに数年を要する。この遅延により、不具合のある仕様が長期間放置され、互換性問題が蓄積する。実機検証環境を用いた早期の相互運用性テストと標準化団体へのタイムリーなフィードバックが可能になれば、このサイクルを大幅に短縮できる。免許不要帯で運用可能なsXGPは、この課題に対する実践的な解決策となる。
\end{itemize}

これらに加えて、相互接続性、運用・再現性、計測基盤の不足といったモバイルシステム全体の横断的課題を踏まえ、とりわけRANとコア間のインターワーキングが標準化・実装検証のボトルネックである点を指摘する。

\section{研究目的}
\begin{itemize}
	\item 実装ベース標準化を支援する実機検証環境の構築方法を示す。
	\item sXGPベースの4G RANと5GCを接続するコンバータの設計・実装を通じて、標準仕様の相互運用性検証を実現する。
	\item 実機特有の問題を早期検出し、標準化へのフィードバックサイクルを短縮する手法を提案する。
	\item 再現可能な検証環境により、継続的インテグレーション・回帰テストを可能にする。
\end{itemize}

\section{本論文の貢献}
\begin{itemize}
	\item 実装ベース標準化を可能にする実機検証環境(sXGP + 5GC)の実証。標準仕様と実装の乖離を早期発見し、標準化団体へタイムリーにフィードバックする基盤を提供。
	\item 4G RANと5GCの信令・ユーザ面の相互接続に関する設計指針と、実装レベルでの相互運用性問題の分類・解決手法の整理。
	\item 再現性の高い検証プロファイル(トポロジ、計測項目、実験手順)の提示により、継続的な相互運用性テストと回帰テストを支援。
	\item 免許不要帯を活用した法令遵守型の実機検証手法の確立。電波法制約下でも実装を動かしながら標準化を進めるアプローチの実現可能性を示す。
\end{itemize}

\section{論文構成}
本論文は以下の構成である。第\ref{chap:background}章で基礎知識と課題整理、第\ref{chap:related}章で関連研究と事例、第\ref{chap:proposal}章で提案手法、第\ref{chap:experiment}章で実験環境と方法、第\ref{chap:evaluation}章で評価、第\ref{chap:conclusion}章で結論と今後の課題を述べる。


% 第2章 基礎知識と課題整理
\chapter{hoge}
\section{宇宙通信におけるインターネット技術の適用性とDTN}
近年、月や火星の宇宙探査ミッションが本格化し、NASA中心のアルテミス計画citenasa2020は2025年から
有人ミッションも予定している。 これらの計画に伴い、 今後は月・火星にある衛星やさまざまな通信機器、 
デバイスなどの数が増加する可能性が高い。 従来までの宇宙ミッションにおいて宇宙のノードと地球との通信は、 
地球上にある各国の大型アンテナを利用し、 一対一の通信を行っていた。 しかしこのような計画でノードの数が増加する場合、
通信ニーズに対応するためには宇宙にも多対多のノードで通信が可能なインターネットが必要となる。 
既存のインターネットはEnd-to-Endの疎通性が確保できている環境で通信を行うことが多いが、 
宇宙で通信を行う際には頻繁な断絶と大きな遅延が問題となる。 中継ノードとなる様々な宇宙機は
宇宙空間での位置が常に変化しており、天体の影に入るなどで断絶が頻繁に起こる。また通信の際には地球月間でも片道1。3秒、 
地球火星間では太陽に対する2天体の公転の状況によって、 片道4分から20分程度の遅延が想定されている。 
End-to-EndでTCPを用いた通信を行う際には、 3-way-handshakeなどを含め
これらの天体間を複数回往復する通信を行う必要があり、遅延はさらに大きな時間になる。 

そのため宇宙のインターネットにはDelay and Disruption Tolerant Networking(DTN)の
技術を利用することが考えられている。 DTNの技術の一つにBundle Protocol(BP)があり、 
BPでは通信されるデータはバンドルという可変長のデータとして転送される。 
中間ノードでは経路上の次のノードへ転送可能なタイミングまでバンドルを蓄積することが可能になっているため、 
End-to-Endの通信疎通性が確保できていない場合でも、 この蓄積による転送を行うことにより断絶に強い通信ができる。
 またトランスポートレイヤにUDPなどのプロトコルを用いることで、 
 比較的遅延を抑えて通信することもできる。 

\subsection{Bundle Protocol}
Bundle Protocol (BP)は、DTNにおける主要な通信技術で遅延・断絶が起きやすい環境でデータを確実に伝送するために設計された。

Bundle Protocolは、 データを「バンドル」と呼ばれる可変長の単位として送信する。このバンドルは、 
送信元から目的地までの途中で複数の中継を経ても、全体としてデータを確実に届けるためのものである。 
また、このプロトコルは「ストア&フォワード」方式を利用しており、 各中継ノードが受け取ったバンドルを一時的に保存し、 
次のノードと通信できるタイミングが来るまで待機する。 これにより、通信が一時的に途絶えてもデータが失われることなく、 次のノードへと送信される。

\subsection{Convergence LayerとLTP}
DTNでは多様なプロトコルがトランスポートレイヤ以下の層で使用することを想定しており、 図 ref中のConvergence Layerは
それらの違いを吸収することを目的としている。 Convergence Layer Protocol(CLP)としては、 
利用する下位レイヤプロトコルにより、
- sTCP-based CLP (TCPCL)
- User Datagram Protocol (UDP)
- based CLP (UDPCL)、 Saratoga CLP
- Licklider Transmission Protocol (LTP)
- based CLP (LTPCL)
- Licklider Transmission Protocol(LTP)などがある。 
LTP citerfc5326はコンバージェンスレイヤのプロトコルの一つであり、 再送制御の機能も実装している。
 LTPをコンバージェンスレイヤに用いる場合、 トランスポートレイヤにUDPを用いることがあるほか、 
 宇宙での通信においてLTPが直接リンク層にアクセスすることも想定されている。
\subsection{既存のDTN実装}
既にいくつかの研究機関などによりDTN技術を実装したソフトウェアがリリースされている。 いくつかの例を以下に示す。 

% Interplanetary Overlay Network DTN(ION-DTN):NASA/JPL
% HDTN:NASA/Glenn research center
% DTN ME:Marshall Space Flight center
% μD3TN:D3TN GmbH
% IONe:Experimental ION Scott Burleigh United States 
% DTN7/Go:University of Marburg German

これらのDTNソフトウェアは、 基本的に通信内容からバンドルへのエンコード・デコード、 
中間ノードでのバンドルのままでの蓄積転送を可能にしているが、 
Convergence Layerが対応しているトランスポートレイヤプロトコルの種類などの点で異なる。 


\section{宇宙インターネットにおけるルーティング}
\subsection{衛星間のContact}
DTNは、 通信の遅延や途絶が発生する環境でデータを確実に届けるための技術である。このDTNにおいて、
 通信が可能な時間やタイミングを「Contact」と呼んでいる。 Contactとは、 
 2つのノード間で直接通信が可能な期間やその条件を指し、 データを送受信できる時間を意味している。

DTNを使用する環境では、 常に接続が確立されている訳ではなく、 ノード間の通信が可能な期間が限られていることが一般的である。
したがって、 このContactを正確に把握することが、 効率的なデータ転送やネットワークの最適化において極めて重要な要素となっている。
例えば、 地球と宇宙探査機の間の通信を考えてみると。 両者が視界に入る時間にだけデータを送信することができる。 この時間がContactにあたる。 

\subsection{Contact Graph Routing}
既に述べた通り既存のDTN実装は複数あるが、これらのDTN実装におけるルーティング手法は、
主にContact Graph Routing(CGR)が用いられ、 
CCSDSではSCHEDULE-AWARE BUNDLE ROUTINGとして標準化されている。 
宇宙におけるノード間の通信可能な機会は物理的な軌道の計算により予測可能であり、 
CGRはノードの通信可能機会とそのスループットなどが書かれたContact Planを用いて
\subsection{IPN address}
IPNアドレス(Interplanetary Networking Address)は、 DTN環境で使用されるアドレス形式で、
宇宙通信のためのネットワーク識別とエンドポイントの識別を可能にするものである。 従来のインターネットプロトコルアドレス(IPアドレス)は、
リアルタイムでの通信や短い遅延を前提とした設計であるため、 宇宙空間における遅延や断絶が発生する環境では適切に機能しない。
DTNのアーキテクチャは、 これらの遅延や断絶を前提としており、 IPネットワークとは異なる方法でデータを伝送するため、
IPNアドレスが必要とされている。 さらに、 IPNアドレスは地上のインターネットや宇宙のネットワークなど、
異なるアドレッシングスキームを持つネットワークの統合する役割としても機能する。

IPNアドレスは「ipn:ノード番号. サービス番号」という形式で記述され、 これにより特定の宇宙船や装置が個別に識別される。 

\section{宇宙でのトポロジーの変化}

% 第3章 関連研究と事例
\chapter{宇宙におけるコンタクト情報に関連した研究と標準化動向}

\label{chap:related_works}
\section{宇宙インターネットにおけるルーティングのアルゴリズム}
\subsection{CGRのアルゴリズム}
先行研究1: Routing in the Space Internet: A contact graph routing tutorial
CGRについて定義した論文
DTNでのrouting schemeの管理方法については、Centralized,Distributed,Source routingがあるとした上で、「in-depth quantitative comparison of the centralized, source routing and distributed routing approach remains a topic for future research.」として、配送方法についての検討の必要性が提唱されている
\subsection{Yen routing algorithm}

\section{宇宙ネットワークのシミュレーション}
宇宙ネットワークの立ち位置
既存研究・関連研究

先行研究2: Tracking Lunar Ring Road Communication
先行研究1で述べられた、DTNにおけるrouting schemeのうち、Source routingによって配送を行う、月のcubesat コンステレーションを想定。
経路を多く把握するノードを決めておき、各ノードは最低限の経路情報のみを把握
評価には配送成功率と平均配送時間のみを使用し、
上記で述べた「宇宙環境下で最も適した」かどうかについては定性的に判断しているのみ
提案(Approach)
先行研究では、Centralized,Distributedの手法についての定量的な比較が存在しない。
Centralized,Distributedの手法のどちらがより想定する宇宙環境におけるコンタクト情報配布方法として適切か、またそのための適切な条件を探索する

\section{宇宙インターネットにおける経路情報管理手法の確立の必要性}
\section{コンタクト情報から経路計算への処理を行うノード}
\subsection{Distributed Method}
コンタクトグラフそのもの
CCSDSの見本(単純な表形式)に沿って作成。

\subsection{Centralized Method}
コンタクトグラフから計算された経路情報


% 第4章 提案手法:sXGP-5Gコンバータ
\chapter{提案手法:sXGP-5Gコンバータ}
% --- 章アウトライン・TODO・参考文献引用例 ---
% この章では、提案するsXGP-5Gコンバータの設計方針・要件・実装案・制約を述べる。
% TODO: docker_open5gs_sXGP-5G の設計思想や工夫点を記述。
% TODO: 参考文献を本文中で引用する(例: \cite{rfc9171})。
% 例: プロトコル変換の設計指針は \cite{rfc5326} などを参照。
% ------------------------------------------
\label{chap:proposal}

\section{要求仕様と設計方針}
\subsection{ユースケース}
提案環境は、実装ベース標準化を支援する以下のユースケースを想定する。

\begin{enumerate}
	\item \textbf{標準仕様の実装検証と相互運用性テスト}: 3GPP仕様に基づくOSS実装を実機RANで動作させ、仕様書上の記述と実装の乖離、異なる実装間の相互運用性問題を早期に発見する。

	\item \textbf{標準化へのタイムリーなフィードバック}: 検出した問題を再現手順・パケットキャプチャとともにドキュメント化し、標準化団体(3GPP等)や関連OSSプロジェクトへ報告・改善提案を行う。

	\item \textbf{継続的インテグレーション・回帰テスト}: 再現性の高い実験環境により、コード変更やプロトコル拡張が既存機能を破壊していないかを自動的に検証する。

	\item \textbf{性能評価とボトルネック分析}: 実機環境での遅延・スループット測定により、プロトコル設計や実装の性能ボトルネックを特定し、標準化・実装の両面から改善を図る。

	\item \textbf{プロトコル拡張の試作・検証}: 新機能や実験的プロトコルを実装し、実機環境で動作検証を行うことで、標準化前の技術検証を加速する。

	\item \textbf{教育・トレーニング用途}: 実機相当の環境で5G/6Gプロトコルスタックの動作を学習し、標準化・実装の実践的スキルを習得する。
\end{enumerate}

\subsection{非機能要件(再現性・法令遵守・安全性)}
\begin{itemize}
	\item \textbf{再現性}: Docker等のコンテナ技術により、環境構築手順を標準化し、異なる環境でも同一の実験を再現可能にする。
	\item \textbf{法令遵守}: 免許不要帯(sXGP)を用いることで、電波法に抵触せず実機検証を実施できる。
	\item \textbf{安全性}: 実験環境を隔離されたネットワークで構築し、商用ネットワークへの影響を排除する。
		\item \textbf{可観測性}: すべての制御/ユーザ面トラフィックをpcap/ログとして取得でき、試験ごとに成果物として保存できること。
		\item \textbf{保守性}: 変換ロジックはメッセージ種別ごとにモジュール化し、追加/無効化が容易であること。
\end{itemize}

\subsection{対象範囲(制御/ユーザ面、認証、セッション管理)}
本提案は、LTE eNBを変更せず5GCへ接続するための\textbf{変換層}(s1n2-converter)である。対象とする範囲と非対象は以下のとおりである。

\paragraph{対象}
\begin{itemize}
	\item 制御面: S1APとNGAPのメッセージ変換およびIEマッピング(Initial UE/Context Setup近傍の手順を中心)
	\item ユーザ面: S1-UとN3間のGTP-Uカプセル化のパススルーおよびTEIDマッピング
	\item コンテキスト: UE識別子(MME-UE-S1AP-ID/ENB-UE-S1AP-ID, AMF-UE-NGAP-ID)の対応管理
\end{itemize}

\paragraph{非対象(現段階)}
\begin{itemize}
	\item 無線(PHY/MAC/RLC/PDCP)の最適化やNR固有機能(本研究はRANにsXGP/TD-LTEを用いる)
	\item EPS/5GS間の移動性最適化(N26インターフェース)\cite{threegpp-23502}
	\item ハンドオーバ、eDRX/PSM等の省電力最適化
\end{itemize}

\section{アーキテクチャ設計}
提案アーキテクチャを図示的に記述すると、\texttt{eNB}—(S1AP,S1-U)→\texttt{Converter}—(NGAP,N3)→\texttt{5GC(AMF/SMF/UPF)} である。Converterは制御面/ユーザ面を分離して処理し、状態を共有する。5GCはOpen5GS\cite{open5gs}を想定するが、NGAP/N3の標準動作に従う実装であれば置換可能である(\cite{threegpp-23501,threegpp-23502})。

\subsection{制御面:S1AP–NGAP変換の設計案}
変換対象とする代表的手順とマッピング方針を示す。
\begin{itemize}
	\item \textbf{Initial UE Message(S1AP)→ Initial UE Message(NGAP)}: UE識別子とTA/PLMN情報、NASコンテナを搬送する。ConverterはeNBからのECGI/TAI等をNGAPのTAI/NR CGI相当へマップし、AMF選定に必要なIEを充足させる。
	\item \textbf{Initial Context Setup(S1AP)↔ Initial Context Setup(NGAP)}: セキュリティ/UE-AMBR等のコンテキストIEを相互に再構成する。E-RAB関連IEは、5GのPDU Session資源手順と分離して取り扱う(後述)。
	\item \textbf{UE Context Release}: 釣り合いの取れた解放手順(S1AP UEContextRelease/NGAP UEContextRelease)を相互に変換する。
\end{itemize}

設計上の原則は、(i) NASは原則透過搬送としConverterで終端しない、(ii) IEの欠落時は安全側(エラー)に倒す、(iii) 5Gで独立した手順(例: PDU Session Resource Setup)はS1APのICSと\textbf{分離}して逐次処理する、である。これにより、AMF/SMF側の前提条件を満たさずに\texttt{unknown-PDU-session-ID}等のエラーとなる事象を回避する。

\subsection{ユーザ面:GTP-U転送・TEID管理}
ユーザ面はパススルーを基本とし、S1-U(eNB⇄Converter)とN3(Converter⇄UPF)で独立したTEID空間を維持する。Converterは方向別に以下の対応表を持つ。
\begin{description}
	\item[UL Map] (ENB-UE-S1AP-ID, eNB\_TEID\_UL) → (UPF\_IP, N3\_TEID\_UL)
	\item[DL Map] (AMF-UE-NGAP-ID, N3\_TEID\_DL) → (eNB\_IP, S1U\_TEID\_DL)
\end{description}
TEIDの学習は、\texttt{E-RAB Setup List}(S1AP)および\texttt{PDUSessionResourceSetupResponseTransfer}(NGAP)に含まれるGTP\_Tunnel情報から行う。学習前パケットはドロップし、ログで検知する。MTUについては、d\_max = min(1500, パス上の最小MTU - GTP/IP/UDPヘッダ) とし、DFビットはクリアする保守的ポリシを採用する。

\subsection{認証・登録(NASメッセージ処理の方針)}
NASは原則透過搬送とし、Converterはカプセル化境界の整合性のみを担保する。具体的には、S1AP/NGAPの\texttt{NAS-PDU}フィールドとして運ばれるバイト列を改変せず、AMF/UEで終端されることを前提とする。学術検証モードとして、事前に加入者・鍵情報を5GCに登録し、認証アルゴリズム不一致による失敗を切り分ける運用を提供する。NASの再暗号化や鍵導出はConverterの責務に含めない(\cite{threegpp-23501})。

\subsection{コンテキスト管理とタイムアウト}
ConverterはUEごとに以下を保持する。
\begin{itemize}
	\item ID束: (ENB-UE-S1AP-ID, MME-UE-S1AP-ID) ↔ (AMF-UE-NGAP-ID)
	\item セッション束: PDU Session ID ↔ E-RAB ID の対応、TEID対応表(UL/DL)
	\item タイマ: T\_ctx(コンテキスト保持), T\_gtp(TEID学習待ち), T\_proc(手順待ち合わせ)
\end{itemize}
T\_ctx満了でUEコンテキストを破棄し、未解放資源をログ出力する。異常系(IE不足/不整合、学習超過)は\texttt{cause}を付して相手側へエラーを返すか、ドロップ+監査ログとする。

\section{sXGP採用の根拠と想定限界}
\subsection{採用の根拠:実装ベース標準化への貢献}

sXGPを採用する最大の理由は、\textbf{免許不要帯での法令遵守により「実装を実際に動かせる」}ことにある。これは、実装ベース標準化の実現において極めて重要である。

\begin{itemize}
	\item \textbf{実機による全スタック検証}: 実UE・実NIC・実時間のプロトコルスタックを動作させることで、シミュレータでは検出できない実装レベルの問題(タイミング、リソース競合、ハードウェア依存の挙動など)を発見できる。これは標準仕様と実装の乖離を早期に検出するために不可欠である。

	\item \textbf{相互運用性問題の早期発見}: 異なるベンダの実装(UE、eNB、コア)を組み合わせた際の相互運用性問題を、実機環境で検証できる。3GPP標準の曖昧な記述や解釈の違いによる非互換性を、標準化プロセスの早い段階でフィードバック可能にする。

	\item \textbf{標準化へのタイムリーなフィードバック}: 実機で問題を再現できることで、再現手順・パケットキャプチャ・ログを含む具体的な報告を標準化団体やOSSプロジェクトに提供できる。これにより、仕様策定 → 実装 → 問題発覚 → 修正のサイクルを数年単位から数ヶ月単位に短縮できる。

	\item \textbf{継続的検証の実現}: 免許不要帯であるため、継続的インテグレーション(CI)環境に組み込んで自動テストを実施できる。コード変更のたびに実機検証を行うことで、回帰テストと品質保証を強化できる。

	\item \textbf{(副次的効果)低コスト・再現性}: 専用周波数ライセンスが不要であり、比較的低コストで実験環境を構築できる。また、Dockerなどの技術と組み合わせることで、再現性の高い検証環境を複数の研究者・組織で共有できる。
\end{itemize}

\subsection{想定限界・外延}
本環境はLTE互換のsXGPをRANとして用いるため、5G NR特有のPHY機能やスケジューリング最適化は対象外となる。しかし、制御面(S1AP/NGAP/NAS)およびユーザ面(GTP-U)のプロトコルレベルでの相互運用性検証は十分に可能であり、標準化フィードバックの主要な価値はこのレイヤーにある。NR固有の無線レイヤーの課題は、今後の拡張として分離して扱う。

\section{実装方針と部品選定}
\subsection{ソフトウェア構成(言語・ライブラリ・依存)}
実装言語はCとし、制御面/ユーザ面は独立プロセスまたはスレッドで分離する。ASN.1の処理は外部ツールで生成したデコーダを用いるか、既存実装(例: Open5GSのエンコーダ/デコーダ参照\cite{open5gs})の表現に合わせてシリアライザを実装する。検証用途として、srsRANのZMQインターフェースを接続し、無線機なしでの反復試験を可能とする(\S\ref{chap:related})。

\subsection{ネットワーク構成(VLAN/VRF/NATの要否)}
最小構成では、ConverterにS1U/N3の2ポート(論理でも可)を割り当て、データプレーンをルーティングで分離する。商用NWとの干渉回避のため、VRFまたは名前空間を用いた論理分離を推奨する。UPFとは静的ルーティングで直結し、アドレス設計はOpen5GSのデフォルト構成に準拠する(\S\ref{chap:experiment})。

\subsection{監視・計測(メトリクス、トレース)}
制御面はメッセージ種別/結果(成功・失敗原因)/処理時間をメトリクス化し、ユーザ面はスループット/RTT/パケット損失率を定期測定する。すべての試験についてpcapと構成スナップショット(コンテナタグ、設定ファイル)を成果物として保存し、回帰の基準とする。

\section{制約と想定される限界}
\subsection{規格差分による制限}
S1APとNGAPはIEの集合と手順が一致しないため、完全な1対1変換は成立しない。特に、5GのPDU Session手順はS1APのE-RAB手順と\textbf{分離}されている。設計では手順の逐次化と、IE欠落時の安全側フェイルにより、相互運用性を最大化しつつ誤動作を抑制する(\cite{threegpp-23501,threegpp-23502})。

\subsection{性能上のボトルネックと回避策}
ユーザ面は二重のGTP終端によるオーバヘッドが生じる可能性がある。最適化として、カーネルオフロード(XDP/TC)やゼロコピーI/Oの導入、TEIDルックアップのハッシュ最適化を段階的に適用する。制御面はASN.1エンコード/デコードのCPU負荷が支配的となるため、ホットパスの事前割付とメモリプールでGC負荷を抑える。


% 第5章 実験環境と方法
\chapter{sXGP-5G環境での実機検証手法}
% --- 章アウトライン・TODO・参考文献引用例 ---
% この章では、実験環境・シナリオ・計測方法・指標を記述する。
% TODO: docker_open5gs_sXGP-5G の実験構成や計測手順を具体的に記述。
% TODO: 参考文献を本文中で引用する(例: \cite{dtn_implementations})。
% 例: DTNの評価手法は \cite{Fraire2021} などを参照。
% ------------------------------------------
\label{chap:experiment}

\section{実験の目的}
本章では、提案したsXGP-5Gコンバータ環境を用いて、以下を実証する:

\begin{itemize}
	\item \textbf{基本的な相互接続動作の確認}: 4G RAN(sXGP eNB + 実UE)と5G Core(Open5GS)間で、登録・PDUセッション確立・データ転送が正常に動作することを検証する。

	\item \textbf{S1AP/NGAP/NAS信令変換の正当性}: コンバータが4G(S1AP)と5G(NGAP)のプロトコル変換を正しく行い、両側のプロトコルスタックが期待通り動作することを確認する。

	\item \textbf{GTP-Uトンネル中継の動作確認}: ユーザ面トラフィックがS1-U(eNB側)とN3(UPF側)間で正しくトンネリングされ、TEID管理が適切に機能することを検証する。

	\item \textbf{実機特有の相互運用性問題の検出能力}: シミュレータでは再現困難な実機依存の挙動(タイミング、NAS再送、端末OS依存の処理など)を検出できることを示す。

	\item \textbf{再現性の確保}: Docker化された環境により、同一の実験を異なる環境で再現可能であることを確認する。
\end{itemize}

\section{実験環境}
\subsection{ハードウェア構成(UE/eNB/sXGP基地局/サーバ)}
本研究の実機検証は、以下の最小構成で実施する。
\begin{itemize}
	\item \textbf{UE}: 市販スマートフォン(4G LTE対応、sXGP対応バンド)またはLTEモデム端末。
	\item \textbf{sXGP基地局(eNB)}: TD-LTE互換のsXGP小型基地局(室内用)。
	\item \textbf{サーバ}: x86\_64または同等性能のホスト1台。ConverterおよびOpen5GSのコンテナを実行する。NICは1~2ポート(管理/データ分離は論理でも可)。
	\item \textbf{スイッチ/ルータ}: 管理ネットワークと実験ネットワークのL2/L3分離に用いる。VRFまたはLinuxネットワーク名前空間で代替可能である。
\end{itemize}

\subsection{ソフトウェア構成(5GC, コンバータ, OS)}
\begin{itemize}
	\item \textbf{5GC}: Open5GS(AMF/SMF/UPF/UDM/AUSF/NRF, WebUI)。Dockerコンテナで起動し、加入者情報をWebUIまたはAPIで登録する\cite{open5gs}。
	\item \textbf{Converter}: s1n2-converter(制御面: S1AP↔NGAP, ユーザ面: S1-U↔N3)。制御/ユーザ面は別プロセス(またはスレッド)として動作し、TEID対応表を共有する(第\ref{chap:proposal}章)。
	\item \textbf{OS/ツール}: Linux(Docker/Compose, tcpdump, tshark, iproute2)。時間同期はNTPで十分である(PPS等の高精度は本検証の範囲外)。
\end{itemize}

\subsection{ネットワークトポロジとIPアドレッシング}
トポロジは、\texttt{UE}—\texttt{eNB(sXGP)}—(S1AP,S1-U)→\texttt{Converter}—(NGAP,N3)→\texttt{5GC} の直列構成である。Converterは制御面(S1AP/NGAP)とユーザ面(S1-U/N3)で論理IFを分ける。5GC側は、AMF/SMF/UPFを同一ホスト内に配置する。UEアドレスプールやAPN等はOpen5GSのデフォルト方針に準拠し、必要に応じてIPv6-only + NAT64/464XLAT, DNS64 構成を併用する(第\ref{chap:related}章)。

\section{実験シナリオ}
\subsection{基本接続(登録・PDUセッション確立)}
\paragraph{目的} ConverterのS1AP↔NGAP変換と5GC連携が成立することを確認する。
\paragraph{前提} 5GC起動済、加入者登録済、eNBのS1接続先がConverterに設定済。
\paragraph{手順}
\begin{enumerate}
	\item UEをsXGPセルに接続し、Attach/Registrationを開始する。
	\item ConverterでInitial UE Message/Initial Context Setupの相互変換が成功し、AMFでUE Contextが生成されることを確認する(ログ/pcap)。
	\item PDU Session(5G)資源手順がSMF/UPF側で成功し、ConverterにTEID情報が学習されることを確認する。
\end{enumerate}
\paragraph{成功基準} AMF/SMF/UPFログにエラーがなく、UEがPDUセッション確立状態となる。Converterの対応表にTEID(UL/DL)が登録される。

\subsection{データ転送(スループット・遅延)}
\paragraph{目的} S1-U/N3間のGTP-U中継とTEIDマッピングが正しく機能することを確認する。
\paragraph{手順}
\begin{enumerate}
	\item UEからコアネットワーク外部(またはUPFデータネット)へのICMP/TCP/UDP通信を発生させる。
	\item ConverterのS1-U/N3でpcapを取得し、TEID・5タプル対応を突合する。
	\item スループット(iperf等)とRTT(ping)を測定し、パスの安定性を確認する。
\end{enumerate}
\paragraph{成功基準} UL/DLともに期待するTEIDでカプセル化され、フローが継続する。損失/再送が異常に多い場合はMTU/MSSやオフロード設定を点検する。

\subsection{ハンドオーバ相当の扱い(必要に応じて)}
本研究はN26等の移動性最適化を対象外とするが、\textbf{再登録/再確立}シナリオにより実運用で近似するイベントを評価する。具体的には、電波減衰やセル再選相当のイベントを与え、RegistrationやPDU Sessionの再確立が円滑に行われるかを、NAS再送やタイマ(T3xx)挙動とあわせて確認する(第\ref{chap:proposal}章)。

\subsection{Androidエミュレータの限界と実UE検証の必要性}
通信機能を有するAndroidアプリケーションの検証において、\textbf{PC上のエミュレータのみでは再現困難な事象}が多い。特に、セルラースタックや事業者設定、端末OSの省電力・バックグラウンド制御など、\textbf{実機依存の挙動}がアプリの通信体験を大きく左右する。本研究では、sXGP+実UE+5GC構成により、以下の観点で\textbf{実機ならではの検証}を行う。

\paragraph{エミュレータで不足しがちな項目}
\begin{itemize}
	\item \textbf{セルラー制御プレーンの実挙動}: 実エミュレータはベースバンドを持たず、Registration/TAU、NAS再送、タイマ境界での挙動差(T3xx/Back-off等)の忠実再現が困難。
	\item \textbf{IMS/音声・SMS連携}: VoLTE/VoNRやSMS over NASなどは端末・キャリア・IMSの三者連携が必要で、エミュレータでは未実装/限定的であることが多い。
	\item \textbf{ネットワーク条件とスタック差分}: IPv6-only + NAT64/464XLAT、DNS64、PCO/APNパラメータ、MTU/MSS、NICオフロード(TSO/GRO)等の影響はエミュレータで再現しづらい。
	\item \textbf{OSによるバックグラウンド制御}: Doze/App Standby、JobScheduler/WorkManagerのネットワーク可用性制御は実機の省電力・無線レイヤと結合しており、エミュレータでは挙動が異なる。
	\item \textbf{移動性・無線イベント}: RRC Idle/Connectedの遷移、電波強度・セルリセレクション、ハンドオーバ相当の切替など、時間軸のイベントは実無線でなければ発現しにくい。
	\item \textbf{ポリシ・スライス連携}: URSPやスライシングは端末/OS/事業者依存が強く、汎用エミュレータでは検証対象外となることが多い(本環境では制御/ユーザ面の相互接続検証を主対象とする)。
\end{itemize}

\paragraph{実UE+sXGP+5GCでの検証手順(例)}
\begin{enumerate}
	\item \textbf{端末側ログ取得}: logcat(特にradioバッファ)とネットワークログを収集し、アプリのAPI呼び出しと無線/NASイベントを時系列で突合する。
	\item \textbf{ネットワーク側トレース}: コンバータ/5GC側でS1AP/NGAP/NAS、GTP-Uのpcapを取得し、再送、原因コード、TEID/フローの対応を解析する。
	\item \textbf{テストシナリオ}:
		\begin{itemize}
			\item IPv6-only + NAT64/464XLAT 下でのアプリ接続(DNS64解決、QUIC/TCPの挙動差)。
			\item RRC Idle復帰や電波減衰を伴う再接続でのセッション継続性(ソケット再確立、タイムアウト)。
			\item 大きなパケットや損失率注入時の再送・バックオフ挙動(アプリ層リトライ設計の妥当性)。
		\end{itemize}
	\item \textbf{再現性確保}: Docker化した5GC/コンバータ構成と固定シナリオにより、回帰テストとして継続的に実行可能にする。
\end{enumerate}

\paragraph{本環境で得られる価値}
実UEを用いることで、\textbf{シミュレータでは顕在化しにくい相互運用性問題や端末OS依存の振る舞い}(例:NAS再送シーケンスのズレ、IPv6-only下でのAPI失敗、バックグラウンド時の接続断)を早期に発見し、\textbf{パケットキャプチャと再現手順}を添えて標準化・OSS実装へフィードバックできる。

\section{計測方法と指標}
\subsection{A/B比較(シミュレータ vs 実機)}
\begin{itemize}
	\item A: RANシミュレータ+5GC(OSS)\quad B: sXGP eNB+実UE+5GC(OSS)
	\item 比較対象: 登録/PDU確立成功率、タイマ境界での失敗率、再送回数、エラーコード、スループット、遅延、リソース使用率
	\item ログ・トレース: S1AP/NGAP/NAS、GTP-U、カーネル/DPDK統計、pcap
\end{itemize}
\subsection{成功基準}
基本機能の安定成立(>99%)、性能劣化の要因特定、実機特有の不具合検出(再現手順付き)を満たすこと。
\subsection{成功判定基準と再現手順}
\paragraph{判定基準}
\begin{itemize}
	\item 制御面: Initial UE/Context SetupおよびPDU Session資源手順がエラーなく完了(AMF/SMFログ, NGAPトレース)
	\item ユーザ面: UE↔UPF間で期待するTEIDによりカプセル化され、連続したシーケンスで転送
	\item 安定性: 連続N回(例: 30回)の接続/切断試験で成功率>99%
\end{itemize}
\paragraph{再現手順(成果物)}
試験ごとに、(i) コンテナタグ/設定ファイル、(ii) Converter/AMF/UPFのログ、(iii) S1AP/NGAP/GTP-Uのpcap、(iv) 計測スクリプトの出力(CSV/JSON)を保存し、成果物として管理する。これにより、実装の回帰有無を定量的に判定できる\cite{dtn_implementations}。

\subsection{メトリクス(接続成功率、遅延、スループット、CPU/メモリ)}
\begin{itemize}
	\item 機能: 登録成功率、PDUセッション確立成功率、再登録時の成功率
	\item 性能: RTT(ICMP/TCP SYN-ACK)、スループット(iperf3)、再送率、フロー継続時間
	\item 資源: Converter/5GCのCPU利用率、メモリ、コンテキスト数、TEIDテーブルサイズ
\end{itemize}

\subsection{ロギング・パケットキャプチャの取得方法}
\begin{itemize}
	\item ConverterのS1AP/NGAP IFでpcap取得(制御面、フィルタ: SCTP, NGAP/S1APポート)
	\item ConverterのS1-U/N3 IFでpcap取得(ユーザ面、フィルタ: UDP dst port 2152)
	\item AMF/SMF/UPFのアプリケーションログ(JSON/テキスト)を保存し、タイムスタンプで突合
\end{itemize}
NAS/NGAP/S1APのメッセージシーケンスは、列挙型(IE)と原因コードで比較する。必要に応じ、メッセージ時系列の可視化にシーケンス図生成ツールを用いる\cite{Fraire2021}。


% 第6章 評価
\chapter{評価}
\label{chap:evaluation}

\section{結果}
\subsection{機能検証の結果(登録・PDUセッション)}
\subsection{性能測定の結果(遅延・スループット)}
\subsection{リソース使用率・スケーラビリティ}

\section{考察}
\subsection{提案手法の有効性と限界}
\subsection{関連研究との比較と位置づけ}
\subsection{実運用への適用可能性}


% 第7章 結論と展望
\chapter{結論と展望}
\label{chap:conclusion}
\section{本研究のまとめ}
本研究では, ContactPlanの更新はDTNにおけるBundleをよりよく配達できるように
する効果があることを確認するとともに, その効果は更新の範囲を天体内に限定した場合にも
十分に有効であることが示された. また天体間にも拡散する既存手法の場合, 
提案手法よりも天体間通信リンクを消費することが確認できた. 地球のインターネットにおける
Byte数を考慮するとかなり小さい値ではあるものの, リソースが制約された宇宙ノードにおいては
このリンク消費を必要としないことはメリットとなりうる. 

\section{今後の課題と展望}
本研究ではシミュレーションに用いるトポロジーを抽象化しており, 
さらにそれぞれのノードのContactはランダムに生成を行っている. 
今後は, 衛星のContactのタイミングについて軌道シミュレーターを用いるなど, 
より実際の宇宙環境におけるトポロジーを考慮したシミュレーションを行うことが必要である. 
また本研究では単一のContactの失敗のみに着目しているが, 
実際のDTNの運用においては, 単一のContactの失敗だけでなく, 
長時間にわたる単一のContactの一定時間だけの障害や, 
複数のContactの連続した失敗なども考慮する必要がある. 
これらまで対象を拡大し, 包括的な検証を行うことが今後の課題である. 
さらに, 本研究はContact Planの臨時更新を天体内に限定することを提案したが, 
今後DTNの本格的な運用に置いては, \ref{section:Contact Planの定期的・継続的な配布}節の
Contact Planの定期的・継続的な更新もさらなる研究・標準化が必要である. 
定期的・継続的な更新においてはContact Planのサイズ増加に対する対策が必要であり, 
天体ごとに管理を行うという本研究の発想は, この問題にも適用できる可能性がある. 


% 謝辞
\chapter*{謝辞}\markboth{謝辞}{謝辞}
\addcontentsline{toc}{chapter}{謝辞}
\label{thanks}
本論文を執筆するにあたり,ご指導賜りました慶應義塾大学教授 村井純博士,慶應義塾大学環境情報学部教授 中村修博士,同学部教授 楠本博之博士,同学部教授 高汐一紀博士,同学部教授 Rodney D.Van Meter 博士,同学部教授 植原啓介博士,同学部教授 三次仁博士,
同学部教授 中澤仁博士,同学部教授 手塚悟博士,同学部教授 武田圭史博士,同学部准教授 大越匡博士,同大学政策・メディア研究科特任教授 鈴木茂哉博士,同研究科特任助教 工藤紀篤博士, 同研究科特任講師 松谷健史博士に感謝いたします.

特に植原啓介博士には rgroot のファカルティとして,日頃から研究面や運用面で指導をしていただきました.
また,1 月に参加した私にとって初めての国際会議に同伴していただき,緊張している私をサポートしていただきました.感謝いたします.

私がコンピュータネットワークの分野に進むきっかけを作っていただいた,慶應義塾大学大学院 豊田安信氏,元慶應義塾大学大学院 (現 NTT コミュニケーションズ) 深川祐太氏に感謝いたします.
私は 2020 年秋学期に開講された,インターネットの設計と運用 という講義でネットワーク技術の面白さを知ることができました.
豊田安信氏,深川祐太氏は TA/SA として私にネットワーク技術の面白さを伝えてくださりました.
また,私を rgroot に誘ってくださったのもこのお二人でした.
ありがとうございます.

東京大学准教授 中村遼博士に感謝いたします.
中村遼博士には,研究面で多大な指導をしていただきました.
研究ネタを一緒に考えてくださり,本論文のアイデアも中村遼博士からいただきました.
また,中村遼博士に指導をしていただきながら執筆した論文は ICOIN 国際会議に採択していただくことができました.感謝いたします.

慶應義塾大学修士課程 石原匠氏に感謝いたします.
石原匠氏は友人として私に接してくれながら,ときには先輩としてその背中を見せてくださりました.
コロナ禍に入学した私には大学に友人が少なかったので,先輩でありながら気軽に話せる存在は大変心の支えになりました.

東京大学大学院 伊藤広記氏,元東京大学大学院 (現 LINE ヤフー株式会社) 金谷光一郎氏に感謝いたします.
伊藤広記氏,金谷光一郎氏は,当時の私と同様にネットワーク運用未経験者として WIDE Project の vSIX ワーキンググループに参加し,共に切磋琢磨しあって頂きました.
伊藤広記氏,金谷光一郎氏は他大学の先輩でありながら,友人としても私に接してくださいました.
ネットワークに入門して日が浅く右も左もわからないとき,わからないなりに共に考え,議論したことはとても良い経験になりました.

父の澤田裕司氏,母の澤田由紀氏に感謝いたします.
家では口数の少ない私ですが,部屋に引きこもってパソコン作業を続けることができたのは家族のサポートあってこそでした.感謝いたします.

東京工業大学附属科学技術高等学校 13 期マイコン制御部 OB に感謝いたします.
コロナ禍で大学に通えず,また新たな友人を作る機会が殆どなかった当時,同期の皆さんと毎晩オンラインゲームに励んだことは心の支えでした.
コロナ禍が明けた今でも,たまに飲みに行ったり,変わらずゲームをしたり,Twitter (X) 上で他愛もないコミュニケーションを取れることは大変嬉しいことです.
本論文執筆に関しても,別の大学,別分野の研究でありながら,互いに鼓舞しあうことでモチベーションを高め合い,書き切ることができました.ありがとうございます.

最後に,全員の名前を書くことはできませんが,村井合同研,WIDE プロジェクト関係者全員に感謝いたします.
私がネットワーク分野に興味を持ち,続けられたのは皆様の力あってこそでした.深く感謝申し上げます.

\renewcommand{\thechapter}{\Alph{chapter}}
\setcounter{chapter}{0}

\vspace{-5mm}
\bibliographystyle{unsrt}
\bibliography{bib/thesis}

\chapter*{付録}\markboth{付録}{付録}
\addcontentsline{toc}{chapter}{付録}

\label{appendix}
\lstset{%
 basicstyle={\tiny\ttfamily},%
 identifierstyle={\tiny},%
 commentstyle={\tiny\itshape},%
 keywordstyle={\tiny\bfseries},%
 ndkeywordstyle={\tiny\ttfamily},%
 stringstyle={\tiny\ttfamily},
 frame={tb},
 framesep=1zw,
 breaklines=true,
 numbers=left,%
 xrightmargin=0zw,%
 xleftmargin=1.5zw,%
 numberstyle={\scriptsize},%
 stepnumber=1,
 numbersep=1zw,%
 lineskip=-0.5ex,%
}
\begin{multicols}{2}
\begin{lstlisting}[caption=kernel config,label=kconfig,]
    #
    # Automatically generated file; DO NOT EDIT.
    # Linux/x86 5.15.106 Kernel Configuration
    #
    CONFIG_CC_VERSION_TEXT="gcc (Ubuntu 11.3.0-1ubuntu1~22.04.1) 11.3.0"
    CONFIG_CC_IS_GCC=y
    CONFIG_GCC_VERSION=110300
    CONFIG_CLANG_VERSION=0
    CONFIG_AS_IS_GNU=y
    CONFIG_AS_VERSION=23800
    CONFIG_LD_IS_BFD=y
    CONFIG_LD_VERSION=23800
    CONFIG_LLD_VERSION=0
    CONFIG_CC_CAN_LINK=y
    CONFIG_CC_CAN_LINK_STATIC=y
    CONFIG_CC_HAS_ASM_GOTO=y
    CONFIG_CC_HAS_ASM_GOTO_OUTPUT=y
    CONFIG_CC_HAS_ASM_GOTO_TIED_OUTPUT=y
    CONFIG_CC_HAS_ASM_INLINE=y
    CONFIG_CC_HAS_NO_PROFILE_FN_ATTR=y
    CONFIG_PAHOLE_VERSION=0
    CONFIG_IRQ_WORK=y
    CONFIG_BUILDTIME_TABLE_SORT=y
    CONFIG_THREAD_INFO_IN_TASK=y
    
    #
    # General setup
    #
    CONFIG_INIT_ENV_ARG_LIMIT=32
    CONFIG_LOCALVERSION=""
    CONFIG_BUILD_SALT=""
    CONFIG_HAVE_KERNEL_GZIP=y
    CONFIG_HAVE_KERNEL_BZIP2=y
    CONFIG_HAVE_KERNEL_LZMA=y
    CONFIG_HAVE_KERNEL_XZ=y
    CONFIG_HAVE_KERNEL_LZO=y
    CONFIG_HAVE_KERNEL_LZ4=y
    CONFIG_HAVE_KERNEL_ZSTD=y
    CONFIG_KERNEL_ZSTD=y
    CONFIG_DEFAULT_INIT=""
    CONFIG_DEFAULT_HOSTNAME="(none)"
    CONFIG_SWAP=y
    CONFIG_SYSVIPC=y
    CONFIG_SYSVIPC_SYSCTL=y
    CONFIG_POSIX_MQUEUE=y
    CONFIG_POSIX_MQUEUE_SYSCTL=y
    CONFIG_WATCH_QUEUE=y
    CONFIG_CROSS_MEMORY_ATTACH=y
    CONFIG_USELIB=y
    CONFIG_AUDIT=y
    CONFIG_HAVE_ARCH_AUDITSYSCALL=y
    CONFIG_AUDITSYSCALL=y
    
    #
    # IRQ subsystem
    #
    CONFIG_GENERIC_IRQ_PROBE=y
    CONFIG_GENERIC_IRQ_SHOW=y
    CONFIG_GENERIC_IRQ_EFFECTIVE_AFF_MASK=y
    CONFIG_GENERIC_PENDING_IRQ=y
    CONFIG_GENERIC_IRQ_MIGRATION=y
    CONFIG_HARDIRQS_SW_RESEND=y
    CONFIG_IRQ_DOMAIN=y
    CONFIG_IRQ_DOMAIN_HIERARCHY=y
    CONFIG_GENERIC_MSI_IRQ=y
    CONFIG_GENERIC_MSI_IRQ_DOMAIN=y
    CONFIG_IRQ_MSI_IOMMU=y
    CONFIG_GENERIC_IRQ_MATRIX_ALLOCATOR=y
    CONFIG_GENERIC_IRQ_RESERVATION_MODE=y
    CONFIG_IRQ_FORCED_THREADING=y
    CONFIG_SPARSE_IRQ=y
    # end of IRQ subsystem
    
    CONFIG_CLOCKSOURCE_WATCHDOG=y
    CONFIG_ARCH_CLOCKSOURCE_INIT=y
    CONFIG_CLOCKSOURCE_VALIDATE_LAST_CYCLE=y
    CONFIG_GENERIC_TIME_VSYSCALL=y
    CONFIG_GENERIC_CLOCKEVENTS=y
    CONFIG_GENERIC_CLOCKEVENTS_BROADCAST=y
    CONFIG_GENERIC_CLOCKEVENTS_MIN_ADJUST=y
    CONFIG_GENERIC_CMOS_UPDATE=y
    CONFIG_HAVE_POSIX_CPU_TIMERS_TASK_WORK=y
    CONFIG_POSIX_CPU_TIMERS_TASK_WORK=y
    
    #
    # Timers subsystem
    #
    CONFIG_TICK_ONESHOT=y
    CONFIG_NO_HZ_COMMON=y
    CONFIG_NO_HZ_IDLE=y
    CONFIG_NO_HZ=y
    CONFIG_HIGH_RES_TIMERS=y
    # end of Timers subsystem
    
    CONFIG_BPF=y
    CONFIG_HAVE_EBPF_JIT=y
    CONFIG_ARCH_WANT_DEFAULT_BPF_JIT=y
    
    #
    # BPF subsystem
    #
    CONFIG_BPF_SYSCALL=y
    CONFIG_BPF_JIT=y
    CONFIG_BPF_JIT_ALWAYS_ON=y
    CONFIG_BPF_JIT_DEFAULT_ON=y
    CONFIG_BPF_UNPRIV_DEFAULT_OFF=y
    CONFIG_USERMODE_DRIVER=y
    CONFIG_BPF_LSM=y
    # end of BPF subsystem
    
    CONFIG_PREEMPT_VOLUNTARY=y
    CONFIG_SCHED_CORE=y
    
    #
    # CPU/Task time and stats accounting
    #
    CONFIG_TICK_CPU_ACCOUNTING=y
    CONFIG_BSD_PROCESS_ACCT=y
    CONFIG_BSD_PROCESS_ACCT_V3=y
    CONFIG_TASKSTATS=y
    CONFIG_TASK_DELAY_ACCT=y
    CONFIG_TASK_XACCT=y
    CONFIG_TASK_IO_ACCOUNTING=y
    CONFIG_PSI=y
    # end of CPU/Task time and stats accounting
    
    CONFIG_CPU_ISOLATION=y
    
    #
    # RCU Subsystem
    #
    CONFIG_TREE_RCU=y
    CONFIG_SRCU=y
    CONFIG_TREE_SRCU=y
    CONFIG_TASKS_RCU_GENERIC=y
    CONFIG_TASKS_RUDE_RCU=y
    CONFIG_TASKS_TRACE_RCU=y
    CONFIG_RCU_STALL_COMMON=y
    CONFIG_RCU_NEED_SEGCBLIST=y
    # end of RCU Subsystem
    
    CONFIG_BUILD_BIN2C=y
    CONFIG_IKCONFIG=m
    CONFIG_LOG_BUF_SHIFT=18
    CONFIG_LOG_CPU_MAX_BUF_SHIFT=12
    CONFIG_PRINTK_SAFE_LOG_BUF_SHIFT=13
    CONFIG_HAVE_UNSTABLE_SCHED_CLOCK=y
    
    #
    # Scheduler features
    #
    CONFIG_UCLAMP_TASK=y
    CONFIG_UCLAMP_BUCKETS_COUNT=5
    # end of Scheduler features
    
    CONFIG_ARCH_SUPPORTS_NUMA_BALANCING=y
    CONFIG_ARCH_WANT_BATCHED_UNMAP_TLB_FLUSH=y
    CONFIG_CC_HAS_INT128=y
    CONFIG_ARCH_SUPPORTS_INT128=y
    CONFIG_NUMA_BALANCING=y
    CONFIG_NUMA_BALANCING_DEFAULT_ENABLED=y
    CONFIG_CGROUPS=y
    CONFIG_PAGE_COUNTER=y
    CONFIG_MEMCG=y
    CONFIG_MEMCG_SWAP=y
    CONFIG_MEMCG_KMEM=y
    CONFIG_BLK_CGROUP=y
    CONFIG_CGROUP_WRITEBACK=y
    CONFIG_CGROUP_SCHED=y
    CONFIG_FAIR_GROUP_SCHED=y
    CONFIG_CFS_BANDWIDTH=y
    CONFIG_UCLAMP_TASK_GROUP=y
    CONFIG_CGROUP_PIDS=y
    CONFIG_CGROUP_RDMA=y
    CONFIG_CGROUP_FREEZER=y
    CONFIG_CGROUP_HUGETLB=y
    CONFIG_CPUSETS=y
    CONFIG_PROC_PID_CPUSET=y
    CONFIG_CGROUP_DEVICE=y
    CONFIG_CGROUP_CPUACCT=y
    CONFIG_CGROUP_PERF=y
    CONFIG_CGROUP_BPF=y
    CONFIG_CGROUP_MISC=y
    CONFIG_SOCK_CGROUP_DATA=y
    CONFIG_NAMESPACES=y
    CONFIG_UTS_NS=y
    CONFIG_TIME_NS=y
    CONFIG_IPC_NS=y
    CONFIG_USER_NS=y
    CONFIG_PID_NS=y
    CONFIG_NET_NS=y
    CONFIG_CHECKPOINT_RESTORE=y
    CONFIG_SCHED_AUTOGROUP=y
    CONFIG_RELAY=y
    CONFIG_BLK_DEV_INITRD=y
    CONFIG_INITRAMFS_SOURCE=""
    CONFIG_RD_GZIP=y
    CONFIG_RD_BZIP2=y
    CONFIG_RD_LZMA=y
    CONFIG_RD_XZ=y
    CONFIG_RD_LZO=y
    CONFIG_RD_LZ4=y
    CONFIG_RD_ZSTD=y
    CONFIG_BOOT_CONFIG=y
    CONFIG_CC_OPTIMIZE_FOR_PERFORMANCE=y
    CONFIG_LD_ORPHAN_WARN=y
    CONFIG_SYSCTL=y
    CONFIG_HAVE_UID16=y
    CONFIG_SYSCTL_EXCEPTION_TRACE=y
    CONFIG_HAVE_PCSPKR_PLATFORM=y
    CONFIG_EXPERT=y
    CONFIG_UID16=y
    CONFIG_MULTIUSER=y
    CONFIG_SGETMASK_SYSCALL=y
    CONFIG_SYSFS_SYSCALL=y
    CONFIG_FHANDLE=y
    CONFIG_POSIX_TIMERS=y
    CONFIG_PRINTK=y
    CONFIG_BUG=y
    CONFIG_ELF_CORE=y
    CONFIG_PCSPKR_PLATFORM=y
    CONFIG_BASE_FULL=y
    CONFIG_FUTEX=y
    CONFIG_FUTEX_PI=y
    CONFIG_EPOLL=y
    CONFIG_SIGNALFD=y
    CONFIG_TIMERFD=y
    CONFIG_EVENTFD=y
    CONFIG_SHMEM=y
    CONFIG_AIO=y
    CONFIG_IO_URING=y
    CONFIG_ADVISE_SYSCALLS=y
    CONFIG_HAVE_ARCH_USERFAULTFD_WP=y
    CONFIG_HAVE_ARCH_USERFAULTFD_MINOR=y
    CONFIG_MEMBARRIER=y
    CONFIG_KALLSYMS=y
    CONFIG_KALLSYMS_ALL=y
    CONFIG_KALLSYMS_ABSOLUTE_PERCPU=y
    CONFIG_KALLSYMS_BASE_RELATIVE=y
    CONFIG_USERFAULTFD=y
    CONFIG_ARCH_HAS_MEMBARRIER_SYNC_CORE=y
    CONFIG_KCMP=y
    CONFIG_RSEQ=y
    CONFIG_HAVE_PERF_EVENTS=y
    CONFIG_PC104=y
    
    #
    # Kernel Performance Events And Counters
    #
    CONFIG_PERF_EVENTS=y
    # end of Kernel Performance Events And Counters
    
    CONFIG_VM_EVENT_COUNTERS=y
    CONFIG_SLUB_DEBUG=y
    CONFIG_SLUB=y
    CONFIG_SLAB_MERGE_DEFAULT=y
    CONFIG_SLAB_FREELIST_RANDOM=y
    CONFIG_SLAB_FREELIST_HARDENED=y
    CONFIG_SHUFFLE_PAGE_ALLOCATOR=y
    CONFIG_SLUB_CPU_PARTIAL=y
    CONFIG_SYSTEM_DATA_VERIFICATION=y
    CONFIG_PROFILING=y
    CONFIG_TRACEPOINTS=y
    # end of General setup
    
    CONFIG_64BIT=y
    CONFIG_X86_64=y
    CONFIG_X86=y
    CONFIG_INSTRUCTION_DECODER=y
    CONFIG_OUTPUT_FORMAT="elf64-x86-64"
    CONFIG_LOCKDEP_SUPPORT=y
    CONFIG_STACKTRACE_SUPPORT=y
    CONFIG_MMU=y
    CONFIG_ARCH_MMAP_RND_BITS_MIN=28
    CONFIG_ARCH_MMAP_RND_BITS_MAX=32
    CONFIG_ARCH_MMAP_RND_COMPAT_BITS_MIN=8
    CONFIG_ARCH_MMAP_RND_COMPAT_BITS_MAX=16
    CONFIG_GENERIC_ISA_DMA=y
    CONFIG_GENERIC_BUG=y
    CONFIG_GENERIC_BUG_RELATIVE_POINTERS=y
    CONFIG_ARCH_MAY_HAVE_PC_FDC=y
    CONFIG_GENERIC_CALIBRATE_DELAY=y
    CONFIG_ARCH_HAS_CPU_RELAX=y
    CONFIG_ARCH_HAS_FILTER_PGPROT=y
    CONFIG_HAVE_SETUP_PER_CPU_AREA=y
    CONFIG_NEED_PER_CPU_EMBED_FIRST_CHUNK=y
    CONFIG_NEED_PER_CPU_PAGE_FIRST_CHUNK=y
    CONFIG_ARCH_HIBERNATION_POSSIBLE=y
    CONFIG_ARCH_NR_GPIO=1024
    CONFIG_ARCH_SUSPEND_POSSIBLE=y
    CONFIG_ARCH_WANT_GENERAL_HUGETLB=y
    CONFIG_AUDIT_ARCH=y
    CONFIG_HAVE_INTEL_TXT=y
    CONFIG_X86_64_SMP=y
    CONFIG_ARCH_SUPPORTS_UPROBES=y
    CONFIG_FIX_EARLYCON_MEM=y
    CONFIG_DYNAMIC_PHYSICAL_MASK=y
    CONFIG_PGTABLE_LEVELS=5
    CONFIG_CC_HAS_SANE_STACKPROTECTOR=y
    
    #
    # Processor type and features
    #
    CONFIG_SMP=y
    CONFIG_X86_FEATURE_NAMES=y
    CONFIG_X86_X2APIC=y
    CONFIG_X86_MPPARSE=y
    CONFIG_X86_CPU_RESCTRL=y
    CONFIG_X86_EXTENDED_PLATFORM=y
    CONFIG_X86_NUMACHIP=y
    CONFIG_X86_UV=y
    CONFIG_X86_INTEL_LPSS=y
    CONFIG_X86_AMD_PLATFORM_DEVICE=y
    CONFIG_IOSF_MBI=y
    CONFIG_IOSF_MBI_DEBUG=y
    CONFIG_X86_SUPPORTS_MEMORY_FAILURE=y
    CONFIG_SCHED_OMIT_FRAME_POINTER=y
    CONFIG_HYPERVISOR_GUEST=y
    CONFIG_PARAVIRT=y
    CONFIG_PARAVIRT_XXL=y
    CONFIG_PARAVIRT_SPINLOCKS=y
    CONFIG_X86_HV_CALLBACK_VECTOR=y
    CONFIG_XEN=y
    CONFIG_XEN_PV=y
    CONFIG_XEN_512GB=y
    CONFIG_XEN_PV_SMP=y
    CONFIG_XEN_PV_DOM0=y
    CONFIG_XEN_PVHVM=y
    CONFIG_XEN_PVHVM_SMP=y
    CONFIG_XEN_PVHVM_GUEST=y
    CONFIG_XEN_SAVE_RESTORE=y
    CONFIG_XEN_PVH=y
    CONFIG_XEN_DOM0=y
    CONFIG_KVM_GUEST=y
    CONFIG_ARCH_CPUIDLE_HALTPOLL=y
    CONFIG_PVH=y
    CONFIG_PARAVIRT_CLOCK=y
    CONFIG_JAILHOUSE_GUEST=y
    CONFIG_ACRN_GUEST=y
    CONFIG_GENERIC_CPU=y
    CONFIG_X86_INTERNODE_CACHE_SHIFT=6
    CONFIG_X86_L1_CACHE_SHIFT=6
    CONFIG_X86_TSC=y
    CONFIG_X86_CMPXCHG64=y
    CONFIG_X86_CMOV=y
    CONFIG_X86_MINIMUM_CPU_FAMILY=64
    CONFIG_X86_DEBUGCTLMSR=y
    CONFIG_IA32_FEAT_CTL=y
    CONFIG_X86_VMX_FEATURE_NAMES=y
    CONFIG_PROCESSOR_SELECT=y
    CONFIG_CPU_SUP_INTEL=y
    CONFIG_CPU_SUP_AMD=y
    CONFIG_CPU_SUP_HYGON=y
    CONFIG_CPU_SUP_CENTAUR=y
    CONFIG_CPU_SUP_ZHAOXIN=y
    CONFIG_HPET_TIMER=y
    CONFIG_HPET_EMULATE_RTC=y
    CONFIG_DMI=y
    CONFIG_GART_IOMMU=y
    CONFIG_MAXSMP=y
    CONFIG_NR_CPUS_RANGE_BEGIN=8192
    CONFIG_NR_CPUS_RANGE_END=8192
    CONFIG_NR_CPUS_DEFAULT=8192
    CONFIG_NR_CPUS=8192
    CONFIG_SCHED_SMT=y
    CONFIG_SCHED_MC=y
    CONFIG_SCHED_MC_PRIO=y
    CONFIG_X86_LOCAL_APIC=y
    CONFIG_X86_IO_APIC=y
    CONFIG_X86_REROUTE_FOR_BROKEN_BOOT_IRQS=y
    CONFIG_X86_MCE=y
    CONFIG_X86_MCELOG_LEGACY=y
    CONFIG_X86_MCE_INTEL=y
    CONFIG_X86_MCE_AMD=y
    CONFIG_X86_MCE_THRESHOLD=y
    
    #
    # Performance monitoring
    #
    CONFIG_PERF_EVENTS_INTEL_UNCORE=y
    CONFIG_PERF_EVENTS_INTEL_RAPL=m
    CONFIG_PERF_EVENTS_INTEL_CSTATE=m
    # end of Performance monitoring
    
    CONFIG_X86_16BIT=y
    CONFIG_X86_ESPFIX64=y
    CONFIG_X86_VSYSCALL_EMULATION=y
    CONFIG_X86_IOPL_IOPERM=y
    CONFIG_MICROCODE=y
    CONFIG_MICROCODE_INTEL=y
    CONFIG_MICROCODE_AMD=y
    CONFIG_X86_MSR=m
    CONFIG_X86_5LEVEL=y
    CONFIG_X86_DIRECT_GBPAGES=y
    CONFIG_AMD_MEM_ENCRYPT=y
    CONFIG_NUMA=y
    CONFIG_AMD_NUMA=y
    CONFIG_X86_64_ACPI_NUMA=y
    CONFIG_NODES_SHIFT=10
    CONFIG_ARCH_SPARSEMEM_ENABLE=y
    CONFIG_ARCH_SPARSEMEM_DEFAULT=y
    CONFIG_ARCH_SELECT_MEMORY_MODEL=y
    CONFIG_ARCH_MEMORY_PROBE=y
    CONFIG_ARCH_PROC_KCORE_TEXT=y
    CONFIG_ILLEGAL_POINTER_VALUE=0xdead000000000000
    CONFIG_X86_PMEM_LEGACY_DEVICE=y
    CONFIG_X86_PMEM_LEGACY=y
    CONFIG_X86_CHECK_BIOS_CORRUPTION=y
    CONFIG_X86_BOOTPARAM_MEMORY_CORRUPTION_CHECK=y
    CONFIG_MTRR=y
    CONFIG_MTRR_SANITIZER=y
    CONFIG_MTRR_SANITIZER_ENABLE_DEFAULT=1
    CONFIG_MTRR_SANITIZER_SPARE_REG_NR_DEFAULT=1
    CONFIG_X86_PAT=y
    CONFIG_ARCH_USES_PG_UNCACHED=y
    CONFIG_ARCH_RANDOM=y
    CONFIG_X86_SMAP=y
    CONFIG_X86_UMIP=y
    CONFIG_X86_INTEL_MEMORY_PROTECTION_KEYS=y
    CONFIG_X86_INTEL_TSX_MODE_OFF=y
    CONFIG_X86_SGX=y
    CONFIG_EFI=y
    CONFIG_EFI_STUB=y
    CONFIG_EFI_MIXED=y
    CONFIG_HZ_250=y
    CONFIG_HZ=250
    CONFIG_SCHED_HRTICK=y
    CONFIG_KEXEC=y
    CONFIG_KEXEC_FILE=y
    CONFIG_ARCH_HAS_KEXEC_PURGATORY=y
    CONFIG_KEXEC_SIG=y
    CONFIG_KEXEC_BZIMAGE_VERIFY_SIG=y
    CONFIG_CRASH_DUMP=y
    CONFIG_KEXEC_JUMP=y
    CONFIG_PHYSICAL_START=0x1000000
    CONFIG_RELOCATABLE=y
    CONFIG_RANDOMIZE_BASE=y
    CONFIG_X86_NEED_RELOCS=y
    CONFIG_PHYSICAL_ALIGN=0x200000
    CONFIG_DYNAMIC_MEMORY_LAYOUT=y
    CONFIG_RANDOMIZE_MEMORY=y
    CONFIG_RANDOMIZE_MEMORY_PHYSICAL_PADDING=0xa
    CONFIG_HOTPLUG_CPU=y
    CONFIG_LEGACY_VSYSCALL_XONLY=y
    CONFIG_MODIFY_LDT_SYSCALL=y
    CONFIG_HAVE_LIVEPATCH=y
    CONFIG_LIVEPATCH=y
    # end of Processor type and features
    
    CONFIG_CC_HAS_SLS=y
    CONFIG_CC_HAS_RETURN_THUNK=y
    CONFIG_SPECULATION_MITIGATIONS=y
    CONFIG_PAGE_TABLE_ISOLATION=y
    CONFIG_RETPOLINE=y
    CONFIG_RETHUNK=y
    CONFIG_CPU_UNRET_ENTRY=y
    CONFIG_CPU_IBPB_ENTRY=y
    CONFIG_CPU_IBRS_ENTRY=y
    CONFIG_SLS=y
    CONFIG_ARCH_HAS_ADD_PAGES=y
    CONFIG_ARCH_MHP_MEMMAP_ON_MEMORY_ENABLE=y
    CONFIG_USE_PERCPU_NUMA_NODE_ID=y
    
    #
    # Power management and ACPI options
    #
    CONFIG_ARCH_HIBERNATION_HEADER=y
    CONFIG_SUSPEND=y
    CONFIG_SUSPEND_FREEZER=y
    CONFIG_HIBERNATE_CALLBACKS=y
    CONFIG_HIBERNATION=y
    CONFIG_HIBERNATION_SNAPSHOT_DEV=y
    CONFIG_PM_STD_PARTITION=""
    CONFIG_PM_SLEEP=y
    CONFIG_PM_SLEEP_SMP=y
    CONFIG_PM_WAKELOCKS=y
    CONFIG_PM_WAKELOCKS_LIMIT=100
    CONFIG_PM_WAKELOCKS_GC=y
    CONFIG_PM=y
    CONFIG_PM_DEBUG=y
    CONFIG_PM_ADVANCED_DEBUG=y
    CONFIG_PM_SLEEP_DEBUG=y
    CONFIG_PM_TRACE=y
    CONFIG_PM_TRACE_RTC=y
    CONFIG_PM_CLK=y
    CONFIG_WQ_POWER_EFFICIENT_DEFAULT=y
    CONFIG_ENERGY_MODEL=y
    CONFIG_ARCH_SUPPORTS_ACPI=y
    CONFIG_ACPI=y
    CONFIG_ACPI_LEGACY_TABLES_LOOKUP=y
    CONFIG_ARCH_MIGHT_HAVE_ACPI_PDC=y
    CONFIG_ACPI_SYSTEM_POWER_STATES_SUPPORT=y
    CONFIG_ACPI_DEBUGGER=y
    CONFIG_ACPI_DEBUGGER_USER=y
    CONFIG_ACPI_SPCR_TABLE=y
    CONFIG_ACPI_FPDT=y
    CONFIG_ACPI_LPIT=y
    CONFIG_ACPI_SLEEP=y
    CONFIG_ACPI_REV_OVERRIDE_POSSIBLE=y
    CONFIG_ACPI_AC=y
    CONFIG_ACPI_BATTERY=y
    CONFIG_ACPI_BUTTON=y
    CONFIG_ACPI_FAN=y
    CONFIG_ACPI_DOCK=y
    CONFIG_ACPI_CPU_FREQ_PSS=y
    CONFIG_ACPI_PROCESSOR_CSTATE=y
    CONFIG_ACPI_PROCESSOR_IDLE=y
    CONFIG_ACPI_CPPC_LIB=y
    CONFIG_ACPI_PROCESSOR=y
    CONFIG_ACPI_IPMI=m
    CONFIG_ACPI_HOTPLUG_CPU=y
    CONFIG_ACPI_THERMAL=y
    CONFIG_ACPI_CUSTOM_DSDT_FILE=""
    CONFIG_ARCH_HAS_ACPI_TABLE_UPGRADE=y
    CONFIG_ACPI_TABLE_UPGRADE=y
    CONFIG_ACPI_DEBUG=y
    CONFIG_ACPI_PCI_SLOT=y
    CONFIG_ACPI_CONTAINER=y
    CONFIG_ACPI_HOTPLUG_MEMORY=y
    CONFIG_ACPI_HOTPLUG_IOAPIC=y
    CONFIG_ACPI_HED=y
    CONFIG_ACPI_BGRT=y
    CONFIG_ACPI_NFIT=m
    CONFIG_ACPI_NUMA=y
    CONFIG_ACPI_HMAT=y
    CONFIG_HAVE_ACPI_APEI=y
    CONFIG_HAVE_ACPI_APEI_NMI=y
    CONFIG_ACPI_APEI=y
    CONFIG_ACPI_APEI_GHES=y
    CONFIG_ACPI_APEI_PCIEAER=y
    CONFIG_ACPI_APEI_MEMORY_FAILURE=y
    CONFIG_ACPI_DPTF=y
    CONFIG_ACPI_ADXL=y
    CONFIG_PMIC_OPREGION=y
    CONFIG_BYTCRC_PMIC_OPREGION=y
    CONFIG_CHTCRC_PMIC_OPREGION=y
    CONFIG_CHT_WC_PMIC_OPREGION=y
    CONFIG_ACPI_VIOT=y
    CONFIG_X86_PM_TIMER=y
    CONFIG_ACPI_PRMT=y
    
    #
    # CPU Frequency scaling
    #
    CONFIG_CPU_FREQ=y
    CONFIG_CPU_FREQ_GOV_ATTR_SET=y
    CONFIG_CPU_FREQ_GOV_COMMON=y
    CONFIG_CPU_FREQ_STAT=y
    CONFIG_CPU_FREQ_DEFAULT_GOV_SCHEDUTIL=y
    CONFIG_CPU_FREQ_GOV_PERFORMANCE=y
    CONFIG_CPU_FREQ_GOV_POWERSAVE=y
    CONFIG_CPU_FREQ_GOV_USERSPACE=y
    CONFIG_CPU_FREQ_GOV_ONDEMAND=y
    CONFIG_CPU_FREQ_GOV_CONSERVATIVE=y
    CONFIG_CPU_FREQ_GOV_SCHEDUTIL=y
    
    #
    # CPU frequency scaling drivers
    #
    CONFIG_X86_INTEL_PSTATE=y
    CONFIG_X86_PCC_CPUFREQ=y
    CONFIG_X86_ACPI_CPUFREQ=y
    CONFIG_X86_ACPI_CPUFREQ_CPB=y
    CONFIG_X86_POWERNOW_K8=y
    CONFIG_X86_SPEEDSTEP_CENTRINO=y
    
    #
    # shared options
    #
    # end of CPU Frequency scaling
    
    #
    # CPU Idle
    #
    CONFIG_CPU_IDLE=y
    CONFIG_CPU_IDLE_GOV_LADDER=y
    CONFIG_CPU_IDLE_GOV_MENU=y
    CONFIG_CPU_IDLE_GOV_TEO=y
    CONFIG_CPU_IDLE_GOV_HALTPOLL=y
    # end of CPU Idle
    
    CONFIG_INTEL_IDLE=y
    # end of Power management and ACPI options
    
    #
    # Bus options (PCI etc.)
    #
    CONFIG_PCI_DIRECT=y
    CONFIG_PCI_MMCONFIG=y
    CONFIG_PCI_XEN=y
    CONFIG_MMCONF_FAM10H=y
    CONFIG_ISA_BUS=y
    CONFIG_ISA_DMA_API=y
    CONFIG_AMD_NB=y
    # end of Bus options (PCI etc.)
    
    #
    # Binary Emulations
    #
    CONFIG_IA32_EMULATION=y
    CONFIG_X86_X32=y
    CONFIG_COMPAT_32=y
    CONFIG_COMPAT=y
    CONFIG_COMPAT_FOR_U64_ALIGNMENT=y
    CONFIG_SYSVIPC_COMPAT=y
    # end of Binary Emulations
    
    CONFIG_HAVE_KVM=y
    CONFIG_VIRTUALIZATION=y
    CONFIG_AS_AVX512=y
    CONFIG_AS_SHA1_NI=y
    CONFIG_AS_SHA256_NI=y
    CONFIG_AS_TPAUSE=y
    
    #
    # General architecture-dependent options
    #
    CONFIG_CRASH_CORE=y
    CONFIG_KEXEC_CORE=y
    CONFIG_HOTPLUG_SMT=y
    CONFIG_GENERIC_ENTRY=y
    CONFIG_KPROBES=y
    CONFIG_JUMP_LABEL=y
    CONFIG_OPTPROBES=y
    CONFIG_KPROBES_ON_FTRACE=y
    CONFIG_UPROBES=y
    CONFIG_HAVE_EFFICIENT_UNALIGNED_ACCESS=y
    CONFIG_ARCH_USE_BUILTIN_BSWAP=y
    CONFIG_KRETPROBES=y
    CONFIG_HAVE_IOREMAP_PROT=y
    CONFIG_HAVE_KPROBES=y
    CONFIG_HAVE_KRETPROBES=y
    CONFIG_HAVE_OPTPROBES=y
    CONFIG_HAVE_KPROBES_ON_FTRACE=y
    CONFIG_HAVE_FUNCTION_ERROR_INJECTION=y
    CONFIG_HAVE_NMI=y
    CONFIG_TRACE_IRQFLAGS_SUPPORT=y
    CONFIG_TRACE_IRQFLAGS_NMI_SUPPORT=y
    CONFIG_HAVE_ARCH_TRACEHOOK=y
    CONFIG_HAVE_DMA_CONTIGUOUS=y
    CONFIG_GENERIC_SMP_IDLE_THREAD=y
    CONFIG_ARCH_HAS_FORTIFY_SOURCE=y
    CONFIG_ARCH_HAS_SET_MEMORY=y
    CONFIG_ARCH_HAS_SET_DIRECT_MAP=y
    CONFIG_HAVE_ARCH_THREAD_STRUCT_WHITELIST=y
    CONFIG_ARCH_WANTS_DYNAMIC_TASK_STRUCT=y
    CONFIG_ARCH_WANTS_NO_INSTR=y
    CONFIG_HAVE_ASM_MODVERSIONS=y
    CONFIG_HAVE_REGS_AND_STACK_ACCESS_API=y
    CONFIG_HAVE_RSEQ=y
    CONFIG_HAVE_FUNCTION_ARG_ACCESS_API=y
    CONFIG_HAVE_HW_BREAKPOINT=y
    CONFIG_HAVE_MIXED_BREAKPOINTS_REGS=y
    CONFIG_HAVE_USER_RETURN_NOTIFIER=y
    CONFIG_HAVE_PERF_EVENTS_NMI=y
    CONFIG_HAVE_HARDLOCKUP_DETECTOR_PERF=y
    CONFIG_HAVE_PERF_REGS=y
    CONFIG_HAVE_PERF_USER_STACK_DUMP=y
    CONFIG_HAVE_ARCH_JUMP_LABEL=y
    CONFIG_HAVE_ARCH_JUMP_LABEL_RELATIVE=y
    CONFIG_MMU_GATHER_TABLE_FREE=y
    CONFIG_MMU_GATHER_RCU_TABLE_FREE=y
    CONFIG_ARCH_HAVE_NMI_SAFE_CMPXCHG=y
    CONFIG_HAVE_ALIGNED_STRUCT_PAGE=y
    CONFIG_HAVE_CMPXCHG_LOCAL=y
    CONFIG_HAVE_CMPXCHG_DOUBLE=y
    CONFIG_ARCH_WANT_COMPAT_IPC_PARSE_VERSION=y
    CONFIG_ARCH_WANT_OLD_COMPAT_IPC=y
    CONFIG_HAVE_ARCH_SECCOMP=y
    CONFIG_HAVE_ARCH_SECCOMP_FILTER=y
    CONFIG_SECCOMP=y
    CONFIG_SECCOMP_FILTER=y
    CONFIG_HAVE_ARCH_STACKLEAK=y
    CONFIG_HAVE_STACKPROTECTOR=y
    CONFIG_STACKPROTECTOR=y
    CONFIG_STACKPROTECTOR_STRONG=y
    CONFIG_ARCH_SUPPORTS_LTO_CLANG=y
    CONFIG_ARCH_SUPPORTS_LTO_CLANG_THIN=y
    CONFIG_LTO_NONE=y
    CONFIG_HAVE_ARCH_WITHIN_STACK_FRAMES=y
    CONFIG_HAVE_CONTEXT_TRACKING=y
    CONFIG_HAVE_CONTEXT_TRACKING_OFFSTACK=y
    CONFIG_HAVE_VIRT_CPU_ACCOUNTING_GEN=y
    CONFIG_HAVE_IRQ_TIME_ACCOUNTING=y
    CONFIG_HAVE_MOVE_PUD=y
    CONFIG_HAVE_MOVE_PMD=y
    CONFIG_HAVE_ARCH_TRANSPARENT_HUGEPAGE=y
    CONFIG_HAVE_ARCH_TRANSPARENT_HUGEPAGE_PUD=y
    CONFIG_HAVE_ARCH_HUGE_VMAP=y
    CONFIG_ARCH_WANT_HUGE_PMD_SHARE=y
    CONFIG_HAVE_ARCH_SOFT_DIRTY=y
    CONFIG_HAVE_MOD_ARCH_SPECIFIC=y
    CONFIG_MODULES_USE_ELF_RELA=y
    CONFIG_HAVE_IRQ_EXIT_ON_IRQ_STACK=y
    CONFIG_HAVE_SOFTIRQ_ON_OWN_STACK=y
    CONFIG_ARCH_HAS_ELF_RANDOMIZE=y
    CONFIG_HAVE_ARCH_MMAP_RND_BITS=y
    CONFIG_HAVE_EXIT_THREAD=y
    CONFIG_ARCH_MMAP_RND_BITS=28
    CONFIG_HAVE_ARCH_MMAP_RND_COMPAT_BITS=y
    CONFIG_ARCH_MMAP_RND_COMPAT_BITS=8
    CONFIG_HAVE_ARCH_COMPAT_MMAP_BASES=y
    CONFIG_HAVE_STACK_VALIDATION=y
    CONFIG_HAVE_RELIABLE_STACKTRACE=y
    CONFIG_OLD_SIGSUSPEND3=y
    CONFIG_COMPAT_OLD_SIGACTION=y
    CONFIG_COMPAT_32BIT_TIME=y
    CONFIG_HAVE_ARCH_VMAP_STACK=y
    CONFIG_VMAP_STACK=y
    CONFIG_HAVE_ARCH_RANDOMIZE_KSTACK_OFFSET=y
    CONFIG_RANDOMIZE_KSTACK_OFFSET_DEFAULT=y
    CONFIG_ARCH_HAS_STRICT_KERNEL_RWX=y
    CONFIG_STRICT_KERNEL_RWX=y
    CONFIG_ARCH_HAS_STRICT_MODULE_RWX=y
    CONFIG_STRICT_MODULE_RWX=y
    CONFIG_HAVE_ARCH_PREL32_RELOCATIONS=y
    CONFIG_ARCH_USE_MEMREMAP_PROT=y
    CONFIG_ARCH_HAS_MEM_ENCRYPT=y
    CONFIG_ARCH_HAS_CC_PLATFORM=y
    CONFIG_HAVE_STATIC_CALL=y
    CONFIG_HAVE_STATIC_CALL_INLINE=y
    CONFIG_HAVE_PREEMPT_DYNAMIC=y
    CONFIG_ARCH_WANT_LD_ORPHAN_WARN=y
    CONFIG_ARCH_SUPPORTS_DEBUG_PAGEALLOC=y
    CONFIG_ARCH_HAS_ELFCORE_COMPAT=y
    CONFIG_ARCH_HAS_PARANOID_L1D_FLUSH=y
    
    #
    # GCOV-based kernel profiling
    #
    CONFIG_ARCH_HAS_GCOV_PROFILE_ALL=y
    # end of GCOV-based kernel profiling
    
    CONFIG_HAVE_GCC_PLUGINS=y
    # end of General architecture-dependent options
    
    CONFIG_RT_MUTEXES=y
    CONFIG_BASE_SMALL=0
    CONFIG_MODULE_SIG_FORMAT=y
    CONFIG_MODULES=y
    CONFIG_MODULE_UNLOAD=y
    CONFIG_MODVERSIONS=y
    CONFIG_ASM_MODVERSIONS=y
    CONFIG_MODULE_SRCVERSION_ALL=y
    CONFIG_MODULE_SIG=y
    CONFIG_MODULE_SIG_ALL=y
    CONFIG_MODULE_SIG_SHA512=y
    CONFIG_MODULE_SIG_HASH="sha512"
    CONFIG_MODULE_COMPRESS_NONE=y
    CONFIG_MODPROBE_PATH="/sbin/modprobe"
    CONFIG_MODULES_TREE_LOOKUP=y
    CONFIG_BLOCK=y
    CONFIG_BLK_RQ_ALLOC_TIME=y
    CONFIG_BLK_CGROUP_RWSTAT=y
    CONFIG_BLK_DEV_BSG_COMMON=y
    CONFIG_BLK_DEV_BSGLIB=y
    CONFIG_BLK_DEV_INTEGRITY=y
    CONFIG_BLK_DEV_INTEGRITY_T10=y
    CONFIG_BLK_DEV_ZONED=y
    CONFIG_BLK_DEV_THROTTLING=y
    CONFIG_BLK_WBT=y
    CONFIG_BLK_WBT_MQ=y
    CONFIG_BLK_CGROUP_IOCOST=y
    CONFIG_BLK_CGROUP_IOPRIO=y
    CONFIG_BLK_DEBUG_FS=y
    CONFIG_BLK_DEBUG_FS_ZONED=y
    CONFIG_BLK_SED_OPAL=y
    CONFIG_BLK_INLINE_ENCRYPTION=y
    CONFIG_BLK_INLINE_ENCRYPTION_FALLBACK=y
    
    #
    # Partition Types
    #
    CONFIG_PARTITION_ADVANCED=y
    CONFIG_AIX_PARTITION=y
    CONFIG_OSF_PARTITION=y
    CONFIG_AMIGA_PARTITION=y
    CONFIG_ATARI_PARTITION=y
    CONFIG_MAC_PARTITION=y
    CONFIG_MSDOS_PARTITION=y
    CONFIG_BSD_DISKLABEL=y
    CONFIG_MINIX_SUBPARTITION=y
    CONFIG_SOLARIS_X86_PARTITION=y
    CONFIG_UNIXWARE_DISKLABEL=y
    CONFIG_LDM_PARTITION=y
    CONFIG_SGI_PARTITION=y
    CONFIG_ULTRIX_PARTITION=y
    CONFIG_SUN_PARTITION=y
    CONFIG_KARMA_PARTITION=y
    CONFIG_EFI_PARTITION=y
    CONFIG_SYSV68_PARTITION=y
    CONFIG_CMDLINE_PARTITION=y
    # end of Partition Types
    
    CONFIG_BLOCK_COMPAT=y
    CONFIG_BLK_MQ_PCI=y
    CONFIG_BLK_MQ_VIRTIO=y
    CONFIG_BLK_MQ_RDMA=y
    CONFIG_BLK_PM=y
    CONFIG_BLOCK_HOLDER_DEPRECATED=y
    
    #
    # IO Schedulers
    #
    CONFIG_MQ_IOSCHED_DEADLINE=y
    # end of IO Schedulers
    
    CONFIG_ASN1=y
    CONFIG_INLINE_SPIN_UNLOCK_IRQ=y
    CONFIG_INLINE_READ_UNLOCK=y
    CONFIG_INLINE_READ_UNLOCK_IRQ=y
    CONFIG_INLINE_WRITE_UNLOCK=y
    CONFIG_INLINE_WRITE_UNLOCK_IRQ=y
    CONFIG_ARCH_SUPPORTS_ATOMIC_RMW=y
    CONFIG_MUTEX_SPIN_ON_OWNER=y
    CONFIG_RWSEM_SPIN_ON_OWNER=y
    CONFIG_LOCK_SPIN_ON_OWNER=y
    CONFIG_ARCH_USE_QUEUED_SPINLOCKS=y
    CONFIG_QUEUED_SPINLOCKS=y
    CONFIG_ARCH_USE_QUEUED_RWLOCKS=y
    CONFIG_QUEUED_RWLOCKS=y
    CONFIG_ARCH_HAS_NON_OVERLAPPING_ADDRESS_SPACE=y
    CONFIG_ARCH_HAS_SYNC_CORE_BEFORE_USERMODE=y
    CONFIG_ARCH_HAS_SYSCALL_WRAPPER=y
    CONFIG_FREEZER=y
    
    #
    # Executable file formats
    #
    CONFIG_BINFMT_ELF=y
    CONFIG_COMPAT_BINFMT_ELF=y
    CONFIG_ELFCORE=y
    CONFIG_CORE_DUMP_DEFAULT_ELF_HEADERS=y
    CONFIG_BINFMT_SCRIPT=y
    CONFIG_BINFMT_MISC=m
    CONFIG_COREDUMP=y
    # end of Executable file formats
    
    #
    # Memory Management options
    #
    CONFIG_SELECT_MEMORY_MODEL=y
    CONFIG_SPARSEMEM_MANUAL=y
    CONFIG_SPARSEMEM=y
    CONFIG_SPARSEMEM_EXTREME=y
    CONFIG_SPARSEMEM_VMEMMAP_ENABLE=y
    CONFIG_SPARSEMEM_VMEMMAP=y
    CONFIG_HAVE_FAST_GUP=y
    CONFIG_NUMA_KEEP_MEMINFO=y
    CONFIG_MEMORY_ISOLATION=y
    CONFIG_HAVE_BOOTMEM_INFO_NODE=y
    CONFIG_ARCH_ENABLE_MEMORY_HOTPLUG=y
    CONFIG_MEMORY_HOTPLUG=y
    CONFIG_MEMORY_HOTPLUG_SPARSE=y
    CONFIG_MEMORY_HOTPLUG_DEFAULT_ONLINE=y
    CONFIG_ARCH_ENABLE_MEMORY_HOTREMOVE=y
    CONFIG_MEMORY_HOTREMOVE=y
    CONFIG_MHP_MEMMAP_ON_MEMORY=y
    CONFIG_SPLIT_PTLOCK_CPUS=4
    CONFIG_ARCH_ENABLE_SPLIT_PMD_PTLOCK=y
    CONFIG_MEMORY_BALLOON=y
    CONFIG_BALLOON_COMPACTION=y
    CONFIG_COMPACTION=y
    CONFIG_PAGE_REPORTING=y
    CONFIG_MIGRATION=y
    CONFIG_ARCH_ENABLE_HUGEPAGE_MIGRATION=y
    CONFIG_ARCH_ENABLE_THP_MIGRATION=y
    CONFIG_CONTIG_ALLOC=y
    CONFIG_PHYS_ADDR_T_64BIT=y
    CONFIG_VIRT_TO_BUS=y
    CONFIG_MMU_NOTIFIER=y
    CONFIG_KSM=y
    CONFIG_DEFAULT_MMAP_MIN_ADDR=65536
    CONFIG_ARCH_SUPPORTS_MEMORY_FAILURE=y
    CONFIG_MEMORY_FAILURE=y
    CONFIG_TRANSPARENT_HUGEPAGE=y
    CONFIG_TRANSPARENT_HUGEPAGE_MADVISE=y
    CONFIG_ARCH_WANTS_THP_SWAP=y
    CONFIG_THP_SWAP=y
    CONFIG_CLEANCACHE=y
    CONFIG_FRONTSWAP=y
    CONFIG_MEM_SOFT_DIRTY=y
    CONFIG_ZSWAP=y
    CONFIG_ZSWAP_COMPRESSOR_DEFAULT_LZO=y
    CONFIG_ZSWAP_COMPRESSOR_DEFAULT="lzo"
    CONFIG_ZSWAP_ZPOOL_DEFAULT_ZBUD=y
    CONFIG_ZSWAP_ZPOOL_DEFAULT="zbud"
    CONFIG_ZPOOL=y
    CONFIG_ZBUD=y
    CONFIG_ZSMALLOC=y
    CONFIG_GENERIC_EARLY_IOREMAP=y
    CONFIG_PAGE_IDLE_FLAG=y
    CONFIG_IDLE_PAGE_TRACKING=y
    CONFIG_ARCH_HAS_CACHE_LINE_SIZE=y
    CONFIG_ARCH_HAS_PTE_DEVMAP=y
    CONFIG_ARCH_HAS_ZONE_DMA_SET=y
    CONFIG_ZONE_DMA=y
    CONFIG_ZONE_DMA32=y
    CONFIG_ZONE_DEVICE=y
    CONFIG_DEV_PAGEMAP_OPS=y
    CONFIG_HMM_MIRROR=y
    CONFIG_DEVICE_PRIVATE=y
    CONFIG_ARCH_USES_HIGH_VMA_FLAGS=y
    CONFIG_ARCH_HAS_PKEYS=y
    CONFIG_ARCH_HAS_PTE_SPECIAL=y
    CONFIG_SECRETMEM=y
    
    #
    # Data Access Monitoring
    #
    # end of Data Access Monitoring
    # end of Memory Management options
    
    CONFIG_NET=y
    CONFIG_NET_INGRESS=y
    CONFIG_SKB_EXTENSIONS=y
    
    #
    # Networking options
    #
    CONFIG_PACKET=y
    CONFIG_UNIX=y
    CONFIG_UNIX_SCM=y
    CONFIG_AF_UNIX_OOB=y
    CONFIG_XDP_SOCKETS=y
    CONFIG_INET=y
    CONFIG_IP_MULTICAST=y
    CONFIG_IP_ADVANCED_ROUTER=y
    CONFIG_IP_FIB_TRIE_STATS=y
    CONFIG_IP_MULTIPLE_TABLES=y
    CONFIG_IP_ROUTE_MULTIPATH=y
    CONFIG_IP_ROUTE_VERBOSE=y
    CONFIG_IP_MROUTE_COMMON=y
    CONFIG_IP_MROUTE=y
    CONFIG_IP_MROUTE_MULTIPLE_TABLES=y
    CONFIG_IP_PIMSM_V1=y
    CONFIG_IP_PIMSM_V2=y
    CONFIG_SYN_COOKIES=y
    CONFIG_INET_TABLE_PERTURB_ORDER=16
    CONFIG_TCP_CONG_ADVANCED=y
    CONFIG_TCP_CONG_CUBIC=y
    CONFIG_DEFAULT_CUBIC=y
    CONFIG_DEFAULT_TCP_CONG="cubic"
    CONFIG_TCP_MD5SIG=y
    CONFIG_IPV6=y
    CONFIG_IPV6_ROUTER_PREF=y
    CONFIG_IPV6_ROUTE_INFO=y
    CONFIG_IPV6_MULTIPLE_TABLES=y
    CONFIG_IPV6_SUBTREES=y
    CONFIG_IPV6_MROUTE=y
    CONFIG_IPV6_MROUTE_MULTIPLE_TABLES=y
    CONFIG_IPV6_PIMSM_V2=y
    CONFIG_IPV6_SEG6_LWTUNNEL=y
    CONFIG_IPV6_SEG6_HMAC=y
    CONFIG_IPV6_SEG6_BPF=y
    CONFIG_IPV6_IOAM6_LWTUNNEL=y
    CONFIG_NETLABEL=y
    CONFIG_MPTCP=y
    CONFIG_MPTCP_IPV6=y
    CONFIG_NETWORK_SECMARK=y
    CONFIG_NET_PTP_CLASSIFY=y
    CONFIG_NETWORK_PHY_TIMESTAMPING=y
    CONFIG_NETFILTER=y
    CONFIG_NETFILTER_ADVANCED=y
    
    #
    # Core Netfilter Configuration
    #
    CONFIG_NETFILTER_INGRESS=y
    CONFIG_NETFILTER_NETLINK=y
    CONFIG_NETFILTER_NETLINK_HOOK=y
    CONFIG_NETFILTER_NETLINK_ACCT=y
    CONFIG_NETFILTER_NETLINK_QUEUE=y
    CONFIG_NETFILTER_NETLINK_LOG=y
    CONFIG_NETFILTER_NETLINK_OSF=y
    CONFIG_NF_CONNTRACK=y
    CONFIG_NF_LOG_SYSLOG=y
    CONFIG_NETFILTER_CONNCOUNT=y
    CONFIG_NF_CONNTRACK_MARK=y
    CONFIG_NF_CONNTRACK_SECMARK=y
    CONFIG_NF_CONNTRACK_ZONES=y
    CONFIG_NF_CONNTRACK_PROCFS=y
    CONFIG_NF_CONNTRACK_EVENTS=y
    CONFIG_NF_CONNTRACK_TIMEOUT=y
    CONFIG_NF_CONNTRACK_TIMESTAMP=y
    CONFIG_NF_CONNTRACK_LABELS=y
    CONFIG_NF_CT_PROTO_DCCP=y
    CONFIG_NF_CT_PROTO_SCTP=y
    CONFIG_NF_CT_PROTO_UDPLITE=y
    CONFIG_NF_NAT=y
    CONFIG_NF_NAT_REDIRECT=y
    CONFIG_NF_NAT_MASQUERADE=y
    CONFIG_NETFILTER_SYNPROXY=y
    CONFIG_NF_TABLES=y
    CONFIG_NF_TABLES_INET=y
    CONFIG_NF_TABLES_NETDEV=y
    CONFIG_NFT_NUMGEN=y
    CONFIG_NFT_CT=y
    CONFIG_NFT_COUNTER=y
    CONFIG_NFT_CONNLIMIT=y
    CONFIG_NFT_LOG=y
    CONFIG_NFT_LIMIT=y
    CONFIG_NFT_MASQ=y
    CONFIG_NFT_REDIR=y
    CONFIG_NFT_NAT=y
    CONFIG_NFT_TUNNEL=y
    CONFIG_NFT_OBJREF=y
    CONFIG_NFT_QUEUE=y
    CONFIG_NFT_QUOTA=y
    CONFIG_NFT_REJECT=y
    CONFIG_NFT_REJECT_INET=y
    CONFIG_NFT_COMPAT=m
    CONFIG_NFT_HASH=y
    CONFIG_NFT_SOCKET=y
    CONFIG_NFT_OSF=y
    CONFIG_NFT_TPROXY=y
    CONFIG_NFT_SYNPROXY=y
    CONFIG_NF_DUP_NETDEV=y
    CONFIG_NFT_DUP_NETDEV=y
    CONFIG_NFT_FWD_NETDEV=y
    CONFIG_NFT_REJECT_NETDEV=y
    CONFIG_NF_FLOW_TABLE=y
    CONFIG_NETFILTER_XTABLES=y
    CONFIG_NETFILTER_XTABLES_COMPAT=y
    
    #
    # Xtables combined modules
    #
    CONFIG_NETFILTER_XT_MARK=m
    
    #
    # Xtables targets
    #
    CONFIG_NETFILTER_XT_TARGET_HL=y
    CONFIG_NETFILTER_XT_NAT=y
    CONFIG_NETFILTER_XT_TARGET_NETMAP=m
    CONFIG_NETFILTER_XT_TARGET_REDIRECT=m
    CONFIG_NETFILTER_XT_TARGET_MASQUERADE=y
    
    #
    # Xtables matches
    #
    CONFIG_NETFILTER_XT_MATCH_HL=y
    # end of Core Netfilter Configuration
    
    
    #
    # IP: Netfilter Configuration
    #
    CONFIG_NF_DEFRAG_IPV4=y
    CONFIG_NF_SOCKET_IPV4=y
    CONFIG_NF_TPROXY_IPV4=y
    CONFIG_NF_TABLES_IPV4=y
    CONFIG_NFT_REJECT_IPV4=y
    CONFIG_NF_REJECT_IPV4=y
    CONFIG_IP_NF_IPTABLES=m
    CONFIG_IP_NF_FILTER=m
    CONFIG_IP_NF_NAT=m
    CONFIG_IP_NF_TARGET_MASQUERADE=m
    CONFIG_IP_NF_TARGET_NETMAP=m
    CONFIG_IP_NF_TARGET_REDIRECT=m
    # end of IP: Netfilter Configuration
    
    #
    # IPv6: Netfilter Configuration
    #
    CONFIG_NF_SOCKET_IPV6=y
    CONFIG_NF_TPROXY_IPV6=y
    CONFIG_NF_TABLES_IPV6=y
    CONFIG_NFT_REJECT_IPV6=y
    CONFIG_NF_REJECT_IPV6=y
    CONFIG_NF_LOG_IPV6=y
    CONFIG_IP6_NF_IPTABLES=y
    CONFIG_IP6_NF_MATCH_AH=y
    CONFIG_IP6_NF_MATCH_EUI64=y
    CONFIG_IP6_NF_MATCH_FRAG=y
    CONFIG_IP6_NF_MATCH_OPTS=y
    CONFIG_IP6_NF_MATCH_HL=y
    CONFIG_IP6_NF_MATCH_IPV6HEADER=y
    CONFIG_IP6_NF_MATCH_MH=y
    CONFIG_IP6_NF_MATCH_RPFILTER=y
    CONFIG_IP6_NF_MATCH_RT=y
    CONFIG_IP6_NF_MATCH_SRH=y
    CONFIG_IP6_NF_TARGET_HL=y
    CONFIG_IP6_NF_FILTER=y
    CONFIG_IP6_NF_TARGET_REJECT=y
    CONFIG_IP6_NF_TARGET_SYNPROXY=y
    CONFIG_IP6_NF_MANGLE=y
    CONFIG_IP6_NF_RAW=y
    CONFIG_IP6_NF_SECURITY=y
    CONFIG_IP6_NF_NAT=y
    CONFIG_IP6_NF_TARGET_MASQUERADE=y
    CONFIG_IP6_NF_TARGET_NPT=y
    # end of IPv6: Netfilter Configuration
    
    CONFIG_NF_DEFRAG_IPV6=y
    CONFIG_BPFILTER=y
    CONFIG_NET_DSA=y
    CONFIG_NET_DSA_TAG_OCELOT_8021Q=y
    CONFIG_VLAN_8021Q=y
    CONFIG_NET_SCHED=y
    
    #
    # Queueing/Scheduling
    #
    CONFIG_NET_SCH_FQ_CODEL=m
    
    #
    # Classification
    #
    CONFIG_NET_CLS=y
    CONFIG_NET_EMATCH=y
    CONFIG_NET_EMATCH_STACK=32
    CONFIG_NET_CLS_ACT=y
    CONFIG_NET_ACT_NAT=y
    CONFIG_NET_TC_SKB_EXT=y
    CONFIG_NET_SCH_FIFO=y
    CONFIG_DCB=y
    CONFIG_DNS_RESOLVER=y
    CONFIG_MPLS=y
    CONFIG_NET_SWITCHDEV=y
    CONFIG_NET_L3_MASTER_DEV=y
    CONFIG_NET_NCSI=y
    CONFIG_NCSI_OEM_CMD_GET_MAC=y
    CONFIG_PCPU_DEV_REFCNT=y
    CONFIG_RPS=y
    CONFIG_RFS_ACCEL=y
    CONFIG_SOCK_RX_QUEUE_MAPPING=y
    CONFIG_XPS=y
    CONFIG_CGROUP_NET_PRIO=y
    CONFIG_CGROUP_NET_CLASSID=y
    CONFIG_NET_RX_BUSY_POLL=y
    CONFIG_BQL=y
    CONFIG_BPF_STREAM_PARSER=y
    CONFIG_NET_FLOW_LIMIT=y
    
    #
    # Network testing
    #
    CONFIG_NET_DROP_MONITOR=y
    # end of Network testing
    # end of Networking options
    
    CONFIG_HAMRADIO=y
    
    #
    # Packet Radio protocols
    #
    CONFIG_STREAM_PARSER=y
    CONFIG_FIB_RULES=y
    CONFIG_WIRELESS=y
    
    #
    # CFG80211 needs to be enabled for MAC80211
    #
    CONFIG_MAC80211_STA_HASH_MAX_SIZE=0
    CONFIG_RFKILL=y
    CONFIG_RFKILL_LEDS=y
    CONFIG_RFKILL_INPUT=y
    CONFIG_LWTUNNEL=y
    CONFIG_LWTUNNEL_BPF=y
    CONFIG_DST_CACHE=y
    CONFIG_GRO_CELLS=y
    CONFIG_NET_SELFTESTS=y
    CONFIG_NET_SOCK_MSG=y
    CONFIG_NET_DEVLINK=y
    CONFIG_PAGE_POOL=y
    CONFIG_ETHTOOL_NETLINK=y
    
    #
    # Device Drivers
    #
    CONFIG_HAVE_EISA=y
    CONFIG_EISA=y
    CONFIG_EISA_VLB_PRIMING=y
    CONFIG_EISA_PCI_EISA=y
    CONFIG_EISA_VIRTUAL_ROOT=y
    CONFIG_EISA_NAMES=y
    CONFIG_HAVE_PCI=y
    CONFIG_PCI=y
    CONFIG_PCI_DOMAINS=y
    CONFIG_PCIEPORTBUS=y
    CONFIG_HOTPLUG_PCI_PCIE=y
    CONFIG_PCIEAER=y
    CONFIG_PCIEASPM=y
    CONFIG_PCIEASPM_DEFAULT=y
    CONFIG_PCIE_PME=y
    CONFIG_PCIE_DPC=y
    CONFIG_PCIE_PTM=y
    CONFIG_PCIE_EDR=y
    CONFIG_PCI_MSI=y
    CONFIG_PCI_MSI_IRQ_DOMAIN=y
    CONFIG_PCI_QUIRKS=y
    CONFIG_PCI_REALLOC_ENABLE_AUTO=y
    CONFIG_PCI_ATS=y
    CONFIG_PCI_LOCKLESS_CONFIG=y
    CONFIG_PCI_IOV=y
    CONFIG_PCI_PRI=y
    CONFIG_PCI_PASID=y
    CONFIG_PCI_LABEL=y
    CONFIG_PCIE_BUS_DEFAULT=y
    CONFIG_HOTPLUG_PCI=y
    CONFIG_HOTPLUG_PCI_ACPI=y
    CONFIG_HOTPLUG_PCI_CPCI=y
    CONFIG_HOTPLUG_PCI_SHPC=y
    
    #
    # PCI controller drivers
    #
    
    #
    # DesignWare PCI Core Support
    #
    CONFIG_PCIE_DW=y
    CONFIG_PCIE_DW_HOST=y
    CONFIG_PCIE_DW_EP=y
    CONFIG_PCIE_DW_PLAT=y
    CONFIG_PCIE_DW_PLAT_HOST=y
    CONFIG_PCIE_DW_PLAT_EP=y
    # end of DesignWare PCI Core Support
    
    #
    # Mobiveil PCIe Core Support
    #
    # end of Mobiveil PCIe Core Support
    
    #
    # Cadence PCIe controllers support
    #
    # end of Cadence PCIe controllers support
    # end of PCI controller drivers
    
    #
    # PCI Endpoint
    #
    CONFIG_PCI_ENDPOINT=y
    CONFIG_PCI_ENDPOINT_CONFIGFS=y
    # end of PCI Endpoint
    
    #
    # PCI switch controller drivers
    #
    # end of PCI switch controller drivers
    
    CONFIG_RAPIDIO=y
    CONFIG_RAPIDIO_DISC_TIMEOUT=30
    CONFIG_RAPIDIO_DMA_ENGINE=y
    
    #
    # RapidIO Switch drivers
    #
    # end of RapidIO Switch drivers
    
    #
    # Generic Driver Options
    #
    CONFIG_AUXILIARY_BUS=y
    CONFIG_UEVENT_HELPER=y
    CONFIG_UEVENT_HELPER_PATH=""
    CONFIG_DEVTMPFS=y
    CONFIG_DEVTMPFS_MOUNT=y
    CONFIG_PREVENT_FIRMWARE_BUILD=y
    
    #
    # Firmware loader
    #
    CONFIG_FW_LOADER=y
    CONFIG_FW_LOADER_PAGED_BUF=y
    CONFIG_EXTRA_FIRMWARE=""
    CONFIG_FW_LOADER_USER_HELPER=y
    CONFIG_FW_LOADER_COMPRESS=y
    CONFIG_FW_CACHE=y
    # end of Firmware loader
    
    CONFIG_WANT_DEV_COREDUMP=y
    CONFIG_ALLOW_DEV_COREDUMP=y
    CONFIG_DEV_COREDUMP=y
    CONFIG_HMEM_REPORTING=y
    CONFIG_SYS_HYPERVISOR=y
    CONFIG_GENERIC_CPU_AUTOPROBE=y
    CONFIG_GENERIC_CPU_VULNERABILITIES=y
    CONFIG_REGMAP=y
    CONFIG_REGMAP_I2C=y
    CONFIG_REGMAP_SPI=y
    CONFIG_REGMAP_MMIO=y
    CONFIG_REGMAP_IRQ=y
    CONFIG_DMA_SHARED_BUFFER=y
    # end of Generic Driver Options
    
    #
    # Bus devices
    #
    # end of Bus devices
    
    CONFIG_CONNECTOR=y
    CONFIG_PROC_EVENTS=y
    
    #
    # Firmware Drivers
    #
    
    #
    # ARM System Control and Management Interface Protocol
    #
    # end of ARM System Control and Management Interface Protocol
    
    CONFIG_EDD=y
    CONFIG_EDD_OFF=y
    CONFIG_FIRMWARE_MEMMAP=y
    CONFIG_DMIID=y
    CONFIG_DMI_SCAN_MACHINE_NON_EFI_FALLBACK=y
    CONFIG_SYSFB=y
    
    #
    # EFI (Extensible Firmware Interface) Support
    #
    CONFIG_EFI_VARS=y
    CONFIG_EFI_ESRT=y
    CONFIG_EFI_VARS_PSTORE=m
    CONFIG_EFI_RUNTIME_MAP=y
    CONFIG_EFI_SOFT_RESERVE=y
    CONFIG_EFI_RUNTIME_WRAPPERS=y
    CONFIG_EFI_GENERIC_STUB_INITRD_CMDLINE_LOADER=y
    CONFIG_APPLE_PROPERTIES=y
    CONFIG_RESET_ATTACK_MITIGATION=y
    CONFIG_EFI_RCI2_TABLE=y
    # end of EFI (Extensible Firmware Interface) Support
    
    CONFIG_UEFI_CPER=y
    CONFIG_UEFI_CPER_X86=y
    CONFIG_EFI_DEV_PATH_PARSER=y
    CONFIG_EFI_EARLYCON=y
    CONFIG_EFI_CUSTOM_SSDT_OVERLAYS=y
    
    #
    # Tegra firmware driver
    #
    # end of Tegra firmware driver
    # end of Firmware Drivers
    
    CONFIG_ARCH_MIGHT_HAVE_PC_PARPORT=y
    CONFIG_PNP=y
    
    #
    # Protocols
    #
    CONFIG_PNPACPI=y
    CONFIG_BLK_DEV=y
    CONFIG_CDROM=y
    CONFIG_BLK_DEV_LOOP=y
    CONFIG_BLK_DEV_LOOP_MIN_COUNT=8
    CONFIG_XEN_BLKDEV_FRONTEND=y
    
    #
    # NVME Support
    #
    # end of NVME Support
    
    #
    # Misc devices
    #
    CONFIG_SRAM=y
    
    #
    # EEPROM support
    #
    # end of EEPROM support
    
    
    #
    # Texas Instruments shared transport line discipline
    #
    # end of Texas Instruments shared transport line discipline
    
    CONFIG_INTEL_MEI=m
    CONFIG_INTEL_MEI_ME=m
    CONFIG_PVPANIC=y
    # end of Misc devices
    
    #
    # SCSI device support
    #
    CONFIG_SCSI_MOD=y
    CONFIG_SCSI_COMMON=y
    CONFIG_SCSI=y
    CONFIG_SCSI_DMA=y
    CONFIG_SCSI_PROC_FS=y
    
    #
    # SCSI support type (disk, tape, CD-ROM)
    #
    CONFIG_BLK_DEV_SD=y
    CONFIG_BLK_DEV_SR=y
    CONFIG_CHR_DEV_SG=y
    CONFIG_BLK_DEV_BSG=y
    CONFIG_SCSI_CONSTANTS=y
    CONFIG_SCSI_LOGGING=y
    CONFIG_SCSI_SCAN_ASYNC=y
    
    #
    # SCSI Transports
    #
    # end of SCSI Transports
    
    CONFIG_SCSI_LOWLEVEL=y
    CONFIG_MEGARAID_NEWGEN=y
    CONFIG_MEGARAID_SAS=m
    CONFIG_SCSI_DH=y
    CONFIG_SCSI_DH_RDAC=m
    CONFIG_SCSI_DH_EMC=m
    CONFIG_SCSI_DH_ALUA=m
    # end of SCSI device support
    
    CONFIG_ATA=y
    CONFIG_SATA_HOST=y
    CONFIG_PATA_TIMINGS=y
    CONFIG_ATA_VERBOSE_ERROR=y
    CONFIG_ATA_FORCE=y
    CONFIG_ATA_ACPI=y
    CONFIG_SATA_ZPODD=y
    CONFIG_SATA_PMP=y
    
    #
    # Controllers with non-SFF native interface
    #
    CONFIG_SATA_AHCI=m
    CONFIG_SATA_MOBILE_LPM_POLICY=3
    CONFIG_SATA_AHCI_PLATFORM=m
    CONFIG_SATA_ACARD_AHCI=m
    CONFIG_ATA_SFF=y
    
    #
    # SFF controllers with custom DMA interface
    #
    CONFIG_ATA_BMDMA=y
    
    #
    # SATA SFF controllers with BMDMA
    #
    CONFIG_ATA_PIIX=y
    
    #
    # PATA SFF controllers with BMDMA
    #
    CONFIG_PATA_SIS=y
    
    #
    # PIO-only SFF controllers
    #
    
    #
    # Generic fallback / legacy drivers
    #
    CONFIG_ATA_GENERIC=y
    CONFIG_MD=y
    CONFIG_BLK_DEV_MD=y
    CONFIG_MD_AUTODETECT=y
    CONFIG_MD_LINEAR=m
    CONFIG_MD_RAID0=m
    CONFIG_MD_RAID1=m
    CONFIG_MD_RAID10=m
    CONFIG_MD_RAID456=m
    CONFIG_MD_MULTIPATH=m
    CONFIG_BLK_DEV_DM_BUILTIN=y
    CONFIG_BLK_DEV_DM=y
    CONFIG_DM_MULTIPATH=m
    CONFIG_DM_INIT=y
    CONFIG_DM_UEVENT=y
    CONFIG_FUSION=y
    CONFIG_FUSION_MAX_SGE=128
    CONFIG_FUSION_LOGGING=y
    
    #
    # IEEE 1394 (FireWire) support
    #
    # end of IEEE 1394 (FireWire) support
    
    CONFIG_MACINTOSH_DRIVERS=y
    CONFIG_MAC_EMUMOUSEBTN=m
    CONFIG_NETDEVICES=y
    CONFIG_NET_CORE=y
    CONFIG_NET_FC=y
    CONFIG_TUN=y
    CONFIG_NET_VRF=y
    
    #
    # Distributed Switch Architecture drivers
    #
    # end of Distributed Switch Architecture drivers
    
    CONFIG_ETHERNET=y
    CONFIG_MDIO=y
    CONFIG_NET_VENDOR_3COM=y
    CONFIG_NET_VENDOR_ADAPTEC=y
    CONFIG_NET_VENDOR_AGERE=y
    CONFIG_NET_VENDOR_ALACRITECH=y
    CONFIG_NET_VENDOR_ALTEON=y
    CONFIG_NET_VENDOR_AMAZON=y
    CONFIG_NET_VENDOR_AMD=y
    CONFIG_NET_VENDOR_AQUANTIA=y
    CONFIG_NET_VENDOR_ARC=y
    CONFIG_NET_VENDOR_ATHEROS=y
    CONFIG_NET_VENDOR_BROADCOM=y
    CONFIG_TIGON3=m
    CONFIG_TIGON3_HWMON=y
    CONFIG_NET_VENDOR_CADENCE=y
    CONFIG_NET_VENDOR_CAVIUM=y
    CONFIG_NET_VENDOR_CHELSIO=y
    CONFIG_NET_VENDOR_CIRRUS=y
    CONFIG_NET_VENDOR_CISCO=y
    CONFIG_NET_VENDOR_CORTINA=y
    CONFIG_NET_VENDOR_DEC=y
    CONFIG_NET_TULIP=y
    CONFIG_NET_VENDOR_DLINK=y
    CONFIG_NET_VENDOR_EMULEX=y
    CONFIG_NET_VENDOR_EZCHIP=y
    CONFIG_NET_VENDOR_GOOGLE=y
    CONFIG_NET_VENDOR_HUAWEI=y
    CONFIG_NET_VENDOR_I825XX=y
    CONFIG_NET_VENDOR_INTEL=y
    CONFIG_IXGBE=y
    CONFIG_IXGBE_HWMON=y
    CONFIG_IXGBE_DCB=y
    CONFIG_IXGBEVF=y
    CONFIG_I40E=m
    CONFIG_I40E_DCB=y
    CONFIG_ICE=m
    CONFIG_NET_VENDOR_LITEX=y
    CONFIG_NET_VENDOR_MARVELL=y
    CONFIG_NET_VENDOR_MELLANOX=y
    CONFIG_NET_VENDOR_MICREL=y
    CONFIG_NET_VENDOR_MICROCHIP=y
    CONFIG_NET_VENDOR_MICROSEMI=y
    CONFIG_NET_VENDOR_MICROSOFT=y
    CONFIG_NET_VENDOR_MYRI=y
    CONFIG_NET_VENDOR_NI=y
    CONFIG_NET_VENDOR_NATSEMI=y
    CONFIG_NET_VENDOR_NETERION=y
    CONFIG_NET_VENDOR_NETRONOME=y
    CONFIG_NET_VENDOR_8390=y
    CONFIG_NET_VENDOR_NVIDIA=y
    CONFIG_NET_VENDOR_OKI=y
    CONFIG_NET_VENDOR_PACKET_ENGINES=y
    CONFIG_NET_VENDOR_PENSANDO=y
    CONFIG_NET_VENDOR_QLOGIC=y
    CONFIG_NET_VENDOR_BROCADE=y
    CONFIG_NET_VENDOR_QUALCOMM=y
    CONFIG_NET_VENDOR_RDC=y
    CONFIG_NET_VENDOR_REALTEK=y
    CONFIG_NET_VENDOR_RENESAS=y
    CONFIG_NET_VENDOR_ROCKER=y
    CONFIG_NET_VENDOR_SAMSUNG=y
    CONFIG_NET_VENDOR_SEEQ=y
    CONFIG_NET_VENDOR_SILAN=y
    CONFIG_NET_VENDOR_SIS=y
    CONFIG_NET_VENDOR_SOLARFLARE=y
    CONFIG_NET_VENDOR_SMSC=y
    CONFIG_NET_VENDOR_SOCIONEXT=y
    CONFIG_NET_VENDOR_STMICRO=y
    CONFIG_NET_VENDOR_SUN=y
    CONFIG_NET_VENDOR_SYNOPSYS=y
    CONFIG_NET_VENDOR_TEHUTI=y
    CONFIG_NET_VENDOR_TI=y
    CONFIG_NET_VENDOR_VIA=y
    CONFIG_NET_VENDOR_WIZNET=y
    CONFIG_NET_VENDOR_XILINX=y
    CONFIG_FDDI=y
    CONFIG_PHYLINK=y
    CONFIG_PHYLIB=y
    CONFIG_SWPHY=y
    CONFIG_LED_TRIGGER_PHY=y
    CONFIG_FIXED_PHY=y
    
    #
    # MII PHY device drivers
    #
    CONFIG_BCM84881_PHY=y
    CONFIG_MDIO_DEVICE=y
    CONFIG_MDIO_BUS=y
    CONFIG_FWNODE_MDIO=y
    CONFIG_ACPI_MDIO=y
    CONFIG_MDIO_DEVRES=y
    
    #
    # MDIO Multiplexers
    #
    
    #
    # PCS device drivers
    #
    # end of PCS device drivers
    
    CONFIG_PPP=y
    CONFIG_PPP_FILTER=y
    CONFIG_PPP_MULTILINK=y
    CONFIG_SLHC=y
    CONFIG_WLAN=y
    CONFIG_WLAN_VENDOR_ADMTEK=y
    CONFIG_WLAN_VENDOR_ATH=y
    CONFIG_ATH5K_PCI=y
    CONFIG_WLAN_VENDOR_ATMEL=y
    CONFIG_WLAN_VENDOR_BROADCOM=y
    CONFIG_WLAN_VENDOR_CISCO=y
    CONFIG_WLAN_VENDOR_INTEL=y
    CONFIG_WLAN_VENDOR_INTERSIL=y
    CONFIG_WLAN_VENDOR_MARVELL=y
    CONFIG_WLAN_VENDOR_MEDIATEK=y
    CONFIG_WLAN_VENDOR_MICROCHIP=y
    CONFIG_WLAN_VENDOR_RALINK=y
    CONFIG_WLAN_VENDOR_REALTEK=y
    CONFIG_WLAN_VENDOR_RSI=y
    CONFIG_WLAN_VENDOR_ST=y
    CONFIG_WLAN_VENDOR_TI=y
    CONFIG_WLAN_VENDOR_ZYDAS=y
    CONFIG_WLAN_VENDOR_QUANTENNA=y
    CONFIG_WAN=y
    
    #
    # Wireless WAN
    #
    CONFIG_WWAN=y
    # end of Wireless WAN
    
    CONFIG_XEN_NETDEV_FRONTEND=y
    CONFIG_ISDN=y
    
    #
    # Input device support
    #
    CONFIG_INPUT=y
    CONFIG_INPUT_LEDS=m
    
    #
    # Userland interfaces
    #
    CONFIG_INPUT_MOUSEDEV=y
    CONFIG_INPUT_MOUSEDEV_PSAUX=y
    CONFIG_INPUT_MOUSEDEV_SCREEN_X=1024
    CONFIG_INPUT_MOUSEDEV_SCREEN_Y=768
    CONFIG_INPUT_JOYDEV=m
    CONFIG_INPUT_EVDEV=y
    
    #
    # Input Device Drivers
    #
    CONFIG_INPUT_KEYBOARD=y
    CONFIG_KEYBOARD_ATKBD=y
    CONFIG_INPUT_MOUSE=y
    CONFIG_INPUT_JOYSTICK=y
    CONFIG_INPUT_TABLET=y
    CONFIG_INPUT_TOUCHSCREEN=y
    CONFIG_TOUCHSCREEN_ELAN=y
    CONFIG_INPUT_MISC=y
    CONFIG_INPUT_UINPUT=y
    
    #
    # Hardware I/O ports
    #
    CONFIG_SERIO=y
    CONFIG_ARCH_MIGHT_HAVE_PC_SERIO=y
    CONFIG_SERIO_I8042=y
    CONFIG_SERIO_LIBPS2=y
    # end of Hardware I/O ports
    # end of Input device support
    
    #
    # Character devices
    #
    CONFIG_TTY=y
    CONFIG_VT=y
    CONFIG_CONSOLE_TRANSLATIONS=y
    CONFIG_VT_CONSOLE=y
    CONFIG_VT_CONSOLE_SLEEP=y
    CONFIG_HW_CONSOLE=y
    CONFIG_VT_HW_CONSOLE_BINDING=y
    CONFIG_UNIX98_PTYS=y
    CONFIG_LEGACY_PTYS=y
    CONFIG_LEGACY_PTY_COUNT=0
    CONFIG_LDISC_AUTOLOAD=y
    
    #
    # Serial drivers
    #
    CONFIG_SERIAL_EARLYCON=y
    CONFIG_SERIAL_8250=y
    CONFIG_SERIAL_8250_PNP=y
    CONFIG_SERIAL_8250_16550A_VARIANTS=y
    CONFIG_SERIAL_8250_FINTEK=y
    CONFIG_SERIAL_8250_CONSOLE=y
    CONFIG_SERIAL_8250_DMA=y
    CONFIG_SERIAL_8250_PCI=y
    CONFIG_SERIAL_8250_NR_UARTS=48
    CONFIG_SERIAL_8250_RUNTIME_UARTS=32
    CONFIG_SERIAL_8250_EXTENDED=y
    CONFIG_SERIAL_8250_MANY_PORTS=y
    CONFIG_SERIAL_8250_SHARE_IRQ=y
    CONFIG_SERIAL_8250_RSA=y
    CONFIG_SERIAL_8250_RT288X=y
    CONFIG_SERIAL_8250_MID=y
    
    #
    # Non-8250 serial port support
    #
    CONFIG_SERIAL_KGDB_NMI=y
    CONFIG_SERIAL_MAX310X=y
    CONFIG_SERIAL_CORE=y
    CONFIG_SERIAL_CORE_CONSOLE=y
    CONFIG_CONSOLE_POLL=y
    CONFIG_SERIAL_SCCNXP=y
    CONFIG_SERIAL_SCCNXP_CONSOLE=y
    # end of Serial drivers
    
    CONFIG_SERIAL_MCTRL_GPIO=y
    CONFIG_SERIAL_NONSTANDARD=y
    CONFIG_HVC_DRIVER=y
    CONFIG_HVC_IRQ=y
    CONFIG_HVC_XEN=y
    CONFIG_HVC_XEN_FRONTEND=y
    CONFIG_SERIAL_DEV_BUS=y
    CONFIG_SERIAL_DEV_CTRL_TTYPORT=y
    CONFIG_TTY_PRINTK=y
    CONFIG_TTY_PRINTK_LEVEL=6
    CONFIG_VIRTIO_CONSOLE=y
    CONFIG_IPMI_HANDLER=m
    CONFIG_IPMI_DMI_DECODE=y
    CONFIG_IPMI_PLAT_DATA=y
    CONFIG_IPMI_DEVICE_INTERFACE=m
    CONFIG_IPMI_SI=m
    CONFIG_IPMI_SSIF=m
    CONFIG_HW_RANDOM=y
    CONFIG_DEVMEM=y
    CONFIG_DEVPORT=y
    CONFIG_HPET=y
    CONFIG_HPET_MMAP=y
    CONFIG_HPET_MMAP_DEFAULT=y
    CONFIG_TCG_TPM=y
    CONFIG_HW_RANDOM_TPM=y
    CONFIG_TCG_TIS_CORE=y
    CONFIG_TCG_TIS=y
    CONFIG_TCG_CRB=y
    CONFIG_RANDOM_TRUST_CPU=y
    CONFIG_RANDOM_TRUST_BOOTLOADER=y
    # end of Character devices
    
    #
    # I2C support
    #
    CONFIG_I2C=y
    CONFIG_ACPI_I2C_OPREGION=y
    CONFIG_I2C_BOARDINFO=y
    CONFIG_I2C_COMPAT=y
    CONFIG_I2C_CHARDEV=y
    CONFIG_I2C_HELPER_AUTO=y
    CONFIG_I2C_SMBUS=m
    CONFIG_I2C_ALGOBIT=m
    
    #
    # I2C Hardware Bus support
    #
    
    #
    # PC SMBus host controller drivers
    #
    CONFIG_I2C_I801=m
    
    #
    # ACPI drivers
    #
    
    #
    # I2C system bus drivers (mostly embedded / system-on-chip)
    #
    CONFIG_I2C_DESIGNWARE_CORE=y
    CONFIG_I2C_DESIGNWARE_PLATFORM=y
    CONFIG_I2C_DESIGNWARE_BAYTRAIL=y
    
    #
    # External I2C/SMBus adapter drivers
    #
    
    #
    # Other I2C/SMBus bus drivers
    #
    # end of I2C Hardware Bus support
    
    # end of I2C support
    
    CONFIG_SPI=y
    CONFIG_SPI_MASTER=y
    CONFIG_SPI_MEM=y
    
    #
    # SPI Master Controller Drivers
    #
    
    #
    # SPI Multiplexer support
    #
    
    #
    # SPI Protocol Masters
    #
    CONFIG_SPI_SLAVE=y
    CONFIG_SPI_DYNAMIC=y
    CONFIG_PPS=y
    
    #
    # PPS clients support
    #
    
    #
    # PPS generators support
    #
    
    #
    # PTP clock support
    #
    CONFIG_PTP_1588_CLOCK=y
    CONFIG_PTP_1588_CLOCK_OPTIONAL=y
    # end of PTP clock support
    
    CONFIG_PINCTRL=y
    CONFIG_PINMUX=y
    CONFIG_PINCONF=y
    CONFIG_GENERIC_PINCONF=y
    CONFIG_PINCTRL_AMD=y
    CONFIG_PINCTRL_SX150X=y
    CONFIG_PINCTRL_BAYTRAIL=y
    CONFIG_PINCTRL_CHERRYVIEW=y
    CONFIG_PINCTRL_INTEL=y
    
    #
    # Renesas pinctrl drivers
    #
    # end of Renesas pinctrl drivers
    
    CONFIG_GPIOLIB=y
    CONFIG_GPIOLIB_FASTPATH_LIMIT=512
    CONFIG_GPIO_ACPI=y
    CONFIG_GPIOLIB_IRQCHIP=y
    CONFIG_GPIO_SYSFS=y
    CONFIG_GPIO_CDEV=y
    CONFIG_GPIO_CDEV_V1=y
    
    #
    # Memory mapped GPIO drivers
    #
    # end of Memory mapped GPIO drivers
    
    #
    # Port-mapped I/O GPIO drivers
    #
    # end of Port-mapped I/O GPIO drivers
    
    #
    # I2C GPIO expanders
    #
    # end of I2C GPIO expanders
    
    #
    # MFD GPIO expanders
    #
    CONFIG_GPIO_CRYSTAL_COVE=y
    CONFIG_GPIO_PALMAS=y
    CONFIG_GPIO_RC5T583=y
    CONFIG_GPIO_TPS6586X=y
    CONFIG_GPIO_TPS65910=y
    # end of MFD GPIO expanders
    
    #
    # PCI GPIO expanders
    #
    # end of PCI GPIO expanders
    
    #
    # SPI GPIO expanders
    #
    # end of SPI GPIO expanders
    
    #
    # USB GPIO expanders
    #
    # end of USB GPIO expanders
    
    #
    # Virtual GPIO drivers
    #
    # end of Virtual GPIO drivers
    
    CONFIG_POWER_RESET=y
    CONFIG_POWER_RESET_RESTART=y
    CONFIG_POWER_SUPPLY=y
    CONFIG_POWER_SUPPLY_HWMON=y
    CONFIG_CHARGER_MANAGER=y
    CONFIG_HWMON=y
    
    #
    # Native drivers
    #
    CONFIG_SENSORS_CORETEMP=m
    
    #
    # ACPI drivers
    #
    CONFIG_SENSORS_ACPI_POWER=m
    CONFIG_THERMAL=y
    CONFIG_THERMAL_NETLINK=y
    CONFIG_THERMAL_STATISTICS=y
    CONFIG_THERMAL_EMERGENCY_POWEROFF_DELAY_MS=0
    CONFIG_THERMAL_HWMON=y
    CONFIG_THERMAL_WRITABLE_TRIPS=y
    CONFIG_THERMAL_DEFAULT_GOV_STEP_WISE=y
    CONFIG_THERMAL_GOV_FAIR_SHARE=y
    CONFIG_THERMAL_GOV_STEP_WISE=y
    CONFIG_THERMAL_GOV_BANG_BANG=y
    CONFIG_THERMAL_GOV_USER_SPACE=y
    CONFIG_THERMAL_GOV_POWER_ALLOCATOR=y
    CONFIG_DEVFREQ_THERMAL=y
    CONFIG_THERMAL_EMULATION=y
    
    #
    # Intel thermal drivers
    #
    CONFIG_INTEL_POWERCLAMP=m
    CONFIG_X86_THERMAL_VECTOR=y
    CONFIG_X86_PKG_TEMP_THERMAL=m
    
    #
    # ACPI INT340X thermal drivers
    #
    # end of ACPI INT340X thermal drivers
    
    CONFIG_INTEL_PCH_THERMAL=m
    # end of Intel thermal drivers
    
    CONFIG_WATCHDOG=y
    CONFIG_WATCHDOG_CORE=y
    CONFIG_WATCHDOG_HANDLE_BOOT_ENABLED=y
    CONFIG_WATCHDOG_OPEN_TIMEOUT=0
    CONFIG_WATCHDOG_SYSFS=y
    
    #
    # Watchdog Pretimeout Governors
    #
    CONFIG_WATCHDOG_PRETIMEOUT_GOV=y
    CONFIG_WATCHDOG_PRETIMEOUT_GOV_SEL=m
    CONFIG_WATCHDOG_PRETIMEOUT_GOV_NOOP=y
    CONFIG_WATCHDOG_PRETIMEOUT_DEFAULT_GOV_NOOP=y
    
    #
    # Watchdog Device Drivers
    #
    
    #
    # PCI-based Watchdog Cards
    #
    
    #
    # USB-based Watchdog Cards
    #
    CONFIG_SSB_POSSIBLE=y
    CONFIG_BCMA_POSSIBLE=y
    
    #
    # Multifunction device drivers
    #
    CONFIG_MFD_CORE=y
    CONFIG_MFD_AS3711=y
    CONFIG_PMIC_ADP5520=y
    CONFIG_MFD_AAT2870_CORE=y
    CONFIG_PMIC_DA903X=y
    CONFIG_PMIC_DA9052=y
    CONFIG_MFD_DA9052_SPI=y
    CONFIG_MFD_DA9052_I2C=y
    CONFIG_MFD_DA9055=y
    CONFIG_MFD_DA9063=y
    CONFIG_HTC_I2CPLD=y
    CONFIG_INTEL_SOC_PMIC=y
    CONFIG_INTEL_SOC_PMIC_CHTWC=y
    CONFIG_MFD_INTEL_PMT=m
    CONFIG_MFD_88PM860X=y
    CONFIG_MFD_MAX14577=y
    CONFIG_MFD_MAX77693=y
    CONFIG_MFD_MAX77843=y
    CONFIG_MFD_MAX8925=y
    CONFIG_MFD_MAX8997=y
    CONFIG_MFD_MAX8998=y
    CONFIG_EZX_PCAP=y
    CONFIG_MFD_RC5T583=y
    CONFIG_MFD_SYSCON=y
    CONFIG_MFD_LP8788=y
    CONFIG_MFD_PALMAS=y
    CONFIG_MFD_TPS65090=y
    CONFIG_MFD_TPS6586X=y
    CONFIG_MFD_TPS65910=y
    CONFIG_MFD_TPS65912=y
    CONFIG_MFD_TPS65912_I2C=y
    CONFIG_MFD_TPS65912_SPI=y
    CONFIG_MFD_TPS80031=y
    CONFIG_TWL4030_CORE=y
    CONFIG_MFD_TWL4030_AUDIO=y
    CONFIG_TWL6040_CORE=y
    CONFIG_MFD_WM8400=y
    CONFIG_MFD_WM831X=y
    CONFIG_MFD_WM831X_I2C=y
    CONFIG_MFD_WM831X_SPI=y
    CONFIG_MFD_WM8350=y
    CONFIG_MFD_WM8350_I2C=y
    # end of Multifunction device drivers
    
    CONFIG_REGULATOR=y
    CONFIG_RC_CORE=m
    CONFIG_LIRC=y
    CONFIG_RC_DECODERS=y
    CONFIG_RC_DEVICES=y
    CONFIG_CEC_CORE=m
    CONFIG_MEDIA_CEC_RC=y
    CONFIG_MEDIA_CEC_SUPPORT=y
    
    #
    # Graphics support
    #
    CONFIG_AGP=y
    CONFIG_AGP_AMD64=y
    CONFIG_AGP_INTEL=y
    CONFIG_AGP_VIA=y
    CONFIG_INTEL_GTT=y
    CONFIG_VGA_ARB=y
    CONFIG_VGA_ARB_MAX_GPUS=16
    CONFIG_VGA_SWITCHEROO=y
    CONFIG_DRM=m
    CONFIG_DRM_DP_AUX_CHARDEV=y
    CONFIG_DRM_KMS_HELPER=m
    CONFIG_DRM_FBDEV_EMULATION=y
    CONFIG_DRM_FBDEV_OVERALLOC=100
    CONFIG_DRM_LOAD_EDID_FIRMWARE=y
    CONFIG_DRM_DP_CEC=y
    CONFIG_DRM_GEM_SHMEM_HELPER=y
    
    #
    # I2C encoder or helper chips
    #
    # end of I2C encoder or helper chips
    
    #
    # ARM devices
    #
    # end of ARM devices
    
    CONFIG_DRM_MGAG200=m
    CONFIG_DRM_PANEL=y
    
    #
    # Display Panels
    #
    # end of Display Panels
    
    CONFIG_DRM_BRIDGE=y
    CONFIG_DRM_PANEL_BRIDGE=y
    
    #
    # Display Interface Bridges
    #
    # end of Display Interface Bridges
    
    CONFIG_DRM_PANEL_ORIENTATION_QUIRKS=y
    
    #
    # Frame buffer Devices
    #
    CONFIG_FB_CMDLINE=y
    CONFIG_FB_NOTIFY=y
    CONFIG_FB=y
    CONFIG_FIRMWARE_EDID=y
    CONFIG_FB_BOOT_VESA_SUPPORT=y
    CONFIG_FB_CFB_FILLRECT=y
    CONFIG_FB_CFB_COPYAREA=y
    CONFIG_FB_CFB_IMAGEBLIT=y
    CONFIG_FB_SYS_FILLRECT=m
    CONFIG_FB_SYS_COPYAREA=m
    CONFIG_FB_SYS_IMAGEBLIT=m
    CONFIG_FB_SYS_FOPS=m
    CONFIG_FB_DEFERRED_IO=y
    CONFIG_FB_MODE_HELPERS=y
    CONFIG_FB_TILEBLITTING=y
    
    #
    # Frame buffer hardware drivers
    #
    CONFIG_FB_ASILIANT=y
    CONFIG_FB_IMSTT=y
    CONFIG_FB_VESA=y
    CONFIG_FB_EFI=y
    # end of Frame buffer Devices
    
    #
    # Backlight & LCD device support
    #
    CONFIG_BACKLIGHT_CLASS_DEVICE=y
    # end of Backlight & LCD device support
    
    CONFIG_HDMI=y
    
    #
    # Console display driver support
    #
    CONFIG_VGA_CONSOLE=y
    CONFIG_DUMMY_CONSOLE=y
    CONFIG_DUMMY_CONSOLE_COLUMNS=80
    CONFIG_DUMMY_CONSOLE_ROWS=25
    CONFIG_FRAMEBUFFER_CONSOLE=y
    CONFIG_FRAMEBUFFER_CONSOLE_DETECT_PRIMARY=y
    CONFIG_FRAMEBUFFER_CONSOLE_ROTATION=y
    CONFIG_FRAMEBUFFER_CONSOLE_DEFERRED_TAKEOVER=y
    # end of Console display driver support
    
    # end of Graphics support
    
    
    #
    # HID support
    #
    CONFIG_HID=m
    CONFIG_HID_BATTERY_STRENGTH=y
    CONFIG_HIDRAW=y
    CONFIG_HID_GENERIC=m
    
    #
    # Special HID drivers
    #
    # end of Special HID drivers
    
    #
    # USB HID support
    #
    CONFIG_USB_HID=m
    CONFIG_HID_PID=y
    CONFIG_USB_HIDDEV=y
    
    #
    # USB HID Boot Protocol drivers
    #
    # end of USB HID Boot Protocol drivers
    # end of USB HID support
    
    #
    # I2C HID support
    #
    # end of I2C HID support
    
    #
    # Intel ISH HID support
    #
    # end of Intel ISH HID support
    
    #
    # AMD SFH HID Support
    #
    # end of AMD SFH HID Support
    # end of HID support
    
    CONFIG_USB_OHCI_LITTLE_ENDIAN=y
    CONFIG_USB_SUPPORT=y
    CONFIG_USB_COMMON=y
    CONFIG_USB_LED_TRIG=y
    CONFIG_USB_ARCH_HAS_HCD=y
    CONFIG_USB=y
    CONFIG_USB_PCI=y
    CONFIG_USB_ANNOUNCE_NEW_DEVICES=y
    
    #
    # Miscellaneous USB options
    #
    CONFIG_USB_DEFAULT_PERSIST=y
    CONFIG_USB_DYNAMIC_MINORS=y
    CONFIG_USB_AUTOSUSPEND_DELAY=2
    
    #
    # USB Host Controller Drivers
    #
    CONFIG_USB_XHCI_HCD=y
    CONFIG_USB_XHCI_DBGCAP=y
    CONFIG_USB_XHCI_PCI=m
    CONFIG_USB_XHCI_PCI_RENESAS=m
    CONFIG_USB_EHCI_HCD=y
    CONFIG_USB_EHCI_ROOT_HUB_TT=y
    CONFIG_USB_EHCI_TT_NEWSCHED=y
    CONFIG_USB_EHCI_PCI=y
    CONFIG_USB_EHCI_HCD_PLATFORM=y
    CONFIG_USB_OHCI_HCD=y
    CONFIG_USB_OHCI_HCD_PCI=y
    CONFIG_USB_OHCI_HCD_PLATFORM=y
    CONFIG_USB_UHCI_HCD=y
    
    #
    # USB Device Class drivers
    #
    
    #
    # NOTE: USB_STORAGE depends on SCSI but BLK_DEV_SD may
    #
    
    #
    # also be needed; see USB_STORAGE Help for more info
    #
    
    #
    # USB Imaging devices
    #
    CONFIG_USB_DWC2=y
    CONFIG_USB_DWC2_HOST=y
    
    #
    # Gadget/Dual-role mode requires USB Gadget support to be enabled
    #
    
    #
    # USB port drivers
    #
    
    #
    # USB Miscellaneous drivers
    #
    
    #
    # USB Physical Layer drivers
    #
    # end of USB Physical Layer drivers
    
    CONFIG_USB_ROLE_SWITCH=y
    CONFIG_MMC=y
    CONFIG_MMC_CRYPTO=y
    
    #
    # MMC/SD/SDIO Host Controller Drivers
    #
    CONFIG_NEW_LEDS=y
    CONFIG_LEDS_CLASS=y
    CONFIG_LEDS_BRIGHTNESS_HW_CHANGED=y
    
    #
    # LED drivers
    #
    
    #
    # LED driver for blink(1) USB RGB LED is under Special HID drivers (HID_THINGM)
    #
    
    #
    # Flash and Torch LED drivers
    #
    
    #
    # LED Triggers
    #
    CONFIG_LEDS_TRIGGERS=y
    CONFIG_LEDS_TRIGGER_DISK=y
    CONFIG_LEDS_TRIGGER_CPU=y
    
    #
    # iptables trigger is under Netfilter config (LED target)
    #
    CONFIG_LEDS_TRIGGER_PANIC=y
    CONFIG_ACCESSIBILITY=y
    
    #
    # Speakup console speech
    #
    # end of Speakup console speech
    
    CONFIG_INFINIBAND=m
    CONFIG_INFINIBAND_USER_ACCESS=m
    CONFIG_INFINIBAND_USER_MEM=y
    CONFIG_INFINIBAND_ON_DEMAND_PAGING=y
    CONFIG_INFINIBAND_ADDR_TRANS=y
    CONFIG_INFINIBAND_ADDR_TRANS_CONFIGFS=y
    CONFIG_INFINIBAND_VIRT_DMA=y
    CONFIG_INFINIBAND_IRDMA=m
    CONFIG_EDAC_ATOMIC_SCRUB=y
    CONFIG_EDAC_SUPPORT=y
    CONFIG_EDAC=y
    CONFIG_EDAC_GHES=y
    CONFIG_EDAC_I10NM=m
    CONFIG_RTC_LIB=y
    CONFIG_RTC_MC146818_LIB=y
    CONFIG_RTC_CLASS=y
    CONFIG_RTC_HCTOSYS=y
    CONFIG_RTC_HCTOSYS_DEVICE="rtc0"
    CONFIG_RTC_SYSTOHC=y
    CONFIG_RTC_SYSTOHC_DEVICE="rtc0"
    CONFIG_RTC_NVMEM=y
    
    #
    # RTC interfaces
    #
    CONFIG_RTC_INTF_SYSFS=y
    CONFIG_RTC_INTF_PROC=y
    CONFIG_RTC_INTF_DEV=y
    
    #
    # I2C RTC drivers
    #
    
    #
    # SPI RTC drivers
    #
    CONFIG_RTC_I2C_AND_SPI=y
    
    #
    # SPI and I2C RTC drivers
    #
    
    #
    # Platform RTC drivers
    #
    CONFIG_RTC_DRV_CMOS=y
    
    #
    # on-CPU RTC drivers
    #
    
    #
    # HID Sensor RTC drivers
    #
    CONFIG_DMADEVICES=y
    
    #
    # DMA Devices
    #
    CONFIG_DMA_ENGINE=y
    CONFIG_DMA_VIRTUAL_CHANNELS=y
    CONFIG_DMA_ACPI=y
    CONFIG_HSU_DMA=y
    CONFIG_INTEL_LDMA=y
    
    #
    # DMA Clients
    #
    CONFIG_ASYNC_TX_DMA=y
    
    #
    # DMABUF options
    #
    CONFIG_SYNC_FILE=y
    CONFIG_SW_SYNC=y
    CONFIG_UDMABUF=y
    CONFIG_DMABUF_HEAPS=y
    CONFIG_DMABUF_HEAPS_SYSTEM=y
    # end of DMABUF options
    
    CONFIG_AUXDISPLAY=y
    CONFIG_CHARLCD_BL_FLASH=y
    CONFIG_VFIO=y
    CONFIG_VFIO_IOMMU_TYPE1=y
    CONFIG_VFIO_VIRQFD=y
    CONFIG_VFIO_NOIOMMU=y
    CONFIG_VFIO_PCI_CORE=y
    CONFIG_VFIO_PCI_MMAP=y
    CONFIG_VFIO_PCI_INTX=y
    CONFIG_VFIO_PCI=y
    CONFIG_VFIO_PCI_VGA=y
    CONFIG_VFIO_PCI_IGD=y
    CONFIG_IRQ_BYPASS_MANAGER=y
    CONFIG_VIRT_DRIVERS=y
    CONFIG_VIRTIO=y
    CONFIG_ARCH_HAS_RESTRICTED_VIRTIO_MEMORY_ACCESS=y
    CONFIG_VIRTIO_PCI_LIB=y
    CONFIG_VIRTIO_MENU=y
    CONFIG_VIRTIO_PCI=y
    CONFIG_VIRTIO_PCI_LEGACY=y
    CONFIG_VIRTIO_BALLOON=y
    CONFIG_VIRTIO_MMIO=y
    CONFIG_VIRTIO_MMIO_CMDLINE_DEVICES=y
    CONFIG_VHOST_MENU=y
    
    #
    # Microsoft Hyper-V guest support
    #
    # end of Microsoft Hyper-V guest support
    
    #
    # Xen driver support
    #
    CONFIG_XEN_BALLOON=y
    CONFIG_XEN_BALLOON_MEMORY_HOTPLUG=y
    CONFIG_XEN_MEMORY_HOTPLUG_LIMIT=512
    CONFIG_XEN_SCRUB_PAGES_DEFAULT=y
    CONFIG_XEN_BACKEND=y
    CONFIG_XEN_SYS_HYPERVISOR=y
    CONFIG_XEN_XENBUS_FRONTEND=y
    CONFIG_XEN_GRANT_DMA_ALLOC=y
    CONFIG_SWIOTLB_XEN=y
    CONFIG_XEN_PRIVCMD=m
    CONFIG_XEN_ACPI_PROCESSOR=y
    CONFIG_XEN_MCE_LOG=y
    CONFIG_XEN_HAVE_PVMMU=y
    CONFIG_XEN_EFI=y
    CONFIG_XEN_AUTO_XLATE=y
    CONFIG_XEN_ACPI=y
    CONFIG_XEN_HAVE_VPMU=y
    CONFIG_XEN_UNPOPULATED_ALLOC=y
    # end of Xen driver support
    
    CONFIG_STAGING=y
    CONFIG_STAGING_MEDIA=y
    
    #
    # Android
    #
    # end of Android
    
    CONFIG_UNISYSSPAR=y
    CONFIG_X86_PLATFORM_DEVICES=y
    CONFIG_ACPI_WMI=m
    CONFIG_WMI_BMOF=m
    CONFIG_X86_PLATFORM_DRIVERS_DELL=y
    CONFIG_DCDBAS=m
    CONFIG_DELL_SMBIOS=m
    CONFIG_DELL_SMBIOS_WMI=y
    CONFIG_DELL_SMBIOS_SMM=y
    CONFIG_DELL_WMI_DESCRIPTOR=m
    CONFIG_INTEL_PMC_CORE=y
    
    #
    # Intel Speed Select Technology interface support
    #
    CONFIG_INTEL_SPEED_SELECT_INTERFACE=m
    # end of Intel Speed Select Technology interface support
    
    CONFIG_INTEL_TURBO_MAX_3=y
    CONFIG_INTEL_SCU_IPC=y
    CONFIG_INTEL_SCU=y
    CONFIG_INTEL_SCU_PCI=y
    CONFIG_PMC_ATOM=y
    CONFIG_CHROME_PLATFORMS=y
    CONFIG_MELLANOX_PLATFORM=y
    CONFIG_SURFACE_PLATFORMS=y
    CONFIG_HAVE_CLK=y
    CONFIG_HAVE_CLK_PREPARE=y
    CONFIG_COMMON_CLK=y
    
    #
    # Clock driver for ARM Reference designs
    #
    CONFIG_ICST=y
    CONFIG_CLK_SP810=y
    # end of Clock driver for ARM Reference designs
    
    CONFIG_HWSPINLOCK=y
    
    #
    # Clock Source drivers
    #
    CONFIG_CLKEVT_I8253=y
    CONFIG_I8253_LOCK=y
    CONFIG_CLKBLD_I8253=y
    # end of Clock Source drivers
    
    CONFIG_MAILBOX=y
    CONFIG_PCC=y
    CONFIG_IOMMU_IOVA=y
    CONFIG_IOASID=y
    CONFIG_IOMMU_API=y
    CONFIG_IOMMU_SUPPORT=y
    
    #
    # Generic IOMMU Pagetable Support
    #
    CONFIG_IOMMU_IO_PGTABLE=y
    # end of Generic IOMMU Pagetable Support
    
    CONFIG_IOMMU_DEFAULT_DMA_LAZY=y
    CONFIG_IOMMU_DMA=y
    CONFIG_IOMMU_SVA_LIB=y
    CONFIG_AMD_IOMMU=y
    CONFIG_DMAR_TABLE=y
    CONFIG_INTEL_IOMMU=y
    CONFIG_INTEL_IOMMU_SVM=y
    CONFIG_INTEL_IOMMU_FLOPPY_WA=y
    CONFIG_IRQ_REMAP=y
    CONFIG_VIRTIO_IOMMU=y
    
    #
    # Remoteproc drivers
    #
    CONFIG_REMOTEPROC=y
    CONFIG_REMOTEPROC_CDEV=y
    # end of Remoteproc drivers
    
    #
    # Rpmsg drivers
    #
    # end of Rpmsg drivers
    
    
    #
    # SOC (System On Chip) specific Drivers
    #
    
    #
    # Amlogic SoC drivers
    #
    # end of Amlogic SoC drivers
    
    #
    # Broadcom SoC drivers
    #
    # end of Broadcom SoC drivers
    
    #
    # NXP/Freescale QorIQ SoC drivers
    #
    # end of NXP/Freescale QorIQ SoC drivers
    
    #
    # i.MX SoC drivers
    #
    # end of i.MX SoC drivers
    
    #
    # Enable LiteX SoC Builder specific drivers
    #
    # end of Enable LiteX SoC Builder specific drivers
    
    #
    # Qualcomm SoC drivers
    #
    # end of Qualcomm SoC drivers
    
    CONFIG_SOC_TI=y
    
    #
    # Xilinx SoC drivers
    #
    # end of Xilinx SoC drivers
    # end of SOC (System On Chip) specific Drivers
    
    CONFIG_PM_DEVFREQ=y
    
    #
    # DEVFREQ Governors
    #
    CONFIG_DEVFREQ_GOV_SIMPLE_ONDEMAND=y
    CONFIG_DEVFREQ_GOV_PERFORMANCE=y
    CONFIG_DEVFREQ_GOV_POWERSAVE=y
    CONFIG_DEVFREQ_GOV_USERSPACE=y
    CONFIG_DEVFREQ_GOV_PASSIVE=y
    
    #
    # DEVFREQ Drivers
    #
    CONFIG_PM_DEVFREQ_EVENT=y
    CONFIG_EXTCON=y
    
    #
    # Extcon Device Drivers
    #
    CONFIG_MEMORY=y
    CONFIG_VME_BUS=y
    
    #
    # VME Bridge Drivers
    #
    
    #
    # VME Board Drivers
    #
    
    #
    # VME Device Drivers
    #
    CONFIG_PWM=y
    CONFIG_PWM_SYSFS=y
    CONFIG_PWM_CRC=y
    CONFIG_PWM_LPSS=y
    CONFIG_PWM_LPSS_PCI=y
    CONFIG_PWM_LPSS_PLATFORM=y
    
    #
    # IRQ chip support
    #
    # end of IRQ chip support
    
    CONFIG_RESET_CONTROLLER=y
    
    #
    # PHY Subsystem
    #
    CONFIG_GENERIC_PHY=y
    # end of PHY Subsystem
    
    CONFIG_POWERCAP=y
    CONFIG_INTEL_RAPL_CORE=m
    CONFIG_INTEL_RAPL=m
    CONFIG_IDLE_INJECT=y
    CONFIG_DTPM=y
    CONFIG_DTPM_CPU=y
    
    #
    # Performance monitor support
    #
    # end of Performance monitor support
    
    CONFIG_RAS=y
    CONFIG_RAS_CEC=y
    
    #
    # Android
    #
    CONFIG_ANDROID=y
    # end of Android
    
    CONFIG_LIBNVDIMM=y
    CONFIG_ND_CLAIM=y
    CONFIG_BTT=y
    CONFIG_NVDIMM_PFN=y
    CONFIG_NVDIMM_DAX=y
    CONFIG_NVDIMM_KEYS=y
    CONFIG_DAX=y
    CONFIG_NVMEM=y
    CONFIG_NVMEM_SYSFS=y
    
    #
    # HW tracing support
    #
    # end of HW tracing support
    
    CONFIG_PM_OPP=y
    CONFIG_INTERCONNECT=y
    # end of Device Drivers
    
    #
    # File systems
    #
    CONFIG_DCACHE_WORD_ACCESS=y
    CONFIG_VALIDATE_FS_PARSER=y
    CONFIG_FS_IOMAP=y
    CONFIG_EXT4_FS=y
    CONFIG_EXT4_USE_FOR_EXT2=y
    CONFIG_EXT4_FS_POSIX_ACL=y
    CONFIG_EXT4_FS_SECURITY=y
    CONFIG_JBD2=y
    CONFIG_FS_MBCACHE=y
    CONFIG_BTRFS_FS=m
    CONFIG_BTRFS_FS_POSIX_ACL=y
    CONFIG_FS_DAX=y
    CONFIG_FS_DAX_PMD=y
    CONFIG_FS_POSIX_ACL=y
    CONFIG_EXPORTFS=y
    CONFIG_EXPORTFS_BLOCK_OPS=y
    CONFIG_FILE_LOCKING=y
    CONFIG_FS_ENCRYPTION=y
    CONFIG_FS_ENCRYPTION_ALGS=y
    CONFIG_FS_ENCRYPTION_INLINE_CRYPT=y
    CONFIG_FS_VERITY=y
    CONFIG_FS_VERITY_BUILTIN_SIGNATURES=y
    CONFIG_FSNOTIFY=y
    CONFIG_DNOTIFY=y
    CONFIG_INOTIFY_USER=y
    CONFIG_FANOTIFY=y
    CONFIG_FANOTIFY_ACCESS_PERMISSIONS=y
    CONFIG_QUOTA=y
    CONFIG_QUOTA_NETLINK_INTERFACE=y
    CONFIG_QUOTACTL=y
    CONFIG_AUTOFS_FS=m
    CONFIG_FUSE_FS=y
    
    #
    # Caches
    #
    # end of Caches
    
    #
    # CD-ROM/DVD Filesystems
    #
    # end of CD-ROM/DVD Filesystems
    
    #
    # DOS/FAT/EXFAT/NT Filesystems
    #
    CONFIG_FAT_FS=y
    CONFIG_VFAT_FS=y
    CONFIG_FAT_DEFAULT_CODEPAGE=437
    CONFIG_FAT_DEFAULT_IOCHARSET="iso8859-1"
    # end of DOS/FAT/EXFAT/NT Filesystems
    
    #
    # Pseudo filesystems
    #
    CONFIG_PROC_FS=y
    CONFIG_PROC_KCORE=y
    CONFIG_PROC_VMCORE=y
    CONFIG_PROC_VMCORE_DEVICE_DUMP=y
    CONFIG_PROC_SYSCTL=y
    CONFIG_PROC_PAGE_MONITOR=y
    CONFIG_PROC_CHILDREN=y
    CONFIG_PROC_PID_ARCH_STATUS=y
    CONFIG_PROC_CPU_RESCTRL=y
    CONFIG_KERNFS=y
    CONFIG_SYSFS=y
    CONFIG_TMPFS=y
    CONFIG_TMPFS_POSIX_ACL=y
    CONFIG_TMPFS_XATTR=y
    CONFIG_TMPFS_INODE64=y
    CONFIG_HUGETLBFS=y
    CONFIG_HUGETLB_PAGE=y
    CONFIG_HUGETLB_PAGE_FREE_VMEMMAP=y
    CONFIG_MEMFD_CREATE=y
    CONFIG_ARCH_HAS_GIGANTIC_PAGE=y
    CONFIG_CONFIGFS_FS=y
    CONFIG_EFIVAR_FS=y
    # end of Pseudo filesystems
    
    CONFIG_MISC_FILESYSTEMS=y
    CONFIG_ECRYPT_FS=y
    CONFIG_ECRYPT_FS_MESSAGING=y
    CONFIG_SQUASHFS=y
    CONFIG_SQUASHFS_FILE_DIRECT=y
    CONFIG_SQUASHFS_DECOMP_SINGLE=y
    CONFIG_SQUASHFS_XATTR=y
    CONFIG_SQUASHFS_ZLIB=y
    CONFIG_SQUASHFS_LZ4=y
    CONFIG_SQUASHFS_LZO=y
    CONFIG_SQUASHFS_XZ=y
    CONFIG_SQUASHFS_ZSTD=y
    CONFIG_SQUASHFS_FRAGMENT_CACHE_SIZE=3
    CONFIG_PSTORE=y
    CONFIG_PSTORE_DEFAULT_KMSG_BYTES=10240
    CONFIG_PSTORE_DEFLATE_COMPRESS=y
    CONFIG_PSTORE_COMPRESS=y
    CONFIG_PSTORE_DEFLATE_COMPRESS_DEFAULT=y
    CONFIG_PSTORE_COMPRESS_DEFAULT="deflate"
    CONFIG_PSTORE_RAM=m
    CONFIG_PSTORE_ZONE=m
    CONFIG_PSTORE_BLK=m
    CONFIG_PSTORE_BLK_BLKDEV=""
    CONFIG_PSTORE_BLK_KMSG_SIZE=64
    CONFIG_PSTORE_BLK_MAX_REASON=2
    CONFIG_NETWORK_FILESYSTEMS=y
    CONFIG_NLS=y
    CONFIG_NLS_DEFAULT="utf8"
    CONFIG_NLS_CODEPAGE_437=y
    CONFIG_NLS_ISO8859_1=m
    CONFIG_UNICODE=y
    CONFIG_IO_WQ=y
    # end of File systems
    
    #
    # Security options
    #
    CONFIG_KEYS=y
    CONFIG_KEYS_REQUEST_CACHE=y
    CONFIG_PERSISTENT_KEYRINGS=y
    CONFIG_TRUSTED_KEYS=y
    CONFIG_ENCRYPTED_KEYS=y
    CONFIG_KEY_DH_OPERATIONS=y
    CONFIG_KEY_NOTIFICATIONS=y
    CONFIG_SECURITY_DMESG_RESTRICT=y
    CONFIG_SECURITY=y
    CONFIG_SECURITYFS=y
    CONFIG_SECURITY_NETWORK=y
    CONFIG_SECURITY_INFINIBAND=y
    CONFIG_SECURITY_PATH=y
    CONFIG_INTEL_TXT=y
    CONFIG_LSM_MMAP_MIN_ADDR=0
    CONFIG_HAVE_HARDENED_USERCOPY_ALLOCATOR=y
    CONFIG_HARDENED_USERCOPY=y
    CONFIG_FORTIFY_SOURCE=y
    CONFIG_SECURITY_SELINUX=y
    CONFIG_SECURITY_SELINUX_BOOTPARAM=y
    CONFIG_SECURITY_SELINUX_DEVELOP=y
    CONFIG_SECURITY_SELINUX_AVC_STATS=y
    CONFIG_SECURITY_SELINUX_CHECKREQPROT_VALUE=1
    CONFIG_SECURITY_SELINUX_SIDTAB_HASH_BITS=9
    CONFIG_SECURITY_SELINUX_SID2STR_CACHE_SIZE=256
    CONFIG_SECURITY_SMACK=y
    CONFIG_SECURITY_SMACK_NETFILTER=y
    CONFIG_SECURITY_SMACK_APPEND_SIGNALS=y
    CONFIG_SECURITY_TOMOYO=y
    CONFIG_SECURITY_TOMOYO_MAX_ACCEPT_ENTRY=2048
    CONFIG_SECURITY_TOMOYO_MAX_AUDIT_LOG=1024
    CONFIG_SECURITY_TOMOYO_POLICY_LOADER="/sbin/tomoyo-init"
    CONFIG_SECURITY_TOMOYO_ACTIVATION_TRIGGER="/sbin/init"
    CONFIG_SECURITY_APPARMOR=y
    CONFIG_SECURITY_APPARMOR_HASH=y
    CONFIG_SECURITY_APPARMOR_HASH_DEFAULT=y
    CONFIG_SECURITY_YAMA=y
    CONFIG_SECURITY_SAFESETID=y
    CONFIG_SECURITY_LOCKDOWN_LSM=y
    CONFIG_SECURITY_LOCKDOWN_LSM_EARLY=y
    CONFIG_LOCK_DOWN_KERNEL_FORCE_NONE=y
    CONFIG_SECURITY_LANDLOCK=y
    CONFIG_INTEGRITY=y
    CONFIG_INTEGRITY_SIGNATURE=y
    CONFIG_INTEGRITY_ASYMMETRIC_KEYS=y
    CONFIG_INTEGRITY_TRUSTED_KEYRING=y
    CONFIG_INTEGRITY_PLATFORM_KEYRING=y
    CONFIG_LOAD_UEFI_KEYS=y
    CONFIG_INTEGRITY_AUDIT=y
    CONFIG_IMA=y
    CONFIG_IMA_MEASURE_PCR_IDX=10
    CONFIG_IMA_LSM_RULES=y
    CONFIG_IMA_NG_TEMPLATE=y
    CONFIG_IMA_DEFAULT_TEMPLATE="ima-ng"
    CONFIG_IMA_DEFAULT_HASH_SHA1=y
    CONFIG_IMA_DEFAULT_HASH="sha1"
    CONFIG_IMA_APPRAISE=y
    CONFIG_IMA_APPRAISE_BOOTPARAM=y
    CONFIG_IMA_APPRAISE_MODSIG=y
    CONFIG_IMA_TRUSTED_KEYRING=y
    CONFIG_IMA_MEASURE_ASYMMETRIC_KEYS=y
    CONFIG_IMA_QUEUE_EARLY_BOOT_KEYS=y
    CONFIG_EVM=y
    CONFIG_EVM_ATTR_FSUUID=y
    CONFIG_EVM_EXTRA_SMACK_XATTRS=y
    CONFIG_EVM_ADD_XATTRS=y
    CONFIG_DEFAULT_SECURITY_APPARMOR=y
    CONFIG_LSM="landlock,lockdown,yama,integrity,apparmor"
    
    #
    # Kernel hardening options
    #
    
    #
    # Memory initialization
    #
    CONFIG_INIT_STACK_NONE=y
    CONFIG_INIT_ON_ALLOC_DEFAULT_ON=y
    CONFIG_CC_HAS_ZERO_CALL_USED_REGS=y
    # end of Memory initialization
    # end of Kernel hardening options
    # end of Security options
    
    CONFIG_XOR_BLOCKS=m
    CONFIG_ASYNC_CORE=m
    CONFIG_ASYNC_MEMCPY=m
    CONFIG_ASYNC_XOR=m
    CONFIG_ASYNC_PQ=m
    CONFIG_ASYNC_RAID6_RECOV=m
    CONFIG_CRYPTO=y
    
    #
    # Crypto core or helper
    #
    CONFIG_CRYPTO_ALGAPI=y
    CONFIG_CRYPTO_ALGAPI2=y
    CONFIG_CRYPTO_AEAD=y
    CONFIG_CRYPTO_AEAD2=y
    CONFIG_CRYPTO_SKCIPHER=y
    CONFIG_CRYPTO_SKCIPHER2=y
    CONFIG_CRYPTO_HASH=y
    CONFIG_CRYPTO_HASH2=y
    CONFIG_CRYPTO_RNG=y
    CONFIG_CRYPTO_RNG2=y
    CONFIG_CRYPTO_RNG_DEFAULT=y
    CONFIG_CRYPTO_AKCIPHER2=y
    CONFIG_CRYPTO_AKCIPHER=y
    CONFIG_CRYPTO_KPP2=y
    CONFIG_CRYPTO_KPP=y
    CONFIG_CRYPTO_ACOMP2=y
    CONFIG_CRYPTO_MANAGER=y
    CONFIG_CRYPTO_MANAGER2=y
    CONFIG_CRYPTO_MANAGER_DISABLE_TESTS=y
    CONFIG_CRYPTO_GF128MUL=y
    CONFIG_CRYPTO_NULL=y
    CONFIG_CRYPTO_NULL2=y
    CONFIG_CRYPTO_CRYPTD=m
    CONFIG_CRYPTO_SIMD=m
    
    #
    # Public-key cryptography
    #
    CONFIG_CRYPTO_RSA=y
    CONFIG_CRYPTO_DH=y
    
    #
    # Authenticated Encryption with Associated Data
    #
    CONFIG_CRYPTO_GCM=y
    CONFIG_CRYPTO_SEQIV=y
    
    #
    # Block modes
    #
    CONFIG_CRYPTO_CBC=y
    CONFIG_CRYPTO_CTR=y
    CONFIG_CRYPTO_CTS=y
    CONFIG_CRYPTO_ECB=y
    CONFIG_CRYPTO_XTS=y
    
    #
    # Hash modes
    #
    CONFIG_CRYPTO_HMAC=y
    
    #
    # Digest
    #
    CONFIG_CRYPTO_CRC32C=y
    CONFIG_CRYPTO_CRC32C_INTEL=y
    CONFIG_CRYPTO_CRC32_PCLMUL=m
    CONFIG_CRYPTO_XXHASH=m
    CONFIG_CRYPTO_BLAKE2B=m
    CONFIG_CRYPTO_BLAKE2S_X86=y
    CONFIG_CRYPTO_CRCT10DIF=y
    CONFIG_CRYPTO_CRCT10DIF_PCLMUL=m
    CONFIG_CRYPTO_GHASH=y
    CONFIG_CRYPTO_MD5=y
    CONFIG_CRYPTO_SHA1=y
    CONFIG_CRYPTO_SHA256=y
    CONFIG_CRYPTO_SHA512=y
    CONFIG_CRYPTO_GHASH_CLMUL_NI_INTEL=m
    
    #
    # Ciphers
    #
    CONFIG_CRYPTO_AES=y
    CONFIG_CRYPTO_AES_NI_INTEL=m
    
    #
    # Compression
    #
    CONFIG_CRYPTO_DEFLATE=y
    CONFIG_CRYPTO_LZO=y
    
    #
    # Random Number Generation
    #
    CONFIG_CRYPTO_DRBG_MENU=y
    CONFIG_CRYPTO_DRBG_HMAC=y
    CONFIG_CRYPTO_DRBG_HASH=y
    CONFIG_CRYPTO_DRBG_CTR=y
    CONFIG_CRYPTO_DRBG=y
    CONFIG_CRYPTO_JITTERENTROPY=y
    CONFIG_CRYPTO_HASH_INFO=y
    CONFIG_CRYPTO_HW=y
    CONFIG_CRYPTO_DEV_PADLOCK=y
    CONFIG_CRYPTO_DEV_CCP=y
    CONFIG_ASYMMETRIC_KEY_TYPE=y
    CONFIG_ASYMMETRIC_PUBLIC_KEY_SUBTYPE=y
    CONFIG_X509_CERTIFICATE_PARSER=y
    CONFIG_PKCS7_MESSAGE_PARSER=y
    CONFIG_SIGNED_PE_FILE_VERIFICATION=y
    
    #
    # Certificates for signature checking
    #
    CONFIG_MODULE_SIG_KEY="certs/signing_key.pem"
    CONFIG_MODULE_SIG_KEY_TYPE_RSA=y
    CONFIG_SYSTEM_TRUSTED_KEYRING=y
    CONFIG_SYSTEM_TRUSTED_KEYS=""
    CONFIG_SYSTEM_EXTRA_CERTIFICATE=y
    CONFIG_SYSTEM_EXTRA_CERTIFICATE_SIZE=4096
    CONFIG_SECONDARY_TRUSTED_KEYRING=y
    CONFIG_SYSTEM_BLACKLIST_KEYRING=y
    CONFIG_SYSTEM_BLACKLIST_HASH_LIST=""
    CONFIG_SYSTEM_REVOCATION_LIST=y
    CONFIG_SYSTEM_REVOCATION_KEYS=""
    # end of Certificates for signature checking
    
    CONFIG_BINARY_PRINTF=y
    
    #
    # Library routines
    #
    CONFIG_RAID6_PQ=m
    CONFIG_RAID6_PQ_BENCHMARK=y
    CONFIG_LINEAR_RANGES=y
    CONFIG_PACKING=y
    CONFIG_BITREVERSE=y
    CONFIG_GENERIC_STRNCPY_FROM_USER=y
    CONFIG_GENERIC_STRNLEN_USER=y
    CONFIG_GENERIC_NET_UTILS=y
    CONFIG_GENERIC_FIND_FIRST_BIT=y
    CONFIG_RATIONAL=y
    CONFIG_GENERIC_PCI_IOMAP=y
    CONFIG_GENERIC_IOMAP=y
    CONFIG_ARCH_USE_CMPXCHG_LOCKREF=y
    CONFIG_ARCH_HAS_FAST_MULTIPLIER=y
    CONFIG_ARCH_USE_SYM_ANNOTATIONS=y
    
    #
    # Crypto library routines
    #
    CONFIG_CRYPTO_LIB_AES=y
    CONFIG_CRYPTO_ARCH_HAVE_LIB_BLAKE2S=y
    CONFIG_CRYPTO_LIB_BLAKE2S_GENERIC=y
    CONFIG_CRYPTO_LIB_POLY1305_RSIZE=11
    CONFIG_CRYPTO_LIB_SHA256=y
    # end of Crypto library routines
    
    CONFIG_LIB_MEMNEQ=y
    CONFIG_CRC_CCITT=y
    CONFIG_CRC16=y
    CONFIG_CRC_T10DIF=y
    CONFIG_CRC32=y
    CONFIG_CRC32_SLICEBY8=y
    CONFIG_LIBCRC32C=y
    CONFIG_XXHASH=y
    CONFIG_ZLIB_INFLATE=y
    CONFIG_ZLIB_DEFLATE=y
    CONFIG_LZO_COMPRESS=y
    CONFIG_LZO_DECOMPRESS=y
    CONFIG_LZ4_DECOMPRESS=y
    CONFIG_ZSTD_COMPRESS=m
    CONFIG_ZSTD_DECOMPRESS=y
    CONFIG_XZ_DEC=y
    CONFIG_XZ_DEC_X86=y
    CONFIG_XZ_DEC_POWERPC=y
    CONFIG_XZ_DEC_IA64=y
    CONFIG_XZ_DEC_ARM=y
    CONFIG_XZ_DEC_ARMTHUMB=y
    CONFIG_XZ_DEC_SPARC=y
    CONFIG_XZ_DEC_BCJ=y
    CONFIG_DECOMPRESS_GZIP=y
    CONFIG_DECOMPRESS_BZIP2=y
    CONFIG_DECOMPRESS_LZMA=y
    CONFIG_DECOMPRESS_XZ=y
    CONFIG_DECOMPRESS_LZO=y
    CONFIG_DECOMPRESS_LZ4=y
    CONFIG_DECOMPRESS_ZSTD=y
    CONFIG_GENERIC_ALLOCATOR=y
    CONFIG_REED_SOLOMON=m
    CONFIG_REED_SOLOMON_ENC8=y
    CONFIG_REED_SOLOMON_DEC8=y
    CONFIG_INTERVAL_TREE=y
    CONFIG_XARRAY_MULTI=y
    CONFIG_ASSOCIATIVE_ARRAY=y
    CONFIG_HAS_IOMEM=y
    CONFIG_HAS_IOPORT_MAP=y
    CONFIG_HAS_DMA=y
    CONFIG_DMA_OPS=y
    CONFIG_NEED_SG_DMA_LENGTH=y
    CONFIG_NEED_DMA_MAP_STATE=y
    CONFIG_ARCH_DMA_ADDR_T_64BIT=y
    CONFIG_ARCH_HAS_FORCE_DMA_UNENCRYPTED=y
    CONFIG_SWIOTLB=y
    CONFIG_DMA_COHERENT_POOL=y
    CONFIG_SGL_ALLOC=y
    CONFIG_IOMMU_HELPER=y
    CONFIG_CHECK_SIGNATURE=y
    CONFIG_CPUMASK_OFFSTACK=y
    CONFIG_CPU_RMAP=y
    CONFIG_DQL=y
    CONFIG_GLOB=y
    CONFIG_NLATTR=y
    CONFIG_CLZ_TAB=y
    CONFIG_IRQ_POLL=y
    CONFIG_MPILIB=y
    CONFIG_SIGNATURE=y
    CONFIG_DIMLIB=y
    CONFIG_OID_REGISTRY=y
    CONFIG_UCS2_STRING=y
    CONFIG_HAVE_GENERIC_VDSO=y
    CONFIG_GENERIC_GETTIMEOFDAY=y
    CONFIG_GENERIC_VDSO_TIME_NS=y
    CONFIG_FONT_SUPPORT=y
    CONFIG_FONTS=y
    CONFIG_FONT_8x8=y
    CONFIG_FONT_8x16=y
    CONFIG_FONT_ACORN_8x8=y
    CONFIG_FONT_6x10=y
    CONFIG_FONT_TER16x32=y
    CONFIG_SG_POOL=y
    CONFIG_ARCH_HAS_PMEM_API=y
    CONFIG_MEMREGION=y
    CONFIG_ARCH_HAS_UACCESS_FLUSHCACHE=y
    CONFIG_ARCH_HAS_COPY_MC=y
    CONFIG_ARCH_STACKWALK=y
    CONFIG_SBITMAP=y
    # end of Library routines
    
    CONFIG_PLDMFW=y
    CONFIG_ASN1_ENCODER=y
    
    #
    # Kernel hacking
    #
    
    #
    # printk and dmesg options
    #
    CONFIG_PRINTK_TIME=y
    CONFIG_CONSOLE_LOGLEVEL_DEFAULT=7
    CONFIG_CONSOLE_LOGLEVEL_QUIET=4
    CONFIG_MESSAGE_LOGLEVEL_DEFAULT=4
    CONFIG_BOOT_PRINTK_DELAY=y
    CONFIG_DYNAMIC_DEBUG=y
    CONFIG_DYNAMIC_DEBUG_CORE=y
    CONFIG_SYMBOLIC_ERRNAME=y
    CONFIG_DEBUG_BUGVERBOSE=y
    # end of printk and dmesg options
    
    CONFIG_AS_HAS_NON_CONST_LEB128=y
    
    #
    # Compile-time checks and compiler options
    #
    CONFIG_DEBUG_INFO=y
    CONFIG_DEBUG_INFO_DWARF_TOOLCHAIN_DEFAULT=y
    CONFIG_GDB_SCRIPTS=y
    CONFIG_FRAME_WARN=1024
    CONFIG_SECTION_MISMATCH_WARN_ONLY=y
    CONFIG_FRAME_POINTER=y
    CONFIG_STACK_VALIDATION=y
    CONFIG_VMLINUX_MAP=y
    # end of Compile-time checks and compiler options
    
    #
    # Generic Kernel Debugging Instruments
    #
    CONFIG_MAGIC_SYSRQ=y
    CONFIG_MAGIC_SYSRQ_DEFAULT_ENABLE=0x01b6
    CONFIG_MAGIC_SYSRQ_SERIAL=y
    CONFIG_MAGIC_SYSRQ_SERIAL_SEQUENCE=""
    CONFIG_DEBUG_FS=y
    CONFIG_DEBUG_FS_ALLOW_ALL=y
    CONFIG_HAVE_ARCH_KGDB=y
    CONFIG_KGDB=y
    CONFIG_KGDB_HONOUR_BLOCKLIST=y
    CONFIG_KGDB_SERIAL_CONSOLE=y
    CONFIG_KGDB_LOW_LEVEL_TRAP=y
    CONFIG_KGDB_KDB=y
    CONFIG_KDB_DEFAULT_ENABLE=0x1
    CONFIG_KDB_KEYBOARD=y
    CONFIG_KDB_CONTINUE_CATASTROPHIC=0
    CONFIG_ARCH_HAS_EARLY_DEBUG=y
    CONFIG_ARCH_HAS_UBSAN_SANITIZE_ALL=y
    CONFIG_UBSAN=y
    CONFIG_CC_HAS_UBSAN_BOUNDS=y
    CONFIG_UBSAN_BOUNDS=y
    CONFIG_UBSAN_ONLY_BOUNDS=y
    CONFIG_UBSAN_SHIFT=y
    CONFIG_UBSAN_BOOL=y
    CONFIG_UBSAN_ENUM=y
    CONFIG_UBSAN_SANITIZE_ALL=y
    CONFIG_HAVE_ARCH_KCSAN=y
    CONFIG_HAVE_KCSAN_COMPILER=y
    # end of Generic Kernel Debugging Instruments
    
    CONFIG_DEBUG_KERNEL=y
    CONFIG_DEBUG_MISC=y
    
    #
    # Memory Debugging
    #
    CONFIG_PAGE_POISONING=y
    CONFIG_ARCH_HAS_DEBUG_WX=y
    CONFIG_DEBUG_WX=y
    CONFIG_GENERIC_PTDUMP=y
    CONFIG_PTDUMP_CORE=y
    CONFIG_HAVE_DEBUG_KMEMLEAK=y
    CONFIG_SCHED_STACK_END_CHECK=y
    CONFIG_ARCH_HAS_DEBUG_VM_PGTABLE=y
    CONFIG_ARCH_HAS_DEBUG_VIRTUAL=y
    CONFIG_HAVE_ARCH_KASAN=y
    CONFIG_HAVE_ARCH_KASAN_VMALLOC=y
    CONFIG_CC_HAS_KASAN_GENERIC=y
    CONFIG_CC_HAS_WORKING_NOSANITIZE_ADDRESS=y
    CONFIG_HAVE_ARCH_KFENCE=y
    CONFIG_KFENCE=y
    CONFIG_KFENCE_SAMPLE_INTERVAL=0
    CONFIG_KFENCE_NUM_OBJECTS=255
    CONFIG_KFENCE_STRESS_TEST_FAULTS=0
    # end of Memory Debugging
    
    
    #
    # Debug Oops, Lockups and Hangs
    #
    CONFIG_PANIC_ON_OOPS_VALUE=0
    CONFIG_PANIC_TIMEOUT=0
    CONFIG_LOCKUP_DETECTOR=y
    CONFIG_SOFTLOCKUP_DETECTOR=y
    CONFIG_BOOTPARAM_SOFTLOCKUP_PANIC_VALUE=0
    CONFIG_HARDLOCKUP_DETECTOR_PERF=y
    CONFIG_HARDLOCKUP_CHECK_TIMESTAMP=y
    CONFIG_HARDLOCKUP_DETECTOR=y
    CONFIG_BOOTPARAM_HARDLOCKUP_PANIC_VALUE=0
    CONFIG_DETECT_HUNG_TASK=y
    CONFIG_DEFAULT_HUNG_TASK_TIMEOUT=120
    CONFIG_BOOTPARAM_HUNG_TASK_PANIC_VALUE=0
    # end of Debug Oops, Lockups and Hangs
    
    #
    # Scheduler Debugging
    #
    CONFIG_SCHED_DEBUG=y
    CONFIG_SCHED_INFO=y
    CONFIG_SCHEDSTATS=y
    # end of Scheduler Debugging
    
    
    #
    # Lock Debugging (spinlocks, mutexes, etc...)
    #
    CONFIG_LOCK_DEBUGGING_SUPPORT=y
    # end of Lock Debugging (spinlocks, mutexes, etc...)
    
    CONFIG_STACKTRACE=y
    
    #
    # Debug kernel data structures
    #
    # end of Debug kernel data structures
    
    
    #
    # RCU Debugging
    #
    CONFIG_RCU_CPU_STALL_TIMEOUT=60
    # end of RCU Debugging
    
    CONFIG_USER_STACKTRACE_SUPPORT=y
    CONFIG_NOP_TRACER=y
    CONFIG_HAVE_FUNCTION_TRACER=y
    CONFIG_HAVE_FUNCTION_GRAPH_TRACER=y
    CONFIG_HAVE_DYNAMIC_FTRACE=y
    CONFIG_HAVE_DYNAMIC_FTRACE_WITH_REGS=y
    CONFIG_HAVE_DYNAMIC_FTRACE_WITH_DIRECT_CALLS=y
    CONFIG_HAVE_DYNAMIC_FTRACE_WITH_ARGS=y
    CONFIG_HAVE_FTRACE_MCOUNT_RECORD=y
    CONFIG_HAVE_SYSCALL_TRACEPOINTS=y
    CONFIG_HAVE_FENTRY=y
    CONFIG_HAVE_OBJTOOL_MCOUNT=y
    CONFIG_HAVE_C_RECORDMCOUNT=y
    CONFIG_TRACER_MAX_TRACE=y
    CONFIG_TRACE_CLOCK=y
    CONFIG_RING_BUFFER=y
    CONFIG_EVENT_TRACING=y
    CONFIG_CONTEXT_SWITCH_TRACER=y
    CONFIG_TRACING=y
    CONFIG_GENERIC_TRACER=y
    CONFIG_TRACING_SUPPORT=y
    CONFIG_FTRACE=y
    CONFIG_BOOTTIME_TRACING=y
    CONFIG_FUNCTION_TRACER=y
    CONFIG_FUNCTION_GRAPH_TRACER=y
    CONFIG_DYNAMIC_FTRACE=y
    CONFIG_DYNAMIC_FTRACE_WITH_REGS=y
    CONFIG_DYNAMIC_FTRACE_WITH_DIRECT_CALLS=y
    CONFIG_DYNAMIC_FTRACE_WITH_ARGS=y
    CONFIG_FUNCTION_PROFILER=y
    CONFIG_STACK_TRACER=y
    CONFIG_SCHED_TRACER=y
    CONFIG_HWLAT_TRACER=y
    CONFIG_MMIOTRACE=y
    CONFIG_FTRACE_SYSCALLS=y
    CONFIG_TRACER_SNAPSHOT=y
    CONFIG_BRANCH_PROFILE_NONE=y
    CONFIG_BLK_DEV_IO_TRACE=y
    CONFIG_KPROBE_EVENTS=y
    CONFIG_UPROBE_EVENTS=y
    CONFIG_BPF_EVENTS=y
    CONFIG_DYNAMIC_EVENTS=y
    CONFIG_PROBE_EVENTS=y
    CONFIG_BPF_KPROBE_OVERRIDE=y
    CONFIG_FTRACE_MCOUNT_RECORD=y
    CONFIG_FTRACE_MCOUNT_USE_CC=y
    CONFIG_TRACING_MAP=y
    CONFIG_SYNTH_EVENTS=y
    CONFIG_HIST_TRIGGERS=y
    CONFIG_TRACE_EVENT_INJECT=y
    CONFIG_SAMPLES=y
    CONFIG_ARCH_HAS_DEVMEM_IS_ALLOWED=y
    CONFIG_STRICT_DEVMEM=y
    
    #
    # x86 Debugging
    #
    CONFIG_EARLY_PRINTK_USB=y
    CONFIG_EARLY_PRINTK=y
    CONFIG_EARLY_PRINTK_DBGP=y
    CONFIG_EARLY_PRINTK_USB_XDBC=y
    CONFIG_HAVE_MMIOTRACE_SUPPORT=y
    CONFIG_IO_DELAY_0XED=y
    CONFIG_X86_DEBUG_FPU=y
    CONFIG_UNWINDER_FRAME_POINTER=y
    # end of x86 Debugging
    
    #
    # Kernel Testing and Coverage
    #
    CONFIG_FUNCTION_ERROR_INJECTION=y
    CONFIG_ARCH_HAS_KCOV=y
    CONFIG_CC_HAS_SANCOV_TRACE_PC=y
    CONFIG_RUNTIME_TESTING_MENU=y
    CONFIG_ARCH_USE_MEMTEST=y
    CONFIG_MEMTEST=y
    # end of Kernel Testing and Coverage
    # end of Kernel hacking
    
\end{lstlisting}
\end{multicols}



%\bibliographystyle{unsrt}\pagestyle{plain}
\bibliography{bib/cites}\pagestyle{plain}
\thispagestyle{plain}%bibtex

% \chapter*{付録}\markboth{付録}{付録}
\addcontentsline{toc}{chapter}{付録}

\label{appendix}
\lstset{%
 basicstyle={\tiny\ttfamily},%
 identifierstyle={\tiny},%
 commentstyle={\tiny\itshape},%
 keywordstyle={\tiny\bfseries},%
 ndkeywordstyle={\tiny\ttfamily},%
 stringstyle={\tiny\ttfamily},
 frame={tb},
 framesep=1zw,
 breaklines=true,
 numbers=left,%
 xrightmargin=0zw,%
 xleftmargin=1.5zw,%
 numberstyle={\scriptsize},%
 stepnumber=1,
 numbersep=1zw,%
 lineskip=-0.5ex,%
}
\begin{multicols}{2}
\begin{lstlisting}[caption=kernel config,label=kconfig,]
    #
    # Automatically generated file; DO NOT EDIT.
    # Linux/x86 5.15.106 Kernel Configuration
    #
    CONFIG_CC_VERSION_TEXT="gcc (Ubuntu 11.3.0-1ubuntu1~22.04.1) 11.3.0"
    CONFIG_CC_IS_GCC=y
    CONFIG_GCC_VERSION=110300
    CONFIG_CLANG_VERSION=0
    CONFIG_AS_IS_GNU=y
    CONFIG_AS_VERSION=23800
    CONFIG_LD_IS_BFD=y
    CONFIG_LD_VERSION=23800
    CONFIG_LLD_VERSION=0
    CONFIG_CC_CAN_LINK=y
    CONFIG_CC_CAN_LINK_STATIC=y
    CONFIG_CC_HAS_ASM_GOTO=y
    CONFIG_CC_HAS_ASM_GOTO_OUTPUT=y
    CONFIG_CC_HAS_ASM_GOTO_TIED_OUTPUT=y
    CONFIG_CC_HAS_ASM_INLINE=y
    CONFIG_CC_HAS_NO_PROFILE_FN_ATTR=y
    CONFIG_PAHOLE_VERSION=0
    CONFIG_IRQ_WORK=y
    CONFIG_BUILDTIME_TABLE_SORT=y
    CONFIG_THREAD_INFO_IN_TASK=y
    
    #
    # General setup
    #
    CONFIG_INIT_ENV_ARG_LIMIT=32
    CONFIG_LOCALVERSION=""
    CONFIG_BUILD_SALT=""
    CONFIG_HAVE_KERNEL_GZIP=y
    CONFIG_HAVE_KERNEL_BZIP2=y
    CONFIG_HAVE_KERNEL_LZMA=y
    CONFIG_HAVE_KERNEL_XZ=y
    CONFIG_HAVE_KERNEL_LZO=y
    CONFIG_HAVE_KERNEL_LZ4=y
    CONFIG_HAVE_KERNEL_ZSTD=y
    CONFIG_KERNEL_ZSTD=y
    CONFIG_DEFAULT_INIT=""
    CONFIG_DEFAULT_HOSTNAME="(none)"
    CONFIG_SWAP=y
    CONFIG_SYSVIPC=y
    CONFIG_SYSVIPC_SYSCTL=y
    CONFIG_POSIX_MQUEUE=y
    CONFIG_POSIX_MQUEUE_SYSCTL=y
    CONFIG_WATCH_QUEUE=y
    CONFIG_CROSS_MEMORY_ATTACH=y
    CONFIG_USELIB=y
    CONFIG_AUDIT=y
    CONFIG_HAVE_ARCH_AUDITSYSCALL=y
    CONFIG_AUDITSYSCALL=y
    
    #
    # IRQ subsystem
    #
    CONFIG_GENERIC_IRQ_PROBE=y
    CONFIG_GENERIC_IRQ_SHOW=y
    CONFIG_GENERIC_IRQ_EFFECTIVE_AFF_MASK=y
    CONFIG_GENERIC_PENDING_IRQ=y
    CONFIG_GENERIC_IRQ_MIGRATION=y
    CONFIG_HARDIRQS_SW_RESEND=y
    CONFIG_IRQ_DOMAIN=y
    CONFIG_IRQ_DOMAIN_HIERARCHY=y
    CONFIG_GENERIC_MSI_IRQ=y
    CONFIG_GENERIC_MSI_IRQ_DOMAIN=y
    CONFIG_IRQ_MSI_IOMMU=y
    CONFIG_GENERIC_IRQ_MATRIX_ALLOCATOR=y
    CONFIG_GENERIC_IRQ_RESERVATION_MODE=y
    CONFIG_IRQ_FORCED_THREADING=y
    CONFIG_SPARSE_IRQ=y
    # end of IRQ subsystem
    
    CONFIG_CLOCKSOURCE_WATCHDOG=y
    CONFIG_ARCH_CLOCKSOURCE_INIT=y
    CONFIG_CLOCKSOURCE_VALIDATE_LAST_CYCLE=y
    CONFIG_GENERIC_TIME_VSYSCALL=y
    CONFIG_GENERIC_CLOCKEVENTS=y
    CONFIG_GENERIC_CLOCKEVENTS_BROADCAST=y
    CONFIG_GENERIC_CLOCKEVENTS_MIN_ADJUST=y
    CONFIG_GENERIC_CMOS_UPDATE=y
    CONFIG_HAVE_POSIX_CPU_TIMERS_TASK_WORK=y
    CONFIG_POSIX_CPU_TIMERS_TASK_WORK=y
    
    #
    # Timers subsystem
    #
    CONFIG_TICK_ONESHOT=y
    CONFIG_NO_HZ_COMMON=y
    CONFIG_NO_HZ_IDLE=y
    CONFIG_NO_HZ=y
    CONFIG_HIGH_RES_TIMERS=y
    # end of Timers subsystem
    
    CONFIG_BPF=y
    CONFIG_HAVE_EBPF_JIT=y
    CONFIG_ARCH_WANT_DEFAULT_BPF_JIT=y
    
    #
    # BPF subsystem
    #
    CONFIG_BPF_SYSCALL=y
    CONFIG_BPF_JIT=y
    CONFIG_BPF_JIT_ALWAYS_ON=y
    CONFIG_BPF_JIT_DEFAULT_ON=y
    CONFIG_BPF_UNPRIV_DEFAULT_OFF=y
    CONFIG_USERMODE_DRIVER=y
    CONFIG_BPF_LSM=y
    # end of BPF subsystem
    
    CONFIG_PREEMPT_VOLUNTARY=y
    CONFIG_SCHED_CORE=y
    
    #
    # CPU/Task time and stats accounting
    #
    CONFIG_TICK_CPU_ACCOUNTING=y
    CONFIG_BSD_PROCESS_ACCT=y
    CONFIG_BSD_PROCESS_ACCT_V3=y
    CONFIG_TASKSTATS=y
    CONFIG_TASK_DELAY_ACCT=y
    CONFIG_TASK_XACCT=y
    CONFIG_TASK_IO_ACCOUNTING=y
    CONFIG_PSI=y
    # end of CPU/Task time and stats accounting
    
    CONFIG_CPU_ISOLATION=y
    
    #
    # RCU Subsystem
    #
    CONFIG_TREE_RCU=y
    CONFIG_SRCU=y
    CONFIG_TREE_SRCU=y
    CONFIG_TASKS_RCU_GENERIC=y
    CONFIG_TASKS_RUDE_RCU=y
    CONFIG_TASKS_TRACE_RCU=y
    CONFIG_RCU_STALL_COMMON=y
    CONFIG_RCU_NEED_SEGCBLIST=y
    # end of RCU Subsystem
    
    CONFIG_BUILD_BIN2C=y
    CONFIG_IKCONFIG=m
    CONFIG_LOG_BUF_SHIFT=18
    CONFIG_LOG_CPU_MAX_BUF_SHIFT=12
    CONFIG_PRINTK_SAFE_LOG_BUF_SHIFT=13
    CONFIG_HAVE_UNSTABLE_SCHED_CLOCK=y
    
    #
    # Scheduler features
    #
    CONFIG_UCLAMP_TASK=y
    CONFIG_UCLAMP_BUCKETS_COUNT=5
    # end of Scheduler features
    
    CONFIG_ARCH_SUPPORTS_NUMA_BALANCING=y
    CONFIG_ARCH_WANT_BATCHED_UNMAP_TLB_FLUSH=y
    CONFIG_CC_HAS_INT128=y
    CONFIG_ARCH_SUPPORTS_INT128=y
    CONFIG_NUMA_BALANCING=y
    CONFIG_NUMA_BALANCING_DEFAULT_ENABLED=y
    CONFIG_CGROUPS=y
    CONFIG_PAGE_COUNTER=y
    CONFIG_MEMCG=y
    CONFIG_MEMCG_SWAP=y
    CONFIG_MEMCG_KMEM=y
    CONFIG_BLK_CGROUP=y
    CONFIG_CGROUP_WRITEBACK=y
    CONFIG_CGROUP_SCHED=y
    CONFIG_FAIR_GROUP_SCHED=y
    CONFIG_CFS_BANDWIDTH=y
    CONFIG_UCLAMP_TASK_GROUP=y
    CONFIG_CGROUP_PIDS=y
    CONFIG_CGROUP_RDMA=y
    CONFIG_CGROUP_FREEZER=y
    CONFIG_CGROUP_HUGETLB=y
    CONFIG_CPUSETS=y
    CONFIG_PROC_PID_CPUSET=y
    CONFIG_CGROUP_DEVICE=y
    CONFIG_CGROUP_CPUACCT=y
    CONFIG_CGROUP_PERF=y
    CONFIG_CGROUP_BPF=y
    CONFIG_CGROUP_MISC=y
    CONFIG_SOCK_CGROUP_DATA=y
    CONFIG_NAMESPACES=y
    CONFIG_UTS_NS=y
    CONFIG_TIME_NS=y
    CONFIG_IPC_NS=y
    CONFIG_USER_NS=y
    CONFIG_PID_NS=y
    CONFIG_NET_NS=y
    CONFIG_CHECKPOINT_RESTORE=y
    CONFIG_SCHED_AUTOGROUP=y
    CONFIG_RELAY=y
    CONFIG_BLK_DEV_INITRD=y
    CONFIG_INITRAMFS_SOURCE=""
    CONFIG_RD_GZIP=y
    CONFIG_RD_BZIP2=y
    CONFIG_RD_LZMA=y
    CONFIG_RD_XZ=y
    CONFIG_RD_LZO=y
    CONFIG_RD_LZ4=y
    CONFIG_RD_ZSTD=y
    CONFIG_BOOT_CONFIG=y
    CONFIG_CC_OPTIMIZE_FOR_PERFORMANCE=y
    CONFIG_LD_ORPHAN_WARN=y
    CONFIG_SYSCTL=y
    CONFIG_HAVE_UID16=y
    CONFIG_SYSCTL_EXCEPTION_TRACE=y
    CONFIG_HAVE_PCSPKR_PLATFORM=y
    CONFIG_EXPERT=y
    CONFIG_UID16=y
    CONFIG_MULTIUSER=y
    CONFIG_SGETMASK_SYSCALL=y
    CONFIG_SYSFS_SYSCALL=y
    CONFIG_FHANDLE=y
    CONFIG_POSIX_TIMERS=y
    CONFIG_PRINTK=y
    CONFIG_BUG=y
    CONFIG_ELF_CORE=y
    CONFIG_PCSPKR_PLATFORM=y
    CONFIG_BASE_FULL=y
    CONFIG_FUTEX=y
    CONFIG_FUTEX_PI=y
    CONFIG_EPOLL=y
    CONFIG_SIGNALFD=y
    CONFIG_TIMERFD=y
    CONFIG_EVENTFD=y
    CONFIG_SHMEM=y
    CONFIG_AIO=y
    CONFIG_IO_URING=y
    CONFIG_ADVISE_SYSCALLS=y
    CONFIG_HAVE_ARCH_USERFAULTFD_WP=y
    CONFIG_HAVE_ARCH_USERFAULTFD_MINOR=y
    CONFIG_MEMBARRIER=y
    CONFIG_KALLSYMS=y
    CONFIG_KALLSYMS_ALL=y
    CONFIG_KALLSYMS_ABSOLUTE_PERCPU=y
    CONFIG_KALLSYMS_BASE_RELATIVE=y
    CONFIG_USERFAULTFD=y
    CONFIG_ARCH_HAS_MEMBARRIER_SYNC_CORE=y
    CONFIG_KCMP=y
    CONFIG_RSEQ=y
    CONFIG_HAVE_PERF_EVENTS=y
    CONFIG_PC104=y
    
    #
    # Kernel Performance Events And Counters
    #
    CONFIG_PERF_EVENTS=y
    # end of Kernel Performance Events And Counters
    
    CONFIG_VM_EVENT_COUNTERS=y
    CONFIG_SLUB_DEBUG=y
    CONFIG_SLUB=y
    CONFIG_SLAB_MERGE_DEFAULT=y
    CONFIG_SLAB_FREELIST_RANDOM=y
    CONFIG_SLAB_FREELIST_HARDENED=y
    CONFIG_SHUFFLE_PAGE_ALLOCATOR=y
    CONFIG_SLUB_CPU_PARTIAL=y
    CONFIG_SYSTEM_DATA_VERIFICATION=y
    CONFIG_PROFILING=y
    CONFIG_TRACEPOINTS=y
    # end of General setup
    
    CONFIG_64BIT=y
    CONFIG_X86_64=y
    CONFIG_X86=y
    CONFIG_INSTRUCTION_DECODER=y
    CONFIG_OUTPUT_FORMAT="elf64-x86-64"
    CONFIG_LOCKDEP_SUPPORT=y
    CONFIG_STACKTRACE_SUPPORT=y
    CONFIG_MMU=y
    CONFIG_ARCH_MMAP_RND_BITS_MIN=28
    CONFIG_ARCH_MMAP_RND_BITS_MAX=32
    CONFIG_ARCH_MMAP_RND_COMPAT_BITS_MIN=8
    CONFIG_ARCH_MMAP_RND_COMPAT_BITS_MAX=16
    CONFIG_GENERIC_ISA_DMA=y
    CONFIG_GENERIC_BUG=y
    CONFIG_GENERIC_BUG_RELATIVE_POINTERS=y
    CONFIG_ARCH_MAY_HAVE_PC_FDC=y
    CONFIG_GENERIC_CALIBRATE_DELAY=y
    CONFIG_ARCH_HAS_CPU_RELAX=y
    CONFIG_ARCH_HAS_FILTER_PGPROT=y
    CONFIG_HAVE_SETUP_PER_CPU_AREA=y
    CONFIG_NEED_PER_CPU_EMBED_FIRST_CHUNK=y
    CONFIG_NEED_PER_CPU_PAGE_FIRST_CHUNK=y
    CONFIG_ARCH_HIBERNATION_POSSIBLE=y
    CONFIG_ARCH_NR_GPIO=1024
    CONFIG_ARCH_SUSPEND_POSSIBLE=y
    CONFIG_ARCH_WANT_GENERAL_HUGETLB=y
    CONFIG_AUDIT_ARCH=y
    CONFIG_HAVE_INTEL_TXT=y
    CONFIG_X86_64_SMP=y
    CONFIG_ARCH_SUPPORTS_UPROBES=y
    CONFIG_FIX_EARLYCON_MEM=y
    CONFIG_DYNAMIC_PHYSICAL_MASK=y
    CONFIG_PGTABLE_LEVELS=5
    CONFIG_CC_HAS_SANE_STACKPROTECTOR=y
    
    #
    # Processor type and features
    #
    CONFIG_SMP=y
    CONFIG_X86_FEATURE_NAMES=y
    CONFIG_X86_X2APIC=y
    CONFIG_X86_MPPARSE=y
    CONFIG_X86_CPU_RESCTRL=y
    CONFIG_X86_EXTENDED_PLATFORM=y
    CONFIG_X86_NUMACHIP=y
    CONFIG_X86_UV=y
    CONFIG_X86_INTEL_LPSS=y
    CONFIG_X86_AMD_PLATFORM_DEVICE=y
    CONFIG_IOSF_MBI=y
    CONFIG_IOSF_MBI_DEBUG=y
    CONFIG_X86_SUPPORTS_MEMORY_FAILURE=y
    CONFIG_SCHED_OMIT_FRAME_POINTER=y
    CONFIG_HYPERVISOR_GUEST=y
    CONFIG_PARAVIRT=y
    CONFIG_PARAVIRT_XXL=y
    CONFIG_PARAVIRT_SPINLOCKS=y
    CONFIG_X86_HV_CALLBACK_VECTOR=y
    CONFIG_XEN=y
    CONFIG_XEN_PV=y
    CONFIG_XEN_512GB=y
    CONFIG_XEN_PV_SMP=y
    CONFIG_XEN_PV_DOM0=y
    CONFIG_XEN_PVHVM=y
    CONFIG_XEN_PVHVM_SMP=y
    CONFIG_XEN_PVHVM_GUEST=y
    CONFIG_XEN_SAVE_RESTORE=y
    CONFIG_XEN_PVH=y
    CONFIG_XEN_DOM0=y
    CONFIG_KVM_GUEST=y
    CONFIG_ARCH_CPUIDLE_HALTPOLL=y
    CONFIG_PVH=y
    CONFIG_PARAVIRT_CLOCK=y
    CONFIG_JAILHOUSE_GUEST=y
    CONFIG_ACRN_GUEST=y
    CONFIG_GENERIC_CPU=y
    CONFIG_X86_INTERNODE_CACHE_SHIFT=6
    CONFIG_X86_L1_CACHE_SHIFT=6
    CONFIG_X86_TSC=y
    CONFIG_X86_CMPXCHG64=y
    CONFIG_X86_CMOV=y
    CONFIG_X86_MINIMUM_CPU_FAMILY=64
    CONFIG_X86_DEBUGCTLMSR=y
    CONFIG_IA32_FEAT_CTL=y
    CONFIG_X86_VMX_FEATURE_NAMES=y
    CONFIG_PROCESSOR_SELECT=y
    CONFIG_CPU_SUP_INTEL=y
    CONFIG_CPU_SUP_AMD=y
    CONFIG_CPU_SUP_HYGON=y
    CONFIG_CPU_SUP_CENTAUR=y
    CONFIG_CPU_SUP_ZHAOXIN=y
    CONFIG_HPET_TIMER=y
    CONFIG_HPET_EMULATE_RTC=y
    CONFIG_DMI=y
    CONFIG_GART_IOMMU=y
    CONFIG_MAXSMP=y
    CONFIG_NR_CPUS_RANGE_BEGIN=8192
    CONFIG_NR_CPUS_RANGE_END=8192
    CONFIG_NR_CPUS_DEFAULT=8192
    CONFIG_NR_CPUS=8192
    CONFIG_SCHED_SMT=y
    CONFIG_SCHED_MC=y
    CONFIG_SCHED_MC_PRIO=y
    CONFIG_X86_LOCAL_APIC=y
    CONFIG_X86_IO_APIC=y
    CONFIG_X86_REROUTE_FOR_BROKEN_BOOT_IRQS=y
    CONFIG_X86_MCE=y
    CONFIG_X86_MCELOG_LEGACY=y
    CONFIG_X86_MCE_INTEL=y
    CONFIG_X86_MCE_AMD=y
    CONFIG_X86_MCE_THRESHOLD=y
    
    #
    # Performance monitoring
    #
    CONFIG_PERF_EVENTS_INTEL_UNCORE=y
    CONFIG_PERF_EVENTS_INTEL_RAPL=m
    CONFIG_PERF_EVENTS_INTEL_CSTATE=m
    # end of Performance monitoring
    
    CONFIG_X86_16BIT=y
    CONFIG_X86_ESPFIX64=y
    CONFIG_X86_VSYSCALL_EMULATION=y
    CONFIG_X86_IOPL_IOPERM=y
    CONFIG_MICROCODE=y
    CONFIG_MICROCODE_INTEL=y
    CONFIG_MICROCODE_AMD=y
    CONFIG_X86_MSR=m
    CONFIG_X86_5LEVEL=y
    CONFIG_X86_DIRECT_GBPAGES=y
    CONFIG_AMD_MEM_ENCRYPT=y
    CONFIG_NUMA=y
    CONFIG_AMD_NUMA=y
    CONFIG_X86_64_ACPI_NUMA=y
    CONFIG_NODES_SHIFT=10
    CONFIG_ARCH_SPARSEMEM_ENABLE=y
    CONFIG_ARCH_SPARSEMEM_DEFAULT=y
    CONFIG_ARCH_SELECT_MEMORY_MODEL=y
    CONFIG_ARCH_MEMORY_PROBE=y
    CONFIG_ARCH_PROC_KCORE_TEXT=y
    CONFIG_ILLEGAL_POINTER_VALUE=0xdead000000000000
    CONFIG_X86_PMEM_LEGACY_DEVICE=y
    CONFIG_X86_PMEM_LEGACY=y
    CONFIG_X86_CHECK_BIOS_CORRUPTION=y
    CONFIG_X86_BOOTPARAM_MEMORY_CORRUPTION_CHECK=y
    CONFIG_MTRR=y
    CONFIG_MTRR_SANITIZER=y
    CONFIG_MTRR_SANITIZER_ENABLE_DEFAULT=1
    CONFIG_MTRR_SANITIZER_SPARE_REG_NR_DEFAULT=1
    CONFIG_X86_PAT=y
    CONFIG_ARCH_USES_PG_UNCACHED=y
    CONFIG_ARCH_RANDOM=y
    CONFIG_X86_SMAP=y
    CONFIG_X86_UMIP=y
    CONFIG_X86_INTEL_MEMORY_PROTECTION_KEYS=y
    CONFIG_X86_INTEL_TSX_MODE_OFF=y
    CONFIG_X86_SGX=y
    CONFIG_EFI=y
    CONFIG_EFI_STUB=y
    CONFIG_EFI_MIXED=y
    CONFIG_HZ_250=y
    CONFIG_HZ=250
    CONFIG_SCHED_HRTICK=y
    CONFIG_KEXEC=y
    CONFIG_KEXEC_FILE=y
    CONFIG_ARCH_HAS_KEXEC_PURGATORY=y
    CONFIG_KEXEC_SIG=y
    CONFIG_KEXEC_BZIMAGE_VERIFY_SIG=y
    CONFIG_CRASH_DUMP=y
    CONFIG_KEXEC_JUMP=y
    CONFIG_PHYSICAL_START=0x1000000
    CONFIG_RELOCATABLE=y
    CONFIG_RANDOMIZE_BASE=y
    CONFIG_X86_NEED_RELOCS=y
    CONFIG_PHYSICAL_ALIGN=0x200000
    CONFIG_DYNAMIC_MEMORY_LAYOUT=y
    CONFIG_RANDOMIZE_MEMORY=y
    CONFIG_RANDOMIZE_MEMORY_PHYSICAL_PADDING=0xa
    CONFIG_HOTPLUG_CPU=y
    CONFIG_LEGACY_VSYSCALL_XONLY=y
    CONFIG_MODIFY_LDT_SYSCALL=y
    CONFIG_HAVE_LIVEPATCH=y
    CONFIG_LIVEPATCH=y
    # end of Processor type and features
    
    CONFIG_CC_HAS_SLS=y
    CONFIG_CC_HAS_RETURN_THUNK=y
    CONFIG_SPECULATION_MITIGATIONS=y
    CONFIG_PAGE_TABLE_ISOLATION=y
    CONFIG_RETPOLINE=y
    CONFIG_RETHUNK=y
    CONFIG_CPU_UNRET_ENTRY=y
    CONFIG_CPU_IBPB_ENTRY=y
    CONFIG_CPU_IBRS_ENTRY=y
    CONFIG_SLS=y
    CONFIG_ARCH_HAS_ADD_PAGES=y
    CONFIG_ARCH_MHP_MEMMAP_ON_MEMORY_ENABLE=y
    CONFIG_USE_PERCPU_NUMA_NODE_ID=y
    
    #
    # Power management and ACPI options
    #
    CONFIG_ARCH_HIBERNATION_HEADER=y
    CONFIG_SUSPEND=y
    CONFIG_SUSPEND_FREEZER=y
    CONFIG_HIBERNATE_CALLBACKS=y
    CONFIG_HIBERNATION=y
    CONFIG_HIBERNATION_SNAPSHOT_DEV=y
    CONFIG_PM_STD_PARTITION=""
    CONFIG_PM_SLEEP=y
    CONFIG_PM_SLEEP_SMP=y
    CONFIG_PM_WAKELOCKS=y
    CONFIG_PM_WAKELOCKS_LIMIT=100
    CONFIG_PM_WAKELOCKS_GC=y
    CONFIG_PM=y
    CONFIG_PM_DEBUG=y
    CONFIG_PM_ADVANCED_DEBUG=y
    CONFIG_PM_SLEEP_DEBUG=y
    CONFIG_PM_TRACE=y
    CONFIG_PM_TRACE_RTC=y
    CONFIG_PM_CLK=y
    CONFIG_WQ_POWER_EFFICIENT_DEFAULT=y
    CONFIG_ENERGY_MODEL=y
    CONFIG_ARCH_SUPPORTS_ACPI=y
    CONFIG_ACPI=y
    CONFIG_ACPI_LEGACY_TABLES_LOOKUP=y
    CONFIG_ARCH_MIGHT_HAVE_ACPI_PDC=y
    CONFIG_ACPI_SYSTEM_POWER_STATES_SUPPORT=y
    CONFIG_ACPI_DEBUGGER=y
    CONFIG_ACPI_DEBUGGER_USER=y
    CONFIG_ACPI_SPCR_TABLE=y
    CONFIG_ACPI_FPDT=y
    CONFIG_ACPI_LPIT=y
    CONFIG_ACPI_SLEEP=y
    CONFIG_ACPI_REV_OVERRIDE_POSSIBLE=y
    CONFIG_ACPI_AC=y
    CONFIG_ACPI_BATTERY=y
    CONFIG_ACPI_BUTTON=y
    CONFIG_ACPI_FAN=y
    CONFIG_ACPI_DOCK=y
    CONFIG_ACPI_CPU_FREQ_PSS=y
    CONFIG_ACPI_PROCESSOR_CSTATE=y
    CONFIG_ACPI_PROCESSOR_IDLE=y
    CONFIG_ACPI_CPPC_LIB=y
    CONFIG_ACPI_PROCESSOR=y
    CONFIG_ACPI_IPMI=m
    CONFIG_ACPI_HOTPLUG_CPU=y
    CONFIG_ACPI_THERMAL=y
    CONFIG_ACPI_CUSTOM_DSDT_FILE=""
    CONFIG_ARCH_HAS_ACPI_TABLE_UPGRADE=y
    CONFIG_ACPI_TABLE_UPGRADE=y
    CONFIG_ACPI_DEBUG=y
    CONFIG_ACPI_PCI_SLOT=y
    CONFIG_ACPI_CONTAINER=y
    CONFIG_ACPI_HOTPLUG_MEMORY=y
    CONFIG_ACPI_HOTPLUG_IOAPIC=y
    CONFIG_ACPI_HED=y
    CONFIG_ACPI_BGRT=y
    CONFIG_ACPI_NFIT=m
    CONFIG_ACPI_NUMA=y
    CONFIG_ACPI_HMAT=y
    CONFIG_HAVE_ACPI_APEI=y
    CONFIG_HAVE_ACPI_APEI_NMI=y
    CONFIG_ACPI_APEI=y
    CONFIG_ACPI_APEI_GHES=y
    CONFIG_ACPI_APEI_PCIEAER=y
    CONFIG_ACPI_APEI_MEMORY_FAILURE=y
    CONFIG_ACPI_DPTF=y
    CONFIG_ACPI_ADXL=y
    CONFIG_PMIC_OPREGION=y
    CONFIG_BYTCRC_PMIC_OPREGION=y
    CONFIG_CHTCRC_PMIC_OPREGION=y
    CONFIG_CHT_WC_PMIC_OPREGION=y
    CONFIG_ACPI_VIOT=y
    CONFIG_X86_PM_TIMER=y
    CONFIG_ACPI_PRMT=y
    
    #
    # CPU Frequency scaling
    #
    CONFIG_CPU_FREQ=y
    CONFIG_CPU_FREQ_GOV_ATTR_SET=y
    CONFIG_CPU_FREQ_GOV_COMMON=y
    CONFIG_CPU_FREQ_STAT=y
    CONFIG_CPU_FREQ_DEFAULT_GOV_SCHEDUTIL=y
    CONFIG_CPU_FREQ_GOV_PERFORMANCE=y
    CONFIG_CPU_FREQ_GOV_POWERSAVE=y
    CONFIG_CPU_FREQ_GOV_USERSPACE=y
    CONFIG_CPU_FREQ_GOV_ONDEMAND=y
    CONFIG_CPU_FREQ_GOV_CONSERVATIVE=y
    CONFIG_CPU_FREQ_GOV_SCHEDUTIL=y
    
    #
    # CPU frequency scaling drivers
    #
    CONFIG_X86_INTEL_PSTATE=y
    CONFIG_X86_PCC_CPUFREQ=y
    CONFIG_X86_ACPI_CPUFREQ=y
    CONFIG_X86_ACPI_CPUFREQ_CPB=y
    CONFIG_X86_POWERNOW_K8=y
    CONFIG_X86_SPEEDSTEP_CENTRINO=y
    
    #
    # shared options
    #
    # end of CPU Frequency scaling
    
    #
    # CPU Idle
    #
    CONFIG_CPU_IDLE=y
    CONFIG_CPU_IDLE_GOV_LADDER=y
    CONFIG_CPU_IDLE_GOV_MENU=y
    CONFIG_CPU_IDLE_GOV_TEO=y
    CONFIG_CPU_IDLE_GOV_HALTPOLL=y
    # end of CPU Idle
    
    CONFIG_INTEL_IDLE=y
    # end of Power management and ACPI options
    
    #
    # Bus options (PCI etc.)
    #
    CONFIG_PCI_DIRECT=y
    CONFIG_PCI_MMCONFIG=y
    CONFIG_PCI_XEN=y
    CONFIG_MMCONF_FAM10H=y
    CONFIG_ISA_BUS=y
    CONFIG_ISA_DMA_API=y
    CONFIG_AMD_NB=y
    # end of Bus options (PCI etc.)
    
    #
    # Binary Emulations
    #
    CONFIG_IA32_EMULATION=y
    CONFIG_X86_X32=y
    CONFIG_COMPAT_32=y
    CONFIG_COMPAT=y
    CONFIG_COMPAT_FOR_U64_ALIGNMENT=y
    CONFIG_SYSVIPC_COMPAT=y
    # end of Binary Emulations
    
    CONFIG_HAVE_KVM=y
    CONFIG_VIRTUALIZATION=y
    CONFIG_AS_AVX512=y
    CONFIG_AS_SHA1_NI=y
    CONFIG_AS_SHA256_NI=y
    CONFIG_AS_TPAUSE=y
    
    #
    # General architecture-dependent options
    #
    CONFIG_CRASH_CORE=y
    CONFIG_KEXEC_CORE=y
    CONFIG_HOTPLUG_SMT=y
    CONFIG_GENERIC_ENTRY=y
    CONFIG_KPROBES=y
    CONFIG_JUMP_LABEL=y
    CONFIG_OPTPROBES=y
    CONFIG_KPROBES_ON_FTRACE=y
    CONFIG_UPROBES=y
    CONFIG_HAVE_EFFICIENT_UNALIGNED_ACCESS=y
    CONFIG_ARCH_USE_BUILTIN_BSWAP=y
    CONFIG_KRETPROBES=y
    CONFIG_HAVE_IOREMAP_PROT=y
    CONFIG_HAVE_KPROBES=y
    CONFIG_HAVE_KRETPROBES=y
    CONFIG_HAVE_OPTPROBES=y
    CONFIG_HAVE_KPROBES_ON_FTRACE=y
    CONFIG_HAVE_FUNCTION_ERROR_INJECTION=y
    CONFIG_HAVE_NMI=y
    CONFIG_TRACE_IRQFLAGS_SUPPORT=y
    CONFIG_TRACE_IRQFLAGS_NMI_SUPPORT=y
    CONFIG_HAVE_ARCH_TRACEHOOK=y
    CONFIG_HAVE_DMA_CONTIGUOUS=y
    CONFIG_GENERIC_SMP_IDLE_THREAD=y
    CONFIG_ARCH_HAS_FORTIFY_SOURCE=y
    CONFIG_ARCH_HAS_SET_MEMORY=y
    CONFIG_ARCH_HAS_SET_DIRECT_MAP=y
    CONFIG_HAVE_ARCH_THREAD_STRUCT_WHITELIST=y
    CONFIG_ARCH_WANTS_DYNAMIC_TASK_STRUCT=y
    CONFIG_ARCH_WANTS_NO_INSTR=y
    CONFIG_HAVE_ASM_MODVERSIONS=y
    CONFIG_HAVE_REGS_AND_STACK_ACCESS_API=y
    CONFIG_HAVE_RSEQ=y
    CONFIG_HAVE_FUNCTION_ARG_ACCESS_API=y
    CONFIG_HAVE_HW_BREAKPOINT=y
    CONFIG_HAVE_MIXED_BREAKPOINTS_REGS=y
    CONFIG_HAVE_USER_RETURN_NOTIFIER=y
    CONFIG_HAVE_PERF_EVENTS_NMI=y
    CONFIG_HAVE_HARDLOCKUP_DETECTOR_PERF=y
    CONFIG_HAVE_PERF_REGS=y
    CONFIG_HAVE_PERF_USER_STACK_DUMP=y
    CONFIG_HAVE_ARCH_JUMP_LABEL=y
    CONFIG_HAVE_ARCH_JUMP_LABEL_RELATIVE=y
    CONFIG_MMU_GATHER_TABLE_FREE=y
    CONFIG_MMU_GATHER_RCU_TABLE_FREE=y
    CONFIG_ARCH_HAVE_NMI_SAFE_CMPXCHG=y
    CONFIG_HAVE_ALIGNED_STRUCT_PAGE=y
    CONFIG_HAVE_CMPXCHG_LOCAL=y
    CONFIG_HAVE_CMPXCHG_DOUBLE=y
    CONFIG_ARCH_WANT_COMPAT_IPC_PARSE_VERSION=y
    CONFIG_ARCH_WANT_OLD_COMPAT_IPC=y
    CONFIG_HAVE_ARCH_SECCOMP=y
    CONFIG_HAVE_ARCH_SECCOMP_FILTER=y
    CONFIG_SECCOMP=y
    CONFIG_SECCOMP_FILTER=y
    CONFIG_HAVE_ARCH_STACKLEAK=y
    CONFIG_HAVE_STACKPROTECTOR=y
    CONFIG_STACKPROTECTOR=y
    CONFIG_STACKPROTECTOR_STRONG=y
    CONFIG_ARCH_SUPPORTS_LTO_CLANG=y
    CONFIG_ARCH_SUPPORTS_LTO_CLANG_THIN=y
    CONFIG_LTO_NONE=y
    CONFIG_HAVE_ARCH_WITHIN_STACK_FRAMES=y
    CONFIG_HAVE_CONTEXT_TRACKING=y
    CONFIG_HAVE_CONTEXT_TRACKING_OFFSTACK=y
    CONFIG_HAVE_VIRT_CPU_ACCOUNTING_GEN=y
    CONFIG_HAVE_IRQ_TIME_ACCOUNTING=y
    CONFIG_HAVE_MOVE_PUD=y
    CONFIG_HAVE_MOVE_PMD=y
    CONFIG_HAVE_ARCH_TRANSPARENT_HUGEPAGE=y
    CONFIG_HAVE_ARCH_TRANSPARENT_HUGEPAGE_PUD=y
    CONFIG_HAVE_ARCH_HUGE_VMAP=y
    CONFIG_ARCH_WANT_HUGE_PMD_SHARE=y
    CONFIG_HAVE_ARCH_SOFT_DIRTY=y
    CONFIG_HAVE_MOD_ARCH_SPECIFIC=y
    CONFIG_MODULES_USE_ELF_RELA=y
    CONFIG_HAVE_IRQ_EXIT_ON_IRQ_STACK=y
    CONFIG_HAVE_SOFTIRQ_ON_OWN_STACK=y
    CONFIG_ARCH_HAS_ELF_RANDOMIZE=y
    CONFIG_HAVE_ARCH_MMAP_RND_BITS=y
    CONFIG_HAVE_EXIT_THREAD=y
    CONFIG_ARCH_MMAP_RND_BITS=28
    CONFIG_HAVE_ARCH_MMAP_RND_COMPAT_BITS=y
    CONFIG_ARCH_MMAP_RND_COMPAT_BITS=8
    CONFIG_HAVE_ARCH_COMPAT_MMAP_BASES=y
    CONFIG_HAVE_STACK_VALIDATION=y
    CONFIG_HAVE_RELIABLE_STACKTRACE=y
    CONFIG_OLD_SIGSUSPEND3=y
    CONFIG_COMPAT_OLD_SIGACTION=y
    CONFIG_COMPAT_32BIT_TIME=y
    CONFIG_HAVE_ARCH_VMAP_STACK=y
    CONFIG_VMAP_STACK=y
    CONFIG_HAVE_ARCH_RANDOMIZE_KSTACK_OFFSET=y
    CONFIG_RANDOMIZE_KSTACK_OFFSET_DEFAULT=y
    CONFIG_ARCH_HAS_STRICT_KERNEL_RWX=y
    CONFIG_STRICT_KERNEL_RWX=y
    CONFIG_ARCH_HAS_STRICT_MODULE_RWX=y
    CONFIG_STRICT_MODULE_RWX=y
    CONFIG_HAVE_ARCH_PREL32_RELOCATIONS=y
    CONFIG_ARCH_USE_MEMREMAP_PROT=y
    CONFIG_ARCH_HAS_MEM_ENCRYPT=y
    CONFIG_ARCH_HAS_CC_PLATFORM=y
    CONFIG_HAVE_STATIC_CALL=y
    CONFIG_HAVE_STATIC_CALL_INLINE=y
    CONFIG_HAVE_PREEMPT_DYNAMIC=y
    CONFIG_ARCH_WANT_LD_ORPHAN_WARN=y
    CONFIG_ARCH_SUPPORTS_DEBUG_PAGEALLOC=y
    CONFIG_ARCH_HAS_ELFCORE_COMPAT=y
    CONFIG_ARCH_HAS_PARANOID_L1D_FLUSH=y
    
    #
    # GCOV-based kernel profiling
    #
    CONFIG_ARCH_HAS_GCOV_PROFILE_ALL=y
    # end of GCOV-based kernel profiling
    
    CONFIG_HAVE_GCC_PLUGINS=y
    # end of General architecture-dependent options
    
    CONFIG_RT_MUTEXES=y
    CONFIG_BASE_SMALL=0
    CONFIG_MODULE_SIG_FORMAT=y
    CONFIG_MODULES=y
    CONFIG_MODULE_UNLOAD=y
    CONFIG_MODVERSIONS=y
    CONFIG_ASM_MODVERSIONS=y
    CONFIG_MODULE_SRCVERSION_ALL=y
    CONFIG_MODULE_SIG=y
    CONFIG_MODULE_SIG_ALL=y
    CONFIG_MODULE_SIG_SHA512=y
    CONFIG_MODULE_SIG_HASH="sha512"
    CONFIG_MODULE_COMPRESS_NONE=y
    CONFIG_MODPROBE_PATH="/sbin/modprobe"
    CONFIG_MODULES_TREE_LOOKUP=y
    CONFIG_BLOCK=y
    CONFIG_BLK_RQ_ALLOC_TIME=y
    CONFIG_BLK_CGROUP_RWSTAT=y
    CONFIG_BLK_DEV_BSG_COMMON=y
    CONFIG_BLK_DEV_BSGLIB=y
    CONFIG_BLK_DEV_INTEGRITY=y
    CONFIG_BLK_DEV_INTEGRITY_T10=y
    CONFIG_BLK_DEV_ZONED=y
    CONFIG_BLK_DEV_THROTTLING=y
    CONFIG_BLK_WBT=y
    CONFIG_BLK_WBT_MQ=y
    CONFIG_BLK_CGROUP_IOCOST=y
    CONFIG_BLK_CGROUP_IOPRIO=y
    CONFIG_BLK_DEBUG_FS=y
    CONFIG_BLK_DEBUG_FS_ZONED=y
    CONFIG_BLK_SED_OPAL=y
    CONFIG_BLK_INLINE_ENCRYPTION=y
    CONFIG_BLK_INLINE_ENCRYPTION_FALLBACK=y
    
    #
    # Partition Types
    #
    CONFIG_PARTITION_ADVANCED=y
    CONFIG_AIX_PARTITION=y
    CONFIG_OSF_PARTITION=y
    CONFIG_AMIGA_PARTITION=y
    CONFIG_ATARI_PARTITION=y
    CONFIG_MAC_PARTITION=y
    CONFIG_MSDOS_PARTITION=y
    CONFIG_BSD_DISKLABEL=y
    CONFIG_MINIX_SUBPARTITION=y
    CONFIG_SOLARIS_X86_PARTITION=y
    CONFIG_UNIXWARE_DISKLABEL=y
    CONFIG_LDM_PARTITION=y
    CONFIG_SGI_PARTITION=y
    CONFIG_ULTRIX_PARTITION=y
    CONFIG_SUN_PARTITION=y
    CONFIG_KARMA_PARTITION=y
    CONFIG_EFI_PARTITION=y
    CONFIG_SYSV68_PARTITION=y
    CONFIG_CMDLINE_PARTITION=y
    # end of Partition Types
    
    CONFIG_BLOCK_COMPAT=y
    CONFIG_BLK_MQ_PCI=y
    CONFIG_BLK_MQ_VIRTIO=y
    CONFIG_BLK_MQ_RDMA=y
    CONFIG_BLK_PM=y
    CONFIG_BLOCK_HOLDER_DEPRECATED=y
    
    #
    # IO Schedulers
    #
    CONFIG_MQ_IOSCHED_DEADLINE=y
    # end of IO Schedulers
    
    CONFIG_ASN1=y
    CONFIG_INLINE_SPIN_UNLOCK_IRQ=y
    CONFIG_INLINE_READ_UNLOCK=y
    CONFIG_INLINE_READ_UNLOCK_IRQ=y
    CONFIG_INLINE_WRITE_UNLOCK=y
    CONFIG_INLINE_WRITE_UNLOCK_IRQ=y
    CONFIG_ARCH_SUPPORTS_ATOMIC_RMW=y
    CONFIG_MUTEX_SPIN_ON_OWNER=y
    CONFIG_RWSEM_SPIN_ON_OWNER=y
    CONFIG_LOCK_SPIN_ON_OWNER=y
    CONFIG_ARCH_USE_QUEUED_SPINLOCKS=y
    CONFIG_QUEUED_SPINLOCKS=y
    CONFIG_ARCH_USE_QUEUED_RWLOCKS=y
    CONFIG_QUEUED_RWLOCKS=y
    CONFIG_ARCH_HAS_NON_OVERLAPPING_ADDRESS_SPACE=y
    CONFIG_ARCH_HAS_SYNC_CORE_BEFORE_USERMODE=y
    CONFIG_ARCH_HAS_SYSCALL_WRAPPER=y
    CONFIG_FREEZER=y
    
    #
    # Executable file formats
    #
    CONFIG_BINFMT_ELF=y
    CONFIG_COMPAT_BINFMT_ELF=y
    CONFIG_ELFCORE=y
    CONFIG_CORE_DUMP_DEFAULT_ELF_HEADERS=y
    CONFIG_BINFMT_SCRIPT=y
    CONFIG_BINFMT_MISC=m
    CONFIG_COREDUMP=y
    # end of Executable file formats
    
    #
    # Memory Management options
    #
    CONFIG_SELECT_MEMORY_MODEL=y
    CONFIG_SPARSEMEM_MANUAL=y
    CONFIG_SPARSEMEM=y
    CONFIG_SPARSEMEM_EXTREME=y
    CONFIG_SPARSEMEM_VMEMMAP_ENABLE=y
    CONFIG_SPARSEMEM_VMEMMAP=y
    CONFIG_HAVE_FAST_GUP=y
    CONFIG_NUMA_KEEP_MEMINFO=y
    CONFIG_MEMORY_ISOLATION=y
    CONFIG_HAVE_BOOTMEM_INFO_NODE=y
    CONFIG_ARCH_ENABLE_MEMORY_HOTPLUG=y
    CONFIG_MEMORY_HOTPLUG=y
    CONFIG_MEMORY_HOTPLUG_SPARSE=y
    CONFIG_MEMORY_HOTPLUG_DEFAULT_ONLINE=y
    CONFIG_ARCH_ENABLE_MEMORY_HOTREMOVE=y
    CONFIG_MEMORY_HOTREMOVE=y
    CONFIG_MHP_MEMMAP_ON_MEMORY=y
    CONFIG_SPLIT_PTLOCK_CPUS=4
    CONFIG_ARCH_ENABLE_SPLIT_PMD_PTLOCK=y
    CONFIG_MEMORY_BALLOON=y
    CONFIG_BALLOON_COMPACTION=y
    CONFIG_COMPACTION=y
    CONFIG_PAGE_REPORTING=y
    CONFIG_MIGRATION=y
    CONFIG_ARCH_ENABLE_HUGEPAGE_MIGRATION=y
    CONFIG_ARCH_ENABLE_THP_MIGRATION=y
    CONFIG_CONTIG_ALLOC=y
    CONFIG_PHYS_ADDR_T_64BIT=y
    CONFIG_VIRT_TO_BUS=y
    CONFIG_MMU_NOTIFIER=y
    CONFIG_KSM=y
    CONFIG_DEFAULT_MMAP_MIN_ADDR=65536
    CONFIG_ARCH_SUPPORTS_MEMORY_FAILURE=y
    CONFIG_MEMORY_FAILURE=y
    CONFIG_TRANSPARENT_HUGEPAGE=y
    CONFIG_TRANSPARENT_HUGEPAGE_MADVISE=y
    CONFIG_ARCH_WANTS_THP_SWAP=y
    CONFIG_THP_SWAP=y
    CONFIG_CLEANCACHE=y
    CONFIG_FRONTSWAP=y
    CONFIG_MEM_SOFT_DIRTY=y
    CONFIG_ZSWAP=y
    CONFIG_ZSWAP_COMPRESSOR_DEFAULT_LZO=y
    CONFIG_ZSWAP_COMPRESSOR_DEFAULT="lzo"
    CONFIG_ZSWAP_ZPOOL_DEFAULT_ZBUD=y
    CONFIG_ZSWAP_ZPOOL_DEFAULT="zbud"
    CONFIG_ZPOOL=y
    CONFIG_ZBUD=y
    CONFIG_ZSMALLOC=y
    CONFIG_GENERIC_EARLY_IOREMAP=y
    CONFIG_PAGE_IDLE_FLAG=y
    CONFIG_IDLE_PAGE_TRACKING=y
    CONFIG_ARCH_HAS_CACHE_LINE_SIZE=y
    CONFIG_ARCH_HAS_PTE_DEVMAP=y
    CONFIG_ARCH_HAS_ZONE_DMA_SET=y
    CONFIG_ZONE_DMA=y
    CONFIG_ZONE_DMA32=y
    CONFIG_ZONE_DEVICE=y
    CONFIG_DEV_PAGEMAP_OPS=y
    CONFIG_HMM_MIRROR=y
    CONFIG_DEVICE_PRIVATE=y
    CONFIG_ARCH_USES_HIGH_VMA_FLAGS=y
    CONFIG_ARCH_HAS_PKEYS=y
    CONFIG_ARCH_HAS_PTE_SPECIAL=y
    CONFIG_SECRETMEM=y
    
    #
    # Data Access Monitoring
    #
    # end of Data Access Monitoring
    # end of Memory Management options
    
    CONFIG_NET=y
    CONFIG_NET_INGRESS=y
    CONFIG_SKB_EXTENSIONS=y
    
    #
    # Networking options
    #
    CONFIG_PACKET=y
    CONFIG_UNIX=y
    CONFIG_UNIX_SCM=y
    CONFIG_AF_UNIX_OOB=y
    CONFIG_XDP_SOCKETS=y
    CONFIG_INET=y
    CONFIG_IP_MULTICAST=y
    CONFIG_IP_ADVANCED_ROUTER=y
    CONFIG_IP_FIB_TRIE_STATS=y
    CONFIG_IP_MULTIPLE_TABLES=y
    CONFIG_IP_ROUTE_MULTIPATH=y
    CONFIG_IP_ROUTE_VERBOSE=y
    CONFIG_IP_MROUTE_COMMON=y
    CONFIG_IP_MROUTE=y
    CONFIG_IP_MROUTE_MULTIPLE_TABLES=y
    CONFIG_IP_PIMSM_V1=y
    CONFIG_IP_PIMSM_V2=y
    CONFIG_SYN_COOKIES=y
    CONFIG_INET_TABLE_PERTURB_ORDER=16
    CONFIG_TCP_CONG_ADVANCED=y
    CONFIG_TCP_CONG_CUBIC=y
    CONFIG_DEFAULT_CUBIC=y
    CONFIG_DEFAULT_TCP_CONG="cubic"
    CONFIG_TCP_MD5SIG=y
    CONFIG_IPV6=y
    CONFIG_IPV6_ROUTER_PREF=y
    CONFIG_IPV6_ROUTE_INFO=y
    CONFIG_IPV6_MULTIPLE_TABLES=y
    CONFIG_IPV6_SUBTREES=y
    CONFIG_IPV6_MROUTE=y
    CONFIG_IPV6_MROUTE_MULTIPLE_TABLES=y
    CONFIG_IPV6_PIMSM_V2=y
    CONFIG_IPV6_SEG6_LWTUNNEL=y
    CONFIG_IPV6_SEG6_HMAC=y
    CONFIG_IPV6_SEG6_BPF=y
    CONFIG_IPV6_IOAM6_LWTUNNEL=y
    CONFIG_NETLABEL=y
    CONFIG_MPTCP=y
    CONFIG_MPTCP_IPV6=y
    CONFIG_NETWORK_SECMARK=y
    CONFIG_NET_PTP_CLASSIFY=y
    CONFIG_NETWORK_PHY_TIMESTAMPING=y
    CONFIG_NETFILTER=y
    CONFIG_NETFILTER_ADVANCED=y
    
    #
    # Core Netfilter Configuration
    #
    CONFIG_NETFILTER_INGRESS=y
    CONFIG_NETFILTER_NETLINK=y
    CONFIG_NETFILTER_NETLINK_HOOK=y
    CONFIG_NETFILTER_NETLINK_ACCT=y
    CONFIG_NETFILTER_NETLINK_QUEUE=y
    CONFIG_NETFILTER_NETLINK_LOG=y
    CONFIG_NETFILTER_NETLINK_OSF=y
    CONFIG_NF_CONNTRACK=y
    CONFIG_NF_LOG_SYSLOG=y
    CONFIG_NETFILTER_CONNCOUNT=y
    CONFIG_NF_CONNTRACK_MARK=y
    CONFIG_NF_CONNTRACK_SECMARK=y
    CONFIG_NF_CONNTRACK_ZONES=y
    CONFIG_NF_CONNTRACK_PROCFS=y
    CONFIG_NF_CONNTRACK_EVENTS=y
    CONFIG_NF_CONNTRACK_TIMEOUT=y
    CONFIG_NF_CONNTRACK_TIMESTAMP=y
    CONFIG_NF_CONNTRACK_LABELS=y
    CONFIG_NF_CT_PROTO_DCCP=y
    CONFIG_NF_CT_PROTO_SCTP=y
    CONFIG_NF_CT_PROTO_UDPLITE=y
    CONFIG_NF_NAT=y
    CONFIG_NF_NAT_REDIRECT=y
    CONFIG_NF_NAT_MASQUERADE=y
    CONFIG_NETFILTER_SYNPROXY=y
    CONFIG_NF_TABLES=y
    CONFIG_NF_TABLES_INET=y
    CONFIG_NF_TABLES_NETDEV=y
    CONFIG_NFT_NUMGEN=y
    CONFIG_NFT_CT=y
    CONFIG_NFT_COUNTER=y
    CONFIG_NFT_CONNLIMIT=y
    CONFIG_NFT_LOG=y
    CONFIG_NFT_LIMIT=y
    CONFIG_NFT_MASQ=y
    CONFIG_NFT_REDIR=y
    CONFIG_NFT_NAT=y
    CONFIG_NFT_TUNNEL=y
    CONFIG_NFT_OBJREF=y
    CONFIG_NFT_QUEUE=y
    CONFIG_NFT_QUOTA=y
    CONFIG_NFT_REJECT=y
    CONFIG_NFT_REJECT_INET=y
    CONFIG_NFT_COMPAT=m
    CONFIG_NFT_HASH=y
    CONFIG_NFT_SOCKET=y
    CONFIG_NFT_OSF=y
    CONFIG_NFT_TPROXY=y
    CONFIG_NFT_SYNPROXY=y
    CONFIG_NF_DUP_NETDEV=y
    CONFIG_NFT_DUP_NETDEV=y
    CONFIG_NFT_FWD_NETDEV=y
    CONFIG_NFT_REJECT_NETDEV=y
    CONFIG_NF_FLOW_TABLE=y
    CONFIG_NETFILTER_XTABLES=y
    CONFIG_NETFILTER_XTABLES_COMPAT=y
    
    #
    # Xtables combined modules
    #
    CONFIG_NETFILTER_XT_MARK=m
    
    #
    # Xtables targets
    #
    CONFIG_NETFILTER_XT_TARGET_HL=y
    CONFIG_NETFILTER_XT_NAT=y
    CONFIG_NETFILTER_XT_TARGET_NETMAP=m
    CONFIG_NETFILTER_XT_TARGET_REDIRECT=m
    CONFIG_NETFILTER_XT_TARGET_MASQUERADE=y
    
    #
    # Xtables matches
    #
    CONFIG_NETFILTER_XT_MATCH_HL=y
    # end of Core Netfilter Configuration
    
    
    #
    # IP: Netfilter Configuration
    #
    CONFIG_NF_DEFRAG_IPV4=y
    CONFIG_NF_SOCKET_IPV4=y
    CONFIG_NF_TPROXY_IPV4=y
    CONFIG_NF_TABLES_IPV4=y
    CONFIG_NFT_REJECT_IPV4=y
    CONFIG_NF_REJECT_IPV4=y
    CONFIG_IP_NF_IPTABLES=m
    CONFIG_IP_NF_FILTER=m
    CONFIG_IP_NF_NAT=m
    CONFIG_IP_NF_TARGET_MASQUERADE=m
    CONFIG_IP_NF_TARGET_NETMAP=m
    CONFIG_IP_NF_TARGET_REDIRECT=m
    # end of IP: Netfilter Configuration
    
    #
    # IPv6: Netfilter Configuration
    #
    CONFIG_NF_SOCKET_IPV6=y
    CONFIG_NF_TPROXY_IPV6=y
    CONFIG_NF_TABLES_IPV6=y
    CONFIG_NFT_REJECT_IPV6=y
    CONFIG_NF_REJECT_IPV6=y
    CONFIG_NF_LOG_IPV6=y
    CONFIG_IP6_NF_IPTABLES=y
    CONFIG_IP6_NF_MATCH_AH=y
    CONFIG_IP6_NF_MATCH_EUI64=y
    CONFIG_IP6_NF_MATCH_FRAG=y
    CONFIG_IP6_NF_MATCH_OPTS=y
    CONFIG_IP6_NF_MATCH_HL=y
    CONFIG_IP6_NF_MATCH_IPV6HEADER=y
    CONFIG_IP6_NF_MATCH_MH=y
    CONFIG_IP6_NF_MATCH_RPFILTER=y
    CONFIG_IP6_NF_MATCH_RT=y
    CONFIG_IP6_NF_MATCH_SRH=y
    CONFIG_IP6_NF_TARGET_HL=y
    CONFIG_IP6_NF_FILTER=y
    CONFIG_IP6_NF_TARGET_REJECT=y
    CONFIG_IP6_NF_TARGET_SYNPROXY=y
    CONFIG_IP6_NF_MANGLE=y
    CONFIG_IP6_NF_RAW=y
    CONFIG_IP6_NF_SECURITY=y
    CONFIG_IP6_NF_NAT=y
    CONFIG_IP6_NF_TARGET_MASQUERADE=y
    CONFIG_IP6_NF_TARGET_NPT=y
    # end of IPv6: Netfilter Configuration
    
    CONFIG_NF_DEFRAG_IPV6=y
    CONFIG_BPFILTER=y
    CONFIG_NET_DSA=y
    CONFIG_NET_DSA_TAG_OCELOT_8021Q=y
    CONFIG_VLAN_8021Q=y
    CONFIG_NET_SCHED=y
    
    #
    # Queueing/Scheduling
    #
    CONFIG_NET_SCH_FQ_CODEL=m
    
    #
    # Classification
    #
    CONFIG_NET_CLS=y
    CONFIG_NET_EMATCH=y
    CONFIG_NET_EMATCH_STACK=32
    CONFIG_NET_CLS_ACT=y
    CONFIG_NET_ACT_NAT=y
    CONFIG_NET_TC_SKB_EXT=y
    CONFIG_NET_SCH_FIFO=y
    CONFIG_DCB=y
    CONFIG_DNS_RESOLVER=y
    CONFIG_MPLS=y
    CONFIG_NET_SWITCHDEV=y
    CONFIG_NET_L3_MASTER_DEV=y
    CONFIG_NET_NCSI=y
    CONFIG_NCSI_OEM_CMD_GET_MAC=y
    CONFIG_PCPU_DEV_REFCNT=y
    CONFIG_RPS=y
    CONFIG_RFS_ACCEL=y
    CONFIG_SOCK_RX_QUEUE_MAPPING=y
    CONFIG_XPS=y
    CONFIG_CGROUP_NET_PRIO=y
    CONFIG_CGROUP_NET_CLASSID=y
    CONFIG_NET_RX_BUSY_POLL=y
    CONFIG_BQL=y
    CONFIG_BPF_STREAM_PARSER=y
    CONFIG_NET_FLOW_LIMIT=y
    
    #
    # Network testing
    #
    CONFIG_NET_DROP_MONITOR=y
    # end of Network testing
    # end of Networking options
    
    CONFIG_HAMRADIO=y
    
    #
    # Packet Radio protocols
    #
    CONFIG_STREAM_PARSER=y
    CONFIG_FIB_RULES=y
    CONFIG_WIRELESS=y
    
    #
    # CFG80211 needs to be enabled for MAC80211
    #
    CONFIG_MAC80211_STA_HASH_MAX_SIZE=0
    CONFIG_RFKILL=y
    CONFIG_RFKILL_LEDS=y
    CONFIG_RFKILL_INPUT=y
    CONFIG_LWTUNNEL=y
    CONFIG_LWTUNNEL_BPF=y
    CONFIG_DST_CACHE=y
    CONFIG_GRO_CELLS=y
    CONFIG_NET_SELFTESTS=y
    CONFIG_NET_SOCK_MSG=y
    CONFIG_NET_DEVLINK=y
    CONFIG_PAGE_POOL=y
    CONFIG_ETHTOOL_NETLINK=y
    
    #
    # Device Drivers
    #
    CONFIG_HAVE_EISA=y
    CONFIG_EISA=y
    CONFIG_EISA_VLB_PRIMING=y
    CONFIG_EISA_PCI_EISA=y
    CONFIG_EISA_VIRTUAL_ROOT=y
    CONFIG_EISA_NAMES=y
    CONFIG_HAVE_PCI=y
    CONFIG_PCI=y
    CONFIG_PCI_DOMAINS=y
    CONFIG_PCIEPORTBUS=y
    CONFIG_HOTPLUG_PCI_PCIE=y
    CONFIG_PCIEAER=y
    CONFIG_PCIEASPM=y
    CONFIG_PCIEASPM_DEFAULT=y
    CONFIG_PCIE_PME=y
    CONFIG_PCIE_DPC=y
    CONFIG_PCIE_PTM=y
    CONFIG_PCIE_EDR=y
    CONFIG_PCI_MSI=y
    CONFIG_PCI_MSI_IRQ_DOMAIN=y
    CONFIG_PCI_QUIRKS=y
    CONFIG_PCI_REALLOC_ENABLE_AUTO=y
    CONFIG_PCI_ATS=y
    CONFIG_PCI_LOCKLESS_CONFIG=y
    CONFIG_PCI_IOV=y
    CONFIG_PCI_PRI=y
    CONFIG_PCI_PASID=y
    CONFIG_PCI_LABEL=y
    CONFIG_PCIE_BUS_DEFAULT=y
    CONFIG_HOTPLUG_PCI=y
    CONFIG_HOTPLUG_PCI_ACPI=y
    CONFIG_HOTPLUG_PCI_CPCI=y
    CONFIG_HOTPLUG_PCI_SHPC=y
    
    #
    # PCI controller drivers
    #
    
    #
    # DesignWare PCI Core Support
    #
    CONFIG_PCIE_DW=y
    CONFIG_PCIE_DW_HOST=y
    CONFIG_PCIE_DW_EP=y
    CONFIG_PCIE_DW_PLAT=y
    CONFIG_PCIE_DW_PLAT_HOST=y
    CONFIG_PCIE_DW_PLAT_EP=y
    # end of DesignWare PCI Core Support
    
    #
    # Mobiveil PCIe Core Support
    #
    # end of Mobiveil PCIe Core Support
    
    #
    # Cadence PCIe controllers support
    #
    # end of Cadence PCIe controllers support
    # end of PCI controller drivers
    
    #
    # PCI Endpoint
    #
    CONFIG_PCI_ENDPOINT=y
    CONFIG_PCI_ENDPOINT_CONFIGFS=y
    # end of PCI Endpoint
    
    #
    # PCI switch controller drivers
    #
    # end of PCI switch controller drivers
    
    CONFIG_RAPIDIO=y
    CONFIG_RAPIDIO_DISC_TIMEOUT=30
    CONFIG_RAPIDIO_DMA_ENGINE=y
    
    #
    # RapidIO Switch drivers
    #
    # end of RapidIO Switch drivers
    
    #
    # Generic Driver Options
    #
    CONFIG_AUXILIARY_BUS=y
    CONFIG_UEVENT_HELPER=y
    CONFIG_UEVENT_HELPER_PATH=""
    CONFIG_DEVTMPFS=y
    CONFIG_DEVTMPFS_MOUNT=y
    CONFIG_PREVENT_FIRMWARE_BUILD=y
    
    #
    # Firmware loader
    #
    CONFIG_FW_LOADER=y
    CONFIG_FW_LOADER_PAGED_BUF=y
    CONFIG_EXTRA_FIRMWARE=""
    CONFIG_FW_LOADER_USER_HELPER=y
    CONFIG_FW_LOADER_COMPRESS=y
    CONFIG_FW_CACHE=y
    # end of Firmware loader
    
    CONFIG_WANT_DEV_COREDUMP=y
    CONFIG_ALLOW_DEV_COREDUMP=y
    CONFIG_DEV_COREDUMP=y
    CONFIG_HMEM_REPORTING=y
    CONFIG_SYS_HYPERVISOR=y
    CONFIG_GENERIC_CPU_AUTOPROBE=y
    CONFIG_GENERIC_CPU_VULNERABILITIES=y
    CONFIG_REGMAP=y
    CONFIG_REGMAP_I2C=y
    CONFIG_REGMAP_SPI=y
    CONFIG_REGMAP_MMIO=y
    CONFIG_REGMAP_IRQ=y
    CONFIG_DMA_SHARED_BUFFER=y
    # end of Generic Driver Options
    
    #
    # Bus devices
    #
    # end of Bus devices
    
    CONFIG_CONNECTOR=y
    CONFIG_PROC_EVENTS=y
    
    #
    # Firmware Drivers
    #
    
    #
    # ARM System Control and Management Interface Protocol
    #
    # end of ARM System Control and Management Interface Protocol
    
    CONFIG_EDD=y
    CONFIG_EDD_OFF=y
    CONFIG_FIRMWARE_MEMMAP=y
    CONFIG_DMIID=y
    CONFIG_DMI_SCAN_MACHINE_NON_EFI_FALLBACK=y
    CONFIG_SYSFB=y
    
    #
    # EFI (Extensible Firmware Interface) Support
    #
    CONFIG_EFI_VARS=y
    CONFIG_EFI_ESRT=y
    CONFIG_EFI_VARS_PSTORE=m
    CONFIG_EFI_RUNTIME_MAP=y
    CONFIG_EFI_SOFT_RESERVE=y
    CONFIG_EFI_RUNTIME_WRAPPERS=y
    CONFIG_EFI_GENERIC_STUB_INITRD_CMDLINE_LOADER=y
    CONFIG_APPLE_PROPERTIES=y
    CONFIG_RESET_ATTACK_MITIGATION=y
    CONFIG_EFI_RCI2_TABLE=y
    # end of EFI (Extensible Firmware Interface) Support
    
    CONFIG_UEFI_CPER=y
    CONFIG_UEFI_CPER_X86=y
    CONFIG_EFI_DEV_PATH_PARSER=y
    CONFIG_EFI_EARLYCON=y
    CONFIG_EFI_CUSTOM_SSDT_OVERLAYS=y
    
    #
    # Tegra firmware driver
    #
    # end of Tegra firmware driver
    # end of Firmware Drivers
    
    CONFIG_ARCH_MIGHT_HAVE_PC_PARPORT=y
    CONFIG_PNP=y
    
    #
    # Protocols
    #
    CONFIG_PNPACPI=y
    CONFIG_BLK_DEV=y
    CONFIG_CDROM=y
    CONFIG_BLK_DEV_LOOP=y
    CONFIG_BLK_DEV_LOOP_MIN_COUNT=8
    CONFIG_XEN_BLKDEV_FRONTEND=y
    
    #
    # NVME Support
    #
    # end of NVME Support
    
    #
    # Misc devices
    #
    CONFIG_SRAM=y
    
    #
    # EEPROM support
    #
    # end of EEPROM support
    
    
    #
    # Texas Instruments shared transport line discipline
    #
    # end of Texas Instruments shared transport line discipline
    
    CONFIG_INTEL_MEI=m
    CONFIG_INTEL_MEI_ME=m
    CONFIG_PVPANIC=y
    # end of Misc devices
    
    #
    # SCSI device support
    #
    CONFIG_SCSI_MOD=y
    CONFIG_SCSI_COMMON=y
    CONFIG_SCSI=y
    CONFIG_SCSI_DMA=y
    CONFIG_SCSI_PROC_FS=y
    
    #
    # SCSI support type (disk, tape, CD-ROM)
    #
    CONFIG_BLK_DEV_SD=y
    CONFIG_BLK_DEV_SR=y
    CONFIG_CHR_DEV_SG=y
    CONFIG_BLK_DEV_BSG=y
    CONFIG_SCSI_CONSTANTS=y
    CONFIG_SCSI_LOGGING=y
    CONFIG_SCSI_SCAN_ASYNC=y
    
    #
    # SCSI Transports
    #
    # end of SCSI Transports
    
    CONFIG_SCSI_LOWLEVEL=y
    CONFIG_MEGARAID_NEWGEN=y
    CONFIG_MEGARAID_SAS=m
    CONFIG_SCSI_DH=y
    CONFIG_SCSI_DH_RDAC=m
    CONFIG_SCSI_DH_EMC=m
    CONFIG_SCSI_DH_ALUA=m
    # end of SCSI device support
    
    CONFIG_ATA=y
    CONFIG_SATA_HOST=y
    CONFIG_PATA_TIMINGS=y
    CONFIG_ATA_VERBOSE_ERROR=y
    CONFIG_ATA_FORCE=y
    CONFIG_ATA_ACPI=y
    CONFIG_SATA_ZPODD=y
    CONFIG_SATA_PMP=y
    
    #
    # Controllers with non-SFF native interface
    #
    CONFIG_SATA_AHCI=m
    CONFIG_SATA_MOBILE_LPM_POLICY=3
    CONFIG_SATA_AHCI_PLATFORM=m
    CONFIG_SATA_ACARD_AHCI=m
    CONFIG_ATA_SFF=y
    
    #
    # SFF controllers with custom DMA interface
    #
    CONFIG_ATA_BMDMA=y
    
    #
    # SATA SFF controllers with BMDMA
    #
    CONFIG_ATA_PIIX=y
    
    #
    # PATA SFF controllers with BMDMA
    #
    CONFIG_PATA_SIS=y
    
    #
    # PIO-only SFF controllers
    #
    
    #
    # Generic fallback / legacy drivers
    #
    CONFIG_ATA_GENERIC=y
    CONFIG_MD=y
    CONFIG_BLK_DEV_MD=y
    CONFIG_MD_AUTODETECT=y
    CONFIG_MD_LINEAR=m
    CONFIG_MD_RAID0=m
    CONFIG_MD_RAID1=m
    CONFIG_MD_RAID10=m
    CONFIG_MD_RAID456=m
    CONFIG_MD_MULTIPATH=m
    CONFIG_BLK_DEV_DM_BUILTIN=y
    CONFIG_BLK_DEV_DM=y
    CONFIG_DM_MULTIPATH=m
    CONFIG_DM_INIT=y
    CONFIG_DM_UEVENT=y
    CONFIG_FUSION=y
    CONFIG_FUSION_MAX_SGE=128
    CONFIG_FUSION_LOGGING=y
    
    #
    # IEEE 1394 (FireWire) support
    #
    # end of IEEE 1394 (FireWire) support
    
    CONFIG_MACINTOSH_DRIVERS=y
    CONFIG_MAC_EMUMOUSEBTN=m
    CONFIG_NETDEVICES=y
    CONFIG_NET_CORE=y
    CONFIG_NET_FC=y
    CONFIG_TUN=y
    CONFIG_NET_VRF=y
    
    #
    # Distributed Switch Architecture drivers
    #
    # end of Distributed Switch Architecture drivers
    
    CONFIG_ETHERNET=y
    CONFIG_MDIO=y
    CONFIG_NET_VENDOR_3COM=y
    CONFIG_NET_VENDOR_ADAPTEC=y
    CONFIG_NET_VENDOR_AGERE=y
    CONFIG_NET_VENDOR_ALACRITECH=y
    CONFIG_NET_VENDOR_ALTEON=y
    CONFIG_NET_VENDOR_AMAZON=y
    CONFIG_NET_VENDOR_AMD=y
    CONFIG_NET_VENDOR_AQUANTIA=y
    CONFIG_NET_VENDOR_ARC=y
    CONFIG_NET_VENDOR_ATHEROS=y
    CONFIG_NET_VENDOR_BROADCOM=y
    CONFIG_TIGON3=m
    CONFIG_TIGON3_HWMON=y
    CONFIG_NET_VENDOR_CADENCE=y
    CONFIG_NET_VENDOR_CAVIUM=y
    CONFIG_NET_VENDOR_CHELSIO=y
    CONFIG_NET_VENDOR_CIRRUS=y
    CONFIG_NET_VENDOR_CISCO=y
    CONFIG_NET_VENDOR_CORTINA=y
    CONFIG_NET_VENDOR_DEC=y
    CONFIG_NET_TULIP=y
    CONFIG_NET_VENDOR_DLINK=y
    CONFIG_NET_VENDOR_EMULEX=y
    CONFIG_NET_VENDOR_EZCHIP=y
    CONFIG_NET_VENDOR_GOOGLE=y
    CONFIG_NET_VENDOR_HUAWEI=y
    CONFIG_NET_VENDOR_I825XX=y
    CONFIG_NET_VENDOR_INTEL=y
    CONFIG_IXGBE=y
    CONFIG_IXGBE_HWMON=y
    CONFIG_IXGBE_DCB=y
    CONFIG_IXGBEVF=y
    CONFIG_I40E=m
    CONFIG_I40E_DCB=y
    CONFIG_ICE=m
    CONFIG_NET_VENDOR_LITEX=y
    CONFIG_NET_VENDOR_MARVELL=y
    CONFIG_NET_VENDOR_MELLANOX=y
    CONFIG_NET_VENDOR_MICREL=y
    CONFIG_NET_VENDOR_MICROCHIP=y
    CONFIG_NET_VENDOR_MICROSEMI=y
    CONFIG_NET_VENDOR_MICROSOFT=y
    CONFIG_NET_VENDOR_MYRI=y
    CONFIG_NET_VENDOR_NI=y
    CONFIG_NET_VENDOR_NATSEMI=y
    CONFIG_NET_VENDOR_NETERION=y
    CONFIG_NET_VENDOR_NETRONOME=y
    CONFIG_NET_VENDOR_8390=y
    CONFIG_NET_VENDOR_NVIDIA=y
    CONFIG_NET_VENDOR_OKI=y
    CONFIG_NET_VENDOR_PACKET_ENGINES=y
    CONFIG_NET_VENDOR_PENSANDO=y
    CONFIG_NET_VENDOR_QLOGIC=y
    CONFIG_NET_VENDOR_BROCADE=y
    CONFIG_NET_VENDOR_QUALCOMM=y
    CONFIG_NET_VENDOR_RDC=y
    CONFIG_NET_VENDOR_REALTEK=y
    CONFIG_NET_VENDOR_RENESAS=y
    CONFIG_NET_VENDOR_ROCKER=y
    CONFIG_NET_VENDOR_SAMSUNG=y
    CONFIG_NET_VENDOR_SEEQ=y
    CONFIG_NET_VENDOR_SILAN=y
    CONFIG_NET_VENDOR_SIS=y
    CONFIG_NET_VENDOR_SOLARFLARE=y
    CONFIG_NET_VENDOR_SMSC=y
    CONFIG_NET_VENDOR_SOCIONEXT=y
    CONFIG_NET_VENDOR_STMICRO=y
    CONFIG_NET_VENDOR_SUN=y
    CONFIG_NET_VENDOR_SYNOPSYS=y
    CONFIG_NET_VENDOR_TEHUTI=y
    CONFIG_NET_VENDOR_TI=y
    CONFIG_NET_VENDOR_VIA=y
    CONFIG_NET_VENDOR_WIZNET=y
    CONFIG_NET_VENDOR_XILINX=y
    CONFIG_FDDI=y
    CONFIG_PHYLINK=y
    CONFIG_PHYLIB=y
    CONFIG_SWPHY=y
    CONFIG_LED_TRIGGER_PHY=y
    CONFIG_FIXED_PHY=y
    
    #
    # MII PHY device drivers
    #
    CONFIG_BCM84881_PHY=y
    CONFIG_MDIO_DEVICE=y
    CONFIG_MDIO_BUS=y
    CONFIG_FWNODE_MDIO=y
    CONFIG_ACPI_MDIO=y
    CONFIG_MDIO_DEVRES=y
    
    #
    # MDIO Multiplexers
    #
    
    #
    # PCS device drivers
    #
    # end of PCS device drivers
    
    CONFIG_PPP=y
    CONFIG_PPP_FILTER=y
    CONFIG_PPP_MULTILINK=y
    CONFIG_SLHC=y
    CONFIG_WLAN=y
    CONFIG_WLAN_VENDOR_ADMTEK=y
    CONFIG_WLAN_VENDOR_ATH=y
    CONFIG_ATH5K_PCI=y
    CONFIG_WLAN_VENDOR_ATMEL=y
    CONFIG_WLAN_VENDOR_BROADCOM=y
    CONFIG_WLAN_VENDOR_CISCO=y
    CONFIG_WLAN_VENDOR_INTEL=y
    CONFIG_WLAN_VENDOR_INTERSIL=y
    CONFIG_WLAN_VENDOR_MARVELL=y
    CONFIG_WLAN_VENDOR_MEDIATEK=y
    CONFIG_WLAN_VENDOR_MICROCHIP=y
    CONFIG_WLAN_VENDOR_RALINK=y
    CONFIG_WLAN_VENDOR_REALTEK=y
    CONFIG_WLAN_VENDOR_RSI=y
    CONFIG_WLAN_VENDOR_ST=y
    CONFIG_WLAN_VENDOR_TI=y
    CONFIG_WLAN_VENDOR_ZYDAS=y
    CONFIG_WLAN_VENDOR_QUANTENNA=y
    CONFIG_WAN=y
    
    #
    # Wireless WAN
    #
    CONFIG_WWAN=y
    # end of Wireless WAN
    
    CONFIG_XEN_NETDEV_FRONTEND=y
    CONFIG_ISDN=y
    
    #
    # Input device support
    #
    CONFIG_INPUT=y
    CONFIG_INPUT_LEDS=m
    
    #
    # Userland interfaces
    #
    CONFIG_INPUT_MOUSEDEV=y
    CONFIG_INPUT_MOUSEDEV_PSAUX=y
    CONFIG_INPUT_MOUSEDEV_SCREEN_X=1024
    CONFIG_INPUT_MOUSEDEV_SCREEN_Y=768
    CONFIG_INPUT_JOYDEV=m
    CONFIG_INPUT_EVDEV=y
    
    #
    # Input Device Drivers
    #
    CONFIG_INPUT_KEYBOARD=y
    CONFIG_KEYBOARD_ATKBD=y
    CONFIG_INPUT_MOUSE=y
    CONFIG_INPUT_JOYSTICK=y
    CONFIG_INPUT_TABLET=y
    CONFIG_INPUT_TOUCHSCREEN=y
    CONFIG_TOUCHSCREEN_ELAN=y
    CONFIG_INPUT_MISC=y
    CONFIG_INPUT_UINPUT=y
    
    #
    # Hardware I/O ports
    #
    CONFIG_SERIO=y
    CONFIG_ARCH_MIGHT_HAVE_PC_SERIO=y
    CONFIG_SERIO_I8042=y
    CONFIG_SERIO_LIBPS2=y
    # end of Hardware I/O ports
    # end of Input device support
    
    #
    # Character devices
    #
    CONFIG_TTY=y
    CONFIG_VT=y
    CONFIG_CONSOLE_TRANSLATIONS=y
    CONFIG_VT_CONSOLE=y
    CONFIG_VT_CONSOLE_SLEEP=y
    CONFIG_HW_CONSOLE=y
    CONFIG_VT_HW_CONSOLE_BINDING=y
    CONFIG_UNIX98_PTYS=y
    CONFIG_LEGACY_PTYS=y
    CONFIG_LEGACY_PTY_COUNT=0
    CONFIG_LDISC_AUTOLOAD=y
    
    #
    # Serial drivers
    #
    CONFIG_SERIAL_EARLYCON=y
    CONFIG_SERIAL_8250=y
    CONFIG_SERIAL_8250_PNP=y
    CONFIG_SERIAL_8250_16550A_VARIANTS=y
    CONFIG_SERIAL_8250_FINTEK=y
    CONFIG_SERIAL_8250_CONSOLE=y
    CONFIG_SERIAL_8250_DMA=y
    CONFIG_SERIAL_8250_PCI=y
    CONFIG_SERIAL_8250_NR_UARTS=48
    CONFIG_SERIAL_8250_RUNTIME_UARTS=32
    CONFIG_SERIAL_8250_EXTENDED=y
    CONFIG_SERIAL_8250_MANY_PORTS=y
    CONFIG_SERIAL_8250_SHARE_IRQ=y
    CONFIG_SERIAL_8250_RSA=y
    CONFIG_SERIAL_8250_RT288X=y
    CONFIG_SERIAL_8250_MID=y
    
    #
    # Non-8250 serial port support
    #
    CONFIG_SERIAL_KGDB_NMI=y
    CONFIG_SERIAL_MAX310X=y
    CONFIG_SERIAL_CORE=y
    CONFIG_SERIAL_CORE_CONSOLE=y
    CONFIG_CONSOLE_POLL=y
    CONFIG_SERIAL_SCCNXP=y
    CONFIG_SERIAL_SCCNXP_CONSOLE=y
    # end of Serial drivers
    
    CONFIG_SERIAL_MCTRL_GPIO=y
    CONFIG_SERIAL_NONSTANDARD=y
    CONFIG_HVC_DRIVER=y
    CONFIG_HVC_IRQ=y
    CONFIG_HVC_XEN=y
    CONFIG_HVC_XEN_FRONTEND=y
    CONFIG_SERIAL_DEV_BUS=y
    CONFIG_SERIAL_DEV_CTRL_TTYPORT=y
    CONFIG_TTY_PRINTK=y
    CONFIG_TTY_PRINTK_LEVEL=6
    CONFIG_VIRTIO_CONSOLE=y
    CONFIG_IPMI_HANDLER=m
    CONFIG_IPMI_DMI_DECODE=y
    CONFIG_IPMI_PLAT_DATA=y
    CONFIG_IPMI_DEVICE_INTERFACE=m
    CONFIG_IPMI_SI=m
    CONFIG_IPMI_SSIF=m
    CONFIG_HW_RANDOM=y
    CONFIG_DEVMEM=y
    CONFIG_DEVPORT=y
    CONFIG_HPET=y
    CONFIG_HPET_MMAP=y
    CONFIG_HPET_MMAP_DEFAULT=y
    CONFIG_TCG_TPM=y
    CONFIG_HW_RANDOM_TPM=y
    CONFIG_TCG_TIS_CORE=y
    CONFIG_TCG_TIS=y
    CONFIG_TCG_CRB=y
    CONFIG_RANDOM_TRUST_CPU=y
    CONFIG_RANDOM_TRUST_BOOTLOADER=y
    # end of Character devices
    
    #
    # I2C support
    #
    CONFIG_I2C=y
    CONFIG_ACPI_I2C_OPREGION=y
    CONFIG_I2C_BOARDINFO=y
    CONFIG_I2C_COMPAT=y
    CONFIG_I2C_CHARDEV=y
    CONFIG_I2C_HELPER_AUTO=y
    CONFIG_I2C_SMBUS=m
    CONFIG_I2C_ALGOBIT=m
    
    #
    # I2C Hardware Bus support
    #
    
    #
    # PC SMBus host controller drivers
    #
    CONFIG_I2C_I801=m
    
    #
    # ACPI drivers
    #
    
    #
    # I2C system bus drivers (mostly embedded / system-on-chip)
    #
    CONFIG_I2C_DESIGNWARE_CORE=y
    CONFIG_I2C_DESIGNWARE_PLATFORM=y
    CONFIG_I2C_DESIGNWARE_BAYTRAIL=y
    
    #
    # External I2C/SMBus adapter drivers
    #
    
    #
    # Other I2C/SMBus bus drivers
    #
    # end of I2C Hardware Bus support
    
    # end of I2C support
    
    CONFIG_SPI=y
    CONFIG_SPI_MASTER=y
    CONFIG_SPI_MEM=y
    
    #
    # SPI Master Controller Drivers
    #
    
    #
    # SPI Multiplexer support
    #
    
    #
    # SPI Protocol Masters
    #
    CONFIG_SPI_SLAVE=y
    CONFIG_SPI_DYNAMIC=y
    CONFIG_PPS=y
    
    #
    # PPS clients support
    #
    
    #
    # PPS generators support
    #
    
    #
    # PTP clock support
    #
    CONFIG_PTP_1588_CLOCK=y
    CONFIG_PTP_1588_CLOCK_OPTIONAL=y
    # end of PTP clock support
    
    CONFIG_PINCTRL=y
    CONFIG_PINMUX=y
    CONFIG_PINCONF=y
    CONFIG_GENERIC_PINCONF=y
    CONFIG_PINCTRL_AMD=y
    CONFIG_PINCTRL_SX150X=y
    CONFIG_PINCTRL_BAYTRAIL=y
    CONFIG_PINCTRL_CHERRYVIEW=y
    CONFIG_PINCTRL_INTEL=y
    
    #
    # Renesas pinctrl drivers
    #
    # end of Renesas pinctrl drivers
    
    CONFIG_GPIOLIB=y
    CONFIG_GPIOLIB_FASTPATH_LIMIT=512
    CONFIG_GPIO_ACPI=y
    CONFIG_GPIOLIB_IRQCHIP=y
    CONFIG_GPIO_SYSFS=y
    CONFIG_GPIO_CDEV=y
    CONFIG_GPIO_CDEV_V1=y
    
    #
    # Memory mapped GPIO drivers
    #
    # end of Memory mapped GPIO drivers
    
    #
    # Port-mapped I/O GPIO drivers
    #
    # end of Port-mapped I/O GPIO drivers
    
    #
    # I2C GPIO expanders
    #
    # end of I2C GPIO expanders
    
    #
    # MFD GPIO expanders
    #
    CONFIG_GPIO_CRYSTAL_COVE=y
    CONFIG_GPIO_PALMAS=y
    CONFIG_GPIO_RC5T583=y
    CONFIG_GPIO_TPS6586X=y
    CONFIG_GPIO_TPS65910=y
    # end of MFD GPIO expanders
    
    #
    # PCI GPIO expanders
    #
    # end of PCI GPIO expanders
    
    #
    # SPI GPIO expanders
    #
    # end of SPI GPIO expanders
    
    #
    # USB GPIO expanders
    #
    # end of USB GPIO expanders
    
    #
    # Virtual GPIO drivers
    #
    # end of Virtual GPIO drivers
    
    CONFIG_POWER_RESET=y
    CONFIG_POWER_RESET_RESTART=y
    CONFIG_POWER_SUPPLY=y
    CONFIG_POWER_SUPPLY_HWMON=y
    CONFIG_CHARGER_MANAGER=y
    CONFIG_HWMON=y
    
    #
    # Native drivers
    #
    CONFIG_SENSORS_CORETEMP=m
    
    #
    # ACPI drivers
    #
    CONFIG_SENSORS_ACPI_POWER=m
    CONFIG_THERMAL=y
    CONFIG_THERMAL_NETLINK=y
    CONFIG_THERMAL_STATISTICS=y
    CONFIG_THERMAL_EMERGENCY_POWEROFF_DELAY_MS=0
    CONFIG_THERMAL_HWMON=y
    CONFIG_THERMAL_WRITABLE_TRIPS=y
    CONFIG_THERMAL_DEFAULT_GOV_STEP_WISE=y
    CONFIG_THERMAL_GOV_FAIR_SHARE=y
    CONFIG_THERMAL_GOV_STEP_WISE=y
    CONFIG_THERMAL_GOV_BANG_BANG=y
    CONFIG_THERMAL_GOV_USER_SPACE=y
    CONFIG_THERMAL_GOV_POWER_ALLOCATOR=y
    CONFIG_DEVFREQ_THERMAL=y
    CONFIG_THERMAL_EMULATION=y
    
    #
    # Intel thermal drivers
    #
    CONFIG_INTEL_POWERCLAMP=m
    CONFIG_X86_THERMAL_VECTOR=y
    CONFIG_X86_PKG_TEMP_THERMAL=m
    
    #
    # ACPI INT340X thermal drivers
    #
    # end of ACPI INT340X thermal drivers
    
    CONFIG_INTEL_PCH_THERMAL=m
    # end of Intel thermal drivers
    
    CONFIG_WATCHDOG=y
    CONFIG_WATCHDOG_CORE=y
    CONFIG_WATCHDOG_HANDLE_BOOT_ENABLED=y
    CONFIG_WATCHDOG_OPEN_TIMEOUT=0
    CONFIG_WATCHDOG_SYSFS=y
    
    #
    # Watchdog Pretimeout Governors
    #
    CONFIG_WATCHDOG_PRETIMEOUT_GOV=y
    CONFIG_WATCHDOG_PRETIMEOUT_GOV_SEL=m
    CONFIG_WATCHDOG_PRETIMEOUT_GOV_NOOP=y
    CONFIG_WATCHDOG_PRETIMEOUT_DEFAULT_GOV_NOOP=y
    
    #
    # Watchdog Device Drivers
    #
    
    #
    # PCI-based Watchdog Cards
    #
    
    #
    # USB-based Watchdog Cards
    #
    CONFIG_SSB_POSSIBLE=y
    CONFIG_BCMA_POSSIBLE=y
    
    #
    # Multifunction device drivers
    #
    CONFIG_MFD_CORE=y
    CONFIG_MFD_AS3711=y
    CONFIG_PMIC_ADP5520=y
    CONFIG_MFD_AAT2870_CORE=y
    CONFIG_PMIC_DA903X=y
    CONFIG_PMIC_DA9052=y
    CONFIG_MFD_DA9052_SPI=y
    CONFIG_MFD_DA9052_I2C=y
    CONFIG_MFD_DA9055=y
    CONFIG_MFD_DA9063=y
    CONFIG_HTC_I2CPLD=y
    CONFIG_INTEL_SOC_PMIC=y
    CONFIG_INTEL_SOC_PMIC_CHTWC=y
    CONFIG_MFD_INTEL_PMT=m
    CONFIG_MFD_88PM860X=y
    CONFIG_MFD_MAX14577=y
    CONFIG_MFD_MAX77693=y
    CONFIG_MFD_MAX77843=y
    CONFIG_MFD_MAX8925=y
    CONFIG_MFD_MAX8997=y
    CONFIG_MFD_MAX8998=y
    CONFIG_EZX_PCAP=y
    CONFIG_MFD_RC5T583=y
    CONFIG_MFD_SYSCON=y
    CONFIG_MFD_LP8788=y
    CONFIG_MFD_PALMAS=y
    CONFIG_MFD_TPS65090=y
    CONFIG_MFD_TPS6586X=y
    CONFIG_MFD_TPS65910=y
    CONFIG_MFD_TPS65912=y
    CONFIG_MFD_TPS65912_I2C=y
    CONFIG_MFD_TPS65912_SPI=y
    CONFIG_MFD_TPS80031=y
    CONFIG_TWL4030_CORE=y
    CONFIG_MFD_TWL4030_AUDIO=y
    CONFIG_TWL6040_CORE=y
    CONFIG_MFD_WM8400=y
    CONFIG_MFD_WM831X=y
    CONFIG_MFD_WM831X_I2C=y
    CONFIG_MFD_WM831X_SPI=y
    CONFIG_MFD_WM8350=y
    CONFIG_MFD_WM8350_I2C=y
    # end of Multifunction device drivers
    
    CONFIG_REGULATOR=y
    CONFIG_RC_CORE=m
    CONFIG_LIRC=y
    CONFIG_RC_DECODERS=y
    CONFIG_RC_DEVICES=y
    CONFIG_CEC_CORE=m
    CONFIG_MEDIA_CEC_RC=y
    CONFIG_MEDIA_CEC_SUPPORT=y
    
    #
    # Graphics support
    #
    CONFIG_AGP=y
    CONFIG_AGP_AMD64=y
    CONFIG_AGP_INTEL=y
    CONFIG_AGP_VIA=y
    CONFIG_INTEL_GTT=y
    CONFIG_VGA_ARB=y
    CONFIG_VGA_ARB_MAX_GPUS=16
    CONFIG_VGA_SWITCHEROO=y
    CONFIG_DRM=m
    CONFIG_DRM_DP_AUX_CHARDEV=y
    CONFIG_DRM_KMS_HELPER=m
    CONFIG_DRM_FBDEV_EMULATION=y
    CONFIG_DRM_FBDEV_OVERALLOC=100
    CONFIG_DRM_LOAD_EDID_FIRMWARE=y
    CONFIG_DRM_DP_CEC=y
    CONFIG_DRM_GEM_SHMEM_HELPER=y
    
    #
    # I2C encoder or helper chips
    #
    # end of I2C encoder or helper chips
    
    #
    # ARM devices
    #
    # end of ARM devices
    
    CONFIG_DRM_MGAG200=m
    CONFIG_DRM_PANEL=y
    
    #
    # Display Panels
    #
    # end of Display Panels
    
    CONFIG_DRM_BRIDGE=y
    CONFIG_DRM_PANEL_BRIDGE=y
    
    #
    # Display Interface Bridges
    #
    # end of Display Interface Bridges
    
    CONFIG_DRM_PANEL_ORIENTATION_QUIRKS=y
    
    #
    # Frame buffer Devices
    #
    CONFIG_FB_CMDLINE=y
    CONFIG_FB_NOTIFY=y
    CONFIG_FB=y
    CONFIG_FIRMWARE_EDID=y
    CONFIG_FB_BOOT_VESA_SUPPORT=y
    CONFIG_FB_CFB_FILLRECT=y
    CONFIG_FB_CFB_COPYAREA=y
    CONFIG_FB_CFB_IMAGEBLIT=y
    CONFIG_FB_SYS_FILLRECT=m
    CONFIG_FB_SYS_COPYAREA=m
    CONFIG_FB_SYS_IMAGEBLIT=m
    CONFIG_FB_SYS_FOPS=m
    CONFIG_FB_DEFERRED_IO=y
    CONFIG_FB_MODE_HELPERS=y
    CONFIG_FB_TILEBLITTING=y
    
    #
    # Frame buffer hardware drivers
    #
    CONFIG_FB_ASILIANT=y
    CONFIG_FB_IMSTT=y
    CONFIG_FB_VESA=y
    CONFIG_FB_EFI=y
    # end of Frame buffer Devices
    
    #
    # Backlight & LCD device support
    #
    CONFIG_BACKLIGHT_CLASS_DEVICE=y
    # end of Backlight & LCD device support
    
    CONFIG_HDMI=y
    
    #
    # Console display driver support
    #
    CONFIG_VGA_CONSOLE=y
    CONFIG_DUMMY_CONSOLE=y
    CONFIG_DUMMY_CONSOLE_COLUMNS=80
    CONFIG_DUMMY_CONSOLE_ROWS=25
    CONFIG_FRAMEBUFFER_CONSOLE=y
    CONFIG_FRAMEBUFFER_CONSOLE_DETECT_PRIMARY=y
    CONFIG_FRAMEBUFFER_CONSOLE_ROTATION=y
    CONFIG_FRAMEBUFFER_CONSOLE_DEFERRED_TAKEOVER=y
    # end of Console display driver support
    
    # end of Graphics support
    
    
    #
    # HID support
    #
    CONFIG_HID=m
    CONFIG_HID_BATTERY_STRENGTH=y
    CONFIG_HIDRAW=y
    CONFIG_HID_GENERIC=m
    
    #
    # Special HID drivers
    #
    # end of Special HID drivers
    
    #
    # USB HID support
    #
    CONFIG_USB_HID=m
    CONFIG_HID_PID=y
    CONFIG_USB_HIDDEV=y
    
    #
    # USB HID Boot Protocol drivers
    #
    # end of USB HID Boot Protocol drivers
    # end of USB HID support
    
    #
    # I2C HID support
    #
    # end of I2C HID support
    
    #
    # Intel ISH HID support
    #
    # end of Intel ISH HID support
    
    #
    # AMD SFH HID Support
    #
    # end of AMD SFH HID Support
    # end of HID support
    
    CONFIG_USB_OHCI_LITTLE_ENDIAN=y
    CONFIG_USB_SUPPORT=y
    CONFIG_USB_COMMON=y
    CONFIG_USB_LED_TRIG=y
    CONFIG_USB_ARCH_HAS_HCD=y
    CONFIG_USB=y
    CONFIG_USB_PCI=y
    CONFIG_USB_ANNOUNCE_NEW_DEVICES=y
    
    #
    # Miscellaneous USB options
    #
    CONFIG_USB_DEFAULT_PERSIST=y
    CONFIG_USB_DYNAMIC_MINORS=y
    CONFIG_USB_AUTOSUSPEND_DELAY=2
    
    #
    # USB Host Controller Drivers
    #
    CONFIG_USB_XHCI_HCD=y
    CONFIG_USB_XHCI_DBGCAP=y
    CONFIG_USB_XHCI_PCI=m
    CONFIG_USB_XHCI_PCI_RENESAS=m
    CONFIG_USB_EHCI_HCD=y
    CONFIG_USB_EHCI_ROOT_HUB_TT=y
    CONFIG_USB_EHCI_TT_NEWSCHED=y
    CONFIG_USB_EHCI_PCI=y
    CONFIG_USB_EHCI_HCD_PLATFORM=y
    CONFIG_USB_OHCI_HCD=y
    CONFIG_USB_OHCI_HCD_PCI=y
    CONFIG_USB_OHCI_HCD_PLATFORM=y
    CONFIG_USB_UHCI_HCD=y
    
    #
    # USB Device Class drivers
    #
    
    #
    # NOTE: USB_STORAGE depends on SCSI but BLK_DEV_SD may
    #
    
    #
    # also be needed; see USB_STORAGE Help for more info
    #
    
    #
    # USB Imaging devices
    #
    CONFIG_USB_DWC2=y
    CONFIG_USB_DWC2_HOST=y
    
    #
    # Gadget/Dual-role mode requires USB Gadget support to be enabled
    #
    
    #
    # USB port drivers
    #
    
    #
    # USB Miscellaneous drivers
    #
    
    #
    # USB Physical Layer drivers
    #
    # end of USB Physical Layer drivers
    
    CONFIG_USB_ROLE_SWITCH=y
    CONFIG_MMC=y
    CONFIG_MMC_CRYPTO=y
    
    #
    # MMC/SD/SDIO Host Controller Drivers
    #
    CONFIG_NEW_LEDS=y
    CONFIG_LEDS_CLASS=y
    CONFIG_LEDS_BRIGHTNESS_HW_CHANGED=y
    
    #
    # LED drivers
    #
    
    #
    # LED driver for blink(1) USB RGB LED is under Special HID drivers (HID_THINGM)
    #
    
    #
    # Flash and Torch LED drivers
    #
    
    #
    # LED Triggers
    #
    CONFIG_LEDS_TRIGGERS=y
    CONFIG_LEDS_TRIGGER_DISK=y
    CONFIG_LEDS_TRIGGER_CPU=y
    
    #
    # iptables trigger is under Netfilter config (LED target)
    #
    CONFIG_LEDS_TRIGGER_PANIC=y
    CONFIG_ACCESSIBILITY=y
    
    #
    # Speakup console speech
    #
    # end of Speakup console speech
    
    CONFIG_INFINIBAND=m
    CONFIG_INFINIBAND_USER_ACCESS=m
    CONFIG_INFINIBAND_USER_MEM=y
    CONFIG_INFINIBAND_ON_DEMAND_PAGING=y
    CONFIG_INFINIBAND_ADDR_TRANS=y
    CONFIG_INFINIBAND_ADDR_TRANS_CONFIGFS=y
    CONFIG_INFINIBAND_VIRT_DMA=y
    CONFIG_INFINIBAND_IRDMA=m
    CONFIG_EDAC_ATOMIC_SCRUB=y
    CONFIG_EDAC_SUPPORT=y
    CONFIG_EDAC=y
    CONFIG_EDAC_GHES=y
    CONFIG_EDAC_I10NM=m
    CONFIG_RTC_LIB=y
    CONFIG_RTC_MC146818_LIB=y
    CONFIG_RTC_CLASS=y
    CONFIG_RTC_HCTOSYS=y
    CONFIG_RTC_HCTOSYS_DEVICE="rtc0"
    CONFIG_RTC_SYSTOHC=y
    CONFIG_RTC_SYSTOHC_DEVICE="rtc0"
    CONFIG_RTC_NVMEM=y
    
    #
    # RTC interfaces
    #
    CONFIG_RTC_INTF_SYSFS=y
    CONFIG_RTC_INTF_PROC=y
    CONFIG_RTC_INTF_DEV=y
    
    #
    # I2C RTC drivers
    #
    
    #
    # SPI RTC drivers
    #
    CONFIG_RTC_I2C_AND_SPI=y
    
    #
    # SPI and I2C RTC drivers
    #
    
    #
    # Platform RTC drivers
    #
    CONFIG_RTC_DRV_CMOS=y
    
    #
    # on-CPU RTC drivers
    #
    
    #
    # HID Sensor RTC drivers
    #
    CONFIG_DMADEVICES=y
    
    #
    # DMA Devices
    #
    CONFIG_DMA_ENGINE=y
    CONFIG_DMA_VIRTUAL_CHANNELS=y
    CONFIG_DMA_ACPI=y
    CONFIG_HSU_DMA=y
    CONFIG_INTEL_LDMA=y
    
    #
    # DMA Clients
    #
    CONFIG_ASYNC_TX_DMA=y
    
    #
    # DMABUF options
    #
    CONFIG_SYNC_FILE=y
    CONFIG_SW_SYNC=y
    CONFIG_UDMABUF=y
    CONFIG_DMABUF_HEAPS=y
    CONFIG_DMABUF_HEAPS_SYSTEM=y
    # end of DMABUF options
    
    CONFIG_AUXDISPLAY=y
    CONFIG_CHARLCD_BL_FLASH=y
    CONFIG_VFIO=y
    CONFIG_VFIO_IOMMU_TYPE1=y
    CONFIG_VFIO_VIRQFD=y
    CONFIG_VFIO_NOIOMMU=y
    CONFIG_VFIO_PCI_CORE=y
    CONFIG_VFIO_PCI_MMAP=y
    CONFIG_VFIO_PCI_INTX=y
    CONFIG_VFIO_PCI=y
    CONFIG_VFIO_PCI_VGA=y
    CONFIG_VFIO_PCI_IGD=y
    CONFIG_IRQ_BYPASS_MANAGER=y
    CONFIG_VIRT_DRIVERS=y
    CONFIG_VIRTIO=y
    CONFIG_ARCH_HAS_RESTRICTED_VIRTIO_MEMORY_ACCESS=y
    CONFIG_VIRTIO_PCI_LIB=y
    CONFIG_VIRTIO_MENU=y
    CONFIG_VIRTIO_PCI=y
    CONFIG_VIRTIO_PCI_LEGACY=y
    CONFIG_VIRTIO_BALLOON=y
    CONFIG_VIRTIO_MMIO=y
    CONFIG_VIRTIO_MMIO_CMDLINE_DEVICES=y
    CONFIG_VHOST_MENU=y
    
    #
    # Microsoft Hyper-V guest support
    #
    # end of Microsoft Hyper-V guest support
    
    #
    # Xen driver support
    #
    CONFIG_XEN_BALLOON=y
    CONFIG_XEN_BALLOON_MEMORY_HOTPLUG=y
    CONFIG_XEN_MEMORY_HOTPLUG_LIMIT=512
    CONFIG_XEN_SCRUB_PAGES_DEFAULT=y
    CONFIG_XEN_BACKEND=y
    CONFIG_XEN_SYS_HYPERVISOR=y
    CONFIG_XEN_XENBUS_FRONTEND=y
    CONFIG_XEN_GRANT_DMA_ALLOC=y
    CONFIG_SWIOTLB_XEN=y
    CONFIG_XEN_PRIVCMD=m
    CONFIG_XEN_ACPI_PROCESSOR=y
    CONFIG_XEN_MCE_LOG=y
    CONFIG_XEN_HAVE_PVMMU=y
    CONFIG_XEN_EFI=y
    CONFIG_XEN_AUTO_XLATE=y
    CONFIG_XEN_ACPI=y
    CONFIG_XEN_HAVE_VPMU=y
    CONFIG_XEN_UNPOPULATED_ALLOC=y
    # end of Xen driver support
    
    CONFIG_STAGING=y
    CONFIG_STAGING_MEDIA=y
    
    #
    # Android
    #
    # end of Android
    
    CONFIG_UNISYSSPAR=y
    CONFIG_X86_PLATFORM_DEVICES=y
    CONFIG_ACPI_WMI=m
    CONFIG_WMI_BMOF=m
    CONFIG_X86_PLATFORM_DRIVERS_DELL=y
    CONFIG_DCDBAS=m
    CONFIG_DELL_SMBIOS=m
    CONFIG_DELL_SMBIOS_WMI=y
    CONFIG_DELL_SMBIOS_SMM=y
    CONFIG_DELL_WMI_DESCRIPTOR=m
    CONFIG_INTEL_PMC_CORE=y
    
    #
    # Intel Speed Select Technology interface support
    #
    CONFIG_INTEL_SPEED_SELECT_INTERFACE=m
    # end of Intel Speed Select Technology interface support
    
    CONFIG_INTEL_TURBO_MAX_3=y
    CONFIG_INTEL_SCU_IPC=y
    CONFIG_INTEL_SCU=y
    CONFIG_INTEL_SCU_PCI=y
    CONFIG_PMC_ATOM=y
    CONFIG_CHROME_PLATFORMS=y
    CONFIG_MELLANOX_PLATFORM=y
    CONFIG_SURFACE_PLATFORMS=y
    CONFIG_HAVE_CLK=y
    CONFIG_HAVE_CLK_PREPARE=y
    CONFIG_COMMON_CLK=y
    
    #
    # Clock driver for ARM Reference designs
    #
    CONFIG_ICST=y
    CONFIG_CLK_SP810=y
    # end of Clock driver for ARM Reference designs
    
    CONFIG_HWSPINLOCK=y
    
    #
    # Clock Source drivers
    #
    CONFIG_CLKEVT_I8253=y
    CONFIG_I8253_LOCK=y
    CONFIG_CLKBLD_I8253=y
    # end of Clock Source drivers
    
    CONFIG_MAILBOX=y
    CONFIG_PCC=y
    CONFIG_IOMMU_IOVA=y
    CONFIG_IOASID=y
    CONFIG_IOMMU_API=y
    CONFIG_IOMMU_SUPPORT=y
    
    #
    # Generic IOMMU Pagetable Support
    #
    CONFIG_IOMMU_IO_PGTABLE=y
    # end of Generic IOMMU Pagetable Support
    
    CONFIG_IOMMU_DEFAULT_DMA_LAZY=y
    CONFIG_IOMMU_DMA=y
    CONFIG_IOMMU_SVA_LIB=y
    CONFIG_AMD_IOMMU=y
    CONFIG_DMAR_TABLE=y
    CONFIG_INTEL_IOMMU=y
    CONFIG_INTEL_IOMMU_SVM=y
    CONFIG_INTEL_IOMMU_FLOPPY_WA=y
    CONFIG_IRQ_REMAP=y
    CONFIG_VIRTIO_IOMMU=y
    
    #
    # Remoteproc drivers
    #
    CONFIG_REMOTEPROC=y
    CONFIG_REMOTEPROC_CDEV=y
    # end of Remoteproc drivers
    
    #
    # Rpmsg drivers
    #
    # end of Rpmsg drivers
    
    
    #
    # SOC (System On Chip) specific Drivers
    #
    
    #
    # Amlogic SoC drivers
    #
    # end of Amlogic SoC drivers
    
    #
    # Broadcom SoC drivers
    #
    # end of Broadcom SoC drivers
    
    #
    # NXP/Freescale QorIQ SoC drivers
    #
    # end of NXP/Freescale QorIQ SoC drivers
    
    #
    # i.MX SoC drivers
    #
    # end of i.MX SoC drivers
    
    #
    # Enable LiteX SoC Builder specific drivers
    #
    # end of Enable LiteX SoC Builder specific drivers
    
    #
    # Qualcomm SoC drivers
    #
    # end of Qualcomm SoC drivers
    
    CONFIG_SOC_TI=y
    
    #
    # Xilinx SoC drivers
    #
    # end of Xilinx SoC drivers
    # end of SOC (System On Chip) specific Drivers
    
    CONFIG_PM_DEVFREQ=y
    
    #
    # DEVFREQ Governors
    #
    CONFIG_DEVFREQ_GOV_SIMPLE_ONDEMAND=y
    CONFIG_DEVFREQ_GOV_PERFORMANCE=y
    CONFIG_DEVFREQ_GOV_POWERSAVE=y
    CONFIG_DEVFREQ_GOV_USERSPACE=y
    CONFIG_DEVFREQ_GOV_PASSIVE=y
    
    #
    # DEVFREQ Drivers
    #
    CONFIG_PM_DEVFREQ_EVENT=y
    CONFIG_EXTCON=y
    
    #
    # Extcon Device Drivers
    #
    CONFIG_MEMORY=y
    CONFIG_VME_BUS=y
    
    #
    # VME Bridge Drivers
    #
    
    #
    # VME Board Drivers
    #
    
    #
    # VME Device Drivers
    #
    CONFIG_PWM=y
    CONFIG_PWM_SYSFS=y
    CONFIG_PWM_CRC=y
    CONFIG_PWM_LPSS=y
    CONFIG_PWM_LPSS_PCI=y
    CONFIG_PWM_LPSS_PLATFORM=y
    
    #
    # IRQ chip support
    #
    # end of IRQ chip support
    
    CONFIG_RESET_CONTROLLER=y
    
    #
    # PHY Subsystem
    #
    CONFIG_GENERIC_PHY=y
    # end of PHY Subsystem
    
    CONFIG_POWERCAP=y
    CONFIG_INTEL_RAPL_CORE=m
    CONFIG_INTEL_RAPL=m
    CONFIG_IDLE_INJECT=y
    CONFIG_DTPM=y
    CONFIG_DTPM_CPU=y
    
    #
    # Performance monitor support
    #
    # end of Performance monitor support
    
    CONFIG_RAS=y
    CONFIG_RAS_CEC=y
    
    #
    # Android
    #
    CONFIG_ANDROID=y
    # end of Android
    
    CONFIG_LIBNVDIMM=y
    CONFIG_ND_CLAIM=y
    CONFIG_BTT=y
    CONFIG_NVDIMM_PFN=y
    CONFIG_NVDIMM_DAX=y
    CONFIG_NVDIMM_KEYS=y
    CONFIG_DAX=y
    CONFIG_NVMEM=y
    CONFIG_NVMEM_SYSFS=y
    
    #
    # HW tracing support
    #
    # end of HW tracing support
    
    CONFIG_PM_OPP=y
    CONFIG_INTERCONNECT=y
    # end of Device Drivers
    
    #
    # File systems
    #
    CONFIG_DCACHE_WORD_ACCESS=y
    CONFIG_VALIDATE_FS_PARSER=y
    CONFIG_FS_IOMAP=y
    CONFIG_EXT4_FS=y
    CONFIG_EXT4_USE_FOR_EXT2=y
    CONFIG_EXT4_FS_POSIX_ACL=y
    CONFIG_EXT4_FS_SECURITY=y
    CONFIG_JBD2=y
    CONFIG_FS_MBCACHE=y
    CONFIG_BTRFS_FS=m
    CONFIG_BTRFS_FS_POSIX_ACL=y
    CONFIG_FS_DAX=y
    CONFIG_FS_DAX_PMD=y
    CONFIG_FS_POSIX_ACL=y
    CONFIG_EXPORTFS=y
    CONFIG_EXPORTFS_BLOCK_OPS=y
    CONFIG_FILE_LOCKING=y
    CONFIG_FS_ENCRYPTION=y
    CONFIG_FS_ENCRYPTION_ALGS=y
    CONFIG_FS_ENCRYPTION_INLINE_CRYPT=y
    CONFIG_FS_VERITY=y
    CONFIG_FS_VERITY_BUILTIN_SIGNATURES=y
    CONFIG_FSNOTIFY=y
    CONFIG_DNOTIFY=y
    CONFIG_INOTIFY_USER=y
    CONFIG_FANOTIFY=y
    CONFIG_FANOTIFY_ACCESS_PERMISSIONS=y
    CONFIG_QUOTA=y
    CONFIG_QUOTA_NETLINK_INTERFACE=y
    CONFIG_QUOTACTL=y
    CONFIG_AUTOFS_FS=m
    CONFIG_FUSE_FS=y
    
    #
    # Caches
    #
    # end of Caches
    
    #
    # CD-ROM/DVD Filesystems
    #
    # end of CD-ROM/DVD Filesystems
    
    #
    # DOS/FAT/EXFAT/NT Filesystems
    #
    CONFIG_FAT_FS=y
    CONFIG_VFAT_FS=y
    CONFIG_FAT_DEFAULT_CODEPAGE=437
    CONFIG_FAT_DEFAULT_IOCHARSET="iso8859-1"
    # end of DOS/FAT/EXFAT/NT Filesystems
    
    #
    # Pseudo filesystems
    #
    CONFIG_PROC_FS=y
    CONFIG_PROC_KCORE=y
    CONFIG_PROC_VMCORE=y
    CONFIG_PROC_VMCORE_DEVICE_DUMP=y
    CONFIG_PROC_SYSCTL=y
    CONFIG_PROC_PAGE_MONITOR=y
    CONFIG_PROC_CHILDREN=y
    CONFIG_PROC_PID_ARCH_STATUS=y
    CONFIG_PROC_CPU_RESCTRL=y
    CONFIG_KERNFS=y
    CONFIG_SYSFS=y
    CONFIG_TMPFS=y
    CONFIG_TMPFS_POSIX_ACL=y
    CONFIG_TMPFS_XATTR=y
    CONFIG_TMPFS_INODE64=y
    CONFIG_HUGETLBFS=y
    CONFIG_HUGETLB_PAGE=y
    CONFIG_HUGETLB_PAGE_FREE_VMEMMAP=y
    CONFIG_MEMFD_CREATE=y
    CONFIG_ARCH_HAS_GIGANTIC_PAGE=y
    CONFIG_CONFIGFS_FS=y
    CONFIG_EFIVAR_FS=y
    # end of Pseudo filesystems
    
    CONFIG_MISC_FILESYSTEMS=y
    CONFIG_ECRYPT_FS=y
    CONFIG_ECRYPT_FS_MESSAGING=y
    CONFIG_SQUASHFS=y
    CONFIG_SQUASHFS_FILE_DIRECT=y
    CONFIG_SQUASHFS_DECOMP_SINGLE=y
    CONFIG_SQUASHFS_XATTR=y
    CONFIG_SQUASHFS_ZLIB=y
    CONFIG_SQUASHFS_LZ4=y
    CONFIG_SQUASHFS_LZO=y
    CONFIG_SQUASHFS_XZ=y
    CONFIG_SQUASHFS_ZSTD=y
    CONFIG_SQUASHFS_FRAGMENT_CACHE_SIZE=3
    CONFIG_PSTORE=y
    CONFIG_PSTORE_DEFAULT_KMSG_BYTES=10240
    CONFIG_PSTORE_DEFLATE_COMPRESS=y
    CONFIG_PSTORE_COMPRESS=y
    CONFIG_PSTORE_DEFLATE_COMPRESS_DEFAULT=y
    CONFIG_PSTORE_COMPRESS_DEFAULT="deflate"
    CONFIG_PSTORE_RAM=m
    CONFIG_PSTORE_ZONE=m
    CONFIG_PSTORE_BLK=m
    CONFIG_PSTORE_BLK_BLKDEV=""
    CONFIG_PSTORE_BLK_KMSG_SIZE=64
    CONFIG_PSTORE_BLK_MAX_REASON=2
    CONFIG_NETWORK_FILESYSTEMS=y
    CONFIG_NLS=y
    CONFIG_NLS_DEFAULT="utf8"
    CONFIG_NLS_CODEPAGE_437=y
    CONFIG_NLS_ISO8859_1=m
    CONFIG_UNICODE=y
    CONFIG_IO_WQ=y
    # end of File systems
    
    #
    # Security options
    #
    CONFIG_KEYS=y
    CONFIG_KEYS_REQUEST_CACHE=y
    CONFIG_PERSISTENT_KEYRINGS=y
    CONFIG_TRUSTED_KEYS=y
    CONFIG_ENCRYPTED_KEYS=y
    CONFIG_KEY_DH_OPERATIONS=y
    CONFIG_KEY_NOTIFICATIONS=y
    CONFIG_SECURITY_DMESG_RESTRICT=y
    CONFIG_SECURITY=y
    CONFIG_SECURITYFS=y
    CONFIG_SECURITY_NETWORK=y
    CONFIG_SECURITY_INFINIBAND=y
    CONFIG_SECURITY_PATH=y
    CONFIG_INTEL_TXT=y
    CONFIG_LSM_MMAP_MIN_ADDR=0
    CONFIG_HAVE_HARDENED_USERCOPY_ALLOCATOR=y
    CONFIG_HARDENED_USERCOPY=y
    CONFIG_FORTIFY_SOURCE=y
    CONFIG_SECURITY_SELINUX=y
    CONFIG_SECURITY_SELINUX_BOOTPARAM=y
    CONFIG_SECURITY_SELINUX_DEVELOP=y
    CONFIG_SECURITY_SELINUX_AVC_STATS=y
    CONFIG_SECURITY_SELINUX_CHECKREQPROT_VALUE=1
    CONFIG_SECURITY_SELINUX_SIDTAB_HASH_BITS=9
    CONFIG_SECURITY_SELINUX_SID2STR_CACHE_SIZE=256
    CONFIG_SECURITY_SMACK=y
    CONFIG_SECURITY_SMACK_NETFILTER=y
    CONFIG_SECURITY_SMACK_APPEND_SIGNALS=y
    CONFIG_SECURITY_TOMOYO=y
    CONFIG_SECURITY_TOMOYO_MAX_ACCEPT_ENTRY=2048
    CONFIG_SECURITY_TOMOYO_MAX_AUDIT_LOG=1024
    CONFIG_SECURITY_TOMOYO_POLICY_LOADER="/sbin/tomoyo-init"
    CONFIG_SECURITY_TOMOYO_ACTIVATION_TRIGGER="/sbin/init"
    CONFIG_SECURITY_APPARMOR=y
    CONFIG_SECURITY_APPARMOR_HASH=y
    CONFIG_SECURITY_APPARMOR_HASH_DEFAULT=y
    CONFIG_SECURITY_YAMA=y
    CONFIG_SECURITY_SAFESETID=y
    CONFIG_SECURITY_LOCKDOWN_LSM=y
    CONFIG_SECURITY_LOCKDOWN_LSM_EARLY=y
    CONFIG_LOCK_DOWN_KERNEL_FORCE_NONE=y
    CONFIG_SECURITY_LANDLOCK=y
    CONFIG_INTEGRITY=y
    CONFIG_INTEGRITY_SIGNATURE=y
    CONFIG_INTEGRITY_ASYMMETRIC_KEYS=y
    CONFIG_INTEGRITY_TRUSTED_KEYRING=y
    CONFIG_INTEGRITY_PLATFORM_KEYRING=y
    CONFIG_LOAD_UEFI_KEYS=y
    CONFIG_INTEGRITY_AUDIT=y
    CONFIG_IMA=y
    CONFIG_IMA_MEASURE_PCR_IDX=10
    CONFIG_IMA_LSM_RULES=y
    CONFIG_IMA_NG_TEMPLATE=y
    CONFIG_IMA_DEFAULT_TEMPLATE="ima-ng"
    CONFIG_IMA_DEFAULT_HASH_SHA1=y
    CONFIG_IMA_DEFAULT_HASH="sha1"
    CONFIG_IMA_APPRAISE=y
    CONFIG_IMA_APPRAISE_BOOTPARAM=y
    CONFIG_IMA_APPRAISE_MODSIG=y
    CONFIG_IMA_TRUSTED_KEYRING=y
    CONFIG_IMA_MEASURE_ASYMMETRIC_KEYS=y
    CONFIG_IMA_QUEUE_EARLY_BOOT_KEYS=y
    CONFIG_EVM=y
    CONFIG_EVM_ATTR_FSUUID=y
    CONFIG_EVM_EXTRA_SMACK_XATTRS=y
    CONFIG_EVM_ADD_XATTRS=y
    CONFIG_DEFAULT_SECURITY_APPARMOR=y
    CONFIG_LSM="landlock,lockdown,yama,integrity,apparmor"
    
    #
    # Kernel hardening options
    #
    
    #
    # Memory initialization
    #
    CONFIG_INIT_STACK_NONE=y
    CONFIG_INIT_ON_ALLOC_DEFAULT_ON=y
    CONFIG_CC_HAS_ZERO_CALL_USED_REGS=y
    # end of Memory initialization
    # end of Kernel hardening options
    # end of Security options
    
    CONFIG_XOR_BLOCKS=m
    CONFIG_ASYNC_CORE=m
    CONFIG_ASYNC_MEMCPY=m
    CONFIG_ASYNC_XOR=m
    CONFIG_ASYNC_PQ=m
    CONFIG_ASYNC_RAID6_RECOV=m
    CONFIG_CRYPTO=y
    
    #
    # Crypto core or helper
    #
    CONFIG_CRYPTO_ALGAPI=y
    CONFIG_CRYPTO_ALGAPI2=y
    CONFIG_CRYPTO_AEAD=y
    CONFIG_CRYPTO_AEAD2=y
    CONFIG_CRYPTO_SKCIPHER=y
    CONFIG_CRYPTO_SKCIPHER2=y
    CONFIG_CRYPTO_HASH=y
    CONFIG_CRYPTO_HASH2=y
    CONFIG_CRYPTO_RNG=y
    CONFIG_CRYPTO_RNG2=y
    CONFIG_CRYPTO_RNG_DEFAULT=y
    CONFIG_CRYPTO_AKCIPHER2=y
    CONFIG_CRYPTO_AKCIPHER=y
    CONFIG_CRYPTO_KPP2=y
    CONFIG_CRYPTO_KPP=y
    CONFIG_CRYPTO_ACOMP2=y
    CONFIG_CRYPTO_MANAGER=y
    CONFIG_CRYPTO_MANAGER2=y
    CONFIG_CRYPTO_MANAGER_DISABLE_TESTS=y
    CONFIG_CRYPTO_GF128MUL=y
    CONFIG_CRYPTO_NULL=y
    CONFIG_CRYPTO_NULL2=y
    CONFIG_CRYPTO_CRYPTD=m
    CONFIG_CRYPTO_SIMD=m
    
    #
    # Public-key cryptography
    #
    CONFIG_CRYPTO_RSA=y
    CONFIG_CRYPTO_DH=y
    
    #
    # Authenticated Encryption with Associated Data
    #
    CONFIG_CRYPTO_GCM=y
    CONFIG_CRYPTO_SEQIV=y
    
    #
    # Block modes
    #
    CONFIG_CRYPTO_CBC=y
    CONFIG_CRYPTO_CTR=y
    CONFIG_CRYPTO_CTS=y
    CONFIG_CRYPTO_ECB=y
    CONFIG_CRYPTO_XTS=y
    
    #
    # Hash modes
    #
    CONFIG_CRYPTO_HMAC=y
    
    #
    # Digest
    #
    CONFIG_CRYPTO_CRC32C=y
    CONFIG_CRYPTO_CRC32C_INTEL=y
    CONFIG_CRYPTO_CRC32_PCLMUL=m
    CONFIG_CRYPTO_XXHASH=m
    CONFIG_CRYPTO_BLAKE2B=m
    CONFIG_CRYPTO_BLAKE2S_X86=y
    CONFIG_CRYPTO_CRCT10DIF=y
    CONFIG_CRYPTO_CRCT10DIF_PCLMUL=m
    CONFIG_CRYPTO_GHASH=y
    CONFIG_CRYPTO_MD5=y
    CONFIG_CRYPTO_SHA1=y
    CONFIG_CRYPTO_SHA256=y
    CONFIG_CRYPTO_SHA512=y
    CONFIG_CRYPTO_GHASH_CLMUL_NI_INTEL=m
    
    #
    # Ciphers
    #
    CONFIG_CRYPTO_AES=y
    CONFIG_CRYPTO_AES_NI_INTEL=m
    
    #
    # Compression
    #
    CONFIG_CRYPTO_DEFLATE=y
    CONFIG_CRYPTO_LZO=y
    
    #
    # Random Number Generation
    #
    CONFIG_CRYPTO_DRBG_MENU=y
    CONFIG_CRYPTO_DRBG_HMAC=y
    CONFIG_CRYPTO_DRBG_HASH=y
    CONFIG_CRYPTO_DRBG_CTR=y
    CONFIG_CRYPTO_DRBG=y
    CONFIG_CRYPTO_JITTERENTROPY=y
    CONFIG_CRYPTO_HASH_INFO=y
    CONFIG_CRYPTO_HW=y
    CONFIG_CRYPTO_DEV_PADLOCK=y
    CONFIG_CRYPTO_DEV_CCP=y
    CONFIG_ASYMMETRIC_KEY_TYPE=y
    CONFIG_ASYMMETRIC_PUBLIC_KEY_SUBTYPE=y
    CONFIG_X509_CERTIFICATE_PARSER=y
    CONFIG_PKCS7_MESSAGE_PARSER=y
    CONFIG_SIGNED_PE_FILE_VERIFICATION=y
    
    #
    # Certificates for signature checking
    #
    CONFIG_MODULE_SIG_KEY="certs/signing_key.pem"
    CONFIG_MODULE_SIG_KEY_TYPE_RSA=y
    CONFIG_SYSTEM_TRUSTED_KEYRING=y
    CONFIG_SYSTEM_TRUSTED_KEYS=""
    CONFIG_SYSTEM_EXTRA_CERTIFICATE=y
    CONFIG_SYSTEM_EXTRA_CERTIFICATE_SIZE=4096
    CONFIG_SECONDARY_TRUSTED_KEYRING=y
    CONFIG_SYSTEM_BLACKLIST_KEYRING=y
    CONFIG_SYSTEM_BLACKLIST_HASH_LIST=""
    CONFIG_SYSTEM_REVOCATION_LIST=y
    CONFIG_SYSTEM_REVOCATION_KEYS=""
    # end of Certificates for signature checking
    
    CONFIG_BINARY_PRINTF=y
    
    #
    # Library routines
    #
    CONFIG_RAID6_PQ=m
    CONFIG_RAID6_PQ_BENCHMARK=y
    CONFIG_LINEAR_RANGES=y
    CONFIG_PACKING=y
    CONFIG_BITREVERSE=y
    CONFIG_GENERIC_STRNCPY_FROM_USER=y
    CONFIG_GENERIC_STRNLEN_USER=y
    CONFIG_GENERIC_NET_UTILS=y
    CONFIG_GENERIC_FIND_FIRST_BIT=y
    CONFIG_RATIONAL=y
    CONFIG_GENERIC_PCI_IOMAP=y
    CONFIG_GENERIC_IOMAP=y
    CONFIG_ARCH_USE_CMPXCHG_LOCKREF=y
    CONFIG_ARCH_HAS_FAST_MULTIPLIER=y
    CONFIG_ARCH_USE_SYM_ANNOTATIONS=y
    
    #
    # Crypto library routines
    #
    CONFIG_CRYPTO_LIB_AES=y
    CONFIG_CRYPTO_ARCH_HAVE_LIB_BLAKE2S=y
    CONFIG_CRYPTO_LIB_BLAKE2S_GENERIC=y
    CONFIG_CRYPTO_LIB_POLY1305_RSIZE=11
    CONFIG_CRYPTO_LIB_SHA256=y
    # end of Crypto library routines
    
    CONFIG_LIB_MEMNEQ=y
    CONFIG_CRC_CCITT=y
    CONFIG_CRC16=y
    CONFIG_CRC_T10DIF=y
    CONFIG_CRC32=y
    CONFIG_CRC32_SLICEBY8=y
    CONFIG_LIBCRC32C=y
    CONFIG_XXHASH=y
    CONFIG_ZLIB_INFLATE=y
    CONFIG_ZLIB_DEFLATE=y
    CONFIG_LZO_COMPRESS=y
    CONFIG_LZO_DECOMPRESS=y
    CONFIG_LZ4_DECOMPRESS=y
    CONFIG_ZSTD_COMPRESS=m
    CONFIG_ZSTD_DECOMPRESS=y
    CONFIG_XZ_DEC=y
    CONFIG_XZ_DEC_X86=y
    CONFIG_XZ_DEC_POWERPC=y
    CONFIG_XZ_DEC_IA64=y
    CONFIG_XZ_DEC_ARM=y
    CONFIG_XZ_DEC_ARMTHUMB=y
    CONFIG_XZ_DEC_SPARC=y
    CONFIG_XZ_DEC_BCJ=y
    CONFIG_DECOMPRESS_GZIP=y
    CONFIG_DECOMPRESS_BZIP2=y
    CONFIG_DECOMPRESS_LZMA=y
    CONFIG_DECOMPRESS_XZ=y
    CONFIG_DECOMPRESS_LZO=y
    CONFIG_DECOMPRESS_LZ4=y
    CONFIG_DECOMPRESS_ZSTD=y
    CONFIG_GENERIC_ALLOCATOR=y
    CONFIG_REED_SOLOMON=m
    CONFIG_REED_SOLOMON_ENC8=y
    CONFIG_REED_SOLOMON_DEC8=y
    CONFIG_INTERVAL_TREE=y
    CONFIG_XARRAY_MULTI=y
    CONFIG_ASSOCIATIVE_ARRAY=y
    CONFIG_HAS_IOMEM=y
    CONFIG_HAS_IOPORT_MAP=y
    CONFIG_HAS_DMA=y
    CONFIG_DMA_OPS=y
    CONFIG_NEED_SG_DMA_LENGTH=y
    CONFIG_NEED_DMA_MAP_STATE=y
    CONFIG_ARCH_DMA_ADDR_T_64BIT=y
    CONFIG_ARCH_HAS_FORCE_DMA_UNENCRYPTED=y
    CONFIG_SWIOTLB=y
    CONFIG_DMA_COHERENT_POOL=y
    CONFIG_SGL_ALLOC=y
    CONFIG_IOMMU_HELPER=y
    CONFIG_CHECK_SIGNATURE=y
    CONFIG_CPUMASK_OFFSTACK=y
    CONFIG_CPU_RMAP=y
    CONFIG_DQL=y
    CONFIG_GLOB=y
    CONFIG_NLATTR=y
    CONFIG_CLZ_TAB=y
    CONFIG_IRQ_POLL=y
    CONFIG_MPILIB=y
    CONFIG_SIGNATURE=y
    CONFIG_DIMLIB=y
    CONFIG_OID_REGISTRY=y
    CONFIG_UCS2_STRING=y
    CONFIG_HAVE_GENERIC_VDSO=y
    CONFIG_GENERIC_GETTIMEOFDAY=y
    CONFIG_GENERIC_VDSO_TIME_NS=y
    CONFIG_FONT_SUPPORT=y
    CONFIG_FONTS=y
    CONFIG_FONT_8x8=y
    CONFIG_FONT_8x16=y
    CONFIG_FONT_ACORN_8x8=y
    CONFIG_FONT_6x10=y
    CONFIG_FONT_TER16x32=y
    CONFIG_SG_POOL=y
    CONFIG_ARCH_HAS_PMEM_API=y
    CONFIG_MEMREGION=y
    CONFIG_ARCH_HAS_UACCESS_FLUSHCACHE=y
    CONFIG_ARCH_HAS_COPY_MC=y
    CONFIG_ARCH_STACKWALK=y
    CONFIG_SBITMAP=y
    # end of Library routines
    
    CONFIG_PLDMFW=y
    CONFIG_ASN1_ENCODER=y
    
    #
    # Kernel hacking
    #
    
    #
    # printk and dmesg options
    #
    CONFIG_PRINTK_TIME=y
    CONFIG_CONSOLE_LOGLEVEL_DEFAULT=7
    CONFIG_CONSOLE_LOGLEVEL_QUIET=4
    CONFIG_MESSAGE_LOGLEVEL_DEFAULT=4
    CONFIG_BOOT_PRINTK_DELAY=y
    CONFIG_DYNAMIC_DEBUG=y
    CONFIG_DYNAMIC_DEBUG_CORE=y
    CONFIG_SYMBOLIC_ERRNAME=y
    CONFIG_DEBUG_BUGVERBOSE=y
    # end of printk and dmesg options
    
    CONFIG_AS_HAS_NON_CONST_LEB128=y
    
    #
    # Compile-time checks and compiler options
    #
    CONFIG_DEBUG_INFO=y
    CONFIG_DEBUG_INFO_DWARF_TOOLCHAIN_DEFAULT=y
    CONFIG_GDB_SCRIPTS=y
    CONFIG_FRAME_WARN=1024
    CONFIG_SECTION_MISMATCH_WARN_ONLY=y
    CONFIG_FRAME_POINTER=y
    CONFIG_STACK_VALIDATION=y
    CONFIG_VMLINUX_MAP=y
    # end of Compile-time checks and compiler options
    
    #
    # Generic Kernel Debugging Instruments
    #
    CONFIG_MAGIC_SYSRQ=y
    CONFIG_MAGIC_SYSRQ_DEFAULT_ENABLE=0x01b6
    CONFIG_MAGIC_SYSRQ_SERIAL=y
    CONFIG_MAGIC_SYSRQ_SERIAL_SEQUENCE=""
    CONFIG_DEBUG_FS=y
    CONFIG_DEBUG_FS_ALLOW_ALL=y
    CONFIG_HAVE_ARCH_KGDB=y
    CONFIG_KGDB=y
    CONFIG_KGDB_HONOUR_BLOCKLIST=y
    CONFIG_KGDB_SERIAL_CONSOLE=y
    CONFIG_KGDB_LOW_LEVEL_TRAP=y
    CONFIG_KGDB_KDB=y
    CONFIG_KDB_DEFAULT_ENABLE=0x1
    CONFIG_KDB_KEYBOARD=y
    CONFIG_KDB_CONTINUE_CATASTROPHIC=0
    CONFIG_ARCH_HAS_EARLY_DEBUG=y
    CONFIG_ARCH_HAS_UBSAN_SANITIZE_ALL=y
    CONFIG_UBSAN=y
    CONFIG_CC_HAS_UBSAN_BOUNDS=y
    CONFIG_UBSAN_BOUNDS=y
    CONFIG_UBSAN_ONLY_BOUNDS=y
    CONFIG_UBSAN_SHIFT=y
    CONFIG_UBSAN_BOOL=y
    CONFIG_UBSAN_ENUM=y
    CONFIG_UBSAN_SANITIZE_ALL=y
    CONFIG_HAVE_ARCH_KCSAN=y
    CONFIG_HAVE_KCSAN_COMPILER=y
    # end of Generic Kernel Debugging Instruments
    
    CONFIG_DEBUG_KERNEL=y
    CONFIG_DEBUG_MISC=y
    
    #
    # Memory Debugging
    #
    CONFIG_PAGE_POISONING=y
    CONFIG_ARCH_HAS_DEBUG_WX=y
    CONFIG_DEBUG_WX=y
    CONFIG_GENERIC_PTDUMP=y
    CONFIG_PTDUMP_CORE=y
    CONFIG_HAVE_DEBUG_KMEMLEAK=y
    CONFIG_SCHED_STACK_END_CHECK=y
    CONFIG_ARCH_HAS_DEBUG_VM_PGTABLE=y
    CONFIG_ARCH_HAS_DEBUG_VIRTUAL=y
    CONFIG_HAVE_ARCH_KASAN=y
    CONFIG_HAVE_ARCH_KASAN_VMALLOC=y
    CONFIG_CC_HAS_KASAN_GENERIC=y
    CONFIG_CC_HAS_WORKING_NOSANITIZE_ADDRESS=y
    CONFIG_HAVE_ARCH_KFENCE=y
    CONFIG_KFENCE=y
    CONFIG_KFENCE_SAMPLE_INTERVAL=0
    CONFIG_KFENCE_NUM_OBJECTS=255
    CONFIG_KFENCE_STRESS_TEST_FAULTS=0
    # end of Memory Debugging
    
    
    #
    # Debug Oops, Lockups and Hangs
    #
    CONFIG_PANIC_ON_OOPS_VALUE=0
    CONFIG_PANIC_TIMEOUT=0
    CONFIG_LOCKUP_DETECTOR=y
    CONFIG_SOFTLOCKUP_DETECTOR=y
    CONFIG_BOOTPARAM_SOFTLOCKUP_PANIC_VALUE=0
    CONFIG_HARDLOCKUP_DETECTOR_PERF=y
    CONFIG_HARDLOCKUP_CHECK_TIMESTAMP=y
    CONFIG_HARDLOCKUP_DETECTOR=y
    CONFIG_BOOTPARAM_HARDLOCKUP_PANIC_VALUE=0
    CONFIG_DETECT_HUNG_TASK=y
    CONFIG_DEFAULT_HUNG_TASK_TIMEOUT=120
    CONFIG_BOOTPARAM_HUNG_TASK_PANIC_VALUE=0
    # end of Debug Oops, Lockups and Hangs
    
    #
    # Scheduler Debugging
    #
    CONFIG_SCHED_DEBUG=y
    CONFIG_SCHED_INFO=y
    CONFIG_SCHEDSTATS=y
    # end of Scheduler Debugging
    
    
    #
    # Lock Debugging (spinlocks, mutexes, etc...)
    #
    CONFIG_LOCK_DEBUGGING_SUPPORT=y
    # end of Lock Debugging (spinlocks, mutexes, etc...)
    
    CONFIG_STACKTRACE=y
    
    #
    # Debug kernel data structures
    #
    # end of Debug kernel data structures
    
    
    #
    # RCU Debugging
    #
    CONFIG_RCU_CPU_STALL_TIMEOUT=60
    # end of RCU Debugging
    
    CONFIG_USER_STACKTRACE_SUPPORT=y
    CONFIG_NOP_TRACER=y
    CONFIG_HAVE_FUNCTION_TRACER=y
    CONFIG_HAVE_FUNCTION_GRAPH_TRACER=y
    CONFIG_HAVE_DYNAMIC_FTRACE=y
    CONFIG_HAVE_DYNAMIC_FTRACE_WITH_REGS=y
    CONFIG_HAVE_DYNAMIC_FTRACE_WITH_DIRECT_CALLS=y
    CONFIG_HAVE_DYNAMIC_FTRACE_WITH_ARGS=y
    CONFIG_HAVE_FTRACE_MCOUNT_RECORD=y
    CONFIG_HAVE_SYSCALL_TRACEPOINTS=y
    CONFIG_HAVE_FENTRY=y
    CONFIG_HAVE_OBJTOOL_MCOUNT=y
    CONFIG_HAVE_C_RECORDMCOUNT=y
    CONFIG_TRACER_MAX_TRACE=y
    CONFIG_TRACE_CLOCK=y
    CONFIG_RING_BUFFER=y
    CONFIG_EVENT_TRACING=y
    CONFIG_CONTEXT_SWITCH_TRACER=y
    CONFIG_TRACING=y
    CONFIG_GENERIC_TRACER=y
    CONFIG_TRACING_SUPPORT=y
    CONFIG_FTRACE=y
    CONFIG_BOOTTIME_TRACING=y
    CONFIG_FUNCTION_TRACER=y
    CONFIG_FUNCTION_GRAPH_TRACER=y
    CONFIG_DYNAMIC_FTRACE=y
    CONFIG_DYNAMIC_FTRACE_WITH_REGS=y
    CONFIG_DYNAMIC_FTRACE_WITH_DIRECT_CALLS=y
    CONFIG_DYNAMIC_FTRACE_WITH_ARGS=y
    CONFIG_FUNCTION_PROFILER=y
    CONFIG_STACK_TRACER=y
    CONFIG_SCHED_TRACER=y
    CONFIG_HWLAT_TRACER=y
    CONFIG_MMIOTRACE=y
    CONFIG_FTRACE_SYSCALLS=y
    CONFIG_TRACER_SNAPSHOT=y
    CONFIG_BRANCH_PROFILE_NONE=y
    CONFIG_BLK_DEV_IO_TRACE=y
    CONFIG_KPROBE_EVENTS=y
    CONFIG_UPROBE_EVENTS=y
    CONFIG_BPF_EVENTS=y
    CONFIG_DYNAMIC_EVENTS=y
    CONFIG_PROBE_EVENTS=y
    CONFIG_BPF_KPROBE_OVERRIDE=y
    CONFIG_FTRACE_MCOUNT_RECORD=y
    CONFIG_FTRACE_MCOUNT_USE_CC=y
    CONFIG_TRACING_MAP=y
    CONFIG_SYNTH_EVENTS=y
    CONFIG_HIST_TRIGGERS=y
    CONFIG_TRACE_EVENT_INJECT=y
    CONFIG_SAMPLES=y
    CONFIG_ARCH_HAS_DEVMEM_IS_ALLOWED=y
    CONFIG_STRICT_DEVMEM=y
    
    #
    # x86 Debugging
    #
    CONFIG_EARLY_PRINTK_USB=y
    CONFIG_EARLY_PRINTK=y
    CONFIG_EARLY_PRINTK_DBGP=y
    CONFIG_EARLY_PRINTK_USB_XDBC=y
    CONFIG_HAVE_MMIOTRACE_SUPPORT=y
    CONFIG_IO_DELAY_0XED=y
    CONFIG_X86_DEBUG_FPU=y
    CONFIG_UNWINDER_FRAME_POINTER=y
    # end of x86 Debugging
    
    #
    # Kernel Testing and Coverage
    #
    CONFIG_FUNCTION_ERROR_INJECTION=y
    CONFIG_ARCH_HAS_KCOV=y
    CONFIG_CC_HAS_SANCOV_TRACE_PC=y
    CONFIG_RUNTIME_TESTING_MENU=y
    CONFIG_ARCH_USE_MEMTEST=y
    CONFIG_MEMTEST=y
    # end of Kernel Testing and Coverage
    # end of Kernel hacking
    
\end{lstlisting}
\end{multicols}

\end{document}

%%% Local Variables:
%%% mode: japanese-latex
%%% TeX-master: t
%%% End:
